% ---------------------------------------------------------------------
% Basic configuration of Beamera and Jagiellonian
% ---------------------------------------------------------------------
\RequirePackage[l2tabu, orthodox]{nag}



\ifx\PresentationStyle\notset
\def\PresentationStyle{dark}
\fi



\documentclass[10pt,t]{beamer}
\mode<presentation>
\usetheme[style=\PresentationStyle,logoLang=Latin,logoColor=monochromaticJUwhite,JUlogotitle=yes]{jagiellonian}



% ---------------------------------------
% Configuration files of Jagiellonian loceted in catalog preambule
% ---------------------------------------
\input{./preambule/LanguageSettings/JagiellonianEnglishLanguageSettings.tex}
\input{./preambule/TextposConfiguration/TextposConfiguration.tex}

\input{./preambule/JagiellonianCustomizationGeneral.tex}
\input{./preambule/JagiellonianCustomizationCommands.tex}










% ---------------------------------------
% Packages, libraries and their configuration
% ---------------------------------------





% ---------------------------------------
% Configuration for this particular presentation
% ---------------------------------------










% ---------------------------------------------------------------------
\title{QuantumOptics.jl}
\subtitle{Simulating quantum mechanics in modern programming language}

\author{Kamil Ziemian \\
  \texttt{kziemianfvt@gmail.com}}


\institute{Field Theory Department, Jagiellonian University in~Cracow}

\date[4 March 2019]{Seminar of Atomic Optics Department}
% ---------------------------------------------------------------------










% ####################################################################
% Początek dokumentu
\begin{document}
% ####################################################################





% Wyrównanie do lewej z łamaniem wyrazów

\RaggedRight





% ######################################
\maketitle % Tytuł całego tekstu
% ######################################





% ######################################
\begin{frame}
  \frametitle{Table of contents}


  \tableofcontents % Spis treści

\end{frame}
% ######################################










% ######################################
\section{Basic information}
% ######################################



% ##################
\begin{frame}
  \frametitle{Basic information}


  \textbf{QuantumOptics.jl.e}
  Is~open source project developed by Helmut Ritsch's CQED (Cavity
  Quantum Electrodynamics) group from~the~Institute for~Theoretical
  Physics~of the~University~of Innsbruck, with Sebastian Kr\"{a}mer
  as~leader. Takes inspiration from Quantum Optics Toolbox
  for~MATLAB and~the~Python framework Quantum Toolbox in Python
  (QuTiP).

  Description.
  ``Built exclusively in~Julia and~based on~standard quantum optics
  notation, the~toolbox offers speed comparable to~low-level
  statically typed languages, without compromising
  on~the~accessibility and~code readability found in~dynamic
  languages.''

\end{frame}
% ##################





% ##################
\begin{frame}
  \frametitle{Basic information}


  \begin{itemize}
    \RaggedRight

  \item Belongs to the family~of languages which works base on~LLVM
    (previusly \textit{Low Level Virtual Machine}) like~Go (Google),
    Swift (Apple) and~Rust (Mozilla).

  \item Created by~MIT group. Today Julia Lab at~this institute
    is~one~of leading facilities in~research and~development~of
    Julia language.

  \item Release~of version~1.0: 8~August~2018.

  \item Current version, 25 February 2019: 1.1.0.

  \item 3,4 million downloads (February 2019,
    \colorhref{https://juliacomputing.com/}{https://juliacomputing.com/}).

  \item 2,400 registered packages (February 2019,
    \colorhref{https://juliacomputing.com/}{https://juliacomputing.com/}).

  \item Use multiple dispatch (generic functions) and~sophisticated
    type system approach to~object oriented programming (if~someone
    disagree, I~will not protest).

  \item Number~of papers about Julia or~using it~at~Google Scholar
    (25~February 2019): 722.

  \end{itemize}

\end{frame}
% ##################





% ##################
\begin{frame}
  \frametitle{Aims of Julia}


  Advertisment time. Easy as MATLAB, flexible as Python, deep as LISP,
  fast as Fortran.

  \vspace{1em}

  John F.~Gibson,
  \colorhref{https://github.com/johnfgibson/whyjulia/blob/master/1-whyjulia.ipynb}{WhyJulia}


  Aims of language: code that is
  \begin{itemize}
    \RaggedRight

  \item easy to~read (unicode);

  \item fast to~write;

  \item in~most cases can be~optimized to~run at~0.35 to~1.05
    speed~of FORTRAN or~C code;

  \item both compiled and dynamical (not interpreted) in~use,
    by~REPL (shell) or~notebook;

  \item seamless incorporation~of FORTRAN, C, C++, Python, R, etc., code;

  \item give native support for~parallelism and~GPU;

  \item and~easy access to~documentation.

  \end{itemize}

\end{frame}
% ##################





% ##################
\begin{frame}
  \frametitle{There are lies, big lies and benchmarks}


  \begin{figure}

    \centering

    \includegraphics[scale=0.22]
    {./PresentationPictures/Julia_micro_benchmarks.png}


    \caption{From page \textit{Julia Micro-Benchmarks},
      \colorhref{https://julialang.org/benchmarks/}{https://julialang.org/benchmarks/}}

  \end{figure}

  \textbf{Warning!} Python code use \texttt{numpy} libraries that
  is~written 52.8\% in~C (state of GitHub repository at~4
  January~2019).

\end{frame}
% ##################





% ##################
\begin{frame}
  \frametitle{Warnings!!!}


  Outdated. Because version 1.0 is~from August 2018, most~of benchmarks
  and~many tutorials are less or~more outdated.

  Julia is~still to new. Language is~now mature, but~programming tools
  and~package ecosystems as~whole, are~not.

\end{frame}
% ##################





% ##################
\begin{frame}
  \frametitle{About packages}


  Packages ecosystem.
  Is~immature, but~these don't mean that it~isn't rich. With 2,400
  registered packages you can google one for~vast area~of scientific
  topics and~beyond. Like MOO = Milk Output Optimizer for~New
  Zealand diary farming. \\
  \colorhref{https://www.youtube.com/watch?v=LDgNmsOCl1A}
  {https://www.youtube.com/watch?v=LDgNmsOCl1A}

  Proof. Few days ago, while preparing for~that presentation I~decide~to
  Google ``Quantum Informatics Julia'' and~find two packages.

\end{frame}
% ##################





% ##################
\begin{frame}
  \frametitle{About packages}


  Quantum informatics and~Julia
  \begin{itemize}
    \RaggedRight

  \item \texttt{Yao.jl}, \textit{Extensible, Efficient Quantum
      Algorithm Design for Humans}, by~QuantumBFS (Bao~Fu~Si =
    ang.~Tempel). Members are~associated among others with~Chinese
    Academy~of Sciences, Tsinghua University, University~of Chinese
    Academy~of Sciences. There is~also \texttt{CUYao.jl} for~GPU.

  \item \texttt{QuantumInformation.jl}, \textit{a~Julia package
      for~numerical computation in~quantum information theory},
    by~Piotr Gawron, Dariusz Kurzyk, Łukasz Pawela from Institute~of
    Theoretical and Applied Informatics, Polish Academy~of Sciences,
    Gliwice. \textit{Our goal while designing QuantumInformation.jl
      library was to~follow principles presented in~book ``Geometry
      of Quantum States''} by~I.~Bengtsson, K.~Życzkowski.

  \end{itemize}

\end{frame}
% ##################





% ##################
\begin{frame}
  \frametitle{QuantumOptics.jl proper}


  Lies, big lies and benchmarks from~2017
  \begin{figure}

    \centering

    \includegraphics[scale=0.7]{./PresentationPictures/benchmarks_QO.pdf}


    \caption{QuTiP~-- Quantum Toolbox in Python}

  \end{figure}



  More benchmarks on~the~project side
  \colorhref{https://qojulia.org/}{https://qojulia.org/}. I~talk
  with authors about this~benchmarks, they said that both on~picture
  and~page are~outdated both by~release~of Julia~1.x and~changing~of
  internals. Now they expect better performance~of \texttt{QO.jl}.

\end{frame}
% ##################





% ##################
\begin{frame}
  \frametitle{Closing remarks}


  If you use QuantumOptics.jl. They ask you for citing paper
  \textbf{QuantumOptics.jl: A~Julia framework for~simulating open quantum
    systems}, Sebastian Kr\"{a}mer, David Plankensteiner, Laurin Ostermann,
  Helmut Ritsch, Computer Physics Communications, Volume 227, June 2018,
  pages 109-116,
  \colorhref{https://doi.org/10.1016/j.cpc.2018.02.004}{https://doi.org/10.1016/j.cpc.2018.02.004}.

  If you use Julia in~your research. They ask you for citing paper
  \textbf{Julia: A Fresh Approach to Numerical Computing}, Jeff Bezanson,
  Alan Edelman, Stefan Karpinski and~Viral B.~Shah~(2017) SIAM Review,
  59:~65--98. doi:~10.1137/141000671.
  url:~http://julialang.org/publications/julia-fresh-approach-BEKS.pdf,
  \colorhref{https://julialang.org/publications/julia-fresh-approach-BEKS.pdf}{https://julialang.org/publications/julia-fresh-approach-BEKS.pdf}.
  And add you paper to the following list
  https://julialang.org/publications/,
  \colorhref{https://julialang.org/research/}{https://julialang.org/research/}.

\end{frame}
% ##################





% ##################
\begin{frame}
  \frametitle{Closing remarks}


  People who in~various ways help prepare this seminar.
  \begin{itemize}
    \RaggedRight

  \item QuantumOptics group.

  \item Tamas Papp.

  \item John F. Gibson.

  \item Yakir Luc Gagnon.

  \item Antoine Levitt.

    % \item Krzysztof Musiał.

  \end{itemize}

\end{frame}
% ##################










% ######################################
\appendix
% ######################################





% ######################################
\EndingSlide{Thank you! Quastions?}
% ######################################










% ##################
\begin{frame}
  \frametitle{Bibliography and netography}


  Articles
  \begin{itemize}
    \RaggedRight

  \item Sebastian Kr\"{a}mer, David Plankensteiner, Laurin
    Ostermann, Helmut Ritsch, \textit{QuantumOptics.jl: A~Julia
      framework for~simulating open quantum systems}, Computer
    Physics Communications, Volume 227, June 2018, pages 109-116,
    \colorhref{https://doi.org/10.1016/j.cpc.2018.02.004}{https://doi.org/10.1016/j.cpc.2018.02.004}.

  \item Piotr Gawron, Dariusz Kurzyk, Łukasz Pawela,
    \textit{QuantumInformation.jl -- a Julia package for numerical
      computation in quantum information theory}, Public Library of
    Science, volume 13, number 12,
    \colorhref{https://arxiv.org/pdf/1806.11464.pdf}{arXiv: 1806.11464}.

  \item Jeff Bezanson, Alan Edelman, Stefan Karpinski and~Viral
    B.~Shah, \textit{Julia: A Fresh Approach to Numerical Computing},
    (2017) SIAM Review, 59:~65--98. doi:~10.1137/141000671.
    url:~http://julialang.org/publications/julia-fresh-approach-BEKS.pdf,
    \colorhref{https://julialang.org/publications/julia-fresh-approach-BEKS.pdf}
    {https://julialang.org/publications/julia-fresh-approach-BEKS.pdf}.

  \end{itemize}

\end{frame}
% ##################





% ##################
\begin{frame}
  \frametitle{Bibliography and netography}


  Internet pages and code repositories
  \begin{itemize}
    \RaggedRight

  \item \textit{Julia Micro-Benchmarks},
    \colorhref{https://julialang.org/benchmarks/}{https://julialang.org/benchmarks/}.

  \item QuantumBFS, \colorhref{https://github.com/QuantumBFS}{https://github.com/QuantumBFS}.

  \item \texttt{QuantumInformation.jl},
    \colorhref{https://github.com/ZKSI/QuantumInformation.jl}
    {https://github.com/ZKSI/QuantumInformation.jl}.

  \item Milk Output Optimizer (MOO),
    \colorhref{https://juliacomputing.com/case-studies/moo.html}
    {https://juliacomputing.com/case-studies/moo.html}.

  \item numpy/numpy -- \texttt{numpy} GitHub repository, \\
    \colorhref{https://github.com/numpy/numpy}
    {https://github.com/numpy/numpy}.

  \item John F.~Gibson, \textit{julia-pde-benchmark}, \\
    GitHub
    \colorhref{https://github.com/johnfgibson/julia-pde-benchmark}{johnfgibson/julia-pde-benchmark}.

  \item John F.~Gibson, \textit{Why~Julia}, GitHub
    \colorhref{https://github.com/johnfgibson/whyjulia/blob/master/1-whyjulia.ipynb}{johnfgibson/whyjulia}.

  \end{itemize}

\end{frame}
% ##################










% ####################################################################
% ####################################################################
% Bibliografia
\bibliographystyle{plalpha}

\bibliography{}{}





% ############################

% Koniec dokumentu
\end{document}
