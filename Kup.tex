% Autor: Kamil Ziemian

% ---------------------------------------------------------------------
% Podstawowe ustawienia i pakiety
% ---------------------------------------------------------------------
\RequirePackage[l2tabu, orthodox]{nag}  % Wykrywa przestarzałe i niewłaściwe
% sposoby używania LaTeXa. Więcej jest w l2tabu English version.
\documentclass[a4paper,11pt]{article}
% {rozmiar papieru, rozmiar fontu}[klasa dokumentu]
\usepackage[MeX]{polski}  % Polonizacja LaTeXa, bez niej będzie pracował
% w języku angielskim.
\usepackage[utf8]{inputenc}  % Włączenie kodowania UTF-8, co daje dostęp
% do polskich znaków.
\usepackage{lmodern}  % Wprowadza fonty Latin Modern.
\usepackage[T1]{fontenc}  % Potrzebne do używania fontów Latin Modern.



% ---------------------------------------
% Podstawowe pakiety (niezwiązane z ustawieniami języka)
% ---------------------------------------
\usepackage{microtype}  % Twierdzi, że poprawi rozmiar odstępów w tekście.
% \usepackage{graphicx}  % Wprowadza bardzo potrzebne komendy do wstawiania
% % grafiki.
\usepackage{xcolor}
% \usepackage{verbatim}  % Poprawia otoczenie VERBATIME.
% \usepackage{textcomp}  % Dodaje takie symbole jak stopnie Celsiusa,
% % wprowadzane bezpośrednio w tekście.
\usepackage{vmargin}  % Pozwala na prostą kontrolę rozmiaru marginesów,
% za pomocą komend poniżej. Rozmiar odstępów jest mierzony w calach.
% ---------------------------------------
% MARGINS
% ---------------------------------------
\setmarginsrb
{ 0.7in} % left margin
{ 0.6in} % top margin
{ 0.7in} % right margin
{ 0.8in} % bottom margin
{  20pt} % head height
{0.25in} % head sep
{   9pt} % foot height
{ 0.3in} % foot sep



% ---------------------------------------
% Często przydatne pakiety
% ---------------------------------------
% \usepackage{csquotes}  % Pozwala w prosty sposób wstawiać cytaty do tekstu.
% \usepackage{xcolor}  % Pozwala używać kolorowych czcionek (zapewne dużo
% więcej, ale ja nie potrafię nic o tym powiedzieć).



% ---------------------------------------
% Pakiety do tekstów z nauk przyrodniczych
% ---------------------------------------
\let\lll\undefined  % Amsmath gryzie się z pakietami do języka
% polskiego, bo oba definiują komendę \lll. Aby rozwiązać ten problem
% oddefiniowuje tę komendę, ale może tym samym pozbywam się dużego Ł.
\usepackage[intlimits]{amsmath}  % Podstawowe wsparcie od American
% Mathematical Society (w skrócie AMS)
\usepackage{amsfonts, amssymb, amscd, amsthm} % Dalsze wsparcie od AMS
% \usepackage{siunitx}  % Do prostszego pisania jednostek fizycznych
\usepackage{upgreek}  % Ładniejsze greckie litery
% Przykładowa składnia: pi = \uppi
\usepackage{slashed}  % Pozwala w prosty sposób pisać slash Feynmana.
\usepackage{calrsfs}  % Zmienia czcionkę kaligraficzną w \mathcal
% na ładniejszą. Może w innych miejscach robi to samo, ale o tym nic
% nie wiem.



% ##########
% Tworzenie otoczeń "Twierdzenie", "Definicja", "Lemat", etc.
\newtheorem{theorem}{Twierdzenie}  % Komenda wprowadzająca otoczenie
% „theorem” do pisania twierdzeń matematycznych
\newtheorem{definition}{Definicja}  % Analogicznie jak powyżej
\newtheorem{corollary}{Wniosek}



% ---------------------------------------
% Pakiety napisane przez użytkownika.
% Mają być w tym samym katalogu to ten plik .tex
% ---------------------------------------
% \usepackage{latexgeneralcommands}
% \usepackage{mathcommands}
% \usepackage{calculuscommands}
% \usepackage{SchwartzBooksCommands}  % Pakiet napisany m.in. dla tego pliku.




% ---------------------------------------------------------------------
% Dodatkowe ustawienia dla języka polskiego
% ---------------------------------------------------------------------
\renewcommand{\thesection}{\arabic{section}.}
% Kropki po numerach rozdziału (polski zwyczaj topograficzny)
\renewcommand{\thesubsection}{\thesection\arabic{subsection}}
% Brak kropki po numerach podrozdziału



% ---------------------------------------
% Ustawienia różnych parametrów tekstu
% ---------------------------------------
\renewcommand{\arraystretch}{1.2}  % Ustawienie szerokości odstępów między
% wierszami w tabelach.



% ---------------------------------------
% Pakiet „hyperref”
% Polecano by umieszczać go na końcu preambuły.
% ---------------------------------------
\usepackage{hyperref}  % Pozwala tworzyć hiperlinki i zamienia odwołania
% do bibliografii na hiperlinki.










% ---------------------------------------------------------------------
% Tytuł, autor, data
\title{Rzeczy które należy kupić}

% \author{}
% \date{}
% ---------------------------------------------------------------------










% ####################################################################
\begin{document}
% ####################################################################





% ######################################
\maketitle % Tytuł całego tekstu
% ######################################





% ############################
\section{Gry}
% ############################


\begin{itemize}
\item[--] \textit{Oni},

\item[--] \textit{Soul Reaver II},

\item[--] \textit{Curse of Monkey Island},

\item[--] \textit{Escape from Monkey Island},

\item[--] \textit{Black and White},

\item[--] \textit{Chrono Trigger},

\item[--] \textit{Heroes of Might and Magic V},

\item[--] \textit{The Thing},

\item[--] \textit{Diablo II},

\end{itemize}










% ############################
\section{Filmy}
% ############################


\begin{itemize}
\item[--] Kubrick,

\item[--] \textit{Lot nad kukułczym gniazdem},

\item[--] \textit{Trylogia dolarów},

\item[--] \textit{Napoleon Dynamite},

\item[--] \textit{Pan Tadeusz},

\item[--] John Woo,

\item[--] \textit{Citezan Kane},

\item[--] Akira Kurosawa,

\item[--] \textit{Ludzka skaza},

\item[--] Chaplin, \textit{Dyktator},

\item[--] \textit{Poszukiwacze},

\item[--] Fellini,

\item[--] I. Bergman,

\item[--] Dokument, \textit{Shoah},

\end{itemize}










% ############################
\section{Muzyka}
% ############################


\begin{itemize}
\item[--] Bach,

\item[--] Mozart,

\item[--] Iron Savior,

\item[--] The Beatles,

\item[--] Metallica,

\item[--] Angelo Branduardi,

\item[--] Papa dance??, \textit{Panorama Tatr},

\item[--] Beethoven,

\end{itemize}










% ############################
\section{Seriale}
% ############################


\begin{itemize}
\item[--] \textit{Rzym},

\item[--] \textit{True Blood},

\item[--] \textit{Yes, Minister},

\item[--] \textit{Big Bang Theory},

\item[--] \textit{Neon Genesis Evangelion},

\item[--] \textit{Elfen Lied},

\item[--] \textit{Eureka},

\item[--] \textit{The Big Bang Theory},

\item[--] \textit{Chuck},

\item[--] \textit{Walking Dead}, pierwszy sezon,

\end{itemize}










% ############################
\section{Komiksy}
% ############################


\begin{itemize}
\item[--] \textit{Monster} (manga),

\item[--] \textit{Władcy Chmielu},

\item[--] Pierwsze komiksy o Batmanie i Supermenie,

\item[--] \textit{Walking Dead},

\item[--] \textit{XIII},

\end{itemize}










% ############################

% Koniec dokumentu
\end{document}