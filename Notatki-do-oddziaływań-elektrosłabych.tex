% Autor: Kamil Ziemian
% Korekta: Wojciech Dyba

% --------------------------------------------------------------------
% Podstawowe ustawienia i pakiety
% --------------------------------------------------------------------
\RequirePackage[l2tabu, orthodox]{nag} % Wykrywa przestarzałe i niewłaściwe
% sposoby używania LaTeXa. Więcej jest w l2tabu English version.
\documentclass[a4paper,11pt]{article}
% {rozmiar papieru, rozmiar fontu}[klasa dokumentu]
\usepackage[MeX]{polski} % Polonizacja LaTeXa, bez niej będzie pracował
% w języku angielskim.
\usepackage[utf8]{inputenc} % Włączenie kodowania UTF-8, co daje dostęp
% do polskich znaków.
\usepackage{lmodern} % Wprowadza fonty Latin Modern.
\usepackage[T1]{fontenc} % Potrzebne do używania fontów Latin Modern.



% ------------------------------
% Podstawowe pakiety (niezwiązane z ustawieniami języka)
% ------------------------------
\usepackage{microtype} % Twierdzi, że poprawi rozmiar odstępów w tekście.
\usepackage{graphicx} % Wprowadza bardzo potrzebne komendy do wstawiania
% grafiki.
\usepackage{verbatim} % Poprawia otoczenie VERBATIME.
\usepackage{textcomp} % Dodaje takie symbole jak stopnie Celsiusa,
% wprowadzane bezpośrednio w tekście.
\usepackage{vmargin} % Pozwala na prostą kontrolę rozmiaru marginesów,
% za pomocą komend poniżej. Rozmiar odstępów jest mierzony w calach.
% ------------------------------
% MARGINS
% ------------------------------
\setmarginsrb
{ 0.7in}  % left margin
{ 0.6in}  % top margin
{ 0.7in}  % right margin
{ 0.8in}  % bottom margin
{  20pt}  % head height
{0.25in}  % head sep
{   9pt}  % foot height
{ 0.3in}  % foot sep



% ------------------------------
% Często przydatne pakiety
% ------------------------------
\usepackage{csquotes} % Pozwala w prosty sposób wstawiać cytaty do tekstu.
\usepackage{xcolor} % Pozwala używać kolorowych czcionek (zapewne dużo
% więcej, ale ja nie potrafię nic o tym powiedzieć).



% ------------------------------
% Pakiety do tekstów z nauk przyrodniczych
% ------------------------------
\let\lll\undefined % Amsmath gryzie się z językiem pakietami do języka
% polskiego, bo oba definiują komendę \lll. Aby rozwiązać ten problem
% oddefiniowuję tę komendę, ale może tym samym pozbywam się dużego Ł.
\usepackage[intlimits]{amsmath} % Podstawowe wsparcie od American
% Mathematical Society (w skrócie AMS)
\usepackage{amsfonts, amssymb, amscd, amsthm} % Dalsze wsparcie od AMS
% \usepackage{siunitx} % Dla prostszego pisania jednostek fizycznych
\usepackage{upgreek} % Ładniejsze greckie litery
% Przykładowa składnia: pi = \uppi
\usepackage{slashed} % Pozwala w prosty sposób pisać slash Feynmana.
\usepackage{calrsfs} % Zmienia czcionkę kaligraficzną w \mathcal
% na ładniejszą. Może w innych miejscach robi to samo, ale o tym nic
% nie wiem.



% ##########
% Tworzenie otoczeń "Twierdzenie", "Definicja", "Lemat", etc.
\newtheorem{theorem}{Twierdzenie}  % Komenda wprowadzająca otoczenie
% „theorem” do pisania twierdzeń matematycznych
\newtheorem{definition}{Definicja}  % Analogicznie jak powyżej
\newtheorem{corollary}{Wniosek}



% ---------------------------------------
% Pakiety napisane przez użytkownika.
% Mają być w tym samym katalogu to ten plik .tex
% ---------------------------------------
\usepackage{latexgeneralcommands}
\usepackage{mathcommands}
% \usepackage{calculuscommands}
% \usepackage{SchwartzBooksCommands}  % Pakiet napisany m.in. dla tego pliku.



% ---------------------------------------------------------------------
% Dodatkowe ustawienia dla języka polskiego
% ---------------------------------------------------------------------
\renewcommand{\thesection}{\arabic{section}.}
% Kropki po numerach rozdziału (polski zwyczaj topograficzny)
\renewcommand{\thesubsection}{\thesection\arabic{subsection}}
% Brak kropki po numerach podrozdziału



% ------------------------------
% Ustawienia różnych parametrów tekstu
% ------------------------------
\renewcommand{\arraystretch}{1.2} % Ustawienie szerokości odstępów między
% wierszami w tabelach.



% ------------------------------
% Pakiet "hyperref"
% Polecano by umieszczać go na końcu preambuły.
% ------------------------------
\usepackage{hyperref} % Pozwala tworzyć hiperlinki i zamienia odwołania
% do bibliografii na hiperlinki.










% ---------------------------------------------------------------------
% Tytuł, autor, data
\title{Notatki}

% \author{}
% \date{}
% ---------------------------------------------------------------------










% ####################################################################
\begin{document}
% ####################################################################





% ######################################
\maketitle % Tytuł całego tekstu
% ######################################






% ############################
\section{Możliwe człony w lagrangianie}
% ############################


Lagrangiany używane w teorii pola można w relatywnie krótki sposób scharakteryzować. Takie lagrangiany zawsze będą postaci wielomianu wielu zmiennych w polach i ich pochodnych. Przykładowo wielomiany wielu zmiennych
\begin{equation}
  \label{eq:1}
  P_{ 1 }( x ) = x, \quad P_{ 2 }( x, y ) = xy, \quad P_{ 3 }( x, y ) = x^{ 2 } y, \quad
  P_{ 4 }( x, y ) = xy + x^{ 3 },
\end{equation}
prowadzą do następujących członów lagrangianu, gdy dokonamy zamiany $x \mapsto \phi( t, x )$, $y \mapsto \partial_{ \mu } \phi( t, x )$.
\begin{equation}
  \label{eq:2}
  \begin{split}
    &P_{ 1 }( \phi( t, x ) ) = \phi( t, x ), \quad
      P_{ 2 }( \phi( t, x ), \partial_{ \mu } \phi( t, x ) ), \quad
      P_{ 3 }( \phi( t, x ), \partial_{ \mu } \phi( t, x ) )
      = \phi^{ 2 }( t, x ) + \partial_{ \mu } \phi( t, x ), \quad \\
    &P_{ 4 }( \phi( t, x ), \partial_{ \mu } \phi( t, x ) )
      = \phi( t, x ) \partial_{ \mu } \phi( t, x ) + \phi( t, x )^{ 3 }.
  \end{split}
\end{equation}
Kilka uwag porządkowych. W powyższym wzorach $t$ oznacza czas, a $x$ położenie na prostej. Możemy równie dobrze napisać $x$, $y$, $z$, po prostu mniej pisania.

W wielomianie każda zmienna może być \textit{tylko} w potędze $0, 1, 2, 3, \ldots$. Stopień wyrazu $x$ to 1, $xy$ to 2, $x y^{ 2 }$ to 3, etc. Ten sam stopień przypisujemy wyrazowi w którym zmienne $x$, $y$, $z$ zastąpiliśmy odpowiednimi polami lub ich pochodnymi tych pól. To nie powinno sprawiać trudności.

W ogólności każdy człon lagrangianu jest iloczynem pól i~pochodnych tych pól po $x^{ \mu }$. Dla wygody często mówimy, po prostu że człon ten jest iloczynem pól, w domyśle dopuszczając, że mogą tam też być pochodne odpowiednich pól.

Rozważamy tylko \textbf{lokalne teorie pola}. To znaczy, że jeśli mamy dwa pola $\psi( t, x )$ i $\phi( t, x )$ to dopuszczamy tylko iloczyny postaci
\begin{equation}
  \label{eq:3}
  \psi( t, x ), \quad \psi( t, x ) \phi( t, x ), \quad \psi( t, x )^{ 2 } \phi( t, x ), \quad
  \psi( t, x ) \partial_{ \mu }\phi( t, x ), \ldots
\end{equation}
Mówiąc krótko pola w danym członie muszą być obliczone w tym samym punkcie czasoprzestrzeni. Inaczej mówiąc pole może oddziaływać z innym polem wtedy i tylko wtedy gdy znajdują się w tym samym czasie i w tym samym miejscu, stąd mówimy o \textit{lokalnej} teorii pola.

Najprostszy przykład członu nielokalnego to






% ############################
\section{Klasyfikacja członów w lagrangianie ze względu na stopień}
% ############################


Niech $\phi( x )$ będzie zespolonym polem skalarnym i w tym podrozdziale
$x$ oznacza cztery współrzędne czasoprzestrzenne. Możemy teraz
sklasyfikować wszystkie możliwe człony naszych lagrangianów, ze
względu na stopień.

\textbf{Stopień 0.} Wtedy $\phi( x )^{ 0 } = 1$. Taki człon nie wprowadza
żadnych zmian do teorii, bo to przesunięcie wszystkie o stałą, więc
możemy go pominąć.

\textbf{Stopień 1.} Człon postaci $\phi( x )$, $\bar{\phi}( x )$. Dodatnie
takie członu zmienia lagrangian, ale okazuje się, że nie zmienia
działania. Wszystkie wielkości które nie zmieniają działania możemy
usunąć z lagrangianu, bez zmiany fizyki.

\textbf{Stopień 2.} Mogą być to człony postaci $\phi( x )^{ 2 }$,
$\bar{\phi}( x ) \phi( x )$, $\phi( x ) \partial_{ \mu } \phi( x )$,
$\bar{\phi}( x ) \partial_{ \mu } \phi( x )$,
$\partial_{ \mu } \phi( x ) \partial^{ \mu } \phi( x )$. Prawie wszystkie tego typu człony
pojawiają się rozważanych teoriach.

\begin{itemize}
\item $\phi( x )^{ 2 }$, $\bar{\phi}( x ) \phi( x )$ -- człon masowe dla bozonu,
  bądź fermionu,

\item $\partial_{ \mu } \phi( x ) \partial^{ \mu } \phi( x )$ -- człon kinetyczny (odpowiadając
  za ruch swobodny) bozonu,

\item $\bar{\phi}( x ) \partial_{ \mu } \phi( x )$ -- człony tego typu opisują człon
  kinetyczny fermionu.
\end{itemize}

\textbf{Stopień większy od 3.} Przykładowo $\phi( x )^{ 3 }$. Wszystkie
tego typu człony opisują oddziaływanie między cząstkami. Mogą opisywać
oddziaływanie różnych cząstek, np. elektronu, pozytronu i fotonu, lub
cząstek tego samego typu, np. trzech jednakowych gluonów.

W praktyce spotykamy tylko człony stopnia 3 i 4, więcej o tym w
następnym punkcie.

\textbf{Stopień większy od 5.} Przykładowo $\phi( x )^{ 5 }$. Człony tego
typu opisują oddziaływanie w teorii, ale w praktyce ich nie spotykamy.
Nie wiadomo dlaczego, jest to jednak bardzo dobra wiadomość. Człony
tego typu nie są renormalizowalne, co sprowadza się do tego, że nie
umiemy nic sensownego dla tej teorii obliczyć.










% ############################
\section{Typy członów lagrangianu spotykane w fizyce cząstek}
% ############################







% ############################
\section{Zmienne Grassmana}
% ############################


Oprócz liczb zespolony $\Cbb$ będziemy potrzebowali jest antyprzemiennych \textbf{zmiennych (liczb) Grasmanna}. Zaczniemy od najprostszego przykładu gdy mamy tylko dwie \textbf{zmienne Grassmana}: $\theta_{ 1 }$, $\theta_{ 2 }$.

Napije definiujemy iloczyn zmiennej Grasmanna przez liczbę zespoloną $\alpha$:
\begin{equation}
  \label{eq:Zmienne-Grassmana-01}
  \alpha \theta_{ i } = \theta_{ i } \alpha.
\end{equation}
Takie wyrażenie przedstawia zmienną Grasmanna, ale jedyną naprawdę ważną rzeczą jest to, że mnożenie liczby zespolonej przez zmienną Grassmana jest przemienne.

Definiujemy potęgę zerową zmiennych Grasmanna.
\begin{equation}
  \label{eq:Zmienne-Grassmana-02}
  ( \theta_{ 1 } )^{ 0 } = ( \theta_{ 2 } )^{ 0 } = 1,
\end{equation}
gdzie 1 to liczba zespolona (nie musi to być oczywiste). Liczby zespolone traktujemy więc jako szczególny przypadek zmiennej Grassmana.

Teraz określimy iloczyn dwóch zmiennych Grassmana.
\begin{equation}
  \label{eq:Zmienne-Grassmana-03}
  \theta_{ i } \theta_{ j } = -\theta_{ j } \theta_{ i }.
\end{equation}
Widzimy więc, że $\theta_{ 1 } \theta_{ 2 }$ jest zmienną Grassmana, natomiast $\theta_{ 2 } \theta_{ 1 }$ jest zmienną Grassmana liniowo zależną od $\theta_{ 1 } \theta_{ 2 }$
\begin{equation}
  \label{eq:Zmienne-Grassmana-03}
  \theta_{ 2 } \theta_{ 1 } = -\theta_{ 1 } \theta_{ 2 }.
\end{equation}
Z tego też wynika, że kwadrat dowolnej bazowej zmiennej Grassmana jest równy 0.
\begin{equation}
  \label{eq:Zmienne-Grassmana-05}
  \theta_{ 1 } \theta_{ 1 } = ( \theta_{ 1 } )^{ 2 } = -\theta_{ 1 } \theta_{ 1 }.
\end{equation}
Dodając $\theta_{ 1 } \theta_{ 1 }$ do obu stron tej równości dostajemy.
\begin{equation}
  \label{eq:Zmienne-Grassmana-06}
  2 \theta_{ 1 } \theta_{ 1 } = 0.
\end{equation}
Po podzieleniu przez dwa uzyskujemy porządany wynik

Dwie dowolne zmienne Grassmana możemy do siebie dodać.
\begin{subequations}
  \begin{align}
    \label{eq:Zmienne-Grassmana-07-A}
    &\alpha + \theta_{ 1 } \\
    \label{eq:Zmienne-Grassmana-07-B}
    &\alpha \theta_{ 1 } + \beta \theta_{ 1 } = ( \alpha + \beta ) \theta_{ 1 } \\
    \label{eq:Zmienne-Grassmana-07-C}
    &\theta_{ 1 } + \theta_{ 2 } \\
        \label{eq:Zmienne-Grassmana-07-D}
    &\theta_{ 1 } + \theta_{ 1 } \theta_{ 2 }.
  \end{align}
\end{subequations}
Dla dwóch bazowych zmiennych Grassmana mamy więc $4 = 2^2$ liniowo niezależne liczby Grassmana (liczba zespolona też jest zmienną Grassmana).
\begin{equation}
  \label{eq:Zmienne-Grassmana-08}
  \alpha, \theta_{ 1 }, \theta_{ 2 }, \theta_{ 1 } \theta_{ 2 }, \quad \alpha \in \Cbb.
\end{equation}
Najogólniejsze wyrażeni jakie możemy utworzyć w tym przypadku ze zmiennych Grassmana, to
\begin{equation}
  \label{eq:Zmienne-Grassmana-09}
  \alpha + \beta \theta_{ 1 } + \gamma \theta_{ 2 } + \delta \theta_{ 1 } \theta_{ 2 }, \quad
  \alpha, \beta, \gamma, \delta \in \Cbb.
\end{equation}

Na koniec zwróćmy uwagę, że
\begin{equation}
  \label{eq:Zmienne-Grassmana-10}
  ( \theta_{ 1 } \theta_{ 2 } )^{ 2 } = 0.
\end{equation}
Jest tak ponieważ
\begin{equation}
  \label{eq:Zmienne-Grassmana-10}
  ( \theta_{ 1 } \theta_{ 2 } )^{ 2 } =
  ( \theta_{ 1 } \theta_{ 2 } ) ( \theta_{ 1 } \theta_{ 2 } ) =
  \theta_{ 1 } \theta_{ 2 } \theta_{ 1 } \theta_{ 2 } =
  -\theta_{ 2 } \theta_{ 1 } \theta_{ 1 } \theta_{ 2 } =
  -\theta_{ 1 } \theta_{ 2 } \theta_{ 2 } \theta_{ 1 } =
  -\theta_{ 1 } 0 \theta_{ 1 } = 0.
\end{equation}

Przyjrzyjmy się jeszcze przypadkowi, gdy mielibyśmy trzy bazowe zmienne Grassmana: $\theta_{ 1 }$, $\theta_{ 2 }$, $\theta_{ 3 }$. Istniej wówczas $8 = 2^{ 3 }$ liniowo niezależnych zmiennych Grassmana.
\begin{equation}
  \label{eq:Zmienne-Grassmana-11}
  \alpha, \theta_{ 1 }, \theta_{ 2 }, \theta_{ 3 }, \theta_{ 1 } \theta_{ 2 }, \theta_{ 1 } \theta_{ 3 },
  \theta_{ 2 } \theta_{ 3 }, \theta_{ 1 } \theta_{ 2 } \theta_{ 3 }, \quad \alpha \in \Cbb.
\end{equation}

Ogólnie, dla $N$ bazowych zmiennych Grassmana mamy $2^{ N }$ liniowo niezależnych zmiennych Grassmana. Należy pamiętać, że jedną z nich są liczby zespolone.

Na razie musimy przerwać nasze rozważania tutaj, bowiem czeka nas temat spinorowych zmiennych Grassmana.






% ############################
\subsection{Podnoszenie i opuszczenie wskaźników zmiennych Grassmana}
% ############################


W fizyce będą nam potrzebna „standardowa” przestrzeń zmiennych Grassmana, generowana przez dwie bazowe zmienne Grassmana: $\theta_{ 1 }$, $\theta_{ 2 }$. Opuszczanie i podnoszenie wskaźników takich zmiennych definiujemy za pomocą macierzy
\begin{equation}
  \label{eq:Zmienne-Grassmana-12}
  \varepsilon^{ u } =
  \begin{pmatrix}
    \hphantom{-}0 & 1 \\
    -1 & 0
  \end{pmatrix}, \quad
  \varepsilon_{ d } =
  \begin{pmatrix}
    0 & -1 \\
    1 & \hphantom{-}0
  \end{pmatrix}.
\end{equation}
Od razu widzimy, że
\begin{equation}
  \label{eq:Zmienne-Grassmana-13}
  \varepsilon^{ u } = \varepsilon_{ d }^{ \,\: T }, \quad
  ( \varepsilon^{ u } )^{ 2 } = ( \varepsilon_{ d } )^{ 2 }
  = -\mathbf{1}_{ 2 }, \quad
  \varepsilon^{ u } \varepsilon_{ d } = \mathbf{1}_{ 2 }, \quad
  \varepsilon^{ u } = i \sigma_{ P, 2 },
\end{equation}
gdzie $\mathbf{1}_{ 2 }$ to dwuwymiarowa macierz $2 \times 2$, a $\sigma_{ P, 2 }$ to druga macierz Pauliego. Te same relacje w zapisie wskaźnikowym.
\begin{subequations}
  \begin{align}
    \label{eq:Zmienne-Grassmana-14-A}
    \varepsilon^{ \alpha \beta } &= \varepsilon_{ \beta \alpha }, \\
    \varepsilon^{ \alpha \gamma } \varepsilon^{ \gamma \beta } &= \varepsilon_{ \alpha \gamma } \varepsilon_{ \gamma \beta } = -\delta^{ \alpha \beta }, \\
    \varepsilon^{ \alpha \gamma } \varepsilon_{ \gamma \beta } &= \delta^{ \alpha }_{ \beta }.
  \end{align}
\end{subequations}

Przyjmujemy konwekcję, że numery wierszy i kolumn macierzy $\varepsilon^{ u }$ piszemy na górze, zaś $\varepsilon_{ d }$ na dole. Wprowadzamy też oznaczenia
\begin{equation}
  \label{eq:Zmienne-Grassmana-14}
  \varepsilon^{ \alpha \beta } := ( \varepsilon^{ u } )^{ \alpha \beta }, \quad
  \varepsilon_{ \alpha \beta } := ( \varepsilon_{ d } )_{ \alpha \beta }.
\end{equation}
Standardowo $\alpha$ oznacza wiersz, a $\beta$ kolumnę.

Stąd mamy dwie tożsamości
\begin{subequations}
  \begin{align}
    \label{eq:Zmienne-Grassmana-15-A}
    \theta^{ 1 }
    &= ( \varepsilon^{ u } )^{ 1 \beta } \theta_{ \beta } = \varepsilon^{ 1 \beta } \theta_{ \beta }
      = \varepsilon^{ 1 1 } \theta_{ 1 } + \varepsilon^{ 1 2 } \theta_{ 2 }
      = 0 \theta_{ 1 } + 1 \theta_{ 2 } = \theta_{ 2 }, \\
    \label{eq:Zmienne-Grassmana-15-B}
    \theta^{ 2 } &= \varepsilon^{ 2 \beta } \theta_{ \beta } = -1 \theta_{ 1 } + 0 \theta_{ 2 } = -\theta_{ 1 }.
  \end{align}
\end{subequations}
Stąd
\begin{subequations}
  \begin{align}
    \label{eq:Zmienne-Grassmana-16-A}
    \theta^{ 1 } \theta^{ 1 }
    &= \theta_{ 2 } \theta_{ 2 } = 0, \\
    \label{eq:Zmienne-Grassmana-16-B}
    \theta^{ 2 } \theta^{ 2 }
    &= ( -\theta_{ 1 } ) ( -\theta_{ 1 } ) = \theta_{ 1 }^{ 2 } = 0, \\
    \label{eq:Zmienne-Grassmana-16-C}
    \theta^{ 1 } \theta^{ 2 }
    &= \theta_{ 2 } ( -\theta_{ 1 } ) = -\theta_{ 2 } \theta_{ 1 } = \theta_{ 1 } \theta_{ 2 } \neq 0.
  \end{align}
\end{subequations}
Widzimy więc, że $\theta^{ 1 }$ i $\theta^{ 2 }$ to tylko wygodne nazwy dla $\theta_{ 2 }$ i $-\theta_{ 1 }$, nie zaś jakieś nowe byty matematyczne. Można też rozumować odwrotnie i uznać $\theta^{ 1 }$ i $\theta^{ 2 }$ za „wielości pierwotne”, wtedy $\theta_{ 1 }$ to wygodna nazwa na $-\theta^{ 2 }$, a $\theta_{ 2 }$ to wygodna nazwa na $\theta^{ 1 }$.

Najważniejsze związki jakie otrzymaliśmy to
\begin{subequations}
  \begin{align}
    \label{eq:Zmienne-Grassmana-17-A}
    \theta^{ 1 } &= \theta_{ 2 }, \\
    \label{eq:Zmienne-Grassmana-17-B}
    \theta^{ 2 } &= -\theta_{ 1 }.
  \end{align}
\end{subequations}

Wprowadzamy następującą notację.
\begin{equation}
  \label{eq:Zmienne-Grassmana-18}
  \theta \theta := \theta^{ \alpha } \theta_{ \alpha } = \varepsilon^{ \alpha \beta } \theta_{ \beta } \theta_{ \alpha }.
\end{equation}
Wielkość tę możemy teraz łatwo obliczyć.
\begin{equation}
  \label{eq:Zmienne-Grassmana-19}
  \theta \theta = \theta^{ \alpha } \theta_{ \alpha } = \theta^{ 1 } \theta_{ 1 } + \theta^{ 2 } \theta_{ 2 }
  = \theta_{ 2 } \theta_{ 1 } - \theta_{ 1 } \theta_{ 2 } = 2 \theta_{ 2 } \theta_{ 1 }.
\end{equation}
Korzystając z~związku anty-przemienności i~\eqref{eq:Zmienne-Grassmana-17-A} i~\eqref{eq:Zmienne-Grassmana-17-B} szybko otrzymujemy kilka równoważnych postaci.
\begin{equation}
  \label{eq:Zmienne-Grassmana-20}
  \theta \theta = 2 \theta_{ 2 } \theta_{ 1 } = -2 \theta_{ 1 } \theta_{ 2 } = 2 \theta^{ 2 } \theta_{ 2 }
  = 2 \theta^{ 2 } \theta^{ 1 } = -2 \theta^{ 1 } \theta^{ 2 }.
\end{equation}

Teraz powinny być oczywiste następujące wnioski.
\begin{subequations}
  \begin{align}
  \label{eq:13}
    ( \theta \theta )^{ 2 } &= \theta \theta \, \theta \theta = 0, \\
    \theta_{ \alpha } \theta \theta &= \theta \theta \theta_{ \alpha } = 0.
  \end{align}
\end{subequations}










% ############################
\subsection{Ćwiczenie, zmienne Grassmana}
% ############################


Dla lepszego zrozumienia zmiennych Grassmana wykonamy kilka ćwiczeń. Najpierw udowodnimy tożsamość
\begin{equation}
  \label{eq:4}
  \theta^{ \alpha } \theta^{ \beta } = -\frac{ 1 }{ 2 } \varepsilon^{ \alpha \beta } \theta \theta.
\end{equation}

\textbf{Pierwszy dowód wzoru.} Dowód ten będzie polegać na rozważeniu wszystkich możliwych przypadków.
\begin{itemize}
\item $\alpha = 1$, $\beta = 1$. Lewa strona wynosi
  \begin{equation}
    \label{eq:5}
    \theta^{ 1 } \theta^{ 1 } = 0.
  \end{equation}
  Dla prawej strony mamy
  \begin{equation}
    \label{eq:6}
    -\frac{ 1 }{ 2 } \varepsilon^{ 1 1 } \theta \theta =
    -\frac{ 1 }{ 2 } \varepsilon^{ 1 1 } (-2 \theta_{ 1 } \theta_{ 2 } )
    =
    \varepsilon^{ 1 1 } \theta_{ 1 } \theta_{ 2 } = 0 \theta_{ 1 } \theta_{ 2 } = 0.
  \end{equation}

\item $\alpha = 2$, $\beta = 2$. Wszystko jest tak samo jak w poprzednim punkcie.

\item $\alpha = 1$, $\beta = 2$. Lewa strona
  \begin{equation}
    \label{eq:7}
    \theta^{ 1 } \theta^{ 2 }
  \end{equation}
  Prawa strona
  \begin{equation}
    \label{eq:8}
    -\frac{ 1 }{ 2 } \varepsilon^{ 1 2 } \theta \theta
    = -\frac{ 1 }{ 2 } \varepsilon^{ 1 2 } ( -2 \theta_{ 1 } \theta_{ 2 } )
    = \varepsilon^{ 1 2 } \theta_{ 1 } \theta_{ 2 } = 1 \theta_{ 1 } \theta_{ 2 }
    = \theta_{ 1 } \theta_{ 2 }.
  \end{equation}
  Ten przypadek jest więc udowodniony.

\item $\alpha = 2$, $\beta = 1$. Lewa strona
  \begin{equation}
    \label{eq:9}
    \theta^{ 2 } \theta^{ 1 }
  \end{equation}
  Prawa stona
  \begin{equation}
    \label{eq:10}
    -\frac{ 1 }{ 2 } \varepsilon^{ 2 1 } \theta \theta
    = -\frac{ 1 }{ 2 } \varepsilon^{ 2 1 } ( -2 \theta_{ 1 } \theta_{ 2 } )
    = \varepsilon^{ 2 1 } \theta_{ 1 } \theta_{ 2 } = ( -1 ) \theta_{ 1 } \theta_{ 2 }
    = \theta_{ 2 } \theta_{ 1 }.
  \end{equation}
  Dowód zakończony.

\end{itemize}
Powyższe rozumowanie można byłoby uprościć, ale nam zależało by rachunki były możliwie jawne.

\textbf{Drugi dowód wzoru.} W tym dowodzie posłużymy się macierzami $\varepsilon^{ u }$ i~$\varepsilon_{ d }$. Pokażemy, jak możemy przekształcić prawą stronę by otrzymać lewą.
\begin{equation}
  \label{eq:11}
  -\frac{ 1 }{ 2 } \varepsilon^{ \alpha \beta } \theta \theta
  = -\frac{ 1 }{ 2 } \varepsilon^{ \alpha \beta } \theta^{ \gamma } \theta_{ \gamma }
  = -\frac{ 1 }{ 2 } \varepsilon^{ \alpha \beta } \theta^{ \gamma } \varepsilon_{ \gamma \delta } \theta^{ \delta }.
\end{equation}
Musiałem przerwać pisanie tego z braku czasu.
% Potrzebujemy przekształcić w jakiś sposób człon $\varepsilon^{ \alpha \beta } \varepsilon_{ \gamma \delta }$. Tożsamości \eqref{eq:Zmienne-Grassmana-14-A} i dalsze wymagają by co najmniej jeden indeks w obu macierzach był indeksem sumowania, dokonamy więc przekształcenia
% \begin{equation}
%   \label{eq:12}
%   \varepsilon^{ \alpha \beta } \varepsilon_{ \gamma \delta }
%   = \sum_{ \sigma = 1, 2 } \varepsilon^{ \alpha \sigma } \varepsilon_{ \sigma \delta } \delta^{ \beta }_{ \sigma } \delta^{  }
% \end{equation}





% ############################
\subsection{Dwie przestrzenie zmiennych Grassmana}
% ############################


W fizyce zachodzi czasem potrzeba, by rozważać dwie \textit{przestrzenie} zmiennych Grassmana. W tej sytuacji obok bazowych zmiennych Grassmana $\theta_{ 1 }, \theta_{ 2 } \in \Theta$ w jednej przestrzeni, mamy jeszcze dwie bazowe zmienne Grassmana $\lambda_{ 1 }, \lambda_{ 2 } \in \Lambda$. Obowiązują dla nich te same reguły co dla zmiennych $\theta{ i }$.
\begin{subequations}
  \begin{align}
    \alpha \lambda_{ i } = \lambda_{ i } \alpha, \quad \alpha \in \Cbb, \\
    \lambda_{ i } \lambda_{ j } = -\lambda_{ j } \lambda_{ i }, \\
    \lambda^{ 2 } = -\lambda_{ 1 }, \quad \lambda^{ 1 } = \lambda_{ 2 }, \\
    \lambda \lambda = \lambda^{ \alpha } \lambda_{ \alpha } = 2 \lambda_{ 2 } \lambda_{ 1 },
  \end{align}
\end{subequations}
i tak dalej. Możemy teraz rozpatrzyć iloczyn również iloczyn zmiennych Grassmana z różnych przestrzeni.
\begin{equation}
  \label{eq:14}
  \lambda_{ 1 } \theta_{ 1 }, \quad \lambda \lambda \theta_{ 1 } = \lambda^{ \alpha } \lambda_{ \alpha } \theta_{ 1 }
  = 2 \lambda_{ 2 } \lambda_{ 1 } \theta_{ 1 }, \ldots
\end{equation}
Iloczyn typu $\lambda_{ 1 } \theta_{ 1 }$ nie należy ani do $\Theta$ ani do $\Lambda$, jest obiektem nowego typu, możemy ten typ oznaczyć przez $\Theta \times \Lambda$.

Zmienne Grassmana należą do różnych przestrzeni są \textit{przemienne} między sobą
\begin{equation}
  \label{eq:15}
  \theta_{ 1 } \lambda_{ 1 } = \lambda_{ 1 } \theta_{ 1 }.
\end{equation}
Dodatkowa dla liczby zespolonej $\alpha$ zachodzi
\begin{equation}
  \label{eq:16}
  \alpha \lambda_{ 1 } \theta_{ 1 } = \lambda_{ 1 } \alpha \theta_{ 1 } = \lambda_{ 1 } \theta_{ 1 } \alpha.
\end{equation}

Wprowadzamy następujące oznaczenie.
\begin{equation}
  \label{eq:19}
  \theta \lambda := \theta^{ \alpha } \lambda_{ \alpha }.
\end{equation}
Wynosi ono
\begin{equation}
  \label{eq:20}
  \theta \lambda = \theta_{ 2 } \lambda_{ 1 } - \theta_{ 1 } \lambda_{ 2 }.
\end{equation}
Choć zmienne Grassmana $\theta_{ i }$ i $\lambda_{ j }$ są przemienne, to we wzorze \eqref{eq:19} indeksy nie są umieszczone symetrycznie, a tym samym iloczyn $ \theta \lambda$ nie jest przemienny.
\begin{equation}
  \label{eq:21}
  \lambda \theta = \lambda_{ 2 } \theta_{ 1 } - \lambda_{ 1 } \theta_{ 2 }
  = \theta_{ 1 } \lambda_{ 2 } - \theta_{ 2 } \lambda_{ 1 }
  = -\theta \lambda.
\end{equation}
Możemy teraz pokazać \textbf{ale brak mi czasu by to napisać}
\begin{equation}
  \label{eq:22}
  \theta \lambda \theta \lambda = \frac{ 1 }{ 2 } \theta \theta \lambda \lambda \neq 0.
\end{equation}







% ############################
\section{Spinorowe zmienne Grassmana}
% ############################


Spinorowe zmienne Grassmana to zmienne z dodatkowej przestrzeni zmienny Grassmana $S \Theta$, której dwie (może być więcej, ale w fizyce zwykle wystarczą dwa) zmienne bazowe to $\bar{\theta}_{ \dot{ 1 } }$, $\bar{\theta}_{ \dot{ 2 } }$.

Wszystkie własności, poza jedną, są takie jak dla pozostałych zmiennych Grassmana. \textbf{Napiszę je jak będę miał czas.}

Jedynym wyjątkiem jest definicja symbolu $\bar{\theta} \bar{\theta}$
\begin{equation}
  \label{eq:17}
  \bar{\theta} \bar{\theta} \equiv \overline{\theta \theta}
  := \bar{\theta}_{ \dot{ \alpha } } \bar{\theta}^{ \dot{ \alpha } }.
\end{equation}
Porównaj położenie indeksów we wzorze \eqref{eq:Zmienne-Grassmana-18}. Możemy teraz obliczyć \textbf{nie możemy bo wzorów jeszcze nie napisałem}.
\begin{equation}
  \label{eq:18}
  \overline{\theta \theta} = 2 \bar{\theta}_{ \dot{ 1 } } \bar{\theta}_{ \dot{ 2 } }.
\end{equation}







% #####################################################################
% #####################################################################
% Bibliografia
\bibliographystyle{plalpha}

\bibliography{MathComScienceBooks}{}





% ############################

% Koniec dokumentu
\end{document}
