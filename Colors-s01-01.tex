% ---------------------------------------------------------------------
% Podstawowe ustawienia i pakiety
% ---------------------------------------------------------------------
\RequirePackage[l2tabu, orthodox]{nag}  % Wykrywa przestarzałe i niewłaściwe
% sposoby używania LaTeXa. Więcej jest w l2tabu English version.
\documentclass[a4paper,11pt]{article}
% {rozmiar papieru, rozmiar fontu}[klasa dokumentu]
\usepackage[MeX]{polski}  % Polonizacja LaTeXa, bez niej będzie pracował
% w języku angielskim.
\usepackage[utf8]{inputenc} % Włączenie kodowania UTF-8, co daje dostęp
% do polskich znaków.
\usepackage{lmodern}  % Wprowadza fonty Latin Modern.
\usepackage[T1]{fontenc}  % Potrzebne do używania fontów Latin Modern.



% ------------------------------
% Podstawowe pakiety (niezwiązane z ustawieniami języka)
% ------------------------------
\usepackage{microtype}  % Twierdzi, że poprawi rozmiar odstępów w tekście.
% \usepackage{graphicx}  % Wprowadza bardzo potrzebne komendy do wstawiania
% % grafiki.
\usepackage{vmargin}  % Pozwala na prostą kontrolę rozmiaru marginesów,
% za pomocą komend poniżej. Rozmiar odstępów jest mierzony w calach.
% ------------------------------
% MARGINS
% ------------------------------
\setmarginsrb
{ 0.7in}  % left margin
{ 0.6in}  % top margin
{ 0.7in}  % right margin
{ 0.8in}  % bottom margin
{  20pt}  % head height
{0.25in}  % head sep
{   9pt}  % foot height
{ 0.3in}  % foot sep





% ------------------------------
% Paczki, biblioteki i ich ustawienia dla tego pliku
% ------------------------------
\usepackage{tikz}  % Wspaniały pakiet PGF/TikZ.
% \usetikzlibrary{}

\usepackage{./packages/ColorsForTikZPictures}
\usepackage{./packages/JagiellonianColors}
\usepackage{./packages/VariousColors}





% ------------------------------
% TikZ pics i inne ustawienia dla tego pliku
% ------------------------------
\tikzset{
  point/.pic={
    \fill[color=\MathFGColor] (0,0) circle [radius=0.1];
  },
    cube/.pic={
    \fill[gray!40!black] (0,1,0) -- (1,1,0) -- (1,1,-1) -- (0,1,-1) -- cycle;

    \fill[gray!55!black] (0,1,0) -- (1,1,0) -- (1,0,0) -- (0,0,0) -- cycle;

    \fill[gray!70!black] (1,0,-1) -- (1,1,-1)
    -- (1,1,0) -- (1,0,0) -- cycle;

  }
}







% ------------------------------
% Ustawienie dla tego konkretnego pliku
% ------------------------------
% Light
\def\StylPliku{1}
% Dark
% \def\StylPliku{2}


\if\StylPliku1
\newcommand{\backgroundcolor}{jNormalMathTextBackgroundLight}
\newcommand{\MathFGColor}{jMathTextForegroundGrey}
\else
\newcommand{\backgroundcolor}{jNormalMathTextBackgroundDark}
\newcommand{\MathFGColor}{jMathTextForegroundWhite}
\fi







% ------------------------------
% Pakiet „hyperref”
% Polecano by umieszczać go na końcu preambuły
% ------------------------------
\usepackage{hyperref}  % Pozwala tworzyć hiperlinki i zamienia odwołania
% do bibliografii na hiperlinki.










% ---------------------------------------------------------------------
% Tytuł, autor, data
\title{Kolory Jagielloniana, wykładów z~Geometrii~3D i~Programowania Symulacji Fizyki}

\author{}

% \date{}
% ---------------------------------------------------------------------










% ####################################################################
\begin{document}
% ####################################################################





% ######################################
\maketitle % Tytuł całego tekstu
% ######################################





\pagecolor{\backgroundcolor}



\begin{tabular}{l|l}
  \textbf{Kolory trybu light (J)} & \textbf{Kolory trybu dark (J)} \\
  \hline
  \tikz \draw[color=black,fill=jNormalTextForegroundGrey] (0,0) rectangle
  (0.5,0.25); jNormalTextForegroundGrey: 7E7E7E
                                  & \tikz \draw[color=black,fill=jNormalTextForegroundBlueGrey]
                                    (0,0) rectangle (0.5,0.25);
                                    jNormalTextForegroundBlueGrey: 818F9B \\
  \tikz \draw[color=black,fill=jMathTextForegroundGrey] (0,0) rectangle
  (0.5,0.25); jMathTextForegroundGrey: 4E4E4E
                                  & \tikz \draw[color=black,fill=jMathTextForegroundWhite]
                                    (0,0) rectangle (0.5,0.25);
                                    jMathTextForegroundWhite: D6DEE5 \\
  \tikz \draw[color=black,fill=jNormalMathTextBackgroundLight]
  (0,0) rectangle (0.5,0.25); jNormalMathTextBackgroundLight: FAFAFA
                                  & \tikz \draw[color=black,fill=jNormalMathTextBackgroundLight]
                                    (0,0) rectangle (0.5,0.25);
                                    jNormalMathTextBackgroundLight: 023159
\end{tabular}

\vspace{2em}



\begin{tabular}{l|l}
  \textbf{Kolory osi (J)}
  & \textbf{Kolor dla współrzędnych jednorodnych (3D)} \\
  \hline
  \tikz \draw[color=black,fill=jAxisRed] (0,0) rectangle (0.5,0.25);
  jAxisRed: A21C0E
  & \tikz \draw[color=black,fill=homcolor] (0,0) rectangle (0.5,0.25);
    homcolor: DD5345 \\
  \tikz \draw[color=black,fill=jAxisGreen] (0,0) rectangle (0.5,0.25);
  jAxisGreen: 7FAA60
  & \textbf{Kolor światła (3D)} \\
  \tikz \draw[color=black,fill=jAxisBlue] (0,0) rectangle (0.5,0.25);
  jAxisBlue: 0089FF
  & \tikz \draw[color=black,fill=LightYellow] (0,0) rectangle (0.5,0.25);
    LightYellow: FBCA33
\end{tabular}

\vspace{2em}





\begin{tabular}{l|l|l}
  \textbf{Kolory (J)} & \multicolumn{2}{|l}{\textbf{Kolory brył 3D
                        (Symulacje)}} \\
  \hline
  \tikz \draw[color=black,fill=jAxisGreen] (0,0) rectangle (0.5,0.25);
  jAxisGreen: 14B03D
                      & \tikz \draw[color=black,fill=StrongGreen]
                        (0,0) rectangle (0.5,0.25);
                        StrongGreen: 52, 111, 72 (RGB)
                                                & \tikz \draw[color=black,fill=SoftGreen] (0,0) rectangle (0.5,0.25);
                                                  SoftGreen:
                                                  91, 173, 69 (RGB) \\
  \tikz \draw[color=black,fill=jAxisBlue] (0,0) rectangle (0.5,0.25);
  jAxisBlue: 0089FF
                      & \tikz \draw[color=black,fill=StrongBlue]
                        (0,0) rectangle (0.5,0.25);
                        StrongBlue: 29689E
                                                & \tikz \draw[color=black,fill=SoftBlue] (0,0) rectangle (0.5,0.25);
                                                  SoftBlue: 2B8ADB \\
                        \tikz \draw[color=black,fill=jViolet] (0,0) rectangle (0.5,0.25);
                        jViolet: 8E58CB
                                                & \tikz \draw[color=black,fill=StrongViolet]
                                                  (0,0) rectangle (0.5,0.25); Strong Violet: 523886F
                & \tikz \draw[color=black,fill=SoftViolet] (0,0) rectangle (0.5,0.25);
                  SoftViolet: 8E48CB \\
  \tikz \draw[color=black,fill=jDarkOrange] (0,0) rectangle (0.5,0.25);
  jDarkOrange: EB811B
                      & \tikz \draw[color=black,fill=StrongOrange]
                        (0,0) rectangle (0.5,0.25);
                        StrongOrange: A46933
                & \tikz \draw[color=black,fill=SoftOrange]
                  (0,0) rectangle (0.5,0.25); SoftOrange: 219, 144, 79 (RGB) \\
  \tikz \draw[color=black,fill=jGrey] (0,0) rectangle (0.5,0.25);
  jGrey: 8F8E8E & % \tikz \draw[color=black,fill=DarkGrey]
                  % (0,0) rectangle (0.5,0.25); DarkGrey: 757575
                & % \tikz \draw[color=black,fill=SoftGrey]
                  % (0,0) rectangle (0.5,0.25); SoftGrey: BEBEBE
  \\
  \tikz \draw[color=black,fill=jStrongWhite] (0,0) rectangle (0.5,0.25);
  jStrongWhite: FBFBFB & &
\end{tabular}

\vspace{2em}



\begin{tabular}{l|l}
  \multicolumn{2}{l}{\textbf{Kolory jeszcze nie użyte}} \\
  \hline
  \tikz \draw[color=black,fill=SoftGrey] (0,0) rectangle (0.5,0.25);
  SoftGrey: BEBEBE
  & \tikz \draw[color=black,fill=DarkGrey] (0,0) rectangle (0.5,0.25);
    DarkGrey: 757575 \\
  \tikz \draw[color=black,fill=VerySoftBlue] (0,0) rectangle (0.5,0.25);
  VerySoftBlue: D8D3EA & \\
  \tikz \draw[color=black,fill=StrongCone] (0,0) rectangle (0.5,0.25);
  StrongCone: 212C36 & \tikz \draw[color=black,fill=SoftCone]
                       (0,0) rectangle (0.5,0.25); SoftCone: 048FBC
\end{tabular}


\tikz \draw[color=black,fill=alternatywnyStrongViolet] (0,0) rectangle
(0.5,0.25); alternatywnyStrongViolet: 5C3F7D









% ############################

% Koniec dokumentu
\end{document}