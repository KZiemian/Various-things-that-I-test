% Autor: Kamil Ziemian

% ---------------------------------------------------------------------
% Podstawowe ustawienia i pakiety
% ---------------------------------------------------------------------
\RequirePackage[l2tabu, orthodox]{nag} % Wykrywa przestarzałe i niewłaściwe
% sposoby używania LaTeXa. Więcej jest w l2tabu English version.
\documentclass[a4paper,11pt]{article}
% {rozmiar papieru, rozmiar fontu}[klasa dokumentu]
\usepackage[MeX]{polski} % Polonizacja LaTeXa, bez niej będzie pracował
% w języku angielskim.
\usepackage[utf8]{inputenc} % Włączenie kodowania UTF-8, co daje dostęp
% do polskich znaków.
\usepackage{lmodern} % Wprowadza fonty Latin Modern.
\usepackage[T1]{fontenc} % Potrzebne do używania fontów Latin Modern.



% ---------------------------------------
% Podstawowe pakiety (niezwiązane z ustawieniami języka)
% ---------------------------------------
\usepackage{microtype} % Twierdzi, że poprawi rozmiar odstępów w tekście.
% \usepackage{graphicx} % Wprowadza bardzo potrzebne komendy do wstawiania
% grafiki.
% \usepackage{verbatim} % Poprawia otoczenie VERBATIME.
% \usepackage{textcomp} % Dodaje takie symbole jak stopnie Celsiusa,
% wprowadzane bezpośrednio w tekście.
\usepackage{vmargin} % Pozwala na prostą kontrolę rozmiaru marginesów,
% za pomocą komend poniżej. Rozmiar odstępów jest mierzony w calach.
% ---------------------------------------
% MARGINS
% ---------------------------------------
\setmarginsrb
{ 0.7in}  % left margin
{ 0.6in}  % top margin
{ 0.7in}  % right margin
{ 0.8in}  % bottom margin
{  20pt}  % head height
{0.25in}  % head sep
{   9pt}  % foot height
{ 0.3in}  % foot sep



% ---------------------------------------
% Często używane pakiety
% ---------------------------------------
% \usepackage{csquotes} % Pozwala w prosty sposób wstawiać cytaty do tekstu.



% ---------------------------------------
% Pakiety do tekstów z nauk przyrodniczych
% ---------------------------------------
\let\lll\undefined % Amsmath gryzie się z językiem pakietami do języka
% % polskiego, bo oba definiują komendę \lll. Aby rozwiązać ten problem
% % oddefiniowuję tę komendę, ale może tym samym pozbywam się dużego Ł.
\usepackage[intlimits]{amsmath} % Podstawowe wsparcie od American
% Mathematical Society (w skrócie AMS)
\usepackage{amsfonts, amssymb, amscd, amsthm} % Dalsze wsparcie od AMS
% % \usepackage{siunitx} % Dla prostszego pisania jednostek fizycznych
% \usepackage{upgreek} % Ładniejsze greckie litery
% % Przykładowa składnia: pi = \uppi
% \usepackage{slashed} % Pozwala w prosty sposób pisać slash Feynmana.
% \usepackage{calrsfs} % Zmienia czcionkę kaligraficzną w \mathcal
% % na ładniejszą. Może w innych miejscach robi to samo, ale o tym nic
% % nie wiem.





% ---------------------------------------
% Dodatkowe ustawienia dla języka polskiego
% ---------------------------------------
\renewcommand{\thesection}{\arabic{section}.}
% Kropki po numerach rozdziału (polski zwyczaj topograficzny)
\renewcommand{\thesubsection}{\thesection\arabic{subsection}}
% Brak kropki po numerach podrozdziału



% ---------------------------------------
% Pakiety napisane przez użytkownika.
% Mają być w tym samym katalogu to ten plik .tex
% ---------------------------------------
% \usepackage{latexshortcuts}
% \usepackage{mathshortcuts}



% ---------------------------------------
% Ustawienia różnych parametrów tekstu
% ---------------------------------------
\renewcommand{\arraystretch}{1.2} % Ustawienie szerokości odstępów między
% wierszami w tabelach.





% ---------------------------------------
% Pakiet "hyperref"
% Polecano by umieszczać go na końcu preambuły.
% ---------------------------------------
\usepackage{hyperref} % Pozwala tworzyć hiperlinki i zamienia odwołania
% do bibliografii na hiperlinki.










% ---------------------------------------------------------------------
% Tytuł, autor, data
\title{Zadania do zrobienia, fizyka}

% \author{}
% \date{}
% ---------------------------------------------------------------------










% ####################################################################
% Początek dokumentu
\begin{document}
% ####################################################################





% ######################################
\maketitle  % Tytuł całego tekstu
% ######################################





% ######################################
\section{Zrób te zadania}

% \vspace{\spaceTwo}

% ######################################





\begin{enumerate}

\item Wyznaczyć relacje dyspersji dla jednowymiarowej sieci złożonej z
  identycznych atomów przy założeniu, że stałe siłowe opisujące
  oddziaływanie par atomów są na przemian równe $C_{ 1 }$ i~$C_{ 2 }$
  ($C_{ 1 } \neq C_{ 2 }$).

\item Oblicz $c_{ V }( T )$ przyjmujące
  $D( \omega ) = 3 N \delta( \omega - \omega_{ E } )$, gdzie $N$ to
  liczba atomów w~krysztale. Dla $T \to \infty$ pokaż, że
  $c_{ V }( T )$ spełnia prawo Dulonga-Petita.

\item Pokaż, że dla $N$ atomów w sześcianie o~bloku $L$, gęstość stanów w~przestrzeni pędów jest dana przez $\rho( k ) = \left( \frac{ L }{ 2\pi } \right)^{ 3 }$. \\
  Używając zależności dyspersyjnej $\omega = \nu k$, pokaż że
  $D( \omega ) = \frac{ 3 \omega^{ 2 } L^{ 3 } }{ 2\pi^{ 2 } \nu^{ 3 }
  }$ dla
  $\omega < \omega_{ D } = ( 6 \pi^{ 2 } N )^{ \frac{ 1 }{ 3 } }
  \frac{ V }{ L }$. Oblicz $c_{ V }( T )$ i pokaż, że dla
  $T \searrow 0$, $c_{ V }( T ) \propto T^{ 3 }$.

\item Dystrybucję $\delta( x )$ Diraca można określić jako granicę funkcji
  regualarnych
  \begin{equation}
    \label{eq:1}
    \delta( x ) = \lim_{ a \to 0 } \rho_{ a }( x ), \qquad
    \rho_{ a }( x ) = \frac{ 1 }{ a \sqrt{ \pi } } e^{ -\frac{ x^{ 2 } }{ a^{ 2 } } }
  \end{equation}
  Mówimy, że ciąg funkcji $\rho_{ a }( x )$ jest modelem dla funkcji
  $\delta( x )$. Udowodnić, że $\delta( x )$ ma następujące własności
  \begin{align}
    &\int_{ a }^{ b } \delta( x - x_{ 0 } ) \, dx = 1, \qquad
      x_{ 0 } \in ( a, b ) \\
    &\int_{ a }^{ b } \delta( x - x_{ 0 } ) \, dx = 0, \qquad
      x_{ 0 } \notin ( a, b ) \\
    &\int_{ -\infty }^{ +\infty } \delta( x - x_{ 0 } ) \phi( x ) \, dx = \phi( x_{ 0 } )
  \end{align}
  dla dowolnej funkcji próbnej (tj. regularnej) $\phi( x )$. \\
  Wykorzystując ostatnią własność, oraz właściwości odwrotnej
  transformaty Fouriera udowodnić, że
  \begin{equation}
    \label{eq:2}
    \delta( x ) = \frac{ 1 }{ 2\pi } \int_{ -\infty }^{ +\infty } e^{ i p x } \, dp
  \end{equation}
  Zapisując ostatnią całkę jako
  \begin{equation}
    \label{eq:3}
    \lim_{ \varepsilon \to 0 } \frac{ 1 }{ 2\pi } \int_{ -\infty }^{ +\infty } e^{ ipx - \varepsilon | p | } \, dp,
  \end{equation}
  udowodnić inną, równoważną, reprezentację dystrybucji $\delta( x )$
  \begin{equation}
    \label{eq:4}
    \delta( x ) = \lim_{ \varepsilon \to 0 } f_{ \varepsilon }( x ), \qquad
    f_{ \varepsilon }( x ) = \frac{ 1 }{ \pi } \frac{ \varepsilon }{ x^{ 2 } + \varepsilon^{ 2 } }
  \end{equation}

  Wykreślić funkcje modelujące dla kilku wartości $\varepsilon$. Uzasadnić, że
  rzeczywiście jest to dobry model funkcji $\delta$. W~szczególności,
  sprawdzić warunek normalizacji oraz że dla dowolnej funkcji próbnej
  zachodzi
  \begin{equation}
    \label{eq:5}
    \lim_{ \varepsilon \to 0} \int_{ -\infty }^{ +\infty } f_{ \varepsilon }( x - x_{ 0 } ) \phi( x ) \, dx
    =
    \phi( x_{ 0 } )
  \end{equation}
  Ostatnią całkę wykonać metodą residuów zakładając wystarczająco
  szybkie znikanie funkcji próbnej na dużych okręgach.

  Udowodnić, że fale płaskie są znormalizowane do $\delta$ Diraca.

\item Rozwiązać zależne od czasu równanie Schr\"{o}dingera dla cząstki
  swobodnej w jednym i trzech wymiarach, za pomocą separacji
  zmiennych. Jako warunek początkowy przyjąć, że gęstość
  prawdopodobieństwa znalezienia cząstki w~przestrzeni ma rozkład
  gaussowski o~dyspersji $\sigma^{ 2 } = a^{ 2 }$ i że pakiet ma średnią
  prędkość $v_{ 0 }$. Wykonać \emph{explicite} wszystkie transformaty
  Fouriera i~uzyskać jawne wyrażenie na $\Psi( x, t )$. Obliczyć gęstość
  prawdopodobieństwa $\rho( x, t )$ i gęstość strumienia
  prawdopodobieństwa $\vec{ j }( x, t )$. Sprawdzić równanie
  ciągłości. Wyprowadzić analogiczne wzory (np. na
  $\vec{ j }( x, t )$) dla zbioru cząstek klasycznych.


  % \item Andrew Klavan, \emph{Prawdziwa zbrodnia};

  % \item \emph{Teoria pomiarów};

  % \item Wojciech Roszkowski, \emph{Świat Chrystusa. Tom I};

  % \item Michał Heller, Józef Życiński, \emph{Wszechświat --~maszyna
  %   czy~myśl?};

  % \item Red. L. A. Steen, \emph{Matematyka współczesna. Dwanaście
  %   esejów};

  % \item S. J. Gould, \emph{Niewczesny pogrzeb Darwina};

  % \item Olga Tokarczuk, \emph{Bieguni};

  % \item Hanya Yanagihara, \emph{Małe życie};

  % \item Homer, \emph{Iliada}, \emph{Odyseja};

  % \item Ks. Jelonek, \emph{Wprowadzenie do Biblii};

  % \item Gustaw Meyrink, \emph{Golem};

  % \item Tomasz Mann, \emph{Dokotor Faustus};

  % \item Roger Scruton, \emph{Przewodnik po~kulturze współczesnej dla
  %   inteligentnych};

  % \item Radek Rak, \emph{Baśń o wężowym sercu albo wtóre słowo o
  %   Jakubie Szeli};

  % \item Alain Besancon, \emph{Anatomia widma};

  % \item Red. E. Tarkowska, \emph{Zrozumieć biednego. O~dawnej
  %   i~obecnej biedzie w~Polsce};

  % \item Mario Vatgas Lloysa, \emph{Rozmowa w katedrze};

  % \item Juan Gabriel Vasquez, \emph{Kształt ruin};

  % \item Marquez, \emph{Miłość w czasach zarazy};

  % \item Wiesław Myśliwski, \emph{Traktat o łuskaniu fasoli},
  %   \emph{Widnokrąg}, \emph{Kamień na kamieniu};

  % \item Kofman, Roszkowski, \emph{Transformacja i~postkomunizm};

  % \item Bourbaki, \emph{Elementy historii matematyki};

  % \item Platon, \emph{Państwo};

  % \item Arystoteles, \emph{Etyka Nikomochejska};

  % \item \emph{Recepcja w~Polsce nowych kierunków i~teorii
  %   naukowych};

  % \item Herodot;

  % \item H. Steinhaus;

  % \item \emph{Liryka polska. Interpretacje};

  % \item \emph{Zarys dziejów filozofii w~Polsce};

  % \item \emph{Państwo Boże};

  % \item \emph{Drabina Raju};

  % \item R. Krasowski;

  % \item Benedict, \emph{Wzory kultury};

  % \item I. Bokwa, \emph{Wprowadzenie do teologii Karla Rahnera};

  % \item R. Brandstaer, \emph{Patriarchowie};

  % \item G. K. Chesterton, \emph{Ortodoksja};

  % \item M. Davies, \emph{Liturgiczne bomby zegarowe Vaticanum II.
  %   Zniszczenie katolickiej wiary przez zmiany w katolickim kulcie};

  % \item Michał Heller, \emph{Filozofia przyrody};

  % \item A. MacIntyre, \emph{Dziedzictwo cnoty};

  % \item Ks. Józef Tishner, \emph{Nieszczęsny dar wolności};

  % \item M. Takesaki, \emph{Theory of operator algebras};

  % \item R. Penrose, \emph{Droga do rzeczywistości};

  % \item P. Johnson, \emph{Historia świata XX wieku};

  % \item M. Dzielski;

  % \item A. Zamoyski, \emph{Własną drogą};

  % \item P. Johnson, \emph{Narodziny nowoczesności};

  % \item R. Terlecki, \emph{Dzieje sowieckiej kolonii};

  % \item B. Cywiński, \emph{Rodowody niepokornych};

  % \item A. Nowak, \emph{Od Polski do postpolityki};

  % \item P. Gay;

  % \item A. Leder, \emph{Prześniona rewolucja};

  % \item \emph{Oświecenie dzisiaj};

  % \item E. J. Hobsbawn, \emph{Tradycje wynalezione};

  % \item \emph{Słowacki. Szat-Anioł};

  % \item T. Enderson, \emph{Kultura popularna, tożsamość narodowa
  %   i~życie codzienne};

  % \item A. Friszke, \emph{Rok 1989~r.};

  % \item R. Legutko, \emph{Esej o~duszy polskiej};

  % \item K. Brodacki, \emph{Trzy twarze Juliana Haraschina};

  % \item J. Kurtyka, \emph{Z dziejów agonii i~podboju. Prace zebrane
  %   z~zakresu najnowszej historii Polski};

  % \item G. Kucharczyk, \emph{Polska myśl polityczna po 1939~r.};

  % \item A. L. Sowa, \emph{Historia polityczna Polski 1944-1991};

  % \item K. Janicki, \emph{Epoka hipokryzji. Seks i~erotyka
  %   w~przedwojennej Polsce};

  % \item A. Wielomski, \emph{Prawica w~XX wieku};

  % \item N. Ferguson, \emph{Niebezpieczne związki};

  % \item W. Roszkowski, \emph{Najnowsza historia Polski};

  % \item Frank E. Manuel, (książka o~religioznawstwie);

  % \item Urs Alterman;

  % \item \emph{Chrześcijaństwo, demokracja, kapitalizm};

  % \item J. Aumont, M. Marie, \emph{Analiza filmu};

  % \item J. Osterhammel, \emph{Historia XIX wieku. Przebudowa
  %   świata};

  % \item P. Holmes, \emph{Wiek cudów};

  % \item E. Black, \emph{Wojna przeciw słabym};

  % \item \emph{Polska poezja baroku};

  % \item J. Gray, \emph{Liberalizm};

  % \item E. Gellner;

  % \item A. Golicyn, \emph{Nowe kłamstwa w~miejsce starych};

  % \item \emph{Monografie historii nauki \textsc{pau}};

  % \item J. Browne, \emph{Darwin, o~powstawaniu gatunków. Biografia};

  % \item R. Butterwick, \emph{Polska rewolucja a~Kościół Katolicki};

  % \item B. Gogol, \emph{Czerwony Sztandar. Rzecz o~sowietyzacji ziem
  %   Małopolski Wschodniej};

  % \item Timothy Gray Ash, \emph{Polska rewolucja. Solidarność};

  % \item A. Brzezicki, \emph{Tadeusz Mazowiecki. Biografia naszego
  %   premiera};

  % \item I. Berlin, \emph{Korzenie romantyzmu};

  % \item E. Lucas, \emph{Nowa zimna wojna};

  % \item A. R. Hall, \emph{Rewolucja naukowa 1500-1800};

  % \item P. Jenkis, \emph{Historia Stanów Zjednoczonych};

  % \item Lew Gumilow;

  % \item M. Janowski, \emph{Polska myśl liberalna do~1918 roku};

  % \item M. Blondel;

  % \item K. Wyka, \emph{Pan Tadeusz. Studia o~poemacie};

  % \item M. Zaremba, \emph{Wielka trwoga};

  % \item \emph{Eklektyzm, synkretyzm, uniwersa};

  % \item P. Fiegut, \emph{Poezja w fazie krytycznej};

  % \item T. Snyder, \emph{Nacjonalizm, marksizm i współczesna Europa
  %   Środkowa};

  % \item Ziemkiewicz, \emph{Polactwo};

  % \item M. Voelle, et al., \emph{Człowiek Oświecenia};

  % \item Ks. J. Tischner, \emph{Polski kształt dialogu};

  % \item F. Collins, \emph{Język Boga};

  % \item F. Braudel;

  % \item Ks. J. Tischner, \emph{Nieszczęsny dar wolności};

  % \item Ks. J. Tischner, \emph{Ksiądz na manowcach};

  % \item \emph{Darwin, żywot uczonego};

  % \item Roger Kimball, \emph{The rape of the masters: how political
  %   correctness sabotages art};

  % \item Arnold Janssen;

  % \item Ludwik von Mises, \emph{Socjalizm};

  % \item Robert Conquest, \emph{Wielki terror};

  % \item \emph{Wojna przeciw słabym};

  % \item S. Runciman, \emph{Dzieje wypraw krzyżowych};

  % \item R. Graves, \emph{Mity greckie};

  % \item R. Graves, \emph{Mity hebrajskie};

  % \item L. Strauss, \emph{Prawo naturalne w świetle historii};

  % \item Berlinski, \emph{Szatańskie urojnie};

  % \item R. Wiltgen, \emph{Ren wpada do Tybru};

  % \item Sokal, Briemont;

  % \item Paul Davies;

  % \item Wolfgang Schivelbusch, \emph{Culture of Defeat: On National
  %   Trauma, Mourning, and Recovery};

  % \item Bockenheim K., \emph{Dworek, kontusz, karabela};

  % \item Paweł Śpiewak, \emph{Gramsci};

  % \item Polska 1989-2009: ilustrowany komentarz historyczny;

  % \item Spór o Polskę 1989-99: wybór tekstów prasowych;

  % \item J. Tazbir, \emph{Świat panów Pasków};

  % \item A. Sosnowska, \emph{Zrozumieć zacofanie: spory historyków
  %   o~Europę Wschodnią, 1947-1994};

  % \item \emph{Patron i dwór. Magnateria Rzeczypospolitej w XVI-XVIII
  %   wieku.};

  % \item Jon Dover, Helen W.~Kennedy, \emph{Kultura gier
  %   komputerowych};

  % \item Diarmaid MacCulloch, \emph{The Reformation: A History},
  %   alternatywny tytuł to \emph{Reformation: Europe's House
  %   Divided};

  % \item \emph{Skonsumowani: jak rynek psuje dzieci, infantylizuje
  %   dorosłych i~połyka obywateli};

  % \item Bartłomiej Dobroczyński, \emph{New Age};

  % \item Leszek Kołakowski, \emph{Główne nurty marksizmu};

  % \item \emph{Państwo Boże Osiemnastowiecznych Filozofów};

  % \item \emph{Liberty. The god that Failed};

  % \item Ron Jeffery, \emph{Wisła jak krew czerwona};

  % \item R. Browning, \emph{Cesarstwo Bizantyńskie}, \emph{Justynian
  %   i~Teodora};

  % \item H. Chadwick, \emph{Historia rozłamu Kościoła Wschodniego
  %   i~Zachodniego. Od~czasów apostolskich do~soboru florenckiego};

  % \item P. K. Hitti, \emph{Dzieje Arabów};

  % \item H. Kennedy, \emph{Wielkie arabskie podboje};

  % \item \emph{Historia Persji. Tom~I. Od~czasów najważniejszych
  %   do~najazdu arabów};

  % \item Michał Lubina, \emph{Niedźwiedź w~cieniu smoka. Rosja-Chiny
  %   1991--2014};

  % \item Bronisław Wildstein, \emph{Śmieszna dwuznaczność świata,
  %   który oszalał};

  % \item Bronisław Wildstein, \emph{Długi cień PRL-u, czyli
  %   dekomunizacja której nie było};

  % \item D. Góra-Szopiński, \emph{Zakorzenienie wolności. Myśl
  %   polityczna Michaela Novaka};

  % \item J. Grzybowski, \emph{Jacques Maritain i nowa cywilizacja
  %   chrześcijańska};

  % \item Roger Kimball, \emph{Długi marsz: jak rewolucja kulturalna z
  %   lat 60. zmieniła Amerykę};

  % \item Bellantoni Patti, \emph{Jeśli to fiolet, ktoś umrze. Teoria
  %   koloru w~filmie};

  % \item A.~Wolff-Powęzka, \emph{Pamięć --~brzemię i~uwolnienie.
  %   Niemcy wobec nazistowskiej przeszłości (1945--2010)};

  % \item Valentin L. Popov, \emph{Contact Mechanics and Friction:
  %   Physics Principles and Applications};

  % \item L. Ambrosio, N. Dancer, \emph{Calculus of Variations and
  %   Partial Differential Equations: Topics on Geometrical Evolution
  %   Problems and Degree Theory};

  % \item Stephen Wiggins, \emph{Global Bifurcations and Chaos:
  %   Analytical Methods};

  % \item Serbio Albeverio, \emph{Operator Methods in Ordinary and
  %   Partial Differential Equations};

  % \item D. Boccaletti, G. Pucacco, \emph{Theory of Orbits. 1:
  %   Integrable Systems and Non-perturbative Methods};

  % \item Vasil E. Tarasov, \emph{Fractional Dynamics: Applications of
  %   Fractional Calculus to Dynamics of Particles, Fields and Media};

  % \item E. C. Curtius;

  % \item Ch. West, \emph{Teologia ciała dla początkujących};

  % \item R. Hilbert, \emph{Zagłada Żydów Europejskich};

  % \item J. Delumeau, \emph{Cywilizacja odrodzenia};

  % \item P. Manent, \emph{Intelektualna historia liberalizmu};

  % \item B. Baczko, \emph{Filozofia francuskiego oświecenia};

  % \item J. Juszczak, \emph{Ordoliberalizm};

  % \item Albaro Vargas Llosa, \emph{Mit Che a przyszłość wolności};

  % \item T. Snyder, \emph{Rekonstrukcja narodów};

  % \item Red. M. Rechowicz, \emph{Dzieje teologii katolickiej w
  %   Polsce};

  % \item Ricceure, \emph{Symbolika zła};

  % \item J. Śniadecki;

  % \item S. McMeekin, \emph{Największa grabież w historii. Jak
  %   bolszewicy złupili Rosję};

  % \item R. Syme, \emph{Rewolucja rzymska};

  % \item E. von Kuchnelt-Leddhin, \emph{Ślepy tor};

  % \item A. McGrath, \emph{Jan Kalwin. Studium kształtowania się
  %   kultury Zachodu};

  % \item P. Kuncewicz, \emph{Samotni wobec historii};

  % \item C. Ginzburg, \emph{Ser i~robak};

  % \item James Conrayd Martin, \emph{Nie ponaglaj rzeki};

  % \item W. Zajewski, \emph{Czy historycy piszą prawdę};

  % \item S. Węgrzynowicz, \emph{Patrioci i zdrajcy};

  % \item F. Musiał, \emph{Raj grabarzy narodu};

  % \item Red. Marek Kornat, \emph{Pius XII --~papież w~epoce
  %   totalitaryzmów};

  % \item H. Głębocki, \emph{,,Diabeł Asmodeusz'' w~niebieskich
  %   binoklach i~kraj przyszłości. Henryk Gurowski i~Rosja};

  % \item R. Fiegut, \emph{Zaproszenie do „Quidama”};

  % \item M. Goliczak, \emph{Związek Radziecki w~myśli politycznej
  %   polskiej opozycji 1976-1989};

  % \item M. Urbankowski, \emph{Romans z~Polską};

  % \item E. J\"{u}nger, \emph{Węzeł gordyjski. Eseistyka lat
  %   pięćdziesiątych};

  % \item J. Besal, \emph{Stanisław Żółkiewski};

  % \item J. Skowronek, \emph{Adam Jerzy Czartoryski, 1770-1861};

  % \item C. Shindler, \emph{Historia współczesnego Izraela};

  % \item W. Bernacki, \emph{Myśl polityczna I Rzeczpospolitej};

  % \item \emph{Polsko, uwierz w~swoją siłę};

  % \item Encyclopedia of Mathematical Sciences;

  % \item \emph{Open GL. Księga eksperta};

  % \item A. Nowak, \emph{Putin. Źródła imperialnej agresji};

  % \item Red. P. Musiewicz, \emph{Ronald Reagan. Nowa odsłona w
  %   100-lecie urodzin};

  % \item H. Pilus, \emph{Własność i zasady w katolickiej myśli
  %   społecznej};

  % \item N. von Below, \emph{Byłem adiunktem Hitlera};

  % \item G. Kucharczyk, \emph{Czerwone karty Kościoła};

  % \item G. Kucharczyk, \emph{Kielnią i cyrklem. Laicyzacja Francji w
  %   latach 1870-1914};

  % \item F. Koneczny, \emph{Dzieje Polski opowiedziane dla
  %   młodzieży};

  % \item M. Ekstein, \emph{Święto Wiosny. Wielka wojna i narodziny
  %   nowego wieku};

  % \item B. Kiereś, \emph{Tylko rodzina!};

  % \item M. Soska, \emph{Za Świętą Ruś. Współczesny nacjonalizm
  %   Rosyjski -- zarys ideologi};

  % \item H. Pająk, \emph{Rytualna zemsta na~„kolebce” Solidarności
  %   1981-2011};

  % \item M. Skousen, \emph{Narodziny współczesnej ekonomii};

  % \item P. Gontarczyk, \emph{Najnowsze kłopoty z~historią};

  % \item L. de~Whol;

  % \item G. Bardy, \emph{Charles de~Gaulle. Biografia katolika i~męża
  %   stanu};

  % \item J. Garrison, \emph{Ameryka jako imperium. Przywódcy świata
  %   czy bandycka potęga};

  % \item E. Lucas, \emph{Operacja Snowden};

  % \item C. S. Lewis, \emph{Ostatnia noc świata};

  % \item N. Janner SJ, \emph{Krótka historia Kościoła Katolickiego.
  %   Nowe spojrzenie};

  % \item \emph{Eklektyzmy, synkretyzmy, uniwersa};

  % \item \emph{Literahistorica};

  % \item \emph{Bóg Zła};

  % \item J. Wieliczka-Szarkowa, \emph{III Rzesza. Zbrodnia bez kary};

  % \item P. Gontarczyk, \emph{Polska Partia Robotnicza. Droga do
  %   władzy 1941-1944};

  % \item P. Moa, \emph{Mity wojny domowej w Hiszpania 1936--1939};

  % \item Władymir Arnold, \emph{Lectures on Partial Differential
  %   Equations};

  % \item Herman H. Goldstein, \emph{A History of Numerical Analysis.
  %   From the 16th through the 19th century};

  % \item G. Edward Griffin, \emph{Finansowy potwór z Jekyll Island};

  % \item L. Ulicka, \emph{Daniel Stein, tłumacz};

  % \item R. Scruton, \emph{Kultura jest ważna};

  % \item P. Gottfried, \emph{Wojna i demokracja};

  % \item S. Didler, \emph{Rola neofitów w dziejach Polskich};

  % \item A. Wielomski, \emph{Konserwatyzm. Główne idee i postaci};

  % \item C. S. Lewis, \emph{Bóg na ławie oskarżonych};

  % \item R. Spałek, \emph{Komuniści przeciw komunistom};

  % \item Doug Stanton, \emph{Dwunastu odważnych. Odtajniona historia
  %   konnych żołnierzy};

  % \item Wicek Warszawiak, \emph{Humor w~czasie okupacji};

  % \item Lawrence Wright, \emph{Wyniosłe wieże. Al-Kaida i~atak
  %   na~Amerykę};

  % \item Robert Mason, \emph{Powiedz, że się boisz};

  % \item Dla taty: Chufo Llorens;

  % \item Jearl Walker, \emph{Latający cyrk fizyki};

  % \item David J.~Griffits, \emph{Introduction to~Quantum Mechanics};

  % \item B. Dembowski, \emph{O filozofii chrześcijańskiej w Ameryce
  %   Północnej};

  % \item Wiesław Caban, \emph{Powstanie styczniowe. Polacy i Rosjanie
  %   w XIX wieku};

  % \item Jacek Wegner, \emph{Biesy sarmackie};

  % \item Andrzej Józef Kamiński, \emph{Koszmar niewolnictwa. Obozy
  %   koncentracyjne od 1896 do dziś. Analiza};

  % \item Jochen B\"{o}hler, \emph{Wojna domowa. Nowe spojrzenie
  %   na~odrodzenie Polski};

  % \item F. Wesołowski, \emph{Zasady muzyki};

  % \item Red. A. Czarniecka-Stefańska, \emph{Szukając prawdy. Edyta
  %   Stein w~kulturze polskiej};

  % \item E. Stein, \emph{Kobieta. Jej zadanie według natury i~łaski};

  % \item Red. Umberto Eco, \emph{Historia piękna};

  % \item Zdzisław Krasnodębski, \emph{Rozumienie ludzkiego
  %   zachowania. Rozważania o~filozoficznych podstawach nauk
  %   humanistycznych i~społecznych};

  % \item Adam Przechrzta, \emph{Chorągiew Michała Archanioła};

  % \item Anna Sobolewska, \emph{Mapy duchowe współczesności: co~nam
  %   zostało z~Nowej Ery?};

  % \item Blake J. Harris, \emph{Wojny konsolowe};

  % \item Nikołaj Zieńkowicz, \emph{Tajemnice mijającego wieku. Władza
  %   zakulisowe działania zatargi};

  % \item Nikołaj Zieńkowicz, \emph{Od~Lenina do~Jelcyna. Kremlowska
  %   księga zamachów};

  % \item Marek Jan Chodakiewicz, \emph{Transformacja
  %   czy~niepodległość?};

  % \item Helmuth Plessner, \emph{Śmiech i~płacz. Badania
  %   nad~granicami ludzkiego zachowania};

  % \item Samuel M.~Katz, \emph{Aman. Wywiad wojskowy Izraela};

  % \item Praca zbiorowa, \emph{Rosja --Chiny. Dwa modele
  %   transformacji};

  % \item Mariola Marczak, \emph{Poetyka filmu religijnego};

  % \item Lech Bukowski, \emph{Sade, Kafka, Kierkegaard. Między
  %   rozkoszą a opresją};

  % \item Philip Earl Steele, \emph{Nawrócenie i chrzest Mieszka I};

  % \item Henryk Samsonowicz, \emph{My o sobie. Portret własny
  %   mieszkańców ziem polskich u schyłku średniowiecza};

  % \item \emph{Piłsudski (nie)znany. Historia i popkultura};

  % \item Jakub Z.~Lichański, \emph{Niepopularnie o popularnej. O
  %   narzędziach badań literatury};

  % \item Tadeusz Manteuffel, \emph{Historia Powszechna.
  %   Średniowiecze};

  % \item Robert Jung, \emph{Jaśniej niż tysiąc słońc. Losy badaczy
  %   atomu};

  % \item Wiesław Bator, \emph{Religia starożytnego Egiptu.
  %   Perspektywa religioznawcza};

  % \item Gabriela Matuszek, \emph{Maski i demony wczesnego
  %   modernizmu};

  % \item Artur Szarecki, \emph{Kapitalizm somatyczny. Ciało i władza
  %   w kulturze korporacyjnej};

  % \item \emph{Francuskie pisma o dramacie (1537-1631)};

  % \item John A. McClure, \emph{Półwiary};

  % \item Jerzy Axer, Tadeusz Bujnicki, \emph{Wokół "W pustyni i w
  %   puszczy". W stulecie pierwodruku powieści};

  % \item Gerd-Klaus Kaltenbrunner;

  % \item Steve Brusatte, \emph{Era dinozaurów - od narodzin do
  %   upadku. Nowe odkrycia i fakty o zaginionym świecie};

  % \item Robert Fabbri, \emph{Wespazjan, trybun Rzymu};

  % \item Michael Billing, \emph{Banalny nacjonalizm};

  % \item Ernest Gellner, \emph{Narody i~nacjonalizm};

  % \item \emph{Oświecenie, nieoświecone. Człowiek, natura, magia};

  % \item Eric Hobsbawm, Terence Ranger, \emph{Tradycja wynaleziona};

  % \item Aleksander Śpiewakowski, \emph{Samuraje};

  % \item N. Davies, \emph{Serce Europy};

  % \item Thomas Hylland Eriksen, \emph{Etniczność i~nacjonalizm};

  % \item Benedict Andreson, \emph{Wspólnoty wyobrażone};

  % \item Karol Tarnowski, \emph{W~mroku uczonej niewiedzy};

  % \item Barbary Tuchman, \emph{Odległe zwierciadło, czyli
  %   rozlicznymi plagami nękane XIV stulecie};

  % \item Homi K. Bhabha, \emph{Miejsca kultury};

  % \item Tim Edensor, \emph{Tożsamość narodowa, kultura popularna
  %   i~życie codzienne};

  % \item Justyna Balisz-Schmelz, \emph{Przeszłość niepokonana. Sztuka
  %   niemiecka po 1945 roku jako przestrzeń i medium pamięci};

  % \item Anthony D.~Smith, \emph{Etniczne źródła narodów};

  % \item Anthony D.~Smith, \emph{Kulturowe podstawy narodów};

  % \item Piotr Eberhardt, \emph{Rozwój światowej myśli
  %   geopolitycznej};

  % \item P. Bąk, \emph{Gramatyka języka polskiego. Zarys popularny};

  % \item Jacek Wegner, \emph{Rzeczpospolita. Duma i~wstyd};

  % \item Wassily Kandinsky, \emph{Punkt i~linia a~płaszczyzna.
  %   Przyczynek do~analizy elementów malarskich};

  % \item Red. Aneta Pawłowska, Julia Sowińska-Heim, \emph{Afryka
  %   i~(post)kolonializm};

  % \item Robert J. C. Young, \emph{Postkolonializm. Wprowadzenie};

  % \item G.~Michaelson, \emph{An Introduction to~Functional
  %   Programming through Lambda Calculus};

  % \item Gilberto Freyre, \emph{Panowie i niewolnicy};

  % \item H. P. Barendregt, \emph{The Lambda Calculus: Its Syntax and
  %   Semantics};

  % \item Leon Degrelle, \emph{Wiek Hitlera};

  % \item N. D. Jones, \emph{Computability and Complexity: From
  %   a~Programming Perspective};

  % \item Ks. Marcin Worbs, \emph{Człowiek w~misterium liturgii};

  % \item Anna Grześkowiak-Krwawicz, \emph{Dyskurs polityczny
  %   Rzeczypospolitej Obojga Narodów};

  % \item Nowak \emph{Dzieje Polski};

  % \item Margaret Atwood, \emph{Dług. Rozrachunek z~ciemną stroną
  %   bogactwa};

  % \item Jared Diamond, \emph{Strzelby, zarazki, maszyny. Losy
  %   ludzkich społeczeństw};

  % \item Edward E. Evans\dywiz Pritchard, \emph{Czary, wyrocznie
  %   i~magia u~Azande};

  % \item W. Szumowski, \emph{Historia medycyny filozoficznie ujęta};

  % \item Michał Dondzik, Krzysztof Jajko, Emil Swoiński,
  %   \emph{Elementarz Wytwórni Filmów Oświatowych};

  % \item James M. Murray, \emph{Brugia: Kolebka kapitalizmu};

  % \item Maciej Janowski, \emph{Narodziny inteligencji: 1750--1831};

  % \item Jerzy Jedlicki, \emph{Jakiej cywilizacji Polacy potrzebują:
  %   studia z dziejów idei i~wyobraźni XIX wieku};

  % \item Jerzy Jedlicki, \emph{Droga do narodowej klęski};

  % \item Jerzy Jedlicki, \emph{Błędne koło: 1832-1864};

  % \item Jerzy Jedlicki, \emph{Nieudana próba kapitalistycznej
  %   industrializacji: analiza państwowego gospodarstwa przemysłowego
  %   w~Królestwie Polskim XIX w.};

  % \item Jerzy Jedlicki, \emph{Klejnot i~bariery społeczne:
  %   przeobrażenia szlachectwa polskiego w~schyłkowym okresie
  %   feudalizmu};

  % \item Jerzy Jedlicki, \emph{Świat zwyrodniały: lęki i~wyroki
  %   krytyków nowoczesności};

  % \item Rob Riemen, \emph{Wieczny powrót faszyzmu};

  % \item Łukasz A. Plesnar, \emph{Twarze Westernu};

  % \item S. Prat, \emph{Język C++. Szkoła programowania};

  % \item L. Strauss, \emph{Prawo naturalne w~świetle historii};

  % \item Oskar Halecki, \emph{Tysiąc lat polski katolickiej};

  % \item Krzysztof Mazur, \emph{Przekroczyć nowoczesność. Projekt
  %   polityczny ruchu społecznego Solidarność};

  % \item Marian Henryk Serejski, \emph{Europa a rozbiory Polski:
  %   studium historiograficzne};

  % \item Jerzy Łanowski, \emph{Antologia anegdoty antycznej: teraz
  %   trzeci raz wydane historyjki budujące i niebudujące z autorów
  %   greckich i rzymskich};

  % \item Artur Domosłowski, \emph{Kapuściński non-fiction};

  % \item Michael Moran, \emph{Kraj z Księżyca: podróże do serca
  %   Polski};

  % \item Pierre Hadot, \emph{Filozofia jako ćwiczenie duchowe};

  % \item Robert D. Richtmayer, \emph{Principles of advanced
  %   mathematical physics};

  % \item Walter Burkert, \emph{Stwarzanie świętości. Ślady biologii
  %   we~wczesnych wierzeniach religijnych};

  % \item A. I. Anselm, \emph{Podstawy fizyki statystycznej
  %   i~termodynamiki};

  % \item Z. Krasnodębski, \emph{Demokracja peryferii};

  % \item Maria Dzielska, \emph{Hypatia z~Aleksandrii};

  % \item Paweł Śpiewak, \emph{Spór o~Polskę, 1989--99};

  % \item Tadeusz Zieliński, \emph{Religia starożytnej Grecji};

  % \item Ernest Gellner, \emph{Postmodernizm, rozum i~religia};

  % \item Daniel Beauvois, \emph{Polacy na Ukrainie 1831-1863.
  %   Szlachta polska na Wołyniu, Podolu i Kijowszczyźnie};

  % \item \emph{Polskie mity polityczne XIX i XX wieku};

  % \item \emph{O nas bez nas. Historia Polski w historiografiach
  %   obcojęzycznych};

  % \item Leibniz, \emph{Wyznanie wiary filozofa, Rozprawa
  %   metafizyczna; Monadologia; Zasady natury i łaski oraz inne pisma
  %   filozoficzne};

  % \item Hanna Świda-Ziemba, \emph{Człowiek wewnętrznie zniewolony.
  %   Mechanizmy i konsekwencje minionej formacji --~analiza
  %   psychologiczna};

  % \item Paul Johnson, \emph{Historia świata XX wieku};

  % \item Braudel Fernand, \emph{Dynamika kapitalizmu};

  % \item I. Berlin, \emph{Korzenie romantyzmu};

  % \item Bod Rens, \emph{Historia humanistyki};

  % \item \emph{Teologia i filozofia żydowska wobec Holocaustu};

  % \item Acton, \emph{Historia wolności: wybór esejów};

  % \item Andrzej Żbikowski, \emph{Żydzi};

  % \item Stefan Bartkowski, \emph{Pod wspólnym niebem: krótka
  %   historia Żydów w~Polsce i~stosunków polsko-żydowskich};

  % \item Arystoteles, \emph{Retoryka};

  % \item Agnieszka Urbańczyk, Diamentowy Grant, \emph{Polityczność
  %   science fiction w recepcji fanowskie};

  % \item \emph{Lech Wzbudzony};

  % \item \emph{Unintended Reformation};

  % \item J. Polit, \emph{Chiny};

  % \item Paweł Śpiewak, \emph{Teologia i~filozofia żydowska wobec
  %   Holocaustu};

  % \item J.K. Fairbank, \emph{Historia Chin. Nowe spojrzenie};

  % \item \emph{Nowożytna historia Chin}, red. R. Sławiński;

  % \item R. Sławiński, \emph{Geneza Chińskiej Republiki Ludowej};

  % \item K. Seitz, \emph{Chiny. Powrót Olbrzyma};

  % \item A. Bolesta, \emph{Chiny w okresie transformacji};

  % \item \emph{Chiny. Przemiany państwa i społeczeństwa w okresie
  %   reform 1978--2000}, red. K. Tomala;

  % \item Andrzej Napiórkowski OSPPE, \emph{Teologie XX i~XXI wieku};

  % \item Alfred V.~Aho, Jeffrey D.~Ullman, \emph{Wykłady
  %   z~informatyki z~przykładami w~języku~C};

  % \item Jerzy Eisler, \emph{Co nam zostało z tamtych lat.
  %   Dziedzictwo PRL};

  % \item Peter Burke;

  % \item B. Kozera, \emph{Literatura a~religia. Polska współczesna
  %   powieść katolicka};

  % \item \emph{Physics of living systems};

  % \item Jadwiga Staniszkis, \emph{Samoograniczająca~się rewolucja};

  % \item Jadwiga Staniszkis, \emph{Postkomunizm. Próba opisu};

  % \item Z. Wójcik, \emph{Dzikie Pola w~ogniu. O~Kozaczyźnie w~dawnej
  %   Rzeczypospolitej};

  % \item Jan Kofman, Wojciech Roszkowski, \emph{Transformacja
  %   i~postkomunizm};

  % \item M. Gołaszewska, \emph{Estetyka współczesna};

  % \item E. Badinter, \emph{XY --~tożsamość mężczyzny};

  % \item R. Bly, \emph{Żelazny Jan. Rzecz o~mężczyznach};

  % \item Z. Wójcik, \emph{Wojny kozackie w dawnej Polsce};

  % \item Z. Wójcik, \emph{Dzieje Rosji: 1533-1801};

  % \item Vigarello Georges, \emph{Historia gwałtu};

  % \item Tannahill Reay, \emph{Historia kuchni};

  % \item Éliphas Lévi, \emph{Historia magii};

  % \item Meyer Michel (red.), \emph{Historia retoryki od Greków do
  %   dziś};

  % \item Simmel Georg, \emph{Filozofia pieniądza};

  % \item Dahl Robert, \emph{Demokracja i~jej krytycy};

  % \item Z. Wojcik, \emph{Jan Sobieski: 1629-1696};

  % \item Z. Wójcik, \emph{Jan III Sobieski};

  % \item Ariès Philippe, \emph{Historia dzieciństwa. Dziecko i
  %   rodzina w czasach ancien régime’u};

  % \item Bataille Georges, \emph{Historia erotyzmu};

  % \item Vigarello Georges, \emph{Historia otyłości};

  % \item Flandrin Jean-Louis, \emph{Historia rodziny};

  % \item Z. Wójcik, \emph{Jan Kazimierz Waza};

  % \item Z. Wójcik, \emph{Józef Piłsudski 1867-1935};

  % \item \emph{Legendy uświęcone. Twórczość J. R. R. Tolkiena a
  %   chrześcijaństwo};

  % \item Stefan Bratkowski, \emph{Nieco inna historia cywilizacji:
  %   dzieje banków, bankierów i obrotu pieniężnego};

  % \item Izrael Szahad, \emph{Żydowskie dzieje i religia; Żydzi i
  %   goje – XXX wieków historii};

  % \item Izrael Szahak, \emph{Tel Awiw za zamkniętymi drzwiami};

  % \item Andrzej Żbikowski, \emph{Ideologia antysemicka w~Polsce
  %   1848-1918};

  % \item Władysław Bruliński, \emph{Antykościół};

  % \item Władysław Bruliński, \emph{Co to jest marksizm?};

  % \item Władysław Bruliński, \emph{Czerwone palmy historii};

  % \item Władysław Bruliński, \emph{Dokąd idziesz Polsko?};

  % \item \emph{Żydzi i judaizm we współczesnych badaniach polskich};

  % \item Artur Eisenbach, \emph{Emancypacja Żydów na ziemiach
  %   polskich 1785-1870 na tle europejskim};

  % \item Artur Eisenbach, \emph{Z dziejów ludności żydowskiej w
  %   Polsce w XVIII i XIX w.};

  % \item Artur Eisenbach, \emph{Kwestia równouprawnienia Żydów w
  %   Królestwie Polskim};

  % \item Marian Fuks, \emph{Humor Żydów polskich (do 1939 r.)};

  % \item Marian Fuks, \emph{Żydzi w Polsce – Dawniej i dziś};

  % \item Marian Fuks, \emph{Prasa żydowska w Warszawie 1823-1939};

  % \item Marian Fuks, \emph{Z dziejów wielkiej katastrofy narodu
  %   żydowskiego};

  % \item August Grabski, \emph{Studia z dziejów i kultury Żydów w
  %   Polsce po 1945 r.};

  % \item Robert Fabbri, \emph{Wespazjan, trybun Rzymu};

  % \item Michael Hesemann, \emph{Kłamstwa Hitlera};

  % \item David Hockney, \emph{Wiedza tajemna. Sekrety technik
  %   malarskich Dawnych Mistrzów};

  % \item Anna Magdalena Mandrela, \emph{Tomizm Garrigou-Lagrange’a
  %   wobec wizji filozoficznej Teilharda de Chardin};

  % \item Leonie Swann, \emph{Powiększ Sprawiedliwość owiec.
  %   Filozoficzna powieść kryminalna};

  % \item Antoine de Saint-Exupéry, \emph{Twierdza};

  % \item Umberto Eco, \emph{Historia brzydoty};

  % \item Brian Reynolds Myers, \emph{Najczystsza rasa: Propaganda
  %   Korei Północnej};

  % \item Tomasz Strzyżewski, \emph{Wielka księga cenzury PRL w
  %   dokumentach};

  % \item Orlando Figes, \emph{Tragedia narodu. Rewolucja rosyjska
  %   1891-1924};

  % \item Jon Savage, \emph{Teenage: The Creation of Youth Culture};

  % \item Lawrence Weschler, \emph{Mr. Wilson's Cabinet of Wonder:
  %   Pronged Ants, Horned Humans, Mice on Toast, and Other Marvels of
  %   Jurassic Technology};

  % \item Christophe Galfard, \emph{Wszechświat w twojej dłoni};

  % \item Swietłana Aleksijewicz, \emph{Cynkowi chłopcy};

  % \item Aleksander Hertz, \emph{Amerykańskie stronnictwa
  %   polityczne};

  % \item Aleksander Hertz, \emph{Żydzi w kulturze polskiej};

  % \item Aleksander Hertz, \emph{Wyznania starego człowieka};

  % \item Maurycy Horn, \emph{Żydowskie bractwa rzemieślnicze na
  %   ziemiach polskich, litewskich, białoruskich i ukraińskich w
  %   latach 1613-1850};

  % \item Maurycy Horn, \emph{Walka chłopów czerwonoruskich z
  %   wyzyskiem feudalnym w latach 1600-1643};


  % \item Adam Kaźmierczyk, \emph{Sejmy i sejmiki szlacheckie wobec
  %   Żydów w II połowie XVII wieku};

  % \item Adam Kaźmierczyk, \emph{Żydzi w dobrach prywatnych. W
  %   świetle sądowniczej i administracyjnej praktyki dóbr magnackich
  %   w wiekach XVI-XVIII};

  % \item Krystyna Kersten, \emph{Polacy-Żydzi-Komunizm. Anatomia
  %   półprawd 1939-1968};

  % \item Krystyna Kersten, \emph{Pogrom Żydów w Kielcach 4 lipca 1946
  %   r.};

  % \item Andrzej Sulima Kamiński, \emph{Historia Rzeczypospolitej
  %   Wielu Narodów 1505-1795. Obywatele, ich państwa, społeczeństwo,
  %   kultura};

  % \item Cynarski S., \emph{Zygmunt August};

  % \item Cyra A., \emph{Rotmistrz Pilecki. Ochotnik do Auschwitz};

  % \item \emph{Czy ktoś przebije ten mur? Sprawa Pyjasa};

  % \item Dudek A., Zblewski Z., \emph{Utopia nad Wisłą. Historia
  %   Peerelu};

  % \item Czapliński W., \emph{Władysław IV i jego czasy};

  % \item Dybiec J., \emph{Nie tylko szablą. Nauka i kultura polska w
  %   walce o~utrzymanie tożsamości narodowej 1795--1918};

  % \item Eisler J., \emph{Zarys dziejów politycznych Polski
  %   1944--1989};

  % \item Grzybowski S., \emph{Henryk Walezy};

  % \item Ignatowicz I., \emph{Społeczeństwo polskie 1864--1914};

  % \item \emph{Inteligencja polska XIX i XX wieku. Studia}, red. R.
  %   Czapulis-Rastenis, t.1-6;

  % \item Jedynak B., \emph{Obyczaje domu polskiego w~czasach niewoli
  %   1795--1918};

  % \item Kaczmarczyk J., \emph{Bohdan Chmielnicki};

  % \item Kawalec K., \emph{Roman Dmowski};

  % \item \emph{Kobieta i kultura życia codziennego. Wiek XIX i XX.
  %   Zbiór studiów}, red. A. Żarnowska, A. Szwarc;

  % \item \emph{Kobieta i społeczeństwo na ziemiach polskich w XIX
  %   wieku, zbiór studiów}, red. A. Żarnowska, A. Szwarc;

  % \item Konopczyński W., \emph{Dzieje Polski nowożytnej};

  % \item Kowecka E., \emph{W salonie i w kuchni. Opowieść o kulturze
  %   materialnej pałaców i dworów polskich w XIX wieku};

  % \item Krawczak T., \emph{W szlacheckim zaścianku};

  % \item Kuchowicz Z., \emph{Miłość staropolska};

  % \item Kuchowicz Z., \emph{Obyczaje staropolskie XVII-XVIII w.};

  % \item Litak S., \emph{Od reformacji do Oświecenia. Kościół
  %   katolicki w~Polsce nowożytnej};

  % \item Mączak A., \emph{Klientela. Nieformalne systemy władzy w
  %   Polsce Europie XVI-XVIII w.};

  % \item Łuczak C., \emph{Polska i Polacy w drugiej wojnie
  %   światowej};

  % \item Molenda J., \emph{Chłopi. Naród. Niepodległość.
  %   Kształtowanie się postaw narodowych i~obywatelskich chłopów
  %   w~Galicji i~Królestwie polskim w~przededniu odrodzenia Polski};

  % \item Możdżyńska-Nawotka M., \emph{O~modach i~strojach};

  % \item \emph{Obyczaje w Polsce. Od średniowiecza do czasów
  %   współczesnych};

  % \item Olszewski D., \emph{Polska kultura religijna na przełomie
  %   XIX i~XX wieku};

  % \item Olszewski H., \emph{O skutecznym rad sposobie};

  % \item \emph{Polska XVII wieku. Państwo, społeczeństwo, kultura},
  %   red. J. Tazbir;

  % \item Paczkowski A., \emph{Pół wieku dziejów Polski, 1939--1989};

  % \item \emph{Polska na przestrzeni wieków}, red. J. Tazbir;

  % \item Przyboś A., \emph{Michał Korybut Wiśniowiecki 1640-1673};

  % \item Rok B., \emph{Człowiek wobec śmierci w kulturze
  %   staropolskiej};

  % \item \emph{Rzeczpospolita wielu narodów i jej tradycje}, red. M.
  %   Markiewicz, A. Link-Lenczewski;

  % \item \emph{Społeczeństwo polskie od X do XX wieku}, red. I.
  %   Ignatowicz, A. Mączak, B. Zientara, J. Żarnowski;

  % \item Staszewski J., \emph{August II Mocny};

  % \item Staszewski J., \emph{August III Sas};

  % \item Suleja W., \emph{Józef Piłsudski};

  % \item Szubarczyk P., \emph{Inka. Zachowałam się jak trzeba\ldots};

  % \item Topolski J., \emph{Polska w czasach nowożytnych. Od
  %   europejskiej potęgi do utraty niepodległości};

  % \item Terlecki R., \emph{Miecz i tarcza komunizmu. Historia
  %   aparatu bezpieczeństwa 1944--1990};

  % \item \emph{Tradycje polityczne dawnej Polski}, red. A.
  %   Sucheni-Grabowska, A. Dybowska;

  % \item Wandycz P., \emph{Pod zaborami. Ziemie Rzeczypospolitej w
  %   latach 1795--1918};

  % \item Wisner H., \emph{Władysław IV Waza};

  % \item Wisner H., \emph{Zygmunt III Waza};

  % \item Zielińska Z., \emph{Ostatnie lata Pierwszej
  %   Rzeczypospolitej};

  % \item Zblewski Z, \emph{Abecadło Peerelu};

  % \item Zienkowska K., \emph{Stanisław August Poniatowski};

  % \item Zdrada J., \emph{Historia Polski 1795--1914};

  % \item Żarnowski, J., \emph{Polska 1918--1939. Praca, technika,
  %   społeczeństwo};

  % \item Ziejka F., \emph{Złota legenda chłopów polskich};

  % \item Żołądź D., \emph{Ideały edukacyjne doby staropolskiej.
  %   Stanowe modele i potrzeby edukacyjne szesnastego i siedemnastego
  %   wieku};

  % \item R. Wapiński, \emph{Historia polskiej myśli politycznej XIX
  %   i~XX~wieku};

  % \item R.R. Ludwikowski, \emph{Historia polskiej myśli
  %   politycznej};

  % \item W. Bernacki, \emph{Liberalizm polski};

  % \item B. Szlachta, \emph{Z dziejów polskiego konserwatyzmu};

  % \item M. Śliwa, \emph{Polska myśl socjalistyczna 1892--1948};

  % \item Z. Ogonowski, \emph{Filozofia polityczna w~Polsce XVII w.
  %   i~tradycje demokracji europejskiej};

  % \item S. Tarnowski, \emph{Pisarze polityczni XVI wieku};

  % \item S. Tarnowski, \emph{Historia literatury polskiej, t.2};

  % \item W. Konopczyński, \emph{Polscy pisarze polityczni XVIII w.};

  % \item H. Olszewski, \emph{Doktryny prawno-ustrojowe czasów
  %   saskich};
  % \item K. Waliszewski, \emph{Potoccy i Czartoryscy, walka
  %   stronnictw i programów politycznych przed upadkiem
  %   Rzeczypospolitej 1734--1763};

  % \item Red. Tomasz Dołęgowski, \emph{Przewodnik po moralnym
  %   kapitalizmie};

  % \item Besala J., \emph{Stefan Batory};

  % \item Cieślak E., \emph{Stanisław Leszczyński};

  % \item Bogucka M., \emph{Staropolskie obyczaje XVI-XVII w};

  % \item Cz. Michalski, \emph{Western};

  % \item A. Chwalba, \emph{III Rzeczpospolita --~raport specjalny};

  % \item Jean-Paul Bled, \emph{Bismarck. Żelazny kanclerz};

  % \item Marcin Król, \emph{Byliśmy głupi};

  % \item J. Wójcik, \emph{Labirynt światła};

  % \item A. Chwalba, \emph{Historia Polski 1795-1918};

  % \item Brzoza C., Sowa A., \emph{Historia Polski 1918-1945};

  % \item Cz. Michalski, \emph{Western i~jego bohaterowie};

  % \item Rafał Marszałek, \emph{Pamflet na kino codzienne};

  % \item Rafał Marszałek, \emph{Polska wojna w obcym filmie};

  % \item Rafał Marszałek, \emph{Filmowa pop-historia};

  % \item Rafał Marszałek, \emph{Kino rzeczy znalezionych};

  % \item J. Skwara, \emph{Western odrzuca legendę};

  % \item Władysław Konopczyński, \emph{Konfederacja barska};

  % \item Michał Łuczewski, \emph{Odwieczny naród. Polak i~katolik
  %   w~Żmiącej};

  % \item Alan Bullock, \emph{Hitler. Studium tyrani};

  % \item Tadeusz Lubelski, \emph{Historia Kina Polskiego, Twórcy,
  %   Filmy, Konteksty};

  % \item Marek Haltof, \emph{Kino polskie};

  % \item \emph{Kino bez tajemnic};

  % \item David Bordwell, Kristin Thompson, \emph{Film Art. Sztuka
  %   filmowa. Wprowadzenie};

  % \item W. Stróżewski, \emph{Estetyka};

  % \item Vigarello Georges, \emph{Historia czystości i brudu};

  % \item Muchembled Robert, \emph{Orgazm i Zachód};

  % \item Sloterdijk Peter, \emph{Pogarda mas};

  % \item Gately Iain, \emph{Kulturowa historia alkoholu};

  % \item Eco Umberto, \emph{Poszukiwanie języka doskonałego w
  %   kulturze europejskiej};

  % \item Coogan Michael, \emph{Bóg i~seks. Co naprawdę mówi Biblia};

  % \item Secher Reynald, \emph{Ludobójstwo francusko-francuskie};

  % \item Vigarello Georges, \emph{Historia urody};

  % \item Higman B.W., \emph{Historia żywności};

  % \item Tannahill Reay, \emph{Historia seksu};

  % \item Ramamurti Rajaraman, \emph{Solitons and~instantons};

  % \item Wilson Edward O., \emph{Znaczenie ludzkiego istnienia};

  % \item Karl Loewith, \emph{Historia powszechna i dzieje zbawienia};

  % \item Tony Judt, \emph{Historia niedokończona. Francuscy
  %   intelektualiści 1944-1956};

  % \item Karl Loewith, \emph{Od Hegla do Nietzschego. Rewolucyjny
  %   przełom w myśli XIX wieku};

  % \item J. Fiske \emph{Zrozumieć kulturę popularną};

  % \item Anatol Taras, \emph{Anatomia nienawiści};

  % \item R. Rodes, \emph{Jak powstała bomba atomowa?};

  % \item A. Tarski;

  % \item Karol Buczek, \emph{Studia z dziejów ustroju
  %   społeczno-gospodarczego Polski piastowskiej};

  % \item Red. S. Kowalczyk, E. Balawajder, \emph{Jacques Maritain,
  %   prekursor soborowego humanizmu};

  % \item T. M. Jaroszewski, \emph{Osobowość i wspólnota. Problemy
  %   osobowości we współczesnej antropologii filozoficznej
  %   --~marksizm, strukturalizm, egzystencjalizm, personalizm
  %   chrześcijański};

  % \item Roman Graczyk, \emph{Od uwikłania do autentyczności.
  %   Biografia polityczna Tadeusza Mazowieckiego};

  % \item S. Wiggins, \emph{Introduction to Applied Nonlinear
  %   Dynamical Systems and Chaos};

  % \item Władymir Arnold, \emph{Catastrophe Theory};

  % \item

  % \item

  % \item

  % \item

  % \item

  % \item

  % \item

  % \item

  % \item

  % \item

  % \item

  % \item

  % \item

  % \item

  % \item

  % \item Donald Ritchie, \emph{The Films of Akira Kurosawa};

  % \item Martin Konings, \emph{The Emotional Logic of Capitalism.
  %   What Progressives Have Missed};

  % \item David Sloan Wilson, \emph{Darwin's Cathedral: Evolution,
  %   Religion, and the~Nature~of Society};

  % \item Ch.~R.~Browning, \emph{Zwykli ludzie. 101~Policyjny Batalion
  %   Rezerwowy i~„ostateczne rozwiązanie” w~Polsce}

  % \item Viviana A. Zelizer, \emph{The Social Meaning of Money: Pin
  %   Money, Paychecks, Poor Relief, and Other Currencies};

  % \item

  % \item

  % \item

  % \item

  % \item

  % \item

  % \item

  % \item I. Wallerstein, \emph{Europejski uniwersalizm. Retoryka
  %   władzy};

  % \item Abbé Jacques Meinvielle, \emph{De Lamennais ŕ Maritain};

  % \item

  % \item

  % \item

  % \item

  % \item

  % \item

  % \item

  % \item

  % \item












































\end{enumerate}










% % ######################################
% \newpage
% \section{Zaczęte i~nieskończone}

% \vspace{\spaceTwo}
% % ######################################














% % ######################################
% \newpage
% \section{Articles}

% \vspace{\spaceTwo}
% % ######################################



% \begin{enumerate}

% \item Edward Witten, \emph{Notes on Some Entanglement Properties of
%   Quantum Field Theory},
%   \href{https://arxiv.org/abs/1803.04993}{arXiv:1803.04993};

% \item Jeff Bezanson et al, "Julia: dynamism and performance
%   reconciled by design"
%   \href{https://doi.org/10.1145/3276490}{https://doi.org/10.1145/3276490};

% \item Francesco Zappa Nardelli et al., "Julia subtyping: a rational
%   reconstruction",
%   \href{https://doi.org/10.1145/3276914}{https://doi.org/10.1145/3276914};

% \item Artem Pelenitsyn et al., "Type Stability in Julia: Avoiding
%   Performance Pathologies in JIT Compilation",
%   https://doi.org/10.1145/3485527,
%   \href{arXiv:2109.01950}{https://arxiv.org/abs/2109.01950};

% \item \emph{How SQLite Is Tested},
%   \href{https://www.sqlite.org/testing.html}{https://www.sqlite.org/testing.html};

% \item \emph{SpotBugs},
%   \href{https://spotbugs.github.io/}{https://spotbugs.github.io/};

% \item \emph{A tool to detect bugs in Java and C/C++/Objective-C code
%   before it ships},
%   \href{https://fbinfer.com/}{https://fbinfer.com/la};

% \item \emph{Go 1 and the Future of Go Programs},
%   \href{https://go.dev/doc/go1compat}{https://go.dev/doc/go1compat};

% \item \emph{SD-8: Standard Library Compatibility},
%   \href{https://isocpp.org/std/standing-documents/sd-8-standard-library-compatibility}{https://isocpp.org/std/standing-documents/sd-8-standard-library-compatibility};

% \item \emph{GNU General Public License},
%   \href{https://www.gnu.org/licenses/gpl-3.0.html}{https://www.gnu.org/licenses/gpl-3.0.html};

% \item \emph{Reflections on trusting trust},
%   \href{https://dl.acm.org/doi/10.1145/358198.358210}{https://dl.acm.org/doi/10.1145/358198.358210};

% \item \emph{Go \& Versioning},
%   \href{https://research.swtch.com/vgo}{https://research.swtch.com/vgo};

% \item \emph{Why Google stores billions of lines of code in a single
%   repository},
%   \href{https://dl.acm.org/doi/10.1145/2854146}{https://dl.acm.org/doi/10.1145/2854146};

% \item \emph{Testing Chromium: ThreadSanitizer v2, a next-gen data
%   race
%   detector}, \\
%   \href{https://blog.chromium.org/2014/04/testing-chromium-threadsanitizer-v2.html}{https://blog.chromium.org/2014/04/testing-chromium-threadsanitizer-v2.html};

% \item \emph{Search Vulnerability Database},
%   \href{https://nvd.nist.gov/vuln/search}{https://nvd.nist.gov/vuln/search};

% \item \emph{Regular Expression Matching with a Trigram Index or How
%   Google Code Search Worked}, \\
%   \href{https://swtch.com/~rsc/regexp/regexp4.html}{https://swtch.com/~rsc/regexp/regexp4.html};

% \item \emph{Licenses},
%   \href{https://opensource.google/docs/thirdparty/licenses}{https://opensource.google/docs/thirdparty/licenses};

% \item \emph{ImperialViolet},
%   \href{https://www.imperialviolet.org/2009/08/26/seccomp.html}{https://www.imperialviolet.org/2009/08/26/seccomp.html};

% \item \emph{Multi-process Architecture},
%   \href{https://blog.chromium.org/2008/09/multi-process-architecture.html}{https://blog.chromium.org/2008/09/multi-process-architecture.html};

% \item \emph{A single Node of failure},
%   \href{https://lwn.net/Articles/681410/}{https://lwn.net/Articles/681410/};

% \item \emph{Interpreting the Data: Parallel Analysis with Sawzall},
%   \href{https://www.hindawi.com/journals/sp/2005/962135/}{https://www.hindawi.com/journals/sp/2005/962135/};

% \item \emph{Go Proverbs},
%   \href{https://go-proverbs.github.io/}{https://go-proverbs.github.io/};

% \item \emph{RE2: a principled approach to regular expression
%   matching}, \\
%   \href{https://opensource.googleblog.com/2010/03/re2-principled-approach-to-regular.html}{https://opensource.googleblog.com/2010/03/re2-principled-approach-to-regular.html};

% \item \emph{Details about the event-stream incident}, \\
%   \href{https://blog.npmjs.org/post/180565383195/details-about-the-event-stream-incident}{https://blog.npmjs.org/post/180565383195/details-about-the-event-stream-incident};

% \item \emph{Open-sourcing gVisor, a sandboxed container runtime}, \\
%   \href{https://cloud.google.com/blog/products/identity-security/open-sourcing-gvisor-a-sandboxed-container-runtime}{https://cloud.google.com/blog/products/identity-security/open-sourcing-gvisor-a-sandboxed-container-runtime};

% \item Samuel R. Buss, Alexander S. Kechris, Anand Pillay, Richard A.
%   Shore, \emph{The prospects for mathematical logic in the
%   twenty-first century},
%   \href{https://arxiv.org/abs/cs/0205003v1}{arXiv:cs/0205003v1};

% \item Christian Retore, \emph{On the system F as a glue language for
%   natural-language compositional-semantics},
%   \href{https://arxiv.org/abs/1108.5084}{arXiv:1108.5084};

% \item Robert Harper, \emph{An Equational Logical Framework for Type
%   Theories},
%   \href{https://arxiv.org/abs/2106.01484}{arXiv:2106.01484};

% \item Jan Leike, et al., \emph{AI Safety Gridworlds},
%   \href{https://arxiv.org/abs/1711.09883}{arXiv:1711.09883v2};

% \item Sergei Gukov, Edward Witten, \emph{Branes and Quantization},
%   \href{https://arxiv.org/abs/0809.0305}{https://arxiv.org/abs/0809.0305};

% \item R. Estrada, J. M. Gracia-Bondia, J. C. Varilly, \emph{On
%   summability of distributions and spectral geometry},
%   \href{https://arxiv.org/abs/funct-an/9702001v1}{arXiv:funct-an/9702001};

% \item Ghanashyam Date, \emph{Lectures on Constrained Systems},
%   \href{https://arxiv.org/abs/1010.2062v1}{arXiv:1010.2062};

% \item Frank Wilczek, \emph{Quantum Field Theory},
%   \href{https://arxiv.org/abs/hep-th/9803075v2}{arXiv:hep-th/9803075};

% \item Karl Michael Schmidt, Karl Michael Schmidt, \emph{Schnol’s
%   Theorem and Spectral Properties of Massless Dirac Operators with
%   Scalar Potentials};

% \item Peter J. Olver, \emph{Dirac’s theory of constraints in fields
%   theory and the canonical form of Hamiltonian differential
%   operators};


% \item Lorenzo Iorio, \emph{Editorial for the Special Issue 100 Years
%   of Chronogeometrodynamics: The Status of the Einstein's Theory of
%   Gravitation in Its Centennial Year},
%   \href{https://arxiv.org/abs/1504.05789v2}{arXiv:1504.05789};

% \item B. Mutet, P. Grang\'{e}, E. Werner, \emph{Taylor–Lagrange
%   renormalization and gauge theories in four dimensions};


% \item \emph{Surviving Software Dependencies},
%   \href{https://queue.acm.org/detail.cfm?id=3344149}{https://queue.acm.org/detail.cfm?id=3344149};

% \item Clifford M. Will, \emph{The Confrontation between General
%   Relativity and Experiment};

% \item Paul Lopes, \emph{Culture and Stigma: Popular Culture and the
%   Case of Comic Books};

% \item Marina S.Butuzova 1, Alexander B. Pushkarev, \emph{Is OJ 287 a
%   Single Supermassive Black Hole?};

% \item Peter Selinger, \emph{Lecture notes on the lambda calculus},
%   \href{https://arxiv.org/abs/0804.3434v2}{arXiv:0804.3434};

% \item Alex Eskin, Maryam Mirzakhani, \emph{Counting closed geodesics
%   in Moduli space},
%   \href{https://arxiv.org/abs/0811.2362v3}{arXiv:0811.2362};

% \item Jacques Carette, James H. Davenport, \emph{The Power of
%   Vocabulary: The Case of Cyclotomic Polynomials},
%   \href{https://arxiv.org/abs/1002.0012v1}{arXiv:1002.0012};

% \item Chris Kapulkin, Peter LeFanu Lumsdaine, \emph{The Simplicial
%   Model of Univalent Foundations (after Voevodsky)},
%   \href{https://arxiv.org/abs/1211.2851v5}{arXiv:1211.2851};

% \item Christopher J. Fewster, Rainer Verch, \emph{Quantum fields and
%   local measurements},
%   \href{https://arxiv.org/abs/1810.06512}{arXiv:1810.06512};

% \item Paweł Duch, \emph{Infrared problem in perturbative quantum
%   field theory},
%   \href{https://arxiv.org/abs/1906.00940}{arXiv:1906.00940};

% \item Christopher J. Fewster, \emph{A generally covariant
%   measurement scheme for quantum field theory in curved spacetimes},
%   \href{https://arxiv.org/abs/1904.06944v1}{arXiv:1904.06944};

% \item Jacob Lurie, \emph{Higher Topos Theory},
%   \href{https://arxiv.org/abs/math/0608040v4}{arXiv:math/0608040};

% \item Konrad Osterwalder, and Robert Schrader, \emph{Axioms for
%   Euclidean Green's Functions};

% \item Vladimir Voevodsky, \emph{A very short note on homotopy
%   $\lambda$-calculus};

% \item Charles Rezk, \emph{Toposes and homotopy toposes};

% \item Fredrik Nordvall Forsberg, Anton Setzer, \emph{A finite
%   axiomatisation of inductive-inductive definitions};

% \item Egbert Rilke, \emph{Introduction to homotopy type theory};

% \item \'{A}lvaro Pelayo, Michael A. Warren, \emph{Homotopy type
%   theory and Voevodsky’s Univalent Foundations};

% \item H. Simmons, A. Schalk, \emph{An introduction to
%   $\lambda$-calculi and arithmetics};

% \item Roderich Tumulka, \emph{Lecture Notes on Mathematical
%   Statistical Physics};

% \item Martin Hofmann, \emph{Extensional concepts in intensional type
%   theory};

% \item Eugenio Moggi, \emph{Computational $\lambda$-calculus and
%   monads};

% \item \emph{Proof-theoretic semantics. Assessment and Future
%   Perspectives};

% \item Philip Walder, \emph{Propostions as Types};

% \item Egbert Rijke, \emph{Homotopy type theory};

% \item Eugenio Moggi, \emph{Notions of computation and monads};

% \item Bruno Barras, Thierry Coquand and Simon Huber, \emph{A
%   Generalization of Takeuti-Gandy Interpretation};

% \item Michael Alton Warren, \emph{Homotopy Theoretic Aspects of
%   Constructive Type Theory};

% \item Vladimir Voevodsky, \emph{A universe polymorphic type system};

% \item S. Marmi, \emph{An Introduction To Small Divisors},
%   \href{https://arxiv.org/abs/math/0009232v1}{arXiv:math/0009232};

% \item Paweł Duch, Michael Duetsch, Jose M. Gracia-Bondia,
%   \emph{Diphoton decay of the higgs from the Epstein--Glaser
%   viewpoint},
%   \href{https://arxiv.org/abs/2011.12675v2}{arXiv:2011.12675};

% \item Juliette Kennedy, Menachem Magidor, Jouko
%   V\"{a}\"{a}n\"{a}nen, \emph{ Inner Models from Extended Logics:
%   Part 1},
%   \href{https://arxiv.org/abs/2007.10764}{arXiv:2007.10764};

% \item Paul W. Gross, P. Robert Kotiuga, \emph{Electromagnetic Theory
%   and Computation: A Topological Approach};

% \item Andreas R. Blass, Jeffry L. Hirst, and Stephen G. Simpson,
%   Logical analysis of some theorems of combinatorics and topological
%   dynamics, Logic and combinatorics (Arcata, Calif., 1985), Contemp.
%   Math., vol. 65, Amer. Math. Soc., Providence, RI, 1987, pp.
%   125–156.

% \item Jared Corduan, Marcia Groszek, and Joseph Mileti, Draft: A
%   note on reverse mathematics and partitions of trees.

% \item Jennifer Chubb, Jeffry Hirst, and Tim McNichol, Reverse
%   mathematics and partitions of trees. To appear in J. Symbolic
%   Logic.

% \item Damir Dzhafarov and Jeffry Hirst, The polarized Ramsey
%   theorem. Archive for Math. Logic, Online First: 2008.

% \item Neil Hindman, The existence of certain ultra-filters on N and
%   a conjecture of Graham and Rothschild, Proc. Amer. Math. Soc. 36
%   (1972), 341–346.

% \item Jeffry L. Hirst, Hindman’s theorem, ultrafilters, and reverse
%   mathematics, J. Symbolic Logic 69 (2004), no. 1, 65–72.

% \item Carl G. Jockusch Jr., Ramsey’s theorem and recursion theory,
%   J. Symbolic Logic 37 (1972), 268–280.

% \item J. Mileti, Partition theory and computability theory. Ph.D.
%   Thesis.

% \item


































































































































































































































































% \end{enumerate}
% % \bibliographystyle{alpha} \bibliography{Bibliography}{}










% ############################

% Koniec dokumentu
\end{document}
