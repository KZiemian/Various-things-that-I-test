% Autor: Kamil Ziemian

% ---------------------------------------------------------------------
% Podstawowe ustawienia i pakiety
% ---------------------------------------------------------------------
\RequirePackage[l2tabu, orthodox]{nag} % Wykrywa przestarzałe i niewłaściwe
% sposoby używania LaTeXa. Więcej jest w l2tabu English version.
\documentclass[a4paper,11pt]{article}
% {rozmiar papieru, rozmiar fontu}[klasa dokumentu]
\usepackage[MeX]{polski} % Polonizacja LaTeXa, bez niej będzie pracował
% w języku angielskim.
\usepackage[utf8]{inputenc} % Włączenie kodowania UTF-8, co daje dostęp
% do polskich znaków.
\usepackage{lmodern} % Wprowadza fonty Latin Modern.
\usepackage[T1]{fontenc} % Potrzebne do używania fontów Latin Modern.



% ---------------------------------------
% Podstawowe pakiety (niezwiązane z ustawieniami języka)
% ---------------------------------------
\usepackage{microtype} % Twierdzi, że poprawi rozmiar odstępów w tekście.
% \usepackage{graphicx} % Wprowadza bardzo potrzebne komendy do wstawiania
% grafiki.
% \usepackage{verbatim} % Poprawia otoczenie VERBATIME.
% \usepackage{textcomp} % Dodaje takie symbole jak stopnie Celsiusa,
% wprowadzane bezpośrednio w tekście.
\usepackage{vmargin} % Pozwala na prostą kontrolę rozmiaru marginesów,
% za pomocą komend poniżej. Rozmiar odstępów jest mierzony w calach.
% ---------------------------------------
% MARGINS
% ---------------------------------------
\setmarginsrb
{ 0.7in}  % left margin
{ 0.6in}  % top margin
{ 0.7in}  % right margin
{ 0.8in}  % bottom margin
{  20pt}  % head height
{0.25in}  % head sep
{   9pt}  % foot height
{ 0.3in}  % foot sep



% ---------------------------------------
% Często używane pakiety
% ---------------------------------------
% \usepackage{csquotes} % Pozwala w prosty sposób wstawiać cytaty do tekstu.



% ---------------------------------------
% Pakiety do tekstów z nauk przyrodniczych
% ---------------------------------------
\let\lll\undefined % Amsmath gryzie się z językiem pakietami do języka
% % polskiego, bo oba definiują komendę \lll. Aby rozwiązać ten problem
% % oddefiniowuję tę komendę, ale może tym samym pozbywam się dużego Ł.
\usepackage[intlimits]{amsmath} % Podstawowe wsparcie od American
% Mathematical Society (w skrócie AMS)
\usepackage{amsfonts, amssymb, amscd, amsthm} % Dalsze wsparcie od AMS
% % \usepackage{siunitx} % Dla prostszego pisania jednostek fizycznych
% \usepackage{upgreek} % Ładniejsze greckie litery
% % Przykładowa składnia: pi = \uppi
% \usepackage{slashed} % Pozwala w prosty sposób pisać slash Feynmana.
% \usepackage{calrsfs} % Zmienia czcionkę kaligraficzną w \mathcal
% % na ładniejszą. Może w innych miejscach robi to samo, ale o tym nic
% % nie wiem.





% ---------------------------------------
% Dodatkowe ustawienia dla języka polskiego
% ---------------------------------------
\renewcommand{\thesection}{\arabic{section}.}
% Kropki po numerach rozdziału (polski zwyczaj topograficzny)
\renewcommand{\thesubsection}{\thesection\arabic{subsection}}
% Brak kropki po numerach podrozdziału



% ---------------------------------------
% Pakiety napisane przez użytkownika.
% Mają być w tym samym katalogu to ten plik .tex
% ---------------------------------------
% \usepackage{latexshortcuts}
% \usepackage{mathshortcuts}



% ---------------------------------------
% Ustawienia różnych parametrów tekstu
% ---------------------------------------
\renewcommand{\arraystretch}{1.2} % Ustawienie szerokości odstępów między
% wierszami w tabelach.





% ---------------------------------------
% Pakiet "hyperref"
% Polecano by umieszczać go na końcu preambuły.
% ---------------------------------------
\usepackage{hyperref} % Pozwala tworzyć hiperlinki i zamienia odwołania
% do bibliografii na hiperlinki.










% ---------------------------------------------------------------------
% Tytuł, autor, data
\title{Zadania do zrobienia, fizyka}

% \author{}
% \date{}
% ---------------------------------------------------------------------










% ####################################################################
% Początek dokumentu
\begin{document}
% ####################################################################





% ######################################
\maketitle  % Tytuł całego tekstu
% ######################################





% ######################################
\section{Zrób te zadania}

% \vspace{\spaceTwo}

% ######################################





\begin{enumerate}

\item Wyznaczyć relacje dyspersji dla jednowymiarowej sieci złożonej z
  identycznych atomów przy założeniu, że stałe siłowe opisujące
  oddziaływanie par atomów są na przemian równe $C_{ 1 }$ i~$C_{ 2 }$
  ($C_{ 1 } \neq C_{ 2 }$).

\item Oblicz $c_{ V }( T )$ przyjmujące
  $D( \omega ) = 3 N \delta( \omega - \omega_{ E } )$, gdzie $N$ to
  liczba atomów w~krysztale. Dla $T \to \infty$ pokaż, że
  $c_{ V }( T )$ spełnia prawo Dulonga-Petita.

\item Pokaż, że dla $N$ atomów w sześcianie o~bloku $L$, gęstość stanów w~przestrzeni pędów jest dana przez $\rho( k ) = \left( \frac{ L }{ 2\pi } \right)^{ 3 }$. \\
  Używając zależności dyspersyjnej $\omega = \nu k$, pokaż że
  $D( \omega ) = \frac{ 3 \omega^{ 2 } L^{ 3 } }{ 2\pi^{ 2 } \nu^{ 3 }
  }$ dla
  $\omega < \omega_{ D } = ( 6 \pi^{ 2 } N )^{ \frac{ 1 }{ 3 } }
  \frac{ V }{ L }$. Oblicz $c_{ V }( T )$ i pokaż, że dla
  $T \searrow 0$, $c_{ V }( T ) \propto T^{ 3 }$.

\item Dystrybucję $\delta( x )$ Diraca można określić jako granicę funkcji
  regualarnych
  \begin{equation}
    \label{eq:1}
    \delta( x ) = \lim_{ a \to 0 } \rho_{ a }( x ), \qquad
    \rho_{ a }( x ) = \frac{ 1 }{ a \sqrt{ \pi } } e^{ -\frac{ x^{ 2 } }{ a^{ 2 } } }
  \end{equation}
  Mówimy, że ciąg funkcji $\rho_{ a }( x )$ jest modelem dla funkcji
  $\delta( x )$. Udowodnić, że $\delta( x )$ ma następujące własności
  \begin{align}
    &\int_{ a }^{ b } \delta( x - x_{ 0 } ) \, dx = 1, \qquad
      x_{ 0 } \in ( a, b ) \\
    &\int_{ a }^{ b } \delta( x - x_{ 0 } ) \, dx = 0, \qquad
      x_{ 0 } \notin ( a, b ) \\
    &\int_{ -\infty }^{ +\infty } \delta( x - x_{ 0 } ) \phi( x ) \, dx = \phi( x_{ 0 } )
  \end{align}
  dla dowolnej funkcji próbnej (tj. regularnej) $\phi( x )$. \\
  Wykorzystując ostatnią własność, oraz właściwości odwrotnej
  transformaty Fouriera udowodnić, że
  \begin{equation}
    \label{eq:2}
    \delta( x ) = \frac{ 1 }{ 2\pi } \int_{ -\infty }^{ +\infty } e^{ i p x } \, dp
  \end{equation}
  Zapisując ostatnią całkę jako
  \begin{equation}
    \label{eq:3}
    \lim_{ \varepsilon \to 0 } \frac{ 1 }{ 2\pi } \int_{ -\infty }^{ +\infty } e^{ ipx - \varepsilon | p | } \, dp,
  \end{equation}
  udowodnić inną, równoważną, reprezentację dystrybucji $\delta( x )$
  \begin{equation}
    \label{eq:4}
    \delta( x ) = \lim_{ \varepsilon \to 0 } f_{ \varepsilon }( x ), \qquad
    f_{ \varepsilon }( x ) = \frac{ 1 }{ \pi } \frac{ \varepsilon }{ x^{ 2 } + \varepsilon^{ 2 } }
  \end{equation}

  Wykreślić funkcje modelujące dla kilku wartości $\varepsilon$. Uzasadnić, że
  rzeczywiście jest to dobry model funkcji $\delta$. W~szczególności,
  sprawdzić warunek normalizacji oraz że dla dowolnej funkcji próbnej
  zachodzi
  \begin{equation}
    \label{eq:5}
    \lim_{ \varepsilon \to 0} \int_{ -\infty }^{ +\infty } f_{ \varepsilon }( x - x_{ 0 } ) \phi( x ) \, dx
    =
    \phi( x_{ 0 } )
  \end{equation}
  Ostatnią całkę wykonać metodą residuów zakładając wystarczająco
  szybkie znikanie funkcji próbnej na dużych okręgach.

  Udowodnić, że fale płaskie są znormalizowane do $\delta$ Diraca.

\item Rozwiązać zależne od czasu równanie Schr\"{o}dingera dla cząstki
  swobodnej w jednym i trzech wymiarach, za pomocą separacji
  zmiennych. Jako warunek początkowy przyjąć, że gęstość
  prawdopodobieństwa znalezienia cząstki w~przestrzeni ma rozkład
  gaussowski o~dyspersji $\sigma^{ 2 } = a^{ 2 }$ i że pakiet ma
  średnią prędkość $v_{ 0 }$. Wykonać \emph{explicite} wszystkie
  transformaty Fouriera i~uzyskać jawne wyrażenie na $\Psi( x, t )$.
  Obliczyć gęstość prawdopodobieństwa $\rho( x, t )$ i gęstość
  strumienia prawdopodobieństwa $\vec{ j }( x, t )$. Sprawdzić
  równanie ciągłości. Wyprowadzić analogiczne wzory (np. na
  $\vec{ j }( x, t )$) dla zbioru cząstek klasycznych.

\item Obliczyć odległości między sąsiadującymi jonami w sieci CsCl,
  stała sieci: $a = 4.11$ \AA{} (albo $a = 3.55$ \AA), NaCl,
  $a = 5.63$ \AA{} albo $a = 2.82$ \AA{} oraz KBr o strukturze NaCl,
  $a = 6.59$ \AA{} albo $a = 3.29$ \AA{} na podstawie danych struktury
  krystalicznych i~porównać je z~odległościami wyznaczonymi przy
  pomocy odpowiednich promieni standardowych jonów oraz poprawek
  zależnych od liczby koordynacyjnej (zobacz jakieś tabele).

\item Podobnie jak w poprzednim zadaniu, porównać odległość między
  jonami dla związków o~strukturze ZnS (blendy cynkowej): CuF, stała
  sieci $a = 4.26$ \AA, ZnS, stała sieci $a = 5.41$ \AA{} i~InSb, stała
  sieci $a = 6.46$ \AA, biorąc pod uwagę promienie atomów przy
  tetraedrycznym wiązaniu kowalencyjnym.

\item Używając potencjału Lennarda-Jonesa, oblicz stosunek energii
  wiązania kryształu gazu szlachetnego krystalizującego w~strukturach
  \textbf{bcc} i~\textbf{fcc}.

\item Oszacować wartość stałej Madelunga dla struktury NaCl na trzy
  sposoby.

  \begin{itemize}
  \item[a)] Stosując dodawanie energii oddziaływania kolejnych
    ładunków całkowitych, wyliczyć kilka pierwszych wyrazów szeregu
    wprost z~definicji.

  \item[b)] Wyliczyć stałą Madelunga numerycznie.

  \item[c)] Stosując metodę Evjena (ułamkowych ładunków) dla sześcianu
    zawierającego 26 ładunków.

  \end{itemize}

  % \item

  % \item

  % \item

  % \item

  % \item

  % \item

  % \item

  % \item

  % \item

  % \item

  % \item

  % \item

  % \item

  % \item

  % \item

  % \item

  % \item

  % \item

  % \item

  % \item

  % \item

  % \item

  % \item

  % \item

  % \item

  % \item

  % \item

  % \item

  % \item

  % \item

  % \item

  % \item

  % \item

  % \item

  % \item

  % \item

  % \item

  % \item

  % \item

  % \item

  % \item

  % \item

  % \item

  % \item

  % \item

  % \item

  % \item

  % \item

  % \item

  % \item

  % \item

  % \item

  % \item

  % \item

  % \item

  % \item

  % \item

  % \item

  % \item

  % \item

  % \item

  % \item

  % \item

  % \item

  % \item

  % \item

  % \item

  % \item

  % \item

  % \item

  % \item

  % \item

  % \item

  % \item

  % \item

  % \item

  % \item

  % \item

  % \item

  % \item

  % \item

  % \item

  % \item

  % \item

  % \item

  % \item

  % \item

  % \item

  % \item

  % \item

  % \item

  % \item

  % \item

  % \item

  % \item

  % \item

  % \item

  % \item

  % \item

  % \item

  % \item

  % \item

  % \item

  % \item

  % \item

  % \item

  % \item

  % \item

  % \item

  % \item

  % \item

  % \item

  % \item

  % \item

  % \item

  % \item

  % \item

  % \item

  % \item

  % \item

  % \item

  % \item

  % \item

  % \item

  % \item

  % \item

  % \item

  % \item

  % \item

  % \item

  % \item

  % \item

  % \item

  % \item

  % \item

  % \item

  % \item

  % \item

  % \item

  % \item

  % \item

  % \item

  % \item

  % \item

  % \item

  % \item

  % \item

  % \item

  % \item

  % \item

  % \item

  % \item

  % \item

  % \item

  % \item

  % \item

  % \item

  % \item

  % \item

  % \item

  % \item

  % \item

  % \item

  % \item

  % \item

  % \item

  % \item

  % \item

  % \item

  % \item

  % \item

  % \item

  % \item

  % \item

  % \item

  % \item

  % \item

  % \item

  % \item

  % \item

  % \item

  % \item

  % \item

  % \item

  % \item

  % \item

  % \item

  % \item

  % \item

  % \item

  % \item

  % \item

  % \item

  % \item

  % \item

  % \item

  % \item

  % \item

  % \item

  % \item

  % \item

  % \item

  % \item

  % \item

  % \item

  % \item

  % \item

  % \item

  % \item

  % \item

  % \item

  % \item

  % \item

  % \item

  % \item

  % \item

  % \item

  % \item

  % \item

  % \item

  % \item

  % \item

  % \item

  % \item

  % \item

  % \item

  % \item

  % \item

  % \item

  % \item

  % \item

  % \item

  % \item

  % \item

  % \item

  % \item

  % \item

  % \item

  % \item

  % \item

  % \item

  % \item

  % \item

  % \item

  % \item

  % \item

  % \item

  % \item

  % \item

  % \item

  % \item

  % \item

  % \item

  % \item


  % \item

  % \item

  % \item

  % \item

  % \item

  % \item

  % \item

  % \item

  % \item

  % \item

  % \item

  % \item

  % \item

  % \item

  % \item

  % \item

  % \item

  % \item

  % \item

  % \item

  % \item

  % \item

  % \item

  % \item

  % \item

  % \item

  % \item

  % \item

  % \item

  % \item

  % \item

  % \item

  % \item

  % \item Anna Grześkowiak-Krwawicz, \emph{Dyskurs polityczny
  %   Rzeczypospolitej Obojga Narodów};

  % \item Nowak \emph{Dzieje Polski};

  % \item Margaret Atwood, \emph{Dług. Rozrachunek z~ciemną stroną
  %   bogactwa};

  % \item Jared Diamond, \emph{Strzelby, zarazki, maszyny. Losy
  %   ludzkich społeczeństw};

  % \item Edward E. Evans\dywiz Pritchard, \emph{Czary, wyrocznie
  %   i~magia u~Azande};

  % \item W. Szumowski, \emph{Historia medycyny filozoficznie ujęta};

  % \item Michał Dondzik, Krzysztof Jajko, Emil Swoiński,
  %   \emph{Elementarz Wytwórni Filmów Oświatowych};

  % \item James M. Murray, \emph{Brugia: Kolebka kapitalizmu};

  % \item Maciej Janowski, \emph{Narodziny inteligencji: 1750--1831};

  % \item Jerzy Jedlicki, \emph{Jakiej cywilizacji Polacy potrzebują:
  %   studia z dziejów idei i~wyobraźni XIX wieku};

  % \item Jerzy Jedlicki, \emph{Droga do narodowej klęski};

  % \item Jerzy Jedlicki, \emph{Błędne koło: 1832-1864};

  % \item Jerzy Jedlicki, \emph{Nieudana próba kapitalistycznej
  %   industrializacji: analiza państwowego gospodarstwa przemysłowego
  %   w~Królestwie Polskim XIX w.};

  % \item Jerzy Jedlicki, \emph{Klejnot i~bariery społeczne:
  %   przeobrażenia szlachectwa polskiego w~schyłkowym okresie
  %   feudalizmu};

  % \item Jerzy Jedlicki, \emph{Świat zwyrodniały: lęki i~wyroki
  %   krytyków nowoczesności};

  % \item Rob Riemen, \emph{Wieczny powrót faszyzmu};

  % \item Łukasz A. Plesnar, \emph{Twarze Westernu};

  % \item S. Prat, \emph{Język C++. Szkoła programowania};

  % \item L. Strauss, \emph{Prawo naturalne w~świetle historii};

  % \item Oskar Halecki, \emph{Tysiąc lat polski katolickiej};

  % \item Krzysztof Mazur, \emph{Przekroczyć nowoczesność. Projekt
  %   polityczny ruchu społecznego Solidarność};

  % \item Marian Henryk Serejski, \emph{Europa a rozbiory Polski:
  %   studium historiograficzne};

  % \item Jerzy Łanowski, \emph{Antologia anegdoty antycznej: teraz
  %   trzeci raz wydane historyjki budujące i niebudujące z autorów
  %   greckich i rzymskich};

  % \item Artur Domosłowski, \emph{Kapuściński non-fiction};

  % \item Michael Moran, \emph{Kraj z Księżyca: podróże do serca
  %   Polski};

  % \item Pierre Hadot, \emph{Filozofia jako ćwiczenie duchowe};

  % \item Robert D. Richtmayer, \emph{Principles of advanced
  %   mathematical physics};

  % \item Walter Burkert, \emph{Stwarzanie świętości. Ślady biologii
  %   we~wczesnych wierzeniach religijnych};

  % \item A. I. Anselm, \emph{Podstawy fizyki statystycznej
  %   i~termodynamiki};

  % \item Z. Krasnodębski, \emph{Demokracja peryferii};

  % \item Maria Dzielska, \emph{Hypatia z~Aleksandrii};

  % \item Paweł Śpiewak, \emph{Spór o~Polskę, 1989--99};

  % \item Tadeusz Zieliński, \emph{Religia starożytnej Grecji};

  % \item Ernest Gellner, \emph{Postmodernizm, rozum i~religia};

  % \item Daniel Beauvois, \emph{Polacy na Ukrainie 1831-1863.
  %   Szlachta polska na Wołyniu, Podolu i Kijowszczyźnie};

  % \item \emph{Polskie mity polityczne XIX i XX wieku};

  % \item \emph{O nas bez nas. Historia Polski w historiografiach
  %   obcojęzycznych};

  % \item Leibniz, \emph{Wyznanie wiary filozofa, Rozprawa
  %   metafizyczna; Monadologia; Zasady natury i łaski oraz inne pisma
  %   filozoficzne};

  % \item Hanna Świda-Ziemba, \emph{Człowiek wewnętrznie zniewolony.
  %   Mechanizmy i konsekwencje minionej formacji --~analiza
  %   psychologiczna};

  % \item Paul Johnson, \emph{Historia świata XX wieku};

  % \item Braudel Fernand, \emph{Dynamika kapitalizmu};

  % \item I. Berlin, \emph{Korzenie romantyzmu};

  % \item Bod Rens, \emph{Historia humanistyki};

  % \item \emph{Teologia i filozofia żydowska wobec Holocaustu};

  % \item Acton, \emph{Historia wolności: wybór esejów};

  % \item Andrzej Żbikowski, \emph{Żydzi};

  % \item Stefan Bartkowski, \emph{Pod wspólnym niebem: krótka
  %   historia Żydów w~Polsce i~stosunków polsko-żydowskich};

  % \item Arystoteles, \emph{Retoryka};

  % \item Agnieszka Urbańczyk, Diamentowy Grant, \emph{Polityczność
  %   science fiction w recepcji fanowskie};

  % \item \emph{Lech Wzbudzony};

  % \item \emph{Unintended Reformation};

  % \item J. Polit, \emph{Chiny};

  % \item Paweł Śpiewak, \emph{Teologia i~filozofia żydowska wobec
  %   Holocaustu};

  % \item J.K. Fairbank, \emph{Historia Chin. Nowe spojrzenie};

  % \item \emph{Nowożytna historia Chin}, red. R. Sławiński;

  % \item R. Sławiński, \emph{Geneza Chińskiej Republiki Ludowej};

  % \item K. Seitz, \emph{Chiny. Powrót Olbrzyma};

  % \item A. Bolesta, \emph{Chiny w okresie transformacji};

  % \item \emph{Chiny. Przemiany państwa i społeczeństwa w okresie
  %   reform 1978--2000}, red. K. Tomala;

  % \item Andrzej Napiórkowski OSPPE, \emph{Teologie XX i~XXI wieku};

  % \item Alfred V.~Aho, Jeffrey D.~Ullman, \emph{Wykłady
  %   z~informatyki z~przykładami w~języku~C};

  % \item Jerzy Eisler, \emph{Co nam zostało z tamtych lat.
  %   Dziedzictwo PRL};

  % \item Peter Burke;

  % \item B. Kozera, \emph{Literatura a~religia. Polska współczesna
  %   powieść katolicka};

  % \item \emph{Physics of living systems};

  % \item Jadwiga Staniszkis, \emph{Samoograniczająca~się rewolucja};

  % \item Jadwiga Staniszkis, \emph{Postkomunizm. Próba opisu};

  % \item Z. Wójcik, \emph{Dzikie Pola w~ogniu. O~Kozaczyźnie w~dawnej
  %   Rzeczypospolitej};

  % \item Jan Kofman, Wojciech Roszkowski, \emph{Transformacja
  %   i~postkomunizm};

  % \item M. Gołaszewska, \emph{Estetyka współczesna};

  % \item E. Badinter, \emph{XY --~tożsamość mężczyzny};

  % \item R. Bly, \emph{Żelazny Jan. Rzecz o~mężczyznach};

  % \item Z. Wójcik, \emph{Wojny kozackie w dawnej Polsce};

  % \item Z. Wójcik, \emph{Dzieje Rosji: 1533-1801};

  % \item Vigarello Georges, \emph{Historia gwałtu};

  % \item Tannahill Reay, \emph{Historia kuchni};

  % \item Éliphas Lévi, \emph{Historia magii};

  % \item Meyer Michel (red.), \emph{Historia retoryki od Greków do
  %   dziś};

  % \item Simmel Georg, \emph{Filozofia pieniądza};

  % \item Dahl Robert, \emph{Demokracja i~jej krytycy};

  % \item Z. Wojcik, \emph{Jan Sobieski: 1629-1696};

  % \item Z. Wójcik, \emph{Jan III Sobieski};

  % \item Ariès Philippe, \emph{Historia dzieciństwa. Dziecko i
  %   rodzina w czasach ancien régime’u};

  % \item Bataille Georges, \emph{Historia erotyzmu};

  % \item Vigarello Georges, \emph{Historia otyłości};

  % \item Flandrin Jean-Louis, \emph{Historia rodziny};

  % \item Z. Wójcik, \emph{Jan Kazimierz Waza};

  % \item Z. Wójcik, \emph{Józef Piłsudski 1867-1935};

  % \item \emph{Legendy uświęcone. Twórczość J. R. R. Tolkiena a
  %   chrześcijaństwo};

  % \item Stefan Bratkowski, \emph{Nieco inna historia cywilizacji:
  %   dzieje banków, bankierów i obrotu pieniężnego};

  % \item Izrael Szahad, \emph{Żydowskie dzieje i religia; Żydzi i
  %   goje – XXX wieków historii};

  % \item Izrael Szahak, \emph{Tel Awiw za zamkniętymi drzwiami};

  % \item Andrzej Żbikowski, \emph{Ideologia antysemicka w~Polsce
  %   1848-1918};

  % \item Władysław Bruliński, \emph{Antykościół};

  % \item Władysław Bruliński, \emph{Co to jest marksizm?};

  % \item Władysław Bruliński, \emph{Czerwone palmy historii};

  % \item Władysław Bruliński, \emph{Dokąd idziesz Polsko?};

  % \item \emph{Żydzi i judaizm we współczesnych badaniach polskich};

  % \item Artur Eisenbach, \emph{Emancypacja Żydów na ziemiach
  %   polskich 1785-1870 na tle europejskim};

  % \item Artur Eisenbach, \emph{Z dziejów ludności żydowskiej w
  %   Polsce w XVIII i XIX w.};

  % \item Artur Eisenbach, \emph{Kwestia równouprawnienia Żydów w
  %   Królestwie Polskim};

  % \item Marian Fuks, \emph{Humor Żydów polskich (do 1939 r.)};

  % \item Marian Fuks, \emph{Żydzi w Polsce – Dawniej i dziś};

  % \item Marian Fuks, \emph{Prasa żydowska w Warszawie 1823-1939};

  % \item Marian Fuks, \emph{Z dziejów wielkiej katastrofy narodu
  %   żydowskiego};

  % \item August Grabski, \emph{Studia z dziejów i kultury Żydów w
  %   Polsce po 1945 r.};

  % \item Robert Fabbri, \emph{Wespazjan, trybun Rzymu};

  % \item Michael Hesemann, \emph{Kłamstwa Hitlera};

  % \item David Hockney, \emph{Wiedza tajemna. Sekrety technik
  %   malarskich Dawnych Mistrzów};

  % \item Anna Magdalena Mandrela, \emph{Tomizm Garrigou-Lagrange’a
  %   wobec wizji filozoficznej Teilharda de Chardin};

  % \item Leonie Swann, \emph{Powiększ Sprawiedliwość owiec.
  %   Filozoficzna powieść kryminalna};

  % \item Antoine de Saint-Exupéry, \emph{Twierdza};

  % \item Umberto Eco, \emph{Historia brzydoty};

  % \item Brian Reynolds Myers, \emph{Najczystsza rasa: Propaganda
  %   Korei Północnej};

  % \item Tomasz Strzyżewski, \emph{Wielka księga cenzury PRL w
  %   dokumentach};

  % \item Orlando Figes, \emph{Tragedia narodu. Rewolucja rosyjska
  %   1891-1924};

  % \item Jon Savage, \emph{Teenage: The Creation of Youth Culture};

  % \item Lawrence Weschler, \emph{Mr. Wilson's Cabinet of Wonder:
  %   Pronged Ants, Horned Humans, Mice on Toast, and Other Marvels of
  %   Jurassic Technology};

  % \item Christophe Galfard, \emph{Wszechświat w twojej dłoni};

  % \item Swietłana Aleksijewicz, \emph{Cynkowi chłopcy};

  % \item Aleksander Hertz, \emph{Amerykańskie stronnictwa
  %   polityczne};

  % \item Aleksander Hertz, \emph{Żydzi w kulturze polskiej};

  % \item Aleksander Hertz, \emph{Wyznania starego człowieka};

  % \item Maurycy Horn, \emph{Żydowskie bractwa rzemieślnicze na
  %   ziemiach polskich, litewskich, białoruskich i ukraińskich w
  %   latach 1613-1850};

  % \item Maurycy Horn, \emph{Walka chłopów czerwonoruskich z
  %   wyzyskiem feudalnym w latach 1600-1643};


  % \item Adam Kaźmierczyk, \emph{Sejmy i sejmiki szlacheckie wobec
  %   Żydów w II połowie XVII wieku};

  % \item Adam Kaźmierczyk, \emph{Żydzi w dobrach prywatnych. W
  %   świetle sądowniczej i administracyjnej praktyki dóbr magnackich
  %   w wiekach XVI-XVIII};

  % \item Krystyna Kersten, \emph{Polacy-Żydzi-Komunizm. Anatomia
  %   półprawd 1939-1968};

  % \item Krystyna Kersten, \emph{Pogrom Żydów w Kielcach 4 lipca 1946
  %   r.};

  % \item Andrzej Sulima Kamiński, \emph{Historia Rzeczypospolitej
  %   Wielu Narodów 1505-1795. Obywatele, ich państwa, społeczeństwo,
  %   kultura};

  % \item Cynarski S., \emph{Zygmunt August};

  % \item Cyra A., \emph{Rotmistrz Pilecki. Ochotnik do Auschwitz};

  % \item \emph{Czy ktoś przebije ten mur? Sprawa Pyjasa};

  % \item Dudek A., Zblewski Z., \emph{Utopia nad Wisłą. Historia
  %   Peerelu};

  % \item Czapliński W., \emph{Władysław IV i jego czasy};

  % \item Dybiec J., \emph{Nie tylko szablą. Nauka i kultura polska w
  %   walce o~utrzymanie tożsamości narodowej 1795--1918};

  % \item Eisler J., \emph{Zarys dziejów politycznych Polski
  %   1944--1989};

  % \item Grzybowski S., \emph{Henryk Walezy};

  % \item Ignatowicz I., \emph{Społeczeństwo polskie 1864--1914};

  % \item \emph{Inteligencja polska XIX i XX wieku. Studia}, red. R.
  %   Czapulis-Rastenis, t.1-6;

  % \item Jedynak B., \emph{Obyczaje domu polskiego w~czasach niewoli
  %   1795--1918};

  % \item Kaczmarczyk J., \emph{Bohdan Chmielnicki};

  % \item Kawalec K., \emph{Roman Dmowski};

  % \item \emph{Kobieta i kultura życia codziennego. Wiek XIX i XX.
  %   Zbiór studiów}, red. A. Żarnowska, A. Szwarc;

  % \item \emph{Kobieta i społeczeństwo na ziemiach polskich w XIX
  %   wieku, zbiór studiów}, red. A. Żarnowska, A. Szwarc;

  % \item Konopczyński W., \emph{Dzieje Polski nowożytnej};

  % \item Kowecka E., \emph{W salonie i w kuchni. Opowieść o kulturze
  %   materialnej pałaców i dworów polskich w XIX wieku};

  % \item Krawczak T., \emph{W szlacheckim zaścianku};

  % \item Kuchowicz Z., \emph{Miłość staropolska};

  % \item Kuchowicz Z., \emph{Obyczaje staropolskie XVII-XVIII w.};

  % \item Litak S., \emph{Od reformacji do Oświecenia. Kościół
  %   katolicki w~Polsce nowożytnej};

  % \item Mączak A., \emph{Klientela. Nieformalne systemy władzy w
  %   Polsce Europie XVI-XVIII w.};

  % \item Łuczak C., \emph{Polska i Polacy w drugiej wojnie
  %   światowej};

  % \item Molenda J., \emph{Chłopi. Naród. Niepodległość.
  %   Kształtowanie się postaw narodowych i~obywatelskich chłopów
  %   w~Galicji i~Królestwie polskim w~przededniu odrodzenia Polski};

  % \item Możdżyńska-Nawotka M., \emph{O~modach i~strojach};

  % \item \emph{Obyczaje w Polsce. Od średniowiecza do czasów
  %   współczesnych};

  % \item Olszewski D., \emph{Polska kultura religijna na przełomie
  %   XIX i~XX wieku};

  % \item Olszewski H., \emph{O skutecznym rad sposobie};

  % \item \emph{Polska XVII wieku. Państwo, społeczeństwo, kultura},
  %   red. J. Tazbir;

  % \item Paczkowski A., \emph{Pół wieku dziejów Polski, 1939--1989};

  % \item \emph{Polska na przestrzeni wieków}, red. J. Tazbir;

  % \item Przyboś A., \emph{Michał Korybut Wiśniowiecki 1640-1673};

  % \item Rok B., \emph{Człowiek wobec śmierci w kulturze
  %   staropolskiej};

  % \item \emph{Rzeczpospolita wielu narodów i jej tradycje}, red. M.
  %   Markiewicz, A. Link-Lenczewski;

  % \item \emph{Społeczeństwo polskie od X do XX wieku}, red. I.
  %   Ignatowicz, A. Mączak, B. Zientara, J. Żarnowski;

  % \item Staszewski J., \emph{August II Mocny};

  % \item Staszewski J., \emph{August III Sas};

  % \item Suleja W., \emph{Józef Piłsudski};

  % \item Szubarczyk P., \emph{Inka. Zachowałam się jak trzeba\ldots};

  % \item Topolski J., \emph{Polska w czasach nowożytnych. Od
  %   europejskiej potęgi do utraty niepodległości};

  % \item Terlecki R., \emph{Miecz i tarcza komunizmu. Historia
  %   aparatu bezpieczeństwa 1944--1990};

  % \item \emph{Tradycje polityczne dawnej Polski}, red. A.
  %   Sucheni-Grabowska, A. Dybowska;

  % \item Wandycz P., \emph{Pod zaborami. Ziemie Rzeczypospolitej w
  %   latach 1795--1918};

  % \item Wisner H., \emph{Władysław IV Waza};

  % \item Wisner H., \emph{Zygmunt III Waza};

  % \item Zielińska Z., \emph{Ostatnie lata Pierwszej
  %   Rzeczypospolitej};

  % \item

  % \item

  % \item

  % \item

  % \item

  % \item

  % \item

  % \item

  % \item

  % \item

  % \item

  % \item

  % \item

  % \item

  % \item

  % \item

  % \item

  % \item

  % \item

  % \item

  % \item

  % \item

  % \item

  % \item

  % \item

  % \item

  % \item

  % \item

  % \item

  % \item

  % \item

  % \item

  % \item

  % \item

  % \item

  % \item

  % \item

  % \item

  % \item

  % \item

  % \item

  % \item

  % \item

  % \item

  % \item

  % \item

  % \item

  % \item

  % \item

  % \item

  % \item

  % \item

  % \item

  % \item

  % \item

  % \item Tony Judt, \emph{Historia niedokończona. Francuscy
  %   intelektualiści 1944-1956};

  % \item Karl Loewith, \emph{Od Hegla do Nietzschego. Rewolucyjny
  %   przełom w myśli XIX wieku};

  % \item J. Fiske \emph{Zrozumieć kulturę popularną};

  % \item

  % \item

  % \item

  % \item

  % \item

  % \item

  % \item

  % \item

  % \item

  % \item

  % \item

  % \item

  % \item

  % \item

  % \item

  % \item

  % \item

  % \item

  % \item

  % \item

  % \item

  % \item

  % \item

  % \item

  % \item

  % \item

  % \item

  % \item

  % \item

  % \item

  % \item

  % \item

  % \item

  % \item

  % \item

  % \item

  % \item

  % \item

  % \item

  % \item

  % \item

  % \item

  % \item

  % \item

  % \item

  % \item

  % \item












































\end{enumerate}










% % ######################################
% \newpage
% \section{Zaczęte i~nieskończone}

% \vspace{\spaceTwo}
% % ######################################














% % ######################################
% \newpage
% \section{Articles}

% \vspace{\spaceTwo}
% % ######################################



% \begin{enumerate}

% \item Edward Witten, \emph{Notes on Some Entanglement Properties of
%   Quantum Field Theory},
%   \href{https://arxiv.org/abs/1803.04993}{arXiv:1803.04993};

% \item Jeff Bezanson et al, "Julia: dynamism and performance
%   reconciled by design"
%   \href{https://doi.org/10.1145/3276490}{https://doi.org/10.1145/3276490};

% \item Francesco Zappa Nardelli et al., "Julia subtyping: a rational
%   reconstruction",
%   \href{https://doi.org/10.1145/3276914}{https://doi.org/10.1145/3276914};

% \item Artem Pelenitsyn et al., "Type Stability in Julia: Avoiding
%   Performance Pathologies in JIT Compilation",
%   https://doi.org/10.1145/3485527,
%   \href{arXiv:2109.01950}{https://arxiv.org/abs/2109.01950};

% \item \emph{How SQLite Is Tested},
%   \href{https://www.sqlite.org/testing.html}{https://www.sqlite.org/testing.html};

% \item \emph{SpotBugs},
%   \href{https://spotbugs.github.io/}{https://spotbugs.github.io/};

% \item \emph{A tool to detect bugs in Java and C/C++/Objective-C code
%   before it ships},
%   \href{https://fbinfer.com/}{https://fbinfer.com/la};

% \item \emph{Go 1 and the Future of Go Programs},
%   \href{https://go.dev/doc/go1compat}{https://go.dev/doc/go1compat};

% \item \emph{SD-8: Standard Library Compatibility},
%   \href{https://isocpp.org/std/standing-documents/sd-8-standard-library-compatibility}{https://isocpp.org/std/standing-documents/sd-8-standard-library-compatibility};

% \item \emph{GNU General Public License},
%   \href{https://www.gnu.org/licenses/gpl-3.0.html}{https://www.gnu.org/licenses/gpl-3.0.html};

% \item \emph{Reflections on trusting trust},
%   \href{https://dl.acm.org/doi/10.1145/358198.358210}{https://dl.acm.org/doi/10.1145/358198.358210};

% \item \emph{Go \& Versioning},
%   \href{https://research.swtch.com/vgo}{https://research.swtch.com/vgo};

% \item \emph{Why Google stores billions of lines of code in a single
%   repository},
%   \href{https://dl.acm.org/doi/10.1145/2854146}{https://dl.acm.org/doi/10.1145/2854146};

% \item \emph{Testing Chromium: ThreadSanitizer v2, a next-gen data
%   race
%   detector}, \\
%   \href{https://blog.chromium.org/2014/04/testing-chromium-threadsanitizer-v2.html}{https://blog.chromium.org/2014/04/testing-chromium-threadsanitizer-v2.html};

% \item \emph{Search Vulnerability Database},
%   \href{https://nvd.nist.gov/vuln/search}{https://nvd.nist.gov/vuln/search};

% \item \emph{Regular Expression Matching with a Trigram Index or How
%   Google Code Search Worked}, \\
%   \href{https://swtch.com/~rsc/regexp/regexp4.html}{https://swtch.com/~rsc/regexp/regexp4.html};

% \item \emph{Licenses},
%   \href{https://opensource.google/docs/thirdparty/licenses}{https://opensource.google/docs/thirdparty/licenses};

% \item \emph{ImperialViolet},
%   \href{https://www.imperialviolet.org/2009/08/26/seccomp.html}{https://www.imperialviolet.org/2009/08/26/seccomp.html};

% \item \emph{Multi-process Architecture},
%   \href{https://blog.chromium.org/2008/09/multi-process-architecture.html}{https://blog.chromium.org/2008/09/multi-process-architecture.html};

% \item \emph{A single Node of failure},
%   \href{https://lwn.net/Articles/681410/}{https://lwn.net/Articles/681410/};

% \item \emph{Interpreting the Data: Parallel Analysis with Sawzall},
%   \href{https://www.hindawi.com/journals/sp/2005/962135/}{https://www.hindawi.com/journals/sp/2005/962135/};

% \item \emph{Go Proverbs},
%   \href{https://go-proverbs.github.io/}{https://go-proverbs.github.io/};

% \item \emph{RE2: a principled approach to regular expression
%   matching}, \\
%   \href{https://opensource.googleblog.com/2010/03/re2-principled-approach-to-regular.html}{https://opensource.googleblog.com/2010/03/re2-principled-approach-to-regular.html};

% \item \emph{Details about the event-stream incident}, \\
%   \href{https://blog.npmjs.org/post/180565383195/details-about-the-event-stream-incident}{https://blog.npmjs.org/post/180565383195/details-about-the-event-stream-incident};

% \item \emph{Open-sourcing gVisor, a sandboxed container runtime}, \\
%   \href{https://cloud.google.com/blog/products/identity-security/open-sourcing-gvisor-a-sandboxed-container-runtime}{https://cloud.google.com/blog/products/identity-security/open-sourcing-gvisor-a-sandboxed-container-runtime};

% \item Samuel R. Buss, Alexander S. Kechris, Anand Pillay, Richard A.
%   Shore, \emph{The prospects for mathematical logic in the
%   twenty-first century},
%   \href{https://arxiv.org/abs/cs/0205003v1}{arXiv:cs/0205003v1};

% \item Christian Retore, \emph{On the system F as a glue language for
%   natural-language compositional-semantics},
%   \href{https://arxiv.org/abs/1108.5084}{arXiv:1108.5084};

% \item Robert Harper, \emph{An Equational Logical Framework for Type
%   Theories},
%   \href{https://arxiv.org/abs/2106.01484}{arXiv:2106.01484};

% \item Jan Leike, et al., \emph{AI Safety Gridworlds},
%   \href{https://arxiv.org/abs/1711.09883}{arXiv:1711.09883v2};

% \item Sergei Gukov, Edward Witten, \emph{Branes and Quantization},
%   \href{https://arxiv.org/abs/0809.0305}{https://arxiv.org/abs/0809.0305};

% \item R. Estrada, J. M. Gracia-Bondia, J. C. Varilly, \emph{On
%   summability of distributions and spectral geometry},
%   \href{https://arxiv.org/abs/funct-an/9702001v1}{arXiv:funct-an/9702001};

% \item Ghanashyam Date, \emph{Lectures on Constrained Systems},
%   \href{https://arxiv.org/abs/1010.2062v1}{arXiv:1010.2062};

% \item Frank Wilczek, \emph{Quantum Field Theory},
%   \href{https://arxiv.org/abs/hep-th/9803075v2}{arXiv:hep-th/9803075};

% \item Karl Michael Schmidt, Karl Michael Schmidt, \emph{Schnol’s
%   Theorem and Spectral Properties of Massless Dirac Operators with
%   Scalar Potentials};

% \item Peter J. Olver, \emph{Dirac’s theory of constraints in fields
%   theory and the canonical form of Hamiltonian differential
%   operators};


% \item Lorenzo Iorio, \emph{Editorial for the Special Issue 100 Years
%   of Chronogeometrodynamics: The Status of the Einstein's Theory of
%   Gravitation in Its Centennial Year},
%   \href{https://arxiv.org/abs/1504.05789v2}{arXiv:1504.05789};

% \item B. Mutet, P. Grang\'{e}, E. Werner, \emph{Taylor–Lagrange
%   renormalization and gauge theories in four dimensions};


% \item \emph{Surviving Software Dependencies},
%   \href{https://queue.acm.org/detail.cfm?id=3344149}{https://queue.acm.org/detail.cfm?id=3344149};

% \item Clifford M. Will, \emph{The Confrontation between General
%   Relativity and Experiment};

% \item Paul Lopes, \emph{Culture and Stigma: Popular Culture and the
%   Case of Comic Books};

% \item Marina S.Butuzova 1, Alexander B. Pushkarev, \emph{Is OJ 287 a
%   Single Supermassive Black Hole?};

% \item Peter Selinger, \emph{Lecture notes on the lambda calculus},
%   \href{https://arxiv.org/abs/0804.3434v2}{arXiv:0804.3434};

% \item Alex Eskin, Maryam Mirzakhani, \emph{Counting closed geodesics
%   in Moduli space},
%   \href{https://arxiv.org/abs/0811.2362v3}{arXiv:0811.2362};

% \item Jacques Carette, James H. Davenport, \emph{The Power of
%   Vocabulary: The Case of Cyclotomic Polynomials},
%   \href{https://arxiv.org/abs/1002.0012v1}{arXiv:1002.0012};

% \item Chris Kapulkin, Peter LeFanu Lumsdaine, \emph{The Simplicial
%   Model of Univalent Foundations (after Voevodsky)},
%   \href{https://arxiv.org/abs/1211.2851v5}{arXiv:1211.2851};

% \item Christopher J. Fewster, Rainer Verch, \emph{Quantum fields and
%   local measurements},
%   \href{https://arxiv.org/abs/1810.06512}{arXiv:1810.06512};

% \item Paweł Duch, \emph{Infrared problem in perturbative quantum
%   field theory},
%   \href{https://arxiv.org/abs/1906.00940}{arXiv:1906.00940};

% \item Christopher J. Fewster, \emph{A generally covariant
%   measurement scheme for quantum field theory in curved spacetimes},
%   \href{https://arxiv.org/abs/1904.06944v1}{arXiv:1904.06944};

% \item Jacob Lurie, \emph{Higher Topos Theory},
%   \href{https://arxiv.org/abs/math/0608040v4}{arXiv:math/0608040};

% \item Konrad Osterwalder, and Robert Schrader, \emph{Axioms for
%   Euclidean Green's Functions};

% \item Vladimir Voevodsky, \emph{A very short note on homotopy
%   $\lambda$-calculus};

% \item Charles Rezk, \emph{Toposes and homotopy toposes};

% \item Fredrik Nordvall Forsberg, Anton Setzer, \emph{A finite
%   axiomatisation of inductive-inductive definitions};

% \item Egbert Rilke, \emph{Introduction to homotopy type theory};

% \item \'{A}lvaro Pelayo, Michael A. Warren, \emph{Homotopy type
%   theory and Voevodsky’s Univalent Foundations};

% \item H. Simmons, A. Schalk, \emph{An introduction to
%   $\lambda$-calculi and arithmetics};

% \item Roderich Tumulka, \emph{Lecture Notes on Mathematical
%   Statistical Physics};

% \item Martin Hofmann, \emph{Extensional concepts in intensional type
%   theory};

% \item Eugenio Moggi, \emph{Computational $\lambda$-calculus and
%   monads};

% \item \emph{Proof-theoretic semantics. Assessment and Future
%   Perspectives};

% \item Philip Walder, \emph{Propostions as Types};

% \item Egbert Rijke, \emph{Homotopy type theory};

% \item Eugenio Moggi, \emph{Notions of computation and monads};

% \item Bruno Barras, Thierry Coquand and Simon Huber, \emph{A
%   Generalization of Takeuti-Gandy Interpretation};

% \item Michael Alton Warren, \emph{Homotopy Theoretic Aspects of
%   Constructive Type Theory};

% \item Vladimir Voevodsky, \emph{A universe polymorphic type system};

% \item S. Marmi, \emph{An Introduction To Small Divisors},
%   \href{https://arxiv.org/abs/math/0009232v1}{arXiv:math/0009232};

% \item Paweł Duch, Michael Duetsch, Jose M. Gracia-Bondia,
%   \emph{Diphoton decay of the higgs from the Epstein--Glaser
%   viewpoint},
%   \href{https://arxiv.org/abs/2011.12675v2}{arXiv:2011.12675};

% \item Juliette Kennedy, Menachem Magidor, Jouko
%   V\"{a}\"{a}n\"{a}nen, \emph{ Inner Models from Extended Logics:
%   Part 1},
%   \href{https://arxiv.org/abs/2007.10764}{arXiv:2007.10764};

% \item Paul W. Gross, P. Robert Kotiuga, \emph{Electromagnetic Theory
%   and Computation: A Topological Approach};

% \item Andreas R. Blass, Jeffry L. Hirst, and Stephen G. Simpson,
%   Logical analysis of some theorems of combinatorics and topological
%   dynamics, Logic and combinatorics (Arcata, Calif., 1985), Contemp.
%   Math., vol. 65, Amer. Math. Soc., Providence, RI, 1987, pp.
%   125–156.

% \item Jared Corduan, Marcia Groszek, and Joseph Mileti, Draft: A
%   note on reverse mathematics and partitions of trees.

% \item Jennifer Chubb, Jeffry Hirst, and Tim McNichol, Reverse
%   mathematics and partitions of trees. To appear in J. Symbolic
%   Logic.

% \item Damir Dzhafarov and Jeffry Hirst, The polarized Ramsey
%   theorem. Archive for Math. Logic, Online First: 2008.

% \item Neil Hindman, The existence of certain ultra-filters on N and
%   a conjecture of Graham and Rothschild, Proc. Amer. Math. Soc. 36
%   (1972), 341–346.

% \item Jeffry L. Hirst, Hindman’s theorem, ultrafilters, and reverse
%   mathematics, J. Symbolic Logic 69 (2004), no. 1, 65–72.

% \item Carl G. Jockusch Jr., Ramsey’s theorem and recursion theory,
%   J. Symbolic Logic 37 (1972), 268–280.

% \item J. Mileti, Partition theory and computability theory. Ph.D.
%   Thesis.

% \item


































































































































































































































































% \end{enumerate}
% % \bibliographystyle{alpha} \bibliography{Bibliography}{}










% ############################

% Koniec dokumentu
\end{document}
