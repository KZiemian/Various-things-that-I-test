% ---------------------------------------------------------------------
% Podstawowe ustawienia i pakiety
% ---------------------------------------------------------------------
\RequirePackage[l2tabu, orthodox]{nag}  % Wykrywa przestarzałe i niewłaściwe
% sposoby używania LaTeXa. Więcej jest w l2tabu English version.
\documentclass[a4paper,11pt]{article}
% {rozmiar papieru, rozmiar fontu}[klasa dokumentu]
\usepackage[MeX]{polski}  % Polonizacja LaTeXa, bez niej będzie pracował
% w języku angielskim.
\usepackage[utf8]{inputenc} % Włączenie kodowania UTF-8, co daje dostęp
% do polskich znaków.
\usepackage{lmodern}  % Wprowadza fonty Latin Modern.
\usepackage[T1]{fontenc}  % Potrzebne do używania fontów Latin Modern.



% ------------------------------
% Podstawowe pakiety (niezwiązane z ustawieniami języka)
% ------------------------------
\usepackage{microtype}  % Twierdzi, że poprawi rozmiar odstępów w tekście.
\usepackage{graphicx}  % Wprowadza bardzo potrzebne komendy do wstawiania
% grafiki.
% \usepackage{verbatim}  % Poprawia otoczenie VERBATIME.
% \usepackage{textcomp}  % Dodaje takie symbole jak stopnie Celsiusa,
% wprowadzane bezpośrednio w tekście.
\usepackage{vmargin}  % Pozwala na prostą kontrolę rozmiaru marginesów,
% za pomocą komend poniżej. Rozmiar odstępów jest mierzony w calach.
% ------------------------------
% MARGINS
% ------------------------------
\setmarginsrb
{ 0.7in}  % left margin
{ 0.6in}  % top margin
{ 0.7in}  % right margin
{ 0.8in}  % bottom margin
{  20pt}  % head height
{0.25in}  % head sep
{   9pt}  % foot height
{ 0.3in}  % foot sep



% ------------------------------
% Często przydatne pakiety
% ------------------------------
% \usepackage{csquotes}  % Pozwala w prosty sposób wstawiać cytaty do tekstu.




% ------------------------------
% Pakiety do tekstów z nauk przyrodniczych
% ------------------------------
\let\lll\undefined  % Amsmath gryzie się z pakietami do języka
% polskiego, bo oba definiują komendę \lll. Aby rozwiązać ten problem
% oddefiniowuję tę komendę, ale może tym samym pozbywam się dużego Ł.
\usepackage[intlimits]{amsmath}  % Podstawowe wsparcie od American
% Mathematical Society (w skrócie AMS)
\usepackage{amsfonts, amssymb, amscd, amsthm}  % Dalsze wsparcie od AMS
% \usepackage{siunitx}  % Do prostszego pisania jednostek fizycznych
\usepackage{upgreek}  % Ładniejsze greckie litery
% Przykładowa składnia: pi = \uppi
% \usepackage{slashed}  % Pozwala w prosty sposób pisać slash Feynmana.
\usepackage{calrsfs}  % Zmienia czcionkę kaligraficzną w \mathcal
% na ładniejszą. Może w innych miejscach robi to samo, ale o tym nic
% nie wiem.





% ------------------------------
% Paczki, biblioteki i ich ustawienia dla tego pliku
% ------------------------------
\usepackage{tikz}  % Wspaniały pakiet PGF/TikZ

% \listfiles







% ------------------------------
% Pakiet „hyperref”
% Polecano by umieszczać go na końcu preambuły
% ------------------------------
\usepackage{hyperref}  % Pozwala tworzyć hiperlinki i zamienia odwołania
% do bibliografii na hiperlinki










% ---------------------------------------------------------------------
% Tytuł, autor, data
\title{Ti\textit{k}Z \& PGF \\
  \href{http://piotrkosoft.net/pub/mirrors/CTAN/graphics/pgf/base/doc/pgfmanual.pdf}{Manual for version 3.1.9a} \\
  2~A~Picture for Karl's Students, part~I}

\author{}


% \date{}
% ---------------------------------------------------------------------










% ####################################################################
\begin{document}
% ####################################################################





% ######################################
\maketitle % Tytuł całego tekstu
% ######################################










% ######################################
\newpage

\section{2~Tutorial: A~Picture for Karl's Students}

\vspace{2em}

% ######################################





\tikz \draw (0,0) rectangle +(1,1)  (1.5,0) rectangle +(1,1);
% Abc 4 \tikz \draw[very thin,step=0.5] (0,0) grid (2,2); def. \hspace{2em}
% Abc 5 \tikz \draw[very thin,help lines,step=0.5] (0,0) grid (2,2); def.

\vspace{2em}





% ##################
\begin{figure}[ht]

  \centering

  \def\rectanglepath{-- +(1cm,0cm) -- +(1cm,1cm) -- +(0cm,1cm) -- cycle}

  \begin{tikzpicture}

    \draw (0,0) \rectanglepath;

    \draw (1.5,0) \rectanglepath;

  \end{tikzpicture}

  \caption{Ti\textit{k}Z \& PGF Manula. A~Picture for Karl's Students,~1}

\end{figure}
% ##################





% ##################
\begin{figure}[ht]

  \centering

  \begin{tikzpicture}



  \end{tikzpicture}

  \caption{Ti\textit{k}Z \& PGF Manula. A~Picture for Karl's Students,~2}

\end{figure}
% ##################





% ##################
\begin{figure}[ht]

  \centering

  \begin{tikzpicture}








  \end{tikzpicture}

  \caption{Ti\textit{k}Z \& PGF Manula. A~Picture for Karl's Students,~3}

\end{figure}
% ##################





% ##################
\begin{figure}[ht]

  \centering

  \begin{tikzpicture}








  \end{tikzpicture}

  \caption{Ti\textit{k}Z \& PGF Manula. A~Picture for Karl's Students,~4}

\end{figure}
% ##################





% ##################
\begin{figure}[ht]

  \centering

  \begin{tikzpicture}







  \end{tikzpicture}

  \caption{Ti\textit{k}Z \& PGF Manula. A~Picture for Karl's Students,~5}

\end{figure}
% ##################





% ##################
\begin{figure}[ht]

  \centering

  \begin{tikzpicture}











  \end{tikzpicture}

  \caption{Ti\textit{k}Z \& PGF Manula. A~Picture for Karl's Students,~6}

\end{figure}
% ##################





% ##################
\begin{figure}[ht]

  \centering

  \begin{tikzpicture}









  \end{tikzpicture}

  \caption{Ti\textit{k}Z \& PGF Manula. A~Picture for Karl's Students,~7}

\end{figure}
% ##################





% ##################
\begin{figure}[ht]

  \centering

  \begin{tikzpicture}









  \end{tikzpicture}

  \caption{Ti\textit{k}Z \& PGF Manula. A~Picture for Karl's Students,~8}

\end{figure}
% ##################





% ##################
\begin{figure}[ht]
  \centering

  \begin{tikzpicture}



  \end{tikzpicture}

  \caption{Ti\textit{k}Z \& PGF Manula. A~Picture for Karl's Students,~9}

\end{figure}
% ##################





% ##################
\begin{figure}[ht]

  \centering

  \begin{tikzpicture}



  \end{tikzpicture}

  \caption{Ti\textit{k}Z \& PGF Manula. A~Picture for Karl's Students,~10}

\end{figure}
% ##################





% ##################
\begin{figure}[ht]
  \centering

  \begin{tikzpicture}







  \end{tikzpicture}

  \caption{Ti\textit{k}Z \& PGF Manula. A~Picture for Karl's Students,~11}

\end{figure}
% ##################





% ##################
\begin{figure}[ht]

  \centering

  \begin{tikzpicture}

























  \end{tikzpicture}

  \caption{Ti\textit{k}Z \& PGF Manula. A~Picture for Karl's Students,~12}

\end{figure}
% ##################





% % ##################
% \begin{figure}[ht]

%   \centering

%   \begin{tikzpicture}

%     \draw (-3,-3) -- (-0.5,-3) -- (0,-1) -- (2,-1);

%     \node at (-2,-3.25) {Normal};


%     \draw[very thin] (-3,-2) -- (-0.7,-2) -- (0,-0.9) -- (2,-0.9);

%     \node at (-2,-2.25) {Very thin};


%     \draw[thin] (-3,-1) -- (-0.7,-1) -- (0,-0.8) -- (2,-0.8);

%     \node at (-2,-1.25) {Thin};


%     \draw[semithick] (-3,0) -- (-0.7,0) -- (0,-0.7) -- (2,-0.7);

%     \node at (-2,-0.25) {Semithick};


%     \draw[thick] (-3,1) -- (-0.7,1) -- (0,-0.6) -- (2,-0.6);

%     \node at (-2,0.75) {Thick};


%     \draw[very thick] (-3,2) -- (-0.7,2) -- (0.1,-0.5) -- (2,-0.5);

%     \node at (-2,1.75) {Very thick};


%     \draw[ultra thick] (-3,3) -- (-0.7,3) -- (0.2,-0.4) -- (2,-0.4);

%     \node at (-2,2.75) {Ultra thick};

%   \end{tikzpicture}

%   \caption{Ti\textit{k}Z \& PGF Manula. A~Picture for Karl's Students,~13}

% \end{figure}
% % ##################





% % ##################
% \begin{figure}[ht]

%   \centering

%   \begin{tikzpicture}

%     \draw[dashed] (0,-0.5) -- (8,-0.5);

%     \draw (0,-0.65) -- (0,-0.35);

%     \draw (1,-0.65) -- (1,-0.35);

%     \draw (2,-0.65) -- (2,-0.35);

%     \draw (3,-0.65) -- (3,-0.35);

%     \draw (4,-0.65) -- (4,-0.35);

%     \draw (5,-0.65) -- (5,-0.35);

%     \draw (6,-0.65) -- (6,-0.35);

%     \draw (7,-0.65) -- (7,-0.35);

%     \draw (8,-0.65) -- (8,-0.35);


%     \node[scale=0.7] at (0.5,-0.9) {Ultra thin};

%     \node[scale=0.7] at (1.5,-1.3) {Very thin};

%     \node[scale=0.7] at (2.5,-0.9) {Thin};

%     \node[scale=0.7] at (3.5,-1.3) {Normal};

%     \node[scale=0.7] at (4.5,-0.9) {Semithick};

%     \node[scale=0.7] at (5.5,-1.3) {Thick};

%     \node[scale=0.7] at (6.5,-0.9) {Very thick};

%     \node[scale=0.7] at (7.5,-1.3) {Ultra thick};



%     \draw[ultra thin] (0,0) -- (1,0);

%     \draw[very thin] (1,0) -- (2,0);

%     \draw[thin] (2,0) -- (3,0);

%     \draw (3,0) -- (4,0);

%     \draw[semithick] (4,0) -- (5,0);

%     \draw[thick] (5,0) -- (6,0);

%     \draw[very thick] (6,0) -- (7,0);

%     \draw[ultra thick] (7,0) -- (8,0);

%   \end{tikzpicture}

%   \caption{Ti\textit{k}Z \& PGF Manula. A~Picture for Karl's Students,~14}

% \end{figure}
% % ##################





% % ##################
% \begin{figure}[ht]

%   \centering

%   \begin{tikzpicture}[help lines 1/.style={very thin},]

%     \draw[step=0.5] (0,0) grid (1.5,1.5);

%     \node at (0.75,-0.5) {Normal};



%     \begin{scope}[xshift=2cm]

%       \draw[help lines,step=0.5] (0,0) grid (1.5,1.5);

%       \node at (0.75,-0.5) {Help lines};

%     \end{scope}



%     \begin{scope}[xshift=4cm]

%       \draw[help lines,very thin,step=0.5] (0,0) grid (1.5,1.5);

%       \node[align=left] at (0.75,-0.75)
%       {Help lines, \\
%         very thin};

%     \end{scope}



%     \begin{scope}[xshift=6cm]

%       \draw[very thin,help lines,step=0.5] (0,0) grid (1.5,1.5);

%       \node[align=left] at (0.75,-0.75)
%       {Very thin, \\
%         help line};

%     \end{scope}



%     \begin{scope}[xshift=8cm]

%       \draw[very thin,step=0.5] (0,0) grid (1.5,1.5);

%       \node at (0.75,-0.5) {Very thin};

%     \end{scope}



%     \begin{scope}[xshift=10cm]

%       \draw[help lines 1,step=0.5] (0,0) grid (1.5,1.5);

%       \node[align=left] at (0.75,-0.75)
%       {Help \\
%         lines 1};

%     \end{scope}

%   \end{tikzpicture}

%   \caption{Ti\textit{k}Z \& PGF Manula. A~Picture for Karl's Students,~15}

% \end{figure}
% % ##################





% % ##################
% \begin{figure}[ht]

%   \centering

%   \begin{tikzpicture}

%     \draw[step=0.5] (0,0) grid (1.5,1.5);

%     \node at (0.75,-0.25) {Normal};



%     \begin{scope}[xshift=2cm]

%       \draw[help lines,step=0.5] (0,0) grid (1.5,1.5);

%       \node[align=left] at (0.75,-0.5)
%       {Help \\
%         lines};

%     \end{scope}



%     \begin{scope}[xshift=4cm]

%       \draw[ultra thin,step=0.5] (0,0) grid (1.5,1.5);

%       \node at (0.75,-0.5) {Ultra thin};

%     \end{scope}



%     \begin{scope}[xshift=6cm]

%       \draw[very thin,step=0.5] (0,0) grid (1.5,1.5);

%       \node at (0.75,-0.5) {Very thin};

%     \end{scope}





%     \begin{scope}[yshift=-3cm]

%       \draw[thin,step=0.5] (0,0) grid (1.5,1.5);

%       \node at (0.75,-0.25) {Thin};

%     \end{scope}



%     \begin{scope}[xshift=2cm,yshift=-3cm]

%       \draw[semithick,step=0.5] (0,0) grid (1.5,1.5);

%       \node at (0.75,-0.25) {Semithick};

%     \end{scope}



%     \begin{scope}[xshift=4cm,yshift=-3cm]

%       \draw[thick,step=0.5] (0,0) grid (1.5,1.5);

%       \node at (0.75,-0.25) {Thick};

%     \end{scope}



%     \begin{scope}[xshift=6cm,yshift=-3cm]

%       \draw[very thick,step=0.5] (0,0) grid (1.5,1.5);

%       \node at (0.75,-0.25) {Very thick};

%     \end{scope}





%     \begin{scope}[yshift=-6cm]

%       \draw[ultra thick,step=0.5] (0,0) grid (1.5,1.5);

%       \node at (0.75,-0.3) {Ultra thick};

%     \end{scope}

%   \end{tikzpicture}

%   \caption{Ti\textit{k}Z \& PGF Manula. A~Picture for Karl's Students,~16}

% \end{figure}
% % ##################





% % ##################
% \begin{figure}[ht]

%   \centering

%   \begin{tikzpicture}

%     \fill[color=green] (1,0) circle [radius=0.1];

%     \draw (0,0) -- (1,0) arc [start angle=45,end angle=0,radius=1];

%   \end{tikzpicture}

%   \caption{Ti\textit{k}Z \& PGF Manula. A~Picture for Karl's Students,~17}

% \end{figure}
% % ##################





% % ##################
% \begin{figure}[ht]

%   \centering

%   \begin{tikzpicture}

%     \draw[dashed] (0,0) -- (3,0);

%     \node at (1.5,-0.35) {Dashed};


%     \draw[loosely dashed] (0,1) -- (3,1);

%     \node at (1.5,0.65) {Loosely dashed};


%     \draw[densely dashed] (0,2) -- (3,2);

%     \node at (1.5,1.65) {Densely dashed};





%     \draw[dotted] (4,0) -- (7,0);

%     \node at (5.5,-0.35) {Dotted};


%     \draw[loosely dotted] (4,1) -- (7,1);

%     \node at (5.5,0.65) {Loosely dotted};


%     \draw[densely dotted] (4,2) -- (7,2);

%     \node at (5.5,1.65) {Densely dotted};

%   \end{tikzpicture}

%   \caption{Ti\textit{k}Z \& PGF Manula. A~Picture for Karl's Students,~18}

% \end{figure}
% % ##################





% % ##################
% \begin{figure}[ht]

%   \centering

%   \begin{tikzpicture}

%     \draw[step=0.5cm,gray,very thin] (-1.4,-1.4) grid (1.4,1.4);


%     \draw (-1.5,0) -- (1.5,0);

%     \draw (0,-1.5) -- (0,1.5);


%     \draw (0,0) circle [radius=1cm];

%     \draw (3mm,0mm) arc [start angle=0, end angle=30, radius=3mm];

%   \end{tikzpicture}

%   \caption{Ti\textit{k}Z \& PGF Manula. A~Picture for Karl's Students,~19}

% \end{figure}
% % ##################





% % ##################
% \begin{figure}[ht]

%   \centering

%   \begin{tikzpicture}[scale=3]

%     \draw[step=0.5cm,gray,very thin] (-1.4,-1.4) grid (1.4,1.4);

%     \draw (-1.5,0) -- (1.5,0);

%     \draw (0,-1.5) -- (0,1.5);

%     \draw (0,0) circle [radius=1cm];

%     \draw (3mm,0mm) arc [start angle=0, end angle=30, radius=3mm];

%   \end{tikzpicture}

%   \caption{Ti\textit{k}Z \& PGF Manula. A~Picture for Karl's Students,~20}

% \end{figure}
% % ##################





% % ##################
% \begin{figure}[ht]

%   \centering

%   \begin{tikzpicture}[scale=3]

%     \clip (-0.1,-0.2) rectangle (1.1,0.75);



%     \draw[step=0.5cm,gray,very thin] (-1.4,-1.4) grid (1.4,1.4);

%     \draw (-1.5,0) -- (1.5,0);

%     \draw (0,-1.5) -- (0,1.5);

%     \draw (0,0) circle [radius=1cm];

%     \draw (3mm,0mm) arc [start angle=0, end angle=30, radius=3mm];

%   \end{tikzpicture}

%   \caption{Ti\textit{k}Z \& PGF Manula. A~Picture for Karl's Students,~21}

% \end{figure}
% % ##################





% % ##################
% \begin{figure}[ht]

%   \centering

%   \begin{tikzpicture}[scale=3]

%     \clip[draw] (0.5,0.5) circle [radius=0.6cm];



%     \draw[step=0.5cm,gray,very thin] (-1.4,-1.4) grid (1.4,1.4);

%     \draw (-1.5,0) -- (1.5,0);

%     \draw (0,-1.5) -- (0,1.5);

%     \draw (0,0) circle [radius=1cm];

%     \draw (3mm,0mm) arc [start angle=0, end angle=30, radius=3mm];

%   \end{tikzpicture}

%   \caption{Ti\textit{k}Z \& PGF Manula. A~Picture for Karl's Students,~22}

% \end{figure}
% % ##################





% \tikz \draw (0,0) rectangle (1,1)  (0,0) parabola (1,1); \hspace{2em}
% \tikz \draw[x=1pt,y=1pt] (0,0) parabola bend (4,16) (6,12);

% A sine \tikz \draw[x=1ex,y=1ex] (0,0) sin (1.57,1); curve. \hspace{2em}
% \tikz \draw[x=1.57ex,y=1ex] (0,0) sin (1,1) cos (2,0) sin (3,-1) cos (4,0)
% (0,1) cos (1,0) sin (2,-1) cos (3,0) sin (4,1);





% % ##################
% \begin{figure}[ht]

%   \centering

%   \begin{tikzpicture}[scale=3]

%     \clip (-0.1,-0.2) rectangle (1.1,0.75);



%     \draw[step=0.5cm,gray,very thin] (-1.4,-1.4) grid (1.4,1.4);

%     \draw (-1.5,0) -- (1.5,0);

%     \draw (0,-1.5) -- (0,1.5);

%     \draw (0,0) circle [radius=1cm];

%     \fill[green!20!white] (0,0) -- (3mm,0mm)
%     arc [start angle=0, end angle=30, radius=3mm] -- (0,0);

%   \end{tikzpicture}

%   \caption{Ti\textit{k}Z \& PGF Manula. A~Picture for Karl's Students,~23}

% \end{figure}
% % ##################





% % ##################
% \begin{figure}[ht]

%   \centering

%   \begin{tikzpicture}[line width=5pt]

%     \draw (0,0) -- (1,0) -- (1,1) -- (0,0);

%     \draw (2,0) -- (3,0) -- (3,1) -- cycle;

%     \useasboundingbox (0,1.5);

%   \end{tikzpicture}

%   \caption{Ti\textit{k}Z \& PGF Manula. A~Picture for Karl's Students,~24}

% \end{figure}
% % ##################





% % ##################
% \begin{figure}[ht]
%   \centering

%   \begin{tikzpicture}[scale=3]

%     \clip (-0.1,-0.2) rectangle (1.1,0.75);



%     \draw[step=0.5cm,gray,very thin] (-1.4,-1.4) grid (1.4,1.4);

%     \draw (-1.5,0) -- (1.5,0);

%     \draw (0,-1.5) -- (0,1.5);

%     \draw (0,0) circle [radius=1cm];

%     \filldraw[fill=green!20!white, draw=green!50!black] (0,0) -- (3mm,0mm)
%     arc [start angle=0, end angle=30, radius=3mm] -- cycle;

%   \end{tikzpicture}

%   \caption{Ti\textit{k}Z \& PGF Manula. A~Picture for Karl's Students,~25}

% \end{figure}
% % ##################





% \tikz \shade (0,0) rectangle (2,1)  (3,0.5) circle [radius=0.5cm];





% % ##################
% \begin{figure}[ht]

%   \centering

%   \begin{tikzpicture}[rounded corners,ultra thick]

%     \shade[top color=yellow,bottom color=black] (0,0) rectangle +(2,1);

%     \shade[left color=yellow,right color=black] (3,0) rectangle +(2,1);

%     \shadedraw[inner color=yellow,outer color=black,draw=yellow]
%     (6,0) rectangle +(2,1);

%     \shade[ball color=green] (9,0.5) circle [radius=0.5cm];

%   \end{tikzpicture}

%   \caption{Ti\textit{k}Z \& PGF Manula. A~Picture for Karl's Students,~26}

% \end{figure}
% % ##################





% % ##################
% \begin{figure}[ht]

%   \centering

%   \begin{tikzpicture}[scale=3]

%     \clip (-0.1,-0.2) rectangle (1.1,0.75);



%     \draw[step=0.5cm,gray,very thin] (-1.4,-1.4) grid (1.4,1.4);

%     \draw (-1.5,0) -- (1.5,0);

%     \draw (0,-1.5) -- (0,1.5);

%     \draw (0,0) circle [radius=1cm];

%     \shadedraw[left color=gray,right color=green, draw=green!50!black]
%     (0,0) -- (3mm,0mm)
%     arc [start angle=0, end angle=30, radius=3mm] -- cycle;

%   \end{tikzpicture}

%   \caption{Ti\textit{k}Z \& PGF Manula. A~Picture for Karl's Students,~27}

% \end{figure}
% % ##################





% % ##################
% \begin{figure}[ht]

%   \centering

%   \begin{tikzpicture}[scale=3]

%     \clip (-0.1,-0.2) rectangle (1.1,0.75);



%     \draw[step=0.5cm,gray,very thin] (-1.4,-1.4) grid (1.4,1.4);

%     \draw (-1.5,0) -- (1.5,0);

%     \draw (0,-1.5) -- (0,1.5);

%     \draw (0,0) circle [radius=1cm];

%     \filldraw[fill=green!20,draw=green!50!black] (0,0) -- (3mm,0mm)
%     arc [start angle=0, end angle=30, radius=3mm] -- cycle;

%     \draw[red,very thick] (30:1cm) -- +(0,-0.5);

%   \end{tikzpicture}

%   \caption{Ti\textit{k}Z \& PGF Manula. A~Picture for Karl's Students,~28}

% \end{figure}
% % ##################





% % ##################
% \begin{figure}[ht]

%   \centering

%   \begin{tikzpicture}[scale=3]

%     \clip (-0.1,-0.2) rectangle (1.1,0.75);



%     \draw[step=0.5cm,gray,very thin] (-1.4,-1.4) grid (1.4,1.4);

%     \draw (-1.5,0) -- (1.5,0);

%     \draw (0,-1.5) -- (0,1.5);

%     \draw (0,0) circle [radius=1cm];

%     \filldraw[fill=green!20,draw=green!50!black] (0,0) -- (3mm,0mm)
%     arc [start angle=0, end angle=30, radius=3mm] -- cycle;

%     \draw[red,very thick] (30:1cm) -- +(0,-0.5);

%     \draw[blue,very thick] (30:1cm) ++(0,-0.5) -- (0,0);

%   \end{tikzpicture}

%   \caption{Ti\textit{k}Z \& PGF Manula. A~Picture for Karl's Students,~29}

% \end{figure}
% % ##################





% % ##################
% \begin{figure}[ht]

%   \centering

%   \def\rectanglepath{-- ++(1cm,0cm) -- ++(0cm,1cm)  -- ++(-1cm,0cm) -- cycle}

%   \begin{tikzpicture}

%     \draw (0,0) \rectanglepath;

%     \draw (1.5,0) \rectanglepath;

%   \end{tikzpicture}

%   \caption{Ti\textit{k}Z \& PGF Manula. A~Picture for Karl's Students,~30}

% \end{figure}
% % ##################










% ############################

% Koniec dokumentu
\end{document}
