% ---------------------------------------------------------------------
% Podstawowe ustawienia i pakiety
% ---------------------------------------------------------------------
\RequirePackage[l2tabu, orthodox]{nag}  % Wykrywa przestarzałe i niewłaściwe
% sposoby używania LaTeXa. Więcej jest w l2tabu English version.
\documentclass[a4paper,11pt]{article}
% {rozmiar papieru, rozmiar fontu}[klasa dokumentu]
\usepackage[MeX]{polski}  % Polonizacja LaTeXa, bez niej będzie pracował
% w języku angielskim.
\usepackage[utf8]{inputenc} % Włączenie kodowania UTF-8, co daje dostęp
% do polskich znaków.
\usepackage{lmodern}  % Wprowadza fonty Latin Modern.
\usepackage[T1]{fontenc}  % Potrzebne do używania fontów Latin Modern.



% ------------------------------
% Podstawowe pakiety (niezwiązane z ustawieniami języka)
% ------------------------------
\usepackage{microtype}  % Twierdzi, że poprawi rozmiar odstępów w tekście.
\usepackage{graphicx}  % Wprowadza bardzo potrzebne komendy do wstawiania
% grafiki.
% \usepackage{verbatim}  % Poprawia otoczenie VERBATIME.
% \usepackage{textcomp}  % Dodaje takie symbole jak stopnie Celsiusa,
% wprowadzane bezpośrednio w tekście.
\usepackage{vmargin}  % Pozwala na prostą kontrolę rozmiaru marginesów,
% za pomocą komend poniżej. Rozmiar odstępów jest mierzony w calach.
% ------------------------------
% MARGINS
% ------------------------------
\setmarginsrb
{ 0.7in} % left margin
{ 0.6in} % top margin
{ 0.7in} % right margin
{ 0.8in} % bottom margin
{  20pt} % head height
{0.25in} % head sep
{   9pt} % foot height
{ 0.3in} % foot sep



% ------------------------------
% Często przydatne pakiety
% ------------------------------
% \usepackage{csquotes}  % Pozwala w prosty sposób wstawiać cytaty do tekstu.




% ------------------------------
% Pakiety do tekstów z nauk przyrodniczych
% ------------------------------
\let\lll\undefined  % Amsmath gryzie się z pakietami do języka
% polskiego, bo oba definiują komendę \lll. Aby rozwiązać ten problem
% oddefiniowuję tę komendę, ale może tym samym pozbywam się dużego Ł.
\usepackage[intlimits]{amsmath}  % Podstawowe wsparcie od American
% Mathematical Society (w skrócie AMS)
\usepackage{amsfonts, amssymb, amscd, amsthm}  % Dalsze wsparcie od AMS
% \usepackage{siunitx}  % Do prostszego pisania jednostek fizycznych
\usepackage{upgreek}  % Ładniejsze greckie litery
% Przykładowa składnia: pi = \uppi
% \usepackage{slashed}  % Pozwala w prosty sposób pisać slash Feynmana.
\usepackage{calrsfs}  % Zmienia czcionkę kaligraficzną w \mathcal
% na ładniejszą. Może w innych miejscach robi to samo, ale o tym nic
% nie wiem.





% ------------------------------
% Paczki, biblioteki i ich ustawienia dla tego pliku
% ------------------------------
\usepackage{tikz}  % Wspaniały pakiet PGF/TikZ







% ------------------------------
% Pakiet „hyperref”
% Polecano by umieszczać go na końcu preambuły
% ------------------------------
\usepackage{hyperref}  % Pozwala tworzyć hiperlinki i zamienia odwołania
% do bibliografii na hiperlinki










% ---------------------------------------------------------------------
% Tytuł, autor, data
\title{Ti\textit{k}Z \& PGF \\
  \href{http://piotrkosoft.net/pub/mirrors/CTAN/graphics/pgf/base/doc/pgfmanual.pdf}{Manual for version 3.1.9a} \\
  1~Introduction}
% \href{http://piotrkosoft.net/pub/mirrors/CTAN/graphics/pgf/base/doc/pgfmanual.pdf}{Ti\textit{k}Z \& PGF Manual, v. 3.1.8b} \\
%   14~Syntax for Path Specifications, part~I}

\author{}


% \date{}
% ---------------------------------------------------------------------







% ####################################################################
\begin{document}
% ####################################################################





% ######################################
\maketitle % Tytuł całego tekstu
% ######################################






% ##################
\begin{figure}[ht]

  \centering

  \begin{tikzpicture}



  \end{tikzpicture}

  \caption{Ti\textit{k}Z \& PGF Manula. Rysunek ze~strony tytułowej}

\end{figure}
% ##################





\newpage

\tikz \draw (0pt,0pt) -- (20pt,6pt); \hspace{2em}
\tikz \fill[orange] (1ex,1ex) circle [radius=1ex]; \hspace{2em}
\tikz \draw (0,0) -- (1,0) -- (1,1) -- cycle;





% % ##################
% \begin{figure}[ht]

%   \centering

%   \begin{tikzpicture}



%   \end{tikzpicture}

%   \caption{Ti\textit{k}Z \& PGF Manula. Introduction,~1}

% \end{figure}
% % ##################





% % ##################
% \begin{figure}[ht]

%   \centering

%   \begin{tikzpicture}



%   \end{tikzpicture}

%   \caption

% \end{figure}
% % ##################





% % ##################
% \begin{figure}[ht]

%   \centering

%   \begin{tikzpicture}



%   \end{tikzpicture}

%   \caption

% \end{figure}
% % ##################





% % ##################
% \begin{figure}[ht]

%   \centering

%   \begin{tikzpicture}



%   \end{tikzpicture}

%   \caption

% \end{figure}
% % ##################





% % ##################
% \begin{figure}[ht]

%   \centering

%   \begin{tikzpicture}



%   \end{tikzpicture}

%   \caption

% \end{figure}
% % ##################





% % ##################
% \begin{figure}[ht]

%   \centering

%   \begin{tikzpicture}



%   \end{tikzpicture}

%   \caption

% \end{figure}
% % ##################





% % ##################
% \begin{figure}[ht]

%   \centering

%   \begin{tikzpicture}



%   \end{tikzpicture}

%   \caption

% \end{figure}
% % ##################





% % ##################
% \begin{figure}[ht]

%   \centering

%   \begin{tikzpicture}



%   \end{tikzpicture}

%   \caption

% \end{figure}
% % ##################





% % ##################
% \begin{figure}[ht]
%   \centering

%   \begin{tikzpicture}



%   \end{tikzpicture}

%   \caption

% \end{figure}
% % ##################





% % ##################
% \begin{figure}[ht]

%   \centering

%   \begin{tikzpicture}



%   \end{tikzpicture}

%   \caption

% \end{figure}
% % ##################





% % ##################
% \begin{figure}[ht]

%   \centering

%   \begin{tikzpicture}



%   \end{tikzpicture}

%   \caption

% \end{figure}
% % ##################





% % ##################
% \begin{figure}[ht]

%   \centering

%   \begin{tikzpicture}



%   \end{tikzpicture}

%   \caption

% \end{figure}
% % ##################




30 rysunek ma być ostatnim w pliku.
% % ##################
% \begin{figure}[ht]

%   \centering

%   \begin{tikzpicture}



%   \end{tikzpicture}

%   \caption

% \end{figure}
% % ##################










% ############################

% Koniec dokumentu
\end{document}
