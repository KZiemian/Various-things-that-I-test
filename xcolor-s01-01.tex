% ####################################################################
% Author: Kamil Ziemian
% ####################################################################

% ---------------------------------------------------------------------
% Basic configuraton of LaTeX document and Polish language
% ---------------------------------------------------------------------
\RequirePackage[l2tabu, orthodox]{nag}  % Find out deprecated part of LaTeX.
% More information in l2tabu English version.


\documentclass[a4paper,11pt]{article}
% {rozmiar papieru, rozmiar fontu}[klasa dokumentu]
\usepackage[utf8]{inputenc} % Włączenie kodowania UTF-8, co daje dostęp
% do polskich znaków.
\usepackage{lmodern} % Wprowadza fonty Latin Modern.
\usepackage[T1]{fontenc} % Potrzebne do używania fontów Latin Modern.

\usepackage[MeX]{polski} % Polonizacja LaTeXa, bez niej będzie pracował
% w języku angielskim.










% ---------------------------------------
% MARGINS
% ---------------------------------------
\usepackage{vmargin}  % Pozwala na prostą kontrolę rozmiaru marginesów,
% za pomocą komend poniżej. Rozmiar odstępów jest mierzony w calach.
\setmarginsrb
{ 0.7in}  % left margin
{ 0.6in}  % top margin
{ 0.7in}  % right margin
{ 0.8in}  % bottom margin
{  20pt}  % head height
{0.25in}  % head sep
{   9pt}  % foot height
{ 0.3in}  % foot sep







% ---------------------------------------
% Basic packages
% ---------------------------------------







% ---------------------------------------
% Packages for scientific writing
% ---------------------------------------







% ---------------------------------------
% Packages written by us
% ---------------------------------------
% \usepackage{demopack}







% ---------------------------------------
% Configuration for this particular file
% ---------------------------------------
\usepackage{tikz}  % Great package PGF/TikZ.

\usepackage{xcolor}







% ---------------------------------------
% Some useful commands
% ---------------------------------------
\newcommand{\red}[1]{{\color{red} #1}}









% ---------------------------------------
% Packages "hyperref"
% They say that you should put it at the end of the preamble
% ---------------------------------------
\usepackage{hyperref}  % Pozwala tworzyć hiperlinki i zamienia odwołania
% do bibliografii na hiperlinki.










% ---------------------------------------------------------------------
% Tytuł, autor, data
\title{\href{https://repo.skni.umcs.pl/ctan/macros/latex/contrib/xcolor/xcolor.pdf}{\texttt{xcolor}} testy \\
  s. 01, 01}

\author{}



% \date{}
% ---------------------------------------------------------------------










% ####################################################################
\begin{document}
% ####################################################################





% ######################################
\maketitle % Tytuł całego tekstu
% ######################################





40\% \tikz \filldraw[fill=green] (0,0) rectangle (0.485,0.3);
+ 60\% \tikz \filldraw[fill=yellow] (0,0) rectangle (0.485,0.3);
= \tikz \filldraw[fill=green!40!yellow] (0,0) rectangle (0.485,0.3);

\tikz \filldraw[fill=-green!40!yellow] (0,0) rectangle (0.485,0.3);

% \color{}
% \tikz \filldraw[fill=red] (0,0) rectangle (0.485,0.3);

$3 \times$ \tikz \filldraw[fill=-green!40!yellow] (0,0) rectangle (0.485,0.3);
$+ 2 \times$ \tikz \filldraw[fill=green!40!yellow] (0,0) rectangle (0.485,0.3);
$+ 1 \times$ \tikz \filldraw[fill=red] (0,0) rectangle (0.485,0.3);
= \tikz \filldraw[fill={rgb:-green!40!yellow,3;green!40!yellow,2;red,1}]
(0,0) rectangle (0.485,0.3);

% \definecolor{}
Light with wave length $485$ nm has color: % \tikz \filldraw[fill={\color[wave]{485}}] (0,0) rectangle (0.485,0.3);

\textcolor[cmy]{0.7,0.5,0.3}{foo} \textcolor[HTML]{AFFE90}{bar}

\tikz \filldraw[fill=red] (0,0) rectangle (0.485,0.3);
\tikz \filldraw[fill=-red] (0,0) rectangle (0.485,0.3);
\tikz \filldraw[fill={--red!50!green!12.345}] (0,0) rectangle (0.485,0.3);
\tikz \filldraw[fill=red!50!green!12.345] (0,0) rectangle (0.485,0.3);
\tikz \filldraw[fill=-red!50!green!12.345] (0,0) rectangle (0.485,0.3);
\tikz \filldraw[fill=red!50!green!20!blue] (0,0) rectangle (0.485,0.3);
\tikz \filldraw[fill={rgb:red,1}] (0,0) rectangle (0.485,0.3);
\tikz \filldraw[fill={cmyk:red,1;-green!25!blue!60,11.25;blue,-2}]
(0,0) rectangle (0.485,0.3);
\tikz \filldraw[fill={rgb:red,4;green,2;yellow,1}]
(0,0) rectangle (0.485,0.3);
\tikz \filldraw[fill={rgb:red,4;green,2;yellow,-1}]
(0,0) rectangle (0.485,0.3);
\tikz \filldraw[fill={rgb:yellow,1}] (0,0) rectangle (0.485,0.3);
\tikz \filldraw[fill={rgb,9:red,4;green,2;yellow,1}]
(0,0) rectangle (0.485,0.3);

\tikz \filldraw[fill=red]       (0,0) rectangle (0.485,0.3);
\tikz \filldraw[fill=green]     (0,0) rectangle (0.485,0.3);
\tikz \filldraw[fill=blue]      (0,0) rectangle (0.485,0.3);
\tikz \filldraw[fill=cyan]      (0,0) rectangle (0.485,0.3);
\tikz \filldraw[fill=magenta]   (0,0) rectangle (0.485,0.3);
\tikz \filldraw[fill=yellow]    (0,0) rectangle (0.485,0.3);
\tikz \filldraw[fill=black,draw=red] (0,0) rectangle (0.485,0.3);
\tikz \filldraw[fill=gray]      (0,0) rectangle (0.485,0.3);
\tikz \filldraw[fill=white]     (0,0) rectangle (0.485,0.3);
\tikz \filldraw[fill=darkgray]  (0,0) rectangle (0.485,0.3);
\tikz \filldraw[fill=lightgray] (0,0) rectangle (0.485,0.3);
\tikz \filldraw[fill=brown]     (0,0) rectangle (0.485,0.3);
\tikz \filldraw[fill=lime]      (0,0) rectangle (0.485,0.3);
\tikz \filldraw[fill=olive]     (0,0) rectangle (0.485,0.3);
\tikz \filldraw[fill=orange]    (0,0) rectangle (0.485,0.3);
\tikz \filldraw[fill=pink]      (0,0) rectangle (0.485,0.3);
\tikz \filldraw[fill=purple]    (0,0) rectangle (0.485,0.3);
\tikz \filldraw[fill=teal]      (0,0) rectangle (0.485,0.3);
\tikz \filldraw[fill=violet]    (0,0) rectangle (0.485,0.3);

\definecolor{red1}{rgb}{1,0,0}
\tikz \filldraw[fill=red1] (0,0) rectangle (0.485,0.3);
\definecolor{red2}{rgb/cmyk}{1,0,0/0,1,1,0}
\tikz \filldraw[fill=red2] (0,0) rectangle (0.485,0.3);
% \definecolor{red3}{hsb:rgb/cmyk}{1,0,0/0,1,1,0}
% \tikz \filldraw[fill=red3] (0,0) rectangle (0.485,0.3);
% ERROR: Package pgf Error: Unsupported color model `hsb'. Sorry.
\definecolor[named]{Black1}{cmyk}{0,0,0,1}
\tikz \filldraw[fill=Black1,draw=red] (0,0) rectangle (0.485,0.3);
\definecolor{Black2}{named}{Black1}
\tikz \filldraw[fill=Black2,draw=red] (0,0) rectangle (0.485,0.3);
\colorlet{Black3}{Black1}
\tikz \filldraw[fill=Black3,draw=red] (0,0) rectangle (0.485,0.3);

\colorlet{tableheadcolor}{gray!25}
\tikz \filldraw[fill=tableheadcolor] (0,0) rectangle (0.485,0.3);

\definecolorset{rgb}{}{}{red3,1,0,0;green3,0,1,0;blue3,0,0,1}
\tikz \filldraw[fill=red3]   (0,0) rectangle (0.485,0.3);
\tikz \filldraw[fill=green3] (0,0) rectangle (0.485,0.3);
\tikz \filldraw[fill=blue3]  (0,0) rectangle (0.485,0.3);

\definecolorset{rgb}{x}{10}{red,1,0,0;green,0,1,0;blue,0,0,1}
\tikz \filldraw[fill=xred10]   (0,0) rectangle (0.485,0.3);
\tikz \filldraw[fill=xgreen10] (0,0) rectangle (0.485,0.3);
\tikz \filldraw[fill=xblue10]  (0,0) rectangle (0.485,0.3);

\begingroup

\definecolor{color1}{rgb}{0.5,0.6,0.2}

\tikz \filldraw[fill=color1] (0,0) rectangle (0.485,0.3);

\endgroup

% \tikz \filldraw[fill=color1] (0,0) rectangle (0.485,0.3);
% Nie działa, bo color jeden jest zdefiniowany w grupie, która już
% została zakończona.

\begingroup

\color{red} ABCDEFGH

\endgroup

\begingroup

\color[rgb]{1,0,0} ABCDEFGH

\endgroup

\textcolor{green}{IJKLMNO} \textcolor[rgb]{0,1,0}{IJKLMNO}

\colorbox{green}{PQRSTUV} \colorbox[rgb]{0,1,0}{PQRSTUV}

\fcolorbox{gray}{yellow}{test},
\fcolorbox[cmyk]{0,0,0,0.5}{0,0,1,0}{test},
\fcolorbox[gray]{0.5}[wave]{580}{test},
\fcolorbox{gray}[wave]{580}{test}

% !!!!!!!!!!!!!!!!!!!!!!!!!!!!!!
% Sprawdzić jaką mam wersję xcolor i napisać o tych błędach.
\fcolorbox{red}{green}{ABCDEFGH}
% \fcolorbox[rgb]{green}{blue}{ABCDEFGH}
% ERROR: Argument of \c@lor@@rgb has an extra }.
\fcolorbox[rgb]{1,0,0}{0,0,1}{ABCDEFGH}
% \fcolorbox[rgb]{green}[rgb]{blue}{ABCDEFGH}
% ERROR: Argument of \c@lor@@rgb has an extra }.
\fcolorbox[rgb]{0,1,0}{0,0,1}{ABCDEFGH}
% \fcolorbox{green}[rgb]{blue}{ABCDEFGH}
\fcolorbox{green}[rgb]{0,0,1}{ABCDEFGH}





























































































































% ####################################################################
% ####################################################################

% Koniec dokumentu
\end{document}