% ####################################################################
% Author: Kamil Ziemian
% ####################################################################

% ---------------------------------------------------------------------
% Basic configuraton of LaTeX document and Polish language
% ---------------------------------------------------------------------
\RequirePackage[l2tabu, orthodox]{nag}  % Find out deprecated part of LaTeX.
% More information in l2tabu English version.


\documentclass[a4paper,11pt]{article}
% {rozmiar papieru, rozmiar fontu}[klasa dokumentu]
\usepackage[utf8]{inputenc} % Włączenie kodowania UTF-8, co daje dostęp
% do polskich znaków.
\usepackage{lmodern} % Wprowadza fonty Latin Modern.
\usepackage[T1]{fontenc} % Potrzebne do używania fontów Latin Modern.

\usepackage[MeX]{polski} % Polonizacja LaTeXa, bez niej będzie pracował
% w języku angielskim.









% ---------------------------------------
% MARGINS
% ---------------------------------------
\usepackage{vmargin}  % Pozwala na prostą kontrolę rozmiaru marginesów,
% za pomocą komend poniżej. Rozmiar odstępów jest mierzony w calach.
\setmarginsrb
{ 0.7in}  % left margin
{ 0.6in}  % top margin
{ 0.7in}  % right margin
{ 0.8in}  % bottom margin
{  20pt}  % head height
{0.25in}  % head sep
{   9pt}  % foot height
{ 0.3in}  % foot sep







% ---------------------------------------
% Basic packages
% ---------------------------------------










% ---------------------------------------
% Packages written by us
% ---------------------------------------
% \usepackage{demopack}







% ---------------------------------------
% Packages for scientific writing
% ---------------------------------------







% ---------------------------------------
% Configuration for this particular file
% ---------------------------------------
\usepackage{tikz}  % Great package PGF/TikZ.

\usepackage{xcolor}







% ---------------------------------------
% Some useful commands
% ---------------------------------------
\newcommand{\red}[1]{{\color{red} #1}}









% ---------------------------------------
% Packages "hyperref"
% They say that you should put it at the end of the preamble
% ---------------------------------------
\usepackage{hyperref}  % Pozwala tworzyć hiperlinki i zamienia odwołania
% do bibliografii na hiperlinki.










% ---------------------------------------------------------------------
% Tytuł, autor, data
\title{\href{https://repo.skni.umcs.pl/ctan/macros/latex/contrib/xcolor/xcolor.pdf}{\texttt{xcolor}} testy \\
  s. 01, 01}

\author{}



% \date{}
% ---------------------------------------------------------------------










% ####################################################################
\begin{document}
% ####################################################################





% ######################################
\maketitle % Tytuł całego tekstu
% ######################################





40\% \tikz \filldraw[fill=green] (0,0) rectangle (0.485,0.3);
+ 60\% \tikz \filldraw[fill=yellow] (0,0) rectangle (0.485,0.3);
= \tikz \filldraw[fill=green!40!yellow] (0,0) rectangle (0.485,0.3);

\tikz \filldraw[fill=-green!40!yellow] (0,0) rectangle (0.485,0.3);

% \color{rgb:-green!40!yellow:3;green!40!yellow:2;red:1}
% \tikz \filldraw[fill=red] (0,0) rectangle (0.485,0.3);

$3 \times$ \tikz \filldraw[fill=-green!40!yellow] (0,0) rectangle (0.485,0.3);
$+ 2 \times$ \tikz \filldraw[fill=green!40!yellow] (0,0) rectangle (0.485,0.3);
$+ 1 \times$ \tikz \filldraw[fill=red] (0,0) rectangle (0.485,0.3);
= \red{????}

Light with wave length $485$ nm has color: \red{???}% \tikz \filldraw[fill=\color[wave]{485}] (0,0) rectangle (0.485,0.3);




























































% ####################################################################
% ####################################################################
% Koniec dokumentu
\end{document}