% ####################################################################
% Author: Kamil Ziemian
% ####################################################################

% ---------------------------------------------------------------------
% Podstawowe ustawienia i pakiety
% ---------------------------------------------------------------------
\RequirePackage[l2tabu, orthodox]{nag} % Wykrywa przestarzałe i niewłaściwe
% sposoby używania LaTeXa. Więcej jest w l2tabu English version.


\documentclass[a4paper,11pt]{article}
% {rozmiar papieru, rozmiar fontu}[klasa dokumentu]
\usepackage[utf8]{inputenc} % Włączenie kodowania UTF-8, co daje dostęp
% do polskich znaków.
\usepackage{lmodern} % Wprowadza fonty Latin Modern.
\usepackage[T1]{fontenc} % Potrzebne do używania fontów Latin Modern.

\usepackage[MeX]{polski} % Polonizacja LaTeXa, bez niej będzie pracował
% w języku angielskim.




% ------------------------------
% Podstawowe pakiety
% ------------------------------
% \usepackage{textcomp} % Dodaje takie symbole jak stopnie Celciusa,
% wprowadzane bezpośrednio w tekście.
% \usepackage{graphicx} % Wprowadza bardzo potrzebne komendy do wstawiania
% grafiki.
% \usepackage{verbatim} % Poprawia otoczenie VERBATIME.
% \usepackage[all]{xy} % Paczka do rysowania grafów. Opcja ALL jest
% polecana i włącza wszystkie opcje paczki. Należy go dołączyć po paczce
% POLSKI, bo obie definiują polecenie "\ar". W ten sposób "XY-pic"
% nadpisze "\ar" z pakietu POLSKI
% \usepackage{array} % Przedefiniuje otoczenie "tabular"
% \usepackage{booktabs} % Do robienia profesjonalnych tabeli
% \usepackage{multirow} % Do łączenia komórek z różnych wierszy
\usepackage{graphicx} % Podstawowy pakiet do wstawiania grafiki



% ----------------------------
% Zdefiniowana przez nas paczka
% ----------------------------
% \usepackage{demopack}



% ------------------------------
% Pakiety do tekstów naukowych
% ------------------------------
\let\lll\undefined % Amsmath gryzie się z językiem pakietami do języka
% polskiego, bo oba definiują komendę \lll. Aby rozwiązać ten problem
% oddefiniowuję tę komendę, ale może tym samym pozbywam się dużego Ł.
\usepackage[intlimits]{amsmath}
% Wsparcie od American Mathematical Society (w skrócie AMS)
\usepackage{amsfonts, amssymb, amscd, amsthm} % Dalsze wsparcie od AMS
\usepackage{siunitx} % Łatwiejsze używanie jednostek fizycznych
\usepackage{upgreek} % Ładniejsze greckie litery
% Przykładowa składnia: pi = \uppi
\usepackage{slashed} % Pozwala pisać w prosty sposób slash Feynmana.
\usepackage{calrsfs} % Zmienia czcionkę kaligraficzną w \mathcal
% na ładniejszą. Może w innych miejscach robi to samo, ale o tym nic
% nie wiem.
\let\qed\undefined
% \usepackage{mandi}  % Pakiet do symboli w elementarnej fizyce. Ładuje pakiet
% "tensor"

\usepackage[dvipsnames]{xcolor}  % Rekomendowana paczka do zmiany kolorów w LaTeXu

\definecolor{mypink1}{rgb}{0.858, 0.188, 0.478}
\definecolor{mypink2}{RGB}{219, 48, 122}
\definecolor{mypink3}{cmyk}{0, 0.7808, 0.4429, 0.1412}
\definecolor{mygray}{gray}{0.6}

\colorlet{LightRubineRed}{RubineRed!70}



% ---------------------------------------
% Pakiet "hyperref"
% Polecano by umieszczać go na końcu preambuły.
% ---------------------------------------
\usepackage{hyperref}  % Pozwala tworzyć hiperlinki i zamienia odwołania
% do bibliografii na hiperlinki.






% ---------------------------------------
% Configuration for this particular file
% ---------------------------------------










% ---------------------------------------------------------------------
% Tytuł, autor, data
\title{Using colours in \LaTeX{} (Overleaf)}

\author{}



% \date{}
% ---------------------------------------------------------------------










% ####################################################################
\begin{document}
% ####################################################################



% ######################################
\maketitle % Tytuł całego tekstu
% ######################################



% ######################################
\section{\href{https://www.overleaf.com/learn/latex/Using\_colours\_in\_LaTeX}{Using
    colours in \LaTeX}}

\vspace{2em}

% ######################################



This example shows different examples on how to use the
\texttt{xcolor} package to change the colour of elements in \LaTeX.

\begin{itemize}
  \color{blue}
\item First item

\item Second item

\end{itemize}

\noindent {\color{red} \rule{\linewidth}{0.5mm} }



{\color{white} white}, {\color{black} black}, {\color{yellow} yellow},
{\color{green} green}, {\color{blue} blue}, {\color{purple} purple},
{\color{cyan} cyan}, {\color{magenta} magenta}


\begin{itemize}
  \color{ForestGreen}
\item First item

\item Second item

\end{itemize}

\noindent {\color{RubineRed} \rule{\linewidth}{0.5mm}}


The~background colour~of some text can also be~\textcolor{red}{easily}
set. For instance, you can change to~orange the~background~of
\colorbox{BurntOrange}{this text} and then continue typing.


User-defined colours with different colour models:

\begin{enumerate}
\item \textcolor{mypink1}{Pink with rgb}

\item \textcolor{mypink2}{Pink with RGB}

\item \textcolor{mypink3}{Pink with cmyk}

\item \textcolor{mygray}{Gray with gray}
\end{enumerate}



% \pagecolor{black} \color{white}

This document present several examples on how to use
the~\texttt{xcolor} package to~change the~colour~of elements
in~\LaTeX.

\begin{itemize}
\item \textcolor{Mycolor1}{First item}

\item \textcolor{Mycolor2}{Second item}
\end{itemize}


\noindent {\color{LightRubineRed} \rule{\linewidth}{1mm} }

\noindent {\color{RubineRed} \rule{\linewidth}{1mm} }


\noindent {\color{Apricot} Apricot}, {\color{Aquamarine} Aquamarine},
{\color{Bittersweet} Bitersweet}, {\color{Black} Black}, {\color{Blue}
  Blue}, {\color{BlueGreen} BlueGreen}, {\color{BlueViolet}
  BlueViolet}, {\color{BrickRed} BrickRed}, {\color{Brown} Brown},
{\color{BurntOrange} BurntOrange}, {\color{CadetBlue} CadetBlue},
{\color{CarnationPink} CarnationPink}, {\color{Cerulean} Cerulean},
{\color{CornflowerBlue} CornflowerBlue}, {\color{Cyan} Cyan},
{\color{Dandelion} Dandelion}, {\color{DarkOrchid} DarkOrchid},
{\color{Emerald} Emerald}, {\color{ForestGreen} ForestGreen},
{\color{Fuchsia} Fuschia}, {\color{Goldenrod} Goldenrod},
{\color{Gray} Gray}, {\color{Green} Green}, {\color{GreenYellow}
  GreenYellow}, {\color{JungleGreen} JungleGreen}, {\color{Lavender}
  Lavender}, {\color{LimeGreen} LimeGreen}, {\color{Magenta} Magenta},
{\color{Mahogany} Mahogany}, {\color{Maroon} Maroon}, {\color{Melon}
  Melon}, {\color{MidnightBlue} MidnightBlue}, {\color{Mulberry}
  Mulberry}, {\color{NavyBlue} NavyBlue}, {\color{OliveGreen}
  OliveGreen}, {\color{Orange} Orange}, {\color{OrangeRed} OrangeRed},
{\color{Orchid} Orchid}, {\color{Peach} Peach}, {\color{Periwinkle}
  Periwinkle}, {\color{PineGreen} PineGreen}, {\color{Plum} Plum},
{\color{ProcessBlue} ProcessBlue}, {\color{Purple} Purple},
{\color{RawSienna} RawSienna}, {\color{Red} Red}, {\color{RedOrange}
  RedOrange}, {\color{RedViolet} RedViolet}, {\color{Rhodamine}
  Rhodaminie}, {\color{RoyalBlue} RoyalBlue}, {\color{RoyalPurple}
  RoyalPurple}, {\color{RubineRed} RubineRed}, {\color{Salmon}
  Salmon}, {\color{SeaGreen} SeaGreen}, {\color{Sepia} Sepia},
{\color{YellowOrange} YellowOrange}, {\color{SkyBlue} SkyBlue},
{\color{SpringGreen} SpringGreen}, {\color{Tan} Tan},
{\color{TealBlue} TealBlue}, {\color{Thistle} Thistle},
{\color{Turquoise} Turquoise}, {\color{Violet} Violet},
{\color{VioletRed} VioletRed}, {\color{White} White},
{\color{WildStrawberry} WildStrawberry}, {\color{Yellow} Yellow},
{\color{YellowGreen} YellowGreen}.






% ####################################################################
% ####################################################################
% Koniec dokumentu
\end{document}