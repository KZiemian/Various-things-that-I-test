% Autor: Kamil Ziemian

% ---------------------------------------------------------------------
% Podstawowe ustawienia i pakiety
% ---------------------------------------------------------------------
\RequirePackage[l2tabu, orthodox]{nag} % Wykrywa przestarzałe i niewłaściwe
% sposoby używania LaTeXa. Więcej jest w l2tabu English version.
\documentclass[a4paper,11pt]{article}
% {rozmiar papieru, rozmiar fontu}[klasa dokumentu]
\usepackage[MeX]{polski} % Polonizacja LaTeXa, bez niej będzie pracował
% w języku angielskim.
\usepackage[utf8]{inputenc} % Włączenie kodowania UTF-8, co daje dostęp
% do polskich znaków.
\usepackage{lmodern} % Wprowadza fonty Latin Modern.
\usepackage[T1]{fontenc} % Potrzebne do używania fontów Latin Modern.



% ---------------------------------------
% Podstawowe pakiety (niezwiązane z ustawieniami języka)
% ---------------------------------------
\usepackage{microtype} % Twierdzi, że poprawi rozmiar odstępów w tekście.
% \usepackage{graphicx} % Wprowadza bardzo potrzebne komendy do wstawiania
% grafiki.
% \usepackage{verbatim} % Poprawia otoczenie VERBATIME.
% \usepackage{textcomp} % Dodaje takie symbole jak stopnie Celsiusa,
% wprowadzane bezpośrednio w tekście.
\usepackage{vmargin} % Pozwala na prostą kontrolę rozmiaru marginesów,
% za pomocą komend poniżej. Rozmiar odstępów jest mierzony w calach.
% ---------------------------------------
% MARGINS
% ---------------------------------------
\setmarginsrb
{ 0.7in}  % left margin
{ 0.6in}  % top margin
{ 0.7in}  % right margin
{ 0.8in}  % bottom margin
{  20pt}  % head height
{0.25in}  % head sep
{   9pt}  % foot height
{ 0.3in}  % foot sep



% ---------------------------------------
% Często używane pakiety
% ---------------------------------------
% \usepackage{csquotes} % Pozwala w prosty sposób wstawiać cytaty do tekstu.



% ---------------------------------------
% Pakiety do tekstów z nauk przyrodniczych
% ---------------------------------------
\let\lll\undefined % Amsmath gryzie się z językiem pakietami do języka
% % polskiego, bo oba definiują komendę \lll. Aby rozwiązać ten problem
% % oddefiniowuję tę komendę, ale może tym samym pozbywam się dużego Ł.
\usepackage[intlimits]{amsmath} % Podstawowe wsparcie od American
% Mathematical Society (w skrócie AMS)
\usepackage{amsfonts, amssymb, amscd, amsthm} % Dalsze wsparcie od AMS
% % \usepackage{siunitx} % Dla prostszego pisania jednostek fizycznych
\usepackage{upgreek} % Ładniejsze greckie litery
% Przykładowa składnia: pi = \uppi
% \usepackage{slashed} % Pozwala w prosty sposób pisać slash Feynmana.
% \usepackage{calrsfs} % Zmienia czcionkę kaligraficzną w \mathcal
% % na ładniejszą. Może w innych miejscach robi to samo, ale o tym nic
% % nie wiem.





% ---------------------------------------
% Dodatkowe ustawienia dla języka polskiego
% ---------------------------------------
\renewcommand{\thesection}{\arabic{section}.}
% Kropki po numerach rozdziału (polski zwyczaj topograficzny)
\renewcommand{\thesubsection}{\thesection\arabic{subsection}}
% Brak kropki po numerach podrozdziału



% ---------------------------------------
% Pakiety napisane przez użytkownika.
% Mają być w tym samym katalogu to ten plik .tex
% ---------------------------------------
% \usepackage{latexshortcuts}
\usepackage{mathcommands}





% ---------------------------------------
% Pakiet "hyperref"
% Polecano by umieszczać go na końcu preambuły.
% ---------------------------------------
\usepackage{hyperref} % Pozwala tworzyć hiperlinki i zamienia odwołania
% do bibliografii na hiperlinki.










% ---------------------------------------------------------------------
% Tytuł, autor, data
\title{Zadania do zrobienia, matematyka i~informatyka}

% \author{}
% \date{}
% ---------------------------------------------------------------------










% ####################################################################
% Początek dokumentu
\begin{document}
% ####################################################################





% ######################################
\maketitle  % Tytuł całego tekstu
% ######################################





% ######################################
\section{Zrób te zadania}

% \vspace{\spaceTwo}

% ######################################










% ######################################
\section{Algebra symboliczna, zrób te zadania}

% \vspace{\spaceTwo}

% ######################################



% ##################
\begin{enumerate}

\item Obliczyć całki nieoznaczone podanych wyrażeń, narysuj ich
  wykresy oraz sprawdź otrzymane wyniki, poprzez różniczkowanie.
  \begin{align}
    &\int \frac{ 1 }{ x } \, dx \\
    &\int ( xt + 2x^{ 2 } )^{ n } \, dt \\
    &\int \frac{ 1 }{ 1 + \sqrt{ x } } \, dx \\
    &\int \frac{ 1 }{ \sin x } \, dx \\
    &\int \frac{ 1 }{ \sqrt{ a^{ 4 } \pm x^{ 4 } } } \, dx, \quad
      a > 0
  \end{align}

\item Znajdź rozwiązanie równania różniczkowego
  \begin{equation}
    \label{eq:6}
    \frac{ d f( x ) }{ dx } = g( x )
  \end{equation}
  dla następujących przypadków
  \begin{align}
    g( x ) &= x^{ 2 } + x + 1, \quad
             f( 0 ) = 0 \\
    g( x ) &= e^{ -\sqrt{ x } }, \quad
             f( 0 ) = -2 \\
    g( x ) &= \frac{ 1 }{ x }, \quad
             f( -1 ) = -1
  \end{align}
  \textbf{Wskazówka.} Oblicz całkę i wyznacz stałą całkowania, lub/i
  wypróbuj \texttt{DSolve}.


\item Wykazać, że podane wyrażenia
  \begin{align}
    &\tan^{ -1 }\left( \frac{ x^{ 3 } }{ 3 } - \frac{ x^{ 2 } }{ 2 }
      - x + \frac{ 5 }{ 3 } \right)
      - \tan^{ -1 }( 1 - x ) \\
    &\frac{ 1 }{ 2 } \tan^{ -1 }\left( \frac{ x + 1 }{ 4 - x^{ 2 } } \right)
      - \frac{ 1 }{ 2 } \tan^{ -1 }\left( \frac{ x + 1 }{ x^{ 2 } - 4 }
      \right)
  \end{align}
  po zróżniczkowaniu dają taki sam wynik, tzn. obydwa są całką
  nieoznaczoną tej samej funkcji. Wyjaśnić powstały problem.

\item Na podane niże cztery sposoby oblicz całkę nieoznaczoną z~trzech
  funkcji $f$
  \begin{align}
    f( x ) &= | x | \\
    f( x ) &= \frac{ x^{ 2 } + 2x + 4 }{ x^{ 4 } - 7 x^{ 2 } + 2x + 17 } \\
    f( x ) &= \frac{ 1 }{ \sqrt{ a^{ 2 } + x^{ 2 } } }
  \end{align}
  Wyniki porównać odejmując graficznie ich wykresy.

  Cztery sposoby obliczania całki
  \begin{align}
    &\int f( t ) \, dt \\
    &\int_{ 0 }^{ t } f( x ) \, dx \\
    &\int_{ 0 }^{ t_{ n } } f( t ) \, dt, \quad
      t_{ n } = 0, 1, 2, 3, \ldots \\
    &\int_{ 0 }^{ t } f( x ) \, dx \simeq \sum_{ i = 1 }^{ N } f( x_{ i } ) \, dx_{ i },
      \qquad
      x_{ i } = \frac{ t }{ N } i,\; dx_{ i } = \frac{ t }{ N },
  \end{align}
  gdzie $N = 10, 100, 1000, \ldots$

\item Oblicz całki oznaczone
  \begin{align}
    &\int_{ 0 }^{ \pi } \sin^{ k } x \, dx, \quad
      k = 0, 1, 2, 3, 4, \ldots \\
    &\int_{ -\infty }^{ +\infty } e^{ -\frac{ x^{ 2 } }{ a^{ 2 } } } \, dx \\
    &\int_{ 0 }^{ 4 } \int_{ \frac{ y }{ 2 } }^{ \sqrt{ 2 } } x y^{ 2 } \, dx dy \\
    &\int_{ 0 }^{ 2\pi } \int_{ 0 }^{ \alpha } \int_{ 0 }^{ \frac{ 1 }{ \cos \theta } } \cos\theta \sin\theta \,
      dr d\theta d\phi
  \end{align}

\item Oblicz objętość kuli w $2, 3, 4$ i~$n$ wymiarach.

\end{enumerate}
% ##################










% ######################################
\section{Algebra matematyczna, zrób te zadania}

% \vspace{\spaceTwo}

% ######################################



% ##################
\begin{enumerate}

\item Obliczyć całkę
  \begin{equation}
    \label{Calculus:01}
    \int_{ 0 }^{ 1 } \int_{ 0 }^{ 1 } \frac{ ( x y )^{ s } }{ \sqrt{-\log( x y ) } }
      \; dx \, dy, \quad
      s > 0.
  \end{equation}



\end{enumerate}
% ##################










% ######################################
\section{Geometria różniczkowa, zrób te zadania}

% \vspace{\spaceTwo}

% ######################################



% ##################
\begin{enumerate}
\item Wyznaczyć parametr naturalny krzywej
  \begin{equation}
    \label{eq:1}
    \gamma( u ) = \left[ e^{ -u } \cos u, e^{ -u } \sin u, 1 - e^{ -u } \right]
  \end{equation}

\item Dana jest krzywa
  \begin{equation}
    \label{eq:2}
    \delta( t )
    =
    \big[ a ( t - \sin t ), a ( 1 - \cos t ), 4a \sin \frac{ t }{ 2 } \big]
  \end{equation}
  Obliczyć krzywiznę i~skręcenie krzywej
  \begin{equation}
    \label{eq:3}
    \gamma( t )
    =
    \delta( t )
    + a \vecnbold \sqrt{ 1 + \left( \sin \frac{ t }{ 2 } \right)^{ 2 } }
  \end{equation}

\item Wykazać, że dla krzywej $\alpha( s )$ klasy $\Ccal^{ 4 }$ zachodzi
  tożsamość
  \begin{equation}
    \label{eq:4}
    \left( \frac{ d^{ 3 } \alpha }{ d s^{ 3 } } \right)^{ 2 }
    =
    \kappa^{ 4 } + \kappa^{ 2 } \tau^{ 2 } + \left( \frac{ d \kappa }{ d s } \right)^{ 2 }
  \end{equation}

\item Znaleźć repery Freneta $\vectbold$, $\vecnbold$, $\vecbbold$ w
  punkcie $P = (1, 1, 1)$ krzywej powstałej z~przecięcia powierzchni
  \begin{equation}
    \label{eq:5}
    z^{ 2 } = xy, \quad
    x^{ 2 } + y^{ 2 } - z^{ 2 } = 1
  \end{equation}

\item The topologica spaces $X$, $Y$ are homotopic to each other if
  there exist maps $f : X \to Y$ and $f^{ -1 } : Y \to X$ such that
  $f \circ f^{ -1 }$ and $f^{ -1 } \circ f$ are homotopic,
  respectively, to identity on $X$ and $Y$. Show that this is an
  equivalence relation.

\item Show that $\Rbb^{ n }$ is homotopic to a point. Use the result
  to calculate the singular homology groups of $\Rbb^{ n }$.

\item Show that a $n$-dimensional ball
  $\{ x \in \Rbb^{ n }, \abs{ x } < 1 \}$ is homotopic to $\Rbb^{ n }$
  and use the previous result to determine its singualar homology
  (using homotopy equivalence).

\item Consider $S^{ 1 }$ and its open cover by three open sets
  $U_{ 1 }$, $U_{ 2 }$, $U_{ 3 }$:
  \begin{equation}
    \label{eq:7}
    U_{ k } =
    \left\{ z \in S^{ 1 } : \frac{ k - 1 }{ 3 } 2\pi - \frac{ \pi }{ 2 }
      \arg( z ) < \frac{ k - 1 }{ 3 } 2\pi + \frac{ \pi }{ 2 } \right\}
  \end{equation}
  and consider \v{C}ech complex and its cohomology relative to this
  cover.

\item Let $U_{ i }$ be an open cover of a compact manifold $M$ and
  $g_{ i j } : U_{ i } \cap U_{ j } = U_{ i j } \to U( 1 )$ be a
  collection of maps from their intersections. Show the condition
  which the maps need to satisfy in order to define a line bundle or a
  $U( 1 )$ principle fibre bundle. When are these bundles trivial?

\item Consider an open cover $U_{ i }$ of a compact manifold $M$ and a
  collection of one-forms $A_{ i }$ over the open sets. Please find a
  necessary condition for the forms over the intersection in order for
  the curvarture $dA_{ i }$ to be a well-defined global 2-form.

\item Consider an open cover $U_{ i }$ of a compact manifold $M$ and a
  collection of maps $g_{ ij }$ from
  $U_{ ij } = U_{ i } \cap U_{ j } \to U( 1 )$. Find a compatibility
  condition so that over a triple intersection their composition is
  not trivial but is a \v{C}ech cocycle $\lambda_{ i j k }$. Would
  this work for a nonabelian (valued in $\GL( n )$) transition
  functions?

\item Consider an open cover $U_{ i }$. Let $\lambda_{ i j k }$ be the
  \v{C}ech cocyle, and let $A_{ i j }$ be a collections of one forms
  that satisfy:
  \begin{equation}
    \label{eq:8}
    A_{ i j } + A_{ j k } + A_{ k i } = d\lambda_{ i j k },
  \end{equation}
  and $F_{ i }$ collection of two-forms such that
  $F_{ i } = F_{ j } + dA_{ i j }$. In this construction correct? What
  can one say about $dF$?

\item Let $P( M, G )$ be a principal fibre bundle and $\rho$ a
  finite-dimensional representation of $G$ on a vector space $V$. Show
  that $P \times V / \sim$ is a vector bundle over $M$. Relation
  $\sim$ is defined by
  \begin{equation}
    \label{eq:9}
    ( p, v ) \sim \left( pg, \rho\left( g^{ -1 } \right) v \right).
  \end{equation}

\item Let $E( M )$ be a vector bundle over $M$ and $\nabla$ be a
  connection. Show that, if $U_{ i }$ is an open cover which locally
  trivializes $E$ and $g_{ i j }$ are transition functions on the
  intersections then the connection is determined over each $U_{ i }$
  by a one form $A_{ i }$. Show the relation between $A_{ i }$ and
  $A_{ j }$ over the intersection.

\item When a global two-form over a manifold $M$ is a curvature form
  of a connection on a line bundle.

\item If $p( x )$ is a projection over $m$ (seen as matrix acting on
  $\Cbb^{ n }$) then constuct a connection over the vector bundle
  $p( M \times \Cbb^{ n } )$.

\item For the sphere parametrized by $z \in \Cbb$ and the line bundle
  defined through the Bott projector
  \begin{equation}
    \label{eq:10}
    p( z ) =
    \frac{ 1 }{ 1 + \abs{ z }^{ 2 } }
    \begin{pmatrix}
      1 & \bar{z} \\
      z & \abs{ z }^{ 2 }
    \end{pmatrix},
  \end{equation}
  take a connection and compute its curvature and the Chern number.

\item Please verify whether Chern classes for connections over trivial
  vector bundles vanish.

\item Show that skyscraper sheaf is a sheaf.

\item Consider sheaf $X = S^{ 1 }$, $F_{ 0 } = \Zbb$.

  \textbf{a)} Take open cover
  $\Ucal_{ 1 } = S^{ 1 } \setminus \{ -1 \}$,
  $\Ucal_{ 2 } = S^{ 1 } \setminus \{ 1 \}$ and construct explicit
  cochain complex. Try to compute its cohomology.

  \textbf{b)} Try do previous exercise for try sets cover. What will
  the \v{C}ech complex be?

\item Let $G \hookrightarrow H$ ($V \subset W$) be abelian group (vector spaces). Is
  sequence $0 \to G \to H \to G / H \to 0$ ($0 \to V \to W / V \to 0$) exact.

\item Compute $H_{ \dR }^{ * }( \Rbb^{ n } )$.

\item Show that for singular homology $d_{ n - 1 } \circ d_{ n } \equiv 0$.

\item Consider bundle $P( M, F )$ with sets of diffeomorphism
  $\phi_{ \alpha } : \Ucal_{ \alpha } \times F \to \pi^{ -1 }( \Ucal_{
    \alpha } )$. Consider two situation: $F = V$, $G = \GL( V )$ and
  $F = G$. Why also notice that
  $\pi^{ -1 }( \Ucal_{ \alpha } ) \ni p = \phi_{ \alpha }( \pi( p ), g )$.

  \textbf{a)} How define right action on this bundle?

  \textbf{b)} Is this action free and transition?

\item Suppose that we have cover and associated local trivialization
  of $P( M, G )$.

  \textbf{a)} What would be local trivialization of $P( M, F )_{ G }$?

  \textbf{b)} Do we need to assume something of $F$ and action
  $G( \rho )$ on $F$?

  \textbf{c)} Do $F = \{ a \}$ or $F = \{ a, b \}$ with some nontrivial
  action of the group, are allowed?

\item (Treść zadania może być błędna.) We take two bundles
  $P( M, G ) \leadsto E( M, V )$ with connections. Find a relation of
  holonomy $\int_{ \gamma } ( A_{ \alpha } )_{ \mu } \, dx^{ \mu }$ of
  this two bundles.

\item Show that pullback of vector bundle is a vector bundle.

\item Is there example (easy to find) of some $\Rbb$-vector bundle
  with is not trivial but stable trivial?

\item Show that sets of stable vector bundles over point
  $\Vect_{ \Cbb } = \{ V = \Cbb^{ n }, n \geq 0 \}$ is equal to $\Nbb$.

\item Show that suspension of the circle is a sphere
  $s( S^{ 1 } ) = S^{ 2 }$.

\item In definition of the twisted vector bundle we have
  $g_{ kj } g_{ ji } = g_{ ki } \lambda_{ kji }$. What should we require of
  $\lambda_{ kji }$. Is there some cocycle condition for $\lambda_{ kji }$?

\item Show that for a bundle $E \to M$ with fibre $V = \Cbb^{ n }$
  relation below hold:
  \begin{equation}
    \label{eq:11}
    \ch( E ) =
    V + c_{ 1 }( E )
    + \frac{ 1 }{ 2 } ( c_{ 1 }( E )^{ 2 } - 2 c_{ 2 }( E ) )
    + \frac{ 1 }{ 6 } \left( c_{ 1 }( E )^{ 3 } - 3 c_{ 1 }( E ) c_{ 2 }( E )
      + 3 c_{ 3 }( E ) \right) + \ldots
  \end{equation}

\item Show that Bott projector
  $p = \frac{ 1 }{ 1 + \lvert z \rvert^{ 2 } }
  \begin{pmatrix}
    1 & \bar{z} \\
    z & \lvert z \lvert^{ 2 }
  \end{pmatrix}$ is really a projector.

\item (Niejasny tekst.) We define $A = ( 1, \eta )$,
  $a = \alpha 1 + \beta \eta$, $\alpha, \beta \in \Kbb$. Show that
  $A^{ \opp }$, $\eta^{ 2 } = -1$.

\item What would be natural definition of linear functional on super
  vector space?

\item If we take set of morphism between super vector spaces do they
  form a superalgebra

\item How define trace on superalgebra?

\item Show that $\Rbb^{ 1 / 1 }$ form a supergroup.

\item (Rówanania pewnie są błędne.) Using equations
  \begin{align}
    \partial_{ t } \left( f_{ I } \theta^{ I } \right)
    &= \left( \partial_{ t_{ i } } f_{ I } \theta^{ I } \right), \\
    \partial_{ \theta_{ i } } \left( f_{ I } \theta^{ I }
    + f_{ J_{ i } } \theta^{ i } \theta^{ J } \right)
    &= \sum f_{ J_{ i } } \theta^{ J }, \quad
      i \in J
  \end{align}
  construct all derivative. Are they form module over $\Rbb^{ p / q }$?

  \item (Coś tu jest nie tak.) Rewrite Poincar\'{e}-Birkhoff-Witt theorem super Lie algebra.

  % \item

  % \item

  % \item

  % \item

  % \item

  % \item

  % \item

  % \item

  % \item

  % \item

  % \item

  % \item

  % \item

  % \item

  % \item

  % \item

  % \item

  % \item

  % \item

  % \item

  % \item

  % \item

  % \item

  % \item

  % \item

  % \item

  % \item

  % \item

  % \item

  % \item

  % \item

  % \item

  % \item

  % \item

  % \item

  % \item

  % \item

  % \item

  % \item

  % \item

  % \item

  % \item

  % \item

  % \item

  % \item

  % \item

  % \item

  % \item

  % \item

  % \item

  % \item

  % \item

  % \item

  % \item

  % \item

  % \item

  % \item

  % \item

  % \item

  % \item

  % \item

  % \item

  % \item

  % \item

  % \item

  % \item

  % \item

  % \item

  % \item

  % \item

  % \item

  % \item

  % \item

  % \item

  % \item

  % \item

  % \item

  % \item

  % \item

  % \item

  % \item

  % \item

  % \item

  % \item

  % \item

  % \item

  % \item

  % \item

  % \item

  % \item

  % \item

  % \item

  % \item

  % \item

  % \item

  % \item

  % \item

  % \item

  % \item

  % \item

  % \item

  % \item

  % \item

  % \item

  % \item

  % \item

  % \item

  % \item

  % \item

  % \item

  % \item

  % \item

  % \item

  % \item

  % \item

  % \item

  % \item

  % \item

  % \item

  % \item

  % \item

  % \item

  % \item

  % \item

  % \item

  % \item

  % \item

  % \item

  % \item

  % \item

  % \item

  % \item

  % \item

  % \item

  % \item

  % \item

  % \item

  % \item

  % \item

  % \item

  % \item

  % \item

  % \item

  % \item

  % \item

  % \item

  % \item

  % \item

  % \item

  % \item

  % \item

  % \item

  % \item

  % \item

  % \item

  % \item

  % \item

  % \item

  % \item

  % \item

  % \item

  % \item

  % \item

  % \item

  % \item

  % \item

  % \item

  % \item

  % \item

  % \item

  % \item

  % \item

  % \item

  % \item

  % \item

  % \item

  % \item

  % \item

  % \item

  % \item

  % \item

  % \item

  % \item

  % \item

  % \item

  % \item

  % \item

  % \item

  % \item

  % \item

  % \item

  % \item

  % \item

  % \item

  % \item

  % \item

  % \item

  % \item

  % \item

  % \item

  % \item

  % \item

  % \item

  % \item

  % \item

  % \item

  % \item

  % \item

  % \item

  % \item

  % \item

  % \item

  % \item

  % \item

  % \item

  % \item

  % \item


  % \item

  % \item

  % \item

  % \item

  % \item

  % \item

  % \item

  % \item

  % \item

  % \item

  % \item

  % \item

  % \item

  % \item

  % \item

  % \item

  % \item

  % \item

  % \item

  % \item

  % \item

  % \item

  % \item

  % \item

  % \item

  % \item

  % \item

  % \item

  % \item

  % \item

  % \item

  % \item

  % \item

  % \item

  % \item

  % \item

  % \item

  % \item

  % \item

  % \item

  % \item

  % \item

  % \item

  % \item

  % \item

  % \item

  % \item

  % \item

  % \item

  % \item

  % \item

  % \item

  % \item

  % \item

  % \item

  % \item

  % \item

  % \item

  % \item

  % \item

  % \item

  % \item

  % \item

  % \item

  % \item

  % \item

  % \item

  % \item

  % \item

  % \item

  % \item

  % \item

  % \item

  % \item

  % \item

  % \item

  % \item

  % \item

  % \item

  % \item

  % \item

  % \item

  % \item

  % \item

  % \item

  % \item

  % \item

  % \item

  % \item

  % \item

  % \item

  % \item

  % \item

  % \item

  % \item

  % \item

  % \item

  % \item

  % \item

  % \item

  % \item

  % \item

  % \item

  % \item

  % \item

  % \item

  % \item

  % \item

  % \item

  % \item

  % \item

  % \item

  % \item

  % \item

  % \item

  % \item

  % \item

  % \item

  % \item

  % \item

  % \item

  % \item

  % \item

  % \item

  % \item

  % \item

  % \item

  % \item

  % \item

  % \item

  % \item

  % \item

  % \item

  % \item

  % \item

  % \item

  % \item

  % \item

  % \item

  % \item

  % \item

  % \item

  % \item

  % \item

  % \item

  % \item

  % \item

  % \item

  % \item

  % \item

  % \item

  % \item

  % \item

  % \item

  % \item

  % \item

  % \item

  % \item

  % \item

  % \item

  % \item

  % \item

  % \item

  % \item

  % \item

  % \item

  % \item

  % \item

  % \item

  % \item

  % \item

  % \item

  % \item

  % \item

  % \item

  % \item

  % \item

  % \item

  % \item

  % \item

  % \item

  % \item

  % \item

  % \item

  % \item

  % \item

  % \item

  % \item

  % \item

  % \item

  % \item

  % \item

  % \item

  % \item

  % \item

  % \item

  % \item

  % \item

  % \item

  % \item

  % \item

  % \item

  % \item

  % \item

  % \item

  % \item

  % \item

  % \item

  % \item

  % \item

  % \item

  % \item

  % \item

  % \item

  % \item

  % \item

  % \item

  % \item

  % \item

  % \item

  % \item

  % \item

  % \item

  % \item

  % \item

  % \item

  % \item

  % \item

  % \item

  % \item

  % \item

  % \item

  % \item

  % \item

  % \item

  % \item

  % \item

  % \item

  % \item

  % \item

  % \item

  % \item

  % \item

  % \item

  % \item

  % \item

  % \item

  % \item

  % \item

  % \item

  % \item

  % \item

  % \item

  % \item

  % \item

  % \item

  % \item

  % \item

  % \item

  % \item

  % \item

  % \item

  % \item

  % \item

  % \item

  % \item

  % \item

  % \item

  % \item

  % \item

  % \item

  % \item

  % \item

  % \item

  % \item

  % \item

  % \item

  % \item

  % \item

  % \item

  % \item

  % \item

  % \item

  % \item

  % \item

  % \item

  % \item

  % \item

  % \item

  % \item

  % \item

  % \item

  % \item

  % \item

  % \item

  % \item

  % \item

  % \item

  % \item

  % \item

  % \item

  % \item

  % \item

  % \item

  % \item

  % \item

  % \item

  % \item

  % \item

  % \item

  % \item

  % \item

  % \item

  % \item












































\end{enumerate}
% ##################









% % ######################################
% \newpage
% \section{Zaczęte i~nieskończone}

% \vspace{\spaceTwo}
% % ######################################














% % ######################################
% \newpage
% \section{Articles}

% \vspace{\spaceTwo}
% % ######################################



% \begin{enumerate}

% \item Edward Witten, \emph{Notes on Some Entanglement Properties of
%   Quantum Field Theory},
%   \href{https://arxiv.org/abs/1803.04993}{arXiv:1803.04993};

% \item Jeff Bezanson et al, "Julia: dynamism and performance
%   reconciled by design"
%   \href{https://doi.org/10.1145/3276490}{https://doi.org/10.1145/3276490};

% \item Francesco Zappa Nardelli et al., "Julia subtyping: a rational
%   reconstruction",
%   \href{https://doi.org/10.1145/3276914}{https://doi.org/10.1145/3276914};

% \item Artem Pelenitsyn et al., "Type Stability in Julia: Avoiding
%   Performance Pathologies in JIT Compilation",
%   https://doi.org/10.1145/3485527,
%   \href{arXiv:2109.01950}{https://arxiv.org/abs/2109.01950};

% \item \emph{How SQLite Is Tested},
%   \href{https://www.sqlite.org/testing.html}{https://www.sqlite.org/testing.html};

% \item \emph{SpotBugs},
%   \href{https://spotbugs.github.io/}{https://spotbugs.github.io/};

% \item \emph{A tool to detect bugs in Java and C/C++/Objective-C code
%   before it ships},
%   \href{https://fbinfer.com/}{https://fbinfer.com/la};

% \item \emph{Go 1 and the Future of Go Programs},
%   \href{https://go.dev/doc/go1compat}{https://go.dev/doc/go1compat};

% \item \emph{SD-8: Standard Library Compatibility},
%   \href{https://isocpp.org/std/standing-documents/sd-8-standard-library-compatibility}{https://isocpp.org/std/standing-documents/sd-8-standard-library-compatibility};

% \item \emph{GNU General Public License},
%   \href{https://www.gnu.org/licenses/gpl-3.0.html}{https://www.gnu.org/licenses/gpl-3.0.html};

% \item \emph{Reflections on trusting trust},
%   \href{https://dl.acm.org/doi/10.1145/358198.358210}{https://dl.acm.org/doi/10.1145/358198.358210};

% \item \emph{Go \& Versioning},
%   \href{https://research.swtch.com/vgo}{https://research.swtch.com/vgo};

% \item \emph{Why Google stores billions of lines of code in a single
%   repository},
%   \href{https://dl.acm.org/doi/10.1145/2854146}{https://dl.acm.org/doi/10.1145/2854146};

% \item \emph{Testing Chromium: ThreadSanitizer v2, a next-gen data
%   race
%   detector}, \\
%   \href{https://blog.chromium.org/2014/04/testing-chromium-threadsanitizer-v2.html}{https://blog.chromium.org/2014/04/testing-chromium-threadsanitizer-v2.html};

% \item \emph{Search Vulnerability Database},
%   \href{https://nvd.nist.gov/vuln/search}{https://nvd.nist.gov/vuln/search};

% \item \emph{Regular Expression Matching with a Trigram Index or How
%   Google Code Search Worked}, \\
%   \href{https://swtch.com/~rsc/regexp/regexp4.html}{https://swtch.com/~rsc/regexp/regexp4.html};

% \item \emph{Licenses},
%   \href{https://opensource.google/docs/thirdparty/licenses}{https://opensource.google/docs/thirdparty/licenses};

% \item \emph{ImperialViolet},
%   \href{https://www.imperialviolet.org/2009/08/26/seccomp.html}{https://www.imperialviolet.org/2009/08/26/seccomp.html};

% \item \emph{Multi-process Architecture},
%   \href{https://blog.chromium.org/2008/09/multi-process-architecture.html}{https://blog.chromium.org/2008/09/multi-process-architecture.html};

% \item \emph{A single Node of failure},
%   \href{https://lwn.net/Articles/681410/}{https://lwn.net/Articles/681410/};

% \item \emph{Interpreting the Data: Parallel Analysis with Sawzall},
%   \href{https://www.hindawi.com/journals/sp/2005/962135/}{https://www.hindawi.com/journals/sp/2005/962135/};

% \item \emph{Go Proverbs},
%   \href{https://go-proverbs.github.io/}{https://go-proverbs.github.io/};

% \item \emph{RE2: a principled approach to regular expression
%   matching}, \\
%   \href{https://opensource.googleblog.com/2010/03/re2-principled-approach-to-regular.html}{https://opensource.googleblog.com/2010/03/re2-principled-approach-to-regular.html};

% \item \emph{Details about the event-stream incident}, \\
%   \href{https://blog.npmjs.org/post/180565383195/details-about-the-event-stream-incident}{https://blog.npmjs.org/post/180565383195/details-about-the-event-stream-incident};

% \item \emph{Open-sourcing gVisor, a sandboxed container runtime}, \\
%   \href{https://cloud.google.com/blog/products/identity-security/open-sourcing-gvisor-a-sandboxed-container-runtime}{https://cloud.google.com/blog/products/identity-security/open-sourcing-gvisor-a-sandboxed-container-runtime};

% \item Samuel R. Buss, Alexander S. Kechris, Anand Pillay, Richard A.
%   Shore, \emph{The prospects for mathematical logic in the
%   twenty-first century},
%   \href{https://arxiv.org/abs/cs/0205003v1}{arXiv:cs/0205003v1};

% \item Christian Retore, \emph{On the system F as a glue language for
%   natural-language compositional-semantics},
%   \href{https://arxiv.org/abs/1108.5084}{arXiv:1108.5084};

% \item Robert Harper, \emph{An Equational Logical Framework for Type
%   Theories},
%   \href{https://arxiv.org/abs/2106.01484}{arXiv:2106.01484};

% \item Jan Leike, et al., \emph{AI Safety Gridworlds},
%   \href{https://arxiv.org/abs/1711.09883}{arXiv:1711.09883v2};

% \item Sergei Gukov, Edward Witten, \emph{Branes and Quantization},
%   \href{https://arxiv.org/abs/0809.0305}{https://arxiv.org/abs/0809.0305};

% \item R. Estrada, J. M. Gracia-Bondia, J. C. Varilly, \emph{On
%   summability of distributions and spectral geometry},
%   \href{https://arxiv.org/abs/funct-an/9702001v1}{arXiv:funct-an/9702001};

% \item Ghanashyam Date, \emph{Lectures on Constrained Systems},
%   \href{https://arxiv.org/abs/1010.2062v1}{arXiv:1010.2062};

% \item Frank Wilczek, \emph{Quantum Field Theory},
%   \href{https://arxiv.org/abs/hep-th/9803075v2}{arXiv:hep-th/9803075};

% \item Karl Michael Schmidt, Karl Michael Schmidt, \emph{Schnol’s
%   Theorem and Spectral Properties of Massless Dirac Operators with
%   Scalar Potentials};

% \item Peter J. Olver, \emph{Dirac’s theory of constraints in fields
%   theory and the canonical form of Hamiltonian differential
%   operators};


% \item Lorenzo Iorio, \emph{Editorial for the Special Issue 100 Years
%   of Chronogeometrodynamics: The Status of the Einstein's Theory of
%   Gravitation in Its Centennial Year},
%   \href{https://arxiv.org/abs/1504.05789v2}{arXiv:1504.05789};

% \item B. Mutet, P. Grang\'{e}, E. Werner, \emph{Taylor–Lagrange
%   renormalization and gauge theories in four dimensions};


% \item \emph{Surviving Software Dependencies},
%   \href{https://queue.acm.org/detail.cfm?id=3344149}{https://queue.acm.org/detail.cfm?id=3344149};

% \item Clifford M. Will, \emph{The Confrontation between General
%   Relativity and Experiment};

% \item Paul Lopes, \emph{Culture and Stigma: Popular Culture and the
%   Case of Comic Books};

% \item Marina S.Butuzova 1, Alexander B. Pushkarev, \emph{Is OJ 287 a
%   Single Supermassive Black Hole?};

% \item Peter Selinger, \emph{Lecture notes on the lambda calculus},
%   \href{https://arxiv.org/abs/0804.3434v2}{arXiv:0804.3434};

% \item Alex Eskin, Maryam Mirzakhani, \emph{Counting closed geodesics
%   in Moduli space},
%   \href{https://arxiv.org/abs/0811.2362v3}{arXiv:0811.2362};

% \item Jacques Carette, James H. Davenport, \emph{The Power of
%   Vocabulary: The Case of Cyclotomic Polynomials},
%   \href{https://arxiv.org/abs/1002.0012v1}{arXiv:1002.0012};

% \item Chris Kapulkin, Peter LeFanu Lumsdaine, \emph{The Simplicial
%   Model of Univalent Foundations (after Voevodsky)},
%   \href{https://arxiv.org/abs/1211.2851v5}{arXiv:1211.2851};

% \item Christopher J. Fewster, Rainer Verch, \emph{Quantum fields and
%   local measurements},
%   \href{https://arxiv.org/abs/1810.06512}{arXiv:1810.06512};

% \item Paweł Duch, \emph{Infrared problem in perturbative quantum
%   field theory},
%   \href{https://arxiv.org/abs/1906.00940}{arXiv:1906.00940};

% \item Christopher J. Fewster, \emph{A generally covariant
%   measurement scheme for quantum field theory in curved spacetimes},
%   \href{https://arxiv.org/abs/1904.06944v1}{arXiv:1904.06944};

% \item Jacob Lurie, \emph{Higher Topos Theory},
%   \href{https://arxiv.org/abs/math/0608040v4}{arXiv:math/0608040};

% \item Konrad Osterwalder, and Robert Schrader, \emph{Axioms for
%   Euclidean Green's Functions};

% \item Vladimir Voevodsky, \emph{A very short note on homotopy
%   $\lambda$-calculus};

% \item Charles Rezk, \emph{Toposes and homotopy toposes};

% \item Fredrik Nordvall Forsberg, Anton Setzer, \emph{A finite
%   axiomatisation of inductive-inductive definitions};

% \item Egbert Rilke, \emph{Introduction to homotopy type theory};

% \item \'{A}lvaro Pelayo, Michael A. Warren, \emph{Homotopy type
%   theory and Voevodsky’s Univalent Foundations};

% \item H. Simmons, A. Schalk, \emph{An introduction to
%   $\lambda$-calculi and arithmetics};

% \item Roderich Tumulka, \emph{Lecture Notes on Mathematical
%   Statistical Physics};

% \item Martin Hofmann, \emph{Extensional concepts in intensional type
%   theory};

% \item Eugenio Moggi, \emph{Computational $\lambda$-calculus and
%   monads};

% \item \emph{Proof-theoretic semantics. Assessment and Future
%   Perspectives};

% \item Philip Walder, \emph{Propostions as Types};

% \item Egbert Rijke, \emph{Homotopy type theory};

% \item Eugenio Moggi, \emph{Notions of computation and monads};

% \item Bruno Barras, Thierry Coquand and Simon Huber, \emph{A
%   Generalization of Takeuti-Gandy Interpretation};

% \item Michael Alton Warren, \emph{Homotopy Theoretic Aspects of
%   Constructive Type Theory};

% \item Vladimir Voevodsky, \emph{A universe polymorphic type system};

% \item S. Marmi, \emph{An Introduction To Small Divisors},
%   \href{https://arxiv.org/abs/math/0009232v1}{arXiv:math/0009232};

% \item Paweł Duch, Michael Duetsch, Jose M. Gracia-Bondia,
%   \emph{Diphoton decay of the higgs from the Epstein--Glaser
%   viewpoint},
%   \href{https://arxiv.org/abs/2011.12675v2}{arXiv:2011.12675};

% \item Juliette Kennedy, Menachem Magidor, Jouko
%   V\"{a}\"{a}n\"{a}nen, \emph{ Inner Models from Extended Logics:
%   Part 1},
%   \href{https://arxiv.org/abs/2007.10764}{arXiv:2007.10764};

% \item Paul W. Gross, P. Robert Kotiuga, \emph{Electromagnetic Theory
%   and Computation: A Topological Approach};

% \item Andreas R. Blass, Jeffry L. Hirst, and Stephen G. Simpson,
%   Logical analysis of some theorems of combinatorics and topological
%   dynamics, Logic and combinatorics (Arcata, Calif., 1985), Contemp.
%   Math., vol. 65, Amer. Math. Soc., Providence, RI, 1987, pp.
%   125–156.

% \item Jared Corduan, Marcia Groszek, and Joseph Mileti, Draft: A
%   note on reverse mathematics and partitions of trees.

% \item Jennifer Chubb, Jeffry Hirst, and Tim McNichol, Reverse
%   mathematics and partitions of trees. To appear in J. Symbolic
%   Logic.

% \item Damir Dzhafarov and Jeffry Hirst, The polarized Ramsey
%   theorem. Archive for Math. Logic, Online First: 2008.

% \item Neil Hindman, The existence of certain ultra-filters on N and
%   a conjecture of Graham and Rothschild, Proc. Amer. Math. Soc. 36
%   (1972), 341–346.

% \item Jeffry L. Hirst, Hindman’s theorem, ultrafilters, and reverse
%   mathematics, J. Symbolic Logic 69 (2004), no. 1, 65–72.

% \item Carl G. Jockusch Jr., Ramsey’s theorem and recursion theory,
%   J. Symbolic Logic 37 (1972), 268–280.

% \item J. Mileti, Partition theory and computability theory. Ph.D.
%   Thesis.

% \item


































































































































































































































































% \end{enumerate}
% % \bibliographystyle{alpha} \bibliography{Bibliography}{}










% ############################

% Koniec dokumentu
\end{document}
