% ---------------------------------------------------------------------
% Podstawowe ustawienia i pakiety
% ---------------------------------------------------------------------
\RequirePackage[l2tabu, orthodox]{nag} % Wykrywa przestarzałe i niewłaściwe
% sposoby używania LaTeXa. Więcej jest w l2tabu English version.
\documentclass[a4paper,11pt]{article}
% {rozmiar papieru, rozmiar fontu}[klasa dokumentu]
\usepackage[MeX]{polski} % Polonizacja LaTeXa, bez niej będzie pracował
% w języku angielskim.
\usepackage[utf8]{inputenc} % Włączenie kodowania UTF-8, co daje dostęp
% do polskich znaków.
\usepackage{lmodern} % Wprowadza fonty Latin Modern.
\usepackage[T1]{fontenc} % Potrzebne do używania fontów Latin Modern.



% ------------------------------
% Podstawowe pakiety (niezwiązane z ustawieniami języka)
% ------------------------------
\usepackage{microtype} % Twierdzi, że poprawi rozmiar odstępów w tekście.
\usepackage{graphicx} % Wprowadza bardzo potrzebne komendy do wstawiania
% grafiki.
\usepackage{verbatim} % Poprawia otoczenie VERBATIME.
\usepackage{textcomp} % Dodaje takie symbole jak stopnie Celsiusa,
% wprowadzane bezpośrednio w tekście.
\usepackage{vmargin} % Pozwala na prostą kontrolę rozmiaru marginesów,
% za pomocą komend poniżej. Rozmiar odstępów jest mierzony w calach.
% ------------------------------
% MARGINS
% ------------------------------
\setmarginsrb
{ 0.7in}  % left margin
{ 0.6in}  % top margin
{ 0.7in}  % right margin
{ 0.8in}  % bottom margin
{  20pt}  % head height
{0.25in}  % head sep
{   9pt}  % foot height
{ 0.3in}  % foot sep



% ------------------------------
% Często przydatne pakiety
% ------------------------------
\usepackage{csquotes} % Pozwala w prosty sposób wstawiać cytaty do tekstu.
\usepackage{xcolor} % Pozwala używać kolorowych czcionek (zapewne dużo
% więcej, ale ja nie potrafię nic o tym powiedzieć).



% ------------------------------
% Pakiety do tekstów z nauk przyrodniczych
% ------------------------------
\let\lll\undefined % Amsmath gryzie się z językiem pakietami do języka
% polskiego, bo oba definiują komendę \lll. Aby rozwiązać ten problem
% oddefiniowuję tę komendę, ale może tym samym pozbywam się dużego Ł.
\usepackage[intlimits]{amsmath} % Podstawowe wsparcie od American
% Mathematical Society (w skrócie AMS)
\usepackage{amsfonts, amssymb, amscd, amsthm} % Dalsze wsparcie od AMS
% \usepackage{siunitx} % Dla prostszego pisania jednostek fizycznych
\usepackage{upgreek} % Ładniejsze greckie litery
% Przykładowa składnia: pi = \uppi
\usepackage{slashed} % Pozwala w prosty sposób pisać slash Feynmana.
\usepackage{calrsfs} % Zmienia czcionkę kaligraficzną w \mathcal
% na ładniejszą. Może w innych miejscach robi to samo, ale o tym nic
% nie wiem.



% ##########
% Tworzenie otoczeń "Twierdzenie", "Definicja", "Lemat", etc.
\newtheorem{theorem}{Twierdzenie}  % Komenda wprowadzająca otoczenie
% „theorem” do pisania twierdzeń matematycznych
\newtheorem{definition}{Definicja}  % Analogicznie jak powyżej
\newtheorem{corollary}{Wniosek}



% ---------------------------------------
% Pakiety napisane przez użytkownika.
% Mają być w tym samym katalogu to ten plik .tex
% ---------------------------------------
\usepackage{latexgeneralcommands}
% \usepackage{mathcommands}
% \usepackage{calculuscommands}
% \usepackage{SchwartzBooksCommands}  % Pakiet napisany m.in. dla tego pliku.



% ---------------------------------------------------------------------
% Dodatkowe ustawienia dla języka polskiego
% ---------------------------------------------------------------------
\renewcommand{\thesection}{\arabic{section}.}
% Kropki po numerach rozdziału (polski zwyczaj topograficzny)
\renewcommand{\thesubsection}{\thesection\arabic{subsection}}
% Brak kropki po numerach podrozdziału



% ------------------------------
% Ustawienia różnych parametrów tekstu
% ------------------------------
\renewcommand{\arraystretch}{1.2} % Ustawienie szerokości odstępów między
% wierszami w tabelach.



% ------------------------------
% Pakiet "hyperref"
% Polecano by umieszczać go na końcu preambuły.
% ------------------------------
\usepackage{hyperref} % Pozwala tworzyć hiperlinki i zamienia odwołania
% do bibliografii na hiperlinki.










% ---------------------------------------------------------------------
% Tytuł i autor tekstu
\title{Geometria nieprzemienna \\
  Błędy i~uwagi}

\author{Kamil Ziemian}
% \date{}
% ---------------------------------------------------------------------










% ####################################################################
\begin{document}
% ####################################################################





% ######################################
\maketitle % Tytuł całego tekstu
% ######################################




% ##################
\noindent
Grzeszyłem w~latach chłopięcych, gdy niedorzeczności ceniłem
wyżej od~rzeczy pożytecznych. Co mówię! jedne kochałem, drugich
nienawidziłem. Jeden i~jeden -- dwa, dwa i~dwa -- cztery. Jakaż
nienawistna była mi ta śpiewka. A jak błogim widowiskiem przy całej
swej wewnętrznej pustce był ów drewniany, pełen wojowników koń
trojański! I~łuna Troi! I~,,samej Kreuzy cień...''.

% \attribA{Św. Augustyn „Wyznania”, księga I, 13. \\
%   O~swojej miłości do Wergiliusza i~literatury.}
% ##################

\vspace{\spaceThree}



% ##################
\noindent
Straszliwa rzeko społecznego obyczaju! Kto się tobie oprzeć
zdoła? Czy nigdy twoje wody nie~opadną, nie~wyschną? Jak długo
będziesz jeszcze gnać nieszczęsnych ludzi ku morzu wielkiemu
i~groźnemu, trudnemu do przebycia, nawet dla tych którzy do drzewa
Krzyża przywarli?

% \attribA{Św. Augustyn „Wyznania”, księga I, 16}
% ##################

\vspace{\spaceThree}




% ##################
\noindent
Jak to się dzieje, że~jeden w belferskich szatach człowiek może
spokojnie słuchać, jak inny z~takiego samego miotu biedak woła:
„Zmyślił to Homer; ludzkie sprawy do świata bogów przenosił;
o,~czemuż nie boskie do nas...”

% \attribA{Św. Augustyn „Wyznania”, księga I, 16.}
% ##################

\vspace{\spaceThree}



% ##################
\noindent
Trudno się dziwić, że~tak się pogrążyłem w marnościach
i~odchodziłem daleko od Ciebie, skoro jako wzory do naśladowania
przedstawiano mi ludzi, którzy wstydzili się jak hańby tego,
że~opowiadając o dobrych swoich czynach, popełnili błąd gramatyczny
albo~użyli wyrażeń prowincjonalnych, a~dumnie kroczyli w obłoku
pochwały, jeśli o~swoich niegodziwych pasjach mówili zdaniami
zaokrąglonymi, błyszczącymi obfitą ornamentyką.

% \attribA{Św. Augustyn „Wyznania”, księga I, 16}
% ##################

\vspace{\spaceThree}



% ##################
\noindent
W~Kartaginie zaś~panuje ohydna samowola uczniów. Wpadają do sali
gwałtem jak szaleńcy i~niweczą porządek, jaki każdy nauczyciel ustalił
dla dobra studentów. Z~czystej głupoty dopuszczają~się wielu czynów
które byłyby karalne gdyby tych młodzieńców nie~osłaniał obyczaj.

% \attribA{Św. Augustyn „Wyznania”, księga V, 8}
% ##################

\vspace{\spaceThree}



% ##################
\noindent
I~oto dowiaduję się, że~tu, w~Rzymie, są~kłopoty, z~jakimi~się
nie stykałem w~Kartaginie. Wprawdzie przekonałem~się, że~nie ma tu
takich awantur, jakie młodzi rozbójnicy urządzali w Afryce. Ale, mówią
mi znajomi, nieraz tak~się zdarza, że~studenci umawiają się, iż~nie
zapłacą nauczycielowi za lekcje, i~po prostu przenoszą się do
innego[...].

% \attribA{Św. Augustyn „Wyznania”, księga V, 12}
% ##################

\vspace{\spaceThree}



% ##################
\noindent
Stary Wiktoryn był człowiekiem niepospolicie uczonym. Znał do
głębi wszystkie sztuki wyzwolone. Ileż dzieł filozofów przestudiował,
ocenił i~skomentował. Był nauczycielem wielu znakomitych
senatorów. Z~wdzięczności za~pracę profesorską, którą synowie tego
świata uważają za~ogromnie ważną, uczczono go wystawieniem posągu na
Forum w~Rzymie.

% \attribA{Św. Augustyn „Wyznania”, księga VIII, 2.}
% ##################

\vspace{\spaceThree}



% ##################
\noindent
Na co my czekamy? Czy pojąłeś sens tej opowieści? Powstają
prostaczkowie i~zdobywają niebo, a~my, z~całą naszą bezduszną
uczonością tarzamy~się w~ciele i~w~krwi.

% \attribA{Św. Augustyn „Wyznania”, księga VIII, 8. \\
%   Słowa św. Augustyna do swego przyjaciela Alipiusza.}
% ##################

% \noi Jak schlebiający przyjaciele nieraz czynią nas gorszymi, tak
% obrażający nas wrogowie nieraz~się przyczyniają do naszej poprawy.

% \attribA{Św. Augustyn ,,Wyznania'', księga VIII, 8.}

% \noi Potem~się dowiedziałem, że~pewnego dnia podczas pobytu w~Ostii,
% gdy ja nie byłem obecny, do kilku moich przyjaciół z~macierzyńską
% bezpośredniością mówiła, jak godne pogardy jest to życie i~jakim
% dobrem jest śmierć.

% \attribA{Św. Augustyn ,,Wyznania'', księga IX, 11. \\ O~swej matce
%   św. Monice.}

% \noi I~tą nadzieją~się weselę, ilekroć~się weselę zbawiennie. A~inne
% tego życia sprawy? Tym mniej zasługują na łzy, im częściej się na nimi
% płacze. Tym bardziej się powinno nad nimi płakać, im mniej~się nad
% nimi łez leje.

% \attribA{Św. Augustyn ,,Wyznania'', księga X, 1.}

% \noi Jakże skwapliwie ludzie badają cudze życie, a~jak~się opieszale
% zabierają do naprawienia swojego. Czemu chcą~się ode mnie dowiedzieć,
% jakim jestem człowiekiem, skoro nie pragną usłyszeć od Ciebie, jacy
% oni sami naprawdę~są?

% \attribA{Św. Augustyn ,,Wyznania'', księga X, 3.}










% #####################################################################
% #####################################################################
% Bibliografia
\bibliographystyle{plalpha}

\bibliography{MathComScienceBooks}{}





% ############################

% Koniec dokumentu
\end{document}
