% ---------------------------------------------------------------------
% Podstawowe ustawienia i pakiety
% ---------------------------------------------------------------------
\RequirePackage[l2tabu, orthodox]{nag}  % Wykrywa przestarzałe i niewłaściwe
% sposoby używania LaTeXa. Więcej jest w l2tabu English version.
\documentclass[a4paper,11pt]{article}
% {rozmiar papieru, rozmiar fontu}[klasa dokumentu]
\usepackage[MeX]{polski}  % Polonizacja LaTeXa, bez niej będzie pracował
% w języku angielskim.
\usepackage[utf8]{inputenc}  % Włączenie kodowania UTF-8, co daje dostęp
% do polskich znaków.
\usepackage{lmodern}  % Wprowadza fonty Latin Modern.
\usepackage[T1]{fontenc}  % Potrzebne do używania fontów Latin Modern.



% ------------------------------
% Podstawowe pakiety (niezwiązane z ustawieniami języka)
% ------------------------------
\usepackage{microtype}  % Twierdzi, że poprawi rozmiar odstępów w tekście.
\usepackage{graphicx}  % Wprowadza bardzo potrzebne komendy do wstawiania
% grafiki.
\usepackage{verbatim}  % Poprawia otoczenie VERBATIME.
\usepackage{textcomp}  % Dodaje takie symbole jak stopnie Celsiusa,
% wprowadzane bezpośrednio w tekście.
\usepackage{vmargin}  % Pozwala na prostą kontrolę rozmiaru marginesów,
% za pomocą komend poniżej. Rozmiar odstępów jest mierzony w calach.
% ------------------------------
% MARGINS
% ------------------------------
\setmarginsrb
{ 0.7in}  % left margin
{ 0.6in}  % top margin
{ 0.7in}  % right margin
{ 0.8in}  % bottom margin
{  20pt}  % head height
{0.25in}  % head sep
{   9pt}  % foot height
{ 0.3in}  % foot sep



% ------------------------------
% Często przydatne pakiety
% ------------------------------
\usepackage{csquotes}  % Pozwala w prosty sposób wstawiać cytaty do tekstu.
\usepackage{xcolor}  % Pozwala używać kolorowych czcionek (zapewne dużo
% więcej, ale ja nie potrafię nic o tym powiedzieć).



% ------------------------------
% Pakiety do tekstów z nauk przyrodniczych
% ------------------------------
\let\lll\undefined  % Amsmath gryzie się z językiem pakietami do języka
% polskiego, bo oba definiują komendę \lll. Aby rozwiązać ten problem
% oddefiniowuję tę komendę, ale może tym samym pozbywam się dużego Ł.
\usepackage[intlimits]{amsmath}  % Podstawowe wsparcie od American
% Mathematical Society (w skrócie AMS)
\usepackage{amsfonts, amssymb, amscd, amsthm}  % Dalsze wsparcie od AMS
% \usepackage{siunitx}  % Do prostszego pisania jednostek fizycznych
\usepackage{upgreek}  % Ładniejsze greckie litery
% Przykładowa składnia: pi = \uppi
\usepackage{slashed}  % Pozwala w prosty sposób pisać slash Feynmana.
\usepackage{calrsfs}  % Zmienia czcionkę kaligraficzną w \mathcal
% na ładniejszą. Może w innych miejscach robi to samo, ale o tym nic
% nie wiem.



% ------------------------------
% Tworzenie otoczeń „Twierdzenie”, „Definicja”, „Lemat”, etc.
\newtheorem{twr}{Twierdzenie}  % Komenda wprowadzająca otoczenie
% „twr” do pisania twierdzeń matematycznych
\newtheorem{defin}{Definicja}  % Analogicznie jak powyżej
\newtheorem{wni}{Wniosek}



% ------------------------------
% Pakiety napisane przez użytkownika.
% Mają być w tym samym katalogu to ten plik .tex
% ------------------------------
% \usepackage{reedsimon}  % Pakiet napisany konkretnie dla tego pliku.
% \usepackage{latexgeneralcommands}
% \usepackage{mathcommands}

% \usepackage{mechanika}  % Pakiet napisany konkretnie dla tego pliku.

% \usepackage{tensor}



% ---------------------------------------------------------------------
% Dodatkowe ustawienia dla języka polskiego
% ---------------------------------------------------------------------
\renewcommand{\thesection}{\arabic{section}.}
% Kropki po numerach rozdziału (polski zwyczaj topograficzny)
\renewcommand{\thesubsection}{\thesection\arabic{subsection}}
% Brak kropki po numerach podrozdziału



% ------------------------------
% Ustawienia różnych parametrów tekstu
% ------------------------------
\renewcommand{\arraystretch}{1.2}  % Ustawienie szerokości odstępów między
% wierszami w tabelach.



% ------------------------------
% Pakiet „hyperref”
% Polecano by umieszczać go na końcu preambuły.
% ------------------------------
\usepackage{hyperref}  % Pozwala tworzyć hiperlinki i zamienia odwołania
% do bibliografii na hiperlinki.










% ---------------------------------------------------------------------
% Tytuł, autor, data
\title{Spotkania z Arnoldem}

% \author{}
% \date{}
% ---------------------------------------------------------------------










% ####################################################################
\begin{document}
% ####################################################################





% ######################################
\maketitle % Tytuł całego tekstu
% ######################################





Książka W. Arnolda „Metody matematyczne mechaniki klasycznej”
stanowi dzięki głębi swych przemyśleń, naciskowi na~intuicyjne
uzasadnienie stosowanego formalizmu, matematycznemu wyrafinowaniu
i~niezwykłej erudycji autora jedno z~najlepszych dzieł o mechanice Newtona jakie napisano w~XX wieku. Równocześnie jest to pozycja bardzo trudna, w~której elementarne zagadnienia sąsiadują
z~bardzo skomplikowanymi problemami, wiele kluczowych rozumowań jest
tylko naszkicowanych, a~pełne jej zrozumienie wymaga pewnego obycie
z~matematyką.

Celem tych spotkań jest wspólne przestudiowanie i~zrozumienie tego
dzieła na cotygodniowych dwugodzinnych spotkaniach. W~zależności od
preferencji uczestników forma tych spotkań może przybrać postać
wykładów prowadzonych przez jedną osobę na podstawie poszczególnych
rozdziałów, bądź referowania przez zainteresowane osoby kolejnych
fragmentów książki. W~obu wypadkach wspólna dyskusja przedstawionego
materiału ma być centralną częścią spotkań.




% ############################
\section{Plan spotkań}
% ############################



\noindent
Poniżej jest lista fragmentów książki, które będą dyskutowane na
spotkaniach. Zaczniemy od poczynienia kilku uwag na jego temat.

\begin{itemize}
\item[--] Choć pierwsza książki traktuje o~mechanice w~sformułowaniu
  Newtona, co nie pokrywa się z planem kursu „Mechanika teoretyczna”
  profesora Bizonia, to wprowadzone jest w~niej kilka kluczowych dla
  całej książki koncepcji, dlatego też jej część znalazła~się w~planie
  spotkań.

\item[--] Gwiazdką (*) oznaczone są paragrafy których treść można
  w~całości, albo w~większości, pominąć bez szkody dla dalszej części
  książki, jednak ze względu na ciekawy materiał warto rozważyć ich
  dokładne przestudiowanie.

\item[--] Plan może ulec zmianie w~czasie trwania spotkań.

\item[--] Jeżeli ktoś chce zgłosić zastrzeżenie do tego planu proszę
  pisać na adres \texttt{kziemianfvt@gmail.com}.

\end{itemize}



\begin{itemize}
\item[\textbf{Roz. I.}] \textbf{Fakty doświadczalne}

\item[--] 1. Zasada względności i przyczynowości.

\item[--] 2. Grupa Galileusz i~równania Newtona.


\item[\textbf{Roz. II.}] \textbf{Badanie równań ruchu}

\item[--] 4. Układy o~jednym stopniu swobody.

\item[--] 5. Układy o~dwóch stopniach swobody.

\item[--] 11*. Rozumowanie oparte na~podobieństwie.


\item[\textbf{Roz. III.}] \textbf{Zasada wariacyjna (całość)}

\item[--] 12. Rachunek wariacyjny.

\item[--] 13. Równanie Lagrange’a.

\item[--] 14. Przekształcenie Legendre’a.

\item[--] 15. Równania Hamiltona.

\item[--] 16. Twierdzenie Liouville’a.


\item[\textbf{Roz. IV.}] \textbf{Mechanika Lagrange’a na
    rozmaitościach}

\item[--] 17. Więzy holonomiczne.

\item[--] 18. Rozmaitości różniczkowalne.

\item[--] 19. Układy dynamiczne Lagrange'a.

\item[--] 20. Twierdzenie E. Noether.


\item[\textbf{Roz. V.}] \textbf{Drgania}

\item[--] 22. Linearyzacja.

\item[--] 23. Małe drgania.

\item[--] 24*. O~zachowaniu się częstości własnych.

\item[--] 25*. Rezonans parametryczny.


\item[\textbf{Roz. VII.}] \textbf{Formy różniczkowe (całość)}

\item[--] 32. Formy zewnętrzne.

\item[--] 33. Iloczyn zewnętrzny.

\item[--] 34. Formy różniczkowe.

\item[--] 35. Całkowanie form różniczkowych.

\item[--] 36. Różniczkowanie zewnętrzne.


\item[\textbf{Roz. VIII.}] \textbf{Rozmaitości symplektyczne}

\item[--] 37. Struktura symplektyczna na rozmaitości.

\item[--] 38. Hamiltonowskie potoki fazowe i~ich niezmienniki całkowe.

\item[--] 39. Algebry Liego pól wektorowych.

\item[--] 40. Algebra Liego pól Hamiltona.

\item[--] 41. Geometria symplektyczna.

\item[--] 42*. Rezonans parametryczny w~układach o~wielu stopniach
  swobody.

\item[--] 43. Atlas symplektyczny.


\item[\textbf{Roz. IX.}] \textbf{Formalizm kanoniczny}

\item[--] 44*. Niezmiennik całkowy Poincar\'{e}go-Cartana.

\item[--] 45. Konsekwencje twierdzenia o~niezmienniku całkowym
  Poincar\'{e}go-Cartana.

\item[--] 46*. Zasada Huygensa.

\item[--] 47*. Metoda Jacobiego-Hamiltona całkowania równań
  kanonicznych Hamiltona.

\item[--] 48. Funkcje tworzące.


\item[\textbf{Roz. IX.}] \textbf{Wprowadzenie do teorii zaburzeń}

\item[--] 49*. Układy całkowalne.

\item[--] 50*. Współrzędne działanie-kąt.

\item[--] 51*. Uśrednianie.

\item[--] 52*. Uśrednianie zaburzeń


\item[] \textbf{Uzupełnienia.}

\item[--] 5*. Układy dynamiczne wykazujące symetrię.

\item[--] 8*. Teoria zaburzeń dla~ruchów prawie okresowych
  i~twierdzenie Kołmogorowa.

\item[--] 12*. Osobliwości Lagrange'a.

\item[--] Na co nam jeszcze starczy sił ;). O~ile w~ogóle tu
  dotrzemy\ldots

\end{itemize}










% ############################
\section{Bibliografia}
% ############################


\noindent
Poniższa lista zawiera pozycje zarówno skierowane zarówno dla osób
które chcą szerzej zapoznać się z niektórymi omawianymi zagadnieniami, jak
i~tych które chcą zrozumieć w jaki sposób można uogólnić prezentowane metody i~techniki. Niestety, do niektórych zagadnień nie
udało~się znaleźć zadowalającej literatury, zaś pewnych wartościowych pozycji nie~umieszczono na niej, jako mało adekwatnych
do~treści spotkań

\vspace{1em}



\noindent
\textbf{BWMiI} -- Biblioteka Wydziału Matematyki i Informatyki. \\

\vspace{0.5em}



\noindent
\textbf{Cykl Władymira Arnolda.} Książki te optymalnie byłoby czytać
w~podanej poniżej kolejności, stąd obecność tu omawianego na
spotkaniach dzieła.

\begin{itemize}
\item[--] \textit{Równania różniczkowe zwyczajne} (RRZ), BWMiI.

\item[--] \textit{Metody matematyczne mechaniki klasycznej} (MMMK),
  BWMiI. %Biblioteka Wydziału Matematyki i Informatyki.

\item[--] \textit{Teoria równań różniczkowych} (TRR), BWMiI.

\end{itemize}


\begin{itemize}
\item[] \textbf{Podstawy matematyczne.}

\item[--] L. Schwartz, \textit{Kurs analizy matematycznej, tom I} (LSI),
  większość bibliotek np. NKFu. Książka trudna, ale zawiera dowód
  chyba każdego twierdzenia z~analizy jakie będzie potrzebne.

\item[--] A. Herdegen, \textit{Algebra liniowa i~geometria} (AH).
  Głównie twierdzenia o~formach kwadratowych\footnote{Jedyną inną
    pozycją, która o~ile wiem zawiera dowody potrzebnych twierdzeń,
    jest książka „Wykłady z~algebry liniowej” I. M. Gelfanda.}.


\item[] \textbf{Struktura czasoprzestrzeni mechaniki Newtona.}

\item[--] W. Kopczyński, A. Trautman, \textit{Czasoprzestrzeń
    i~grawitacja} (KT), biblioteki FAISu, NKFu etc. Dobra, krótka
  i~trudna pozycja, jedna z~niewielu które zajmują~się tym tematem.


\item[] \textbf{Mechanika klasyczna.}

\item[--] E. T. Whittaker, \textit{Dynamika analityczna} (ETW),
  biblioteka NKFu. Wiekowa, lecz wciąż warta uwagi pozycja.

\item[--] R. Abraham, J. E. Marsden, \textit{Foundations of Mechanics,
    Second Edition} (FoM2),
  \url{http://authors.library.caltech.edu/25029/}. Monumentalne
  dzieło starające~się z~pełną ścisłością przedstawić mechanikę za
  pomocą metod współczesnej matematyki.


\item[] \textbf{Równania różniczkowe zwyczajne.}

\item[--] W. Walter, \textit{Ordinary differential equations} (WWODEs),
  Springer Link. Rozsądny, współczesny wykład podstaw teorii ODEs.

\item[--] E. Hairer, S. P. N\o rsett, G. Warner, \textit{Solving Ordinary
    Differential Equations} (SODEs), Springer Link. Monumentlane
  dzieło o~tym jak analitycznie, a~przede wszystkim numerycznie
  rozwiązać dane równanie.


\item[] \textbf{Rachunek wariacyjny}

\item[--] I. M. Gelfand, S. V. Fomin, \textit{Rachunek wariacyjny} (GF),
  większość bibliotek, np. NKFu i FAISu. Standardowy wykład
  klasycznych osiągnięć rachunku wariacyjnego.

\item[--] J. Jost, X. Li-Jost, \textit{Calculus of Variations} (JLJ),
  BWMiI. Podręcznik zawierający wprowadzenie do~współczesnych metod
  w~rachunku wariacyjnym.

\item[--] M. Giaquinta, St. Hildebrandt, \textit{Calculus of Variations}
  (GHCoV), Springer Link. Monografia starająca~się dać możliwe
  wyczerpujący opis współczesnych metod.


\item[] \textbf{Geometria różniczkowa}

\item[--] J. Gancarzewicz, \textit{Zarys współczesnej geometrii
    różniczkowej} (ZWGR). Abstrakcyjna, długa i~niepozbawiona dużej ilości
  literówek błędów, jednak bardzo dobra pozycja dla średnio zaawansowanych.

\item[--] R. Sulanke, P. Wintgen, \textit{Geometria różniczkowa i~teoria
    wiązek} (SW), BWMiI. Pozycja wprowadzająca, zawierająca dobre
  wprowadzenie do teorii tożsamości geometryczno-całkowych
  na~rozmaitościach.


\item[] \textbf{Teoria form}

\item[--] L. Schwartz, \textit{Kurs analizy matematycznej, tom II}
  (LSII), BWMiI. Bez znajomości teorii całki\linebreak z~I tomu,
  prawie nie do zrozumienia.

\item[--] S. G. Krantz, H. R. Parks, \textit{Geometric Integration
    Theory} (GIT), Springer Link. Zawiera wprowadzenie do teorii
  prądów, pozwalajacej rozważać formy o~wartościach w~dystrybucjach.

\item[] \textbf{Geometria symplektyczna i mechanika}

\item[--] P. Libermann, Ch.-M. Marle, \textit{Symplectic Geometry
    and Analytical Mechanics} (SGAM), BWMiI, Springer Link. Wykład
  geometrii symplektycznej ilustrowany zastosowaniami w mechanice.

\item[] \textbf{Układy nieholonomiczne}

\item[--] J. I. Nejmark, N. A. Fufajew, \textit{Dynamika układów
    nieholonomicznych} (DUN), Allegro\footnote{Prostszy sposób nie
    jest znany.}. Podstawowe, w~dobry sposób staroświeckie, dzieło
  w~tej dziedzinie.

\item[--] E. Massa, E. Paganim \textit{A new look at classical mechanics
    of constrained systems} (EMEP), \\
  \url{https://eudml.org/doc/76747} . Nowoczesna próba zmierzenia~się
  z~zagadnieniem więzów nieholonomicnzych. Dość trudna pozycja.

\item[--] H. Geiges \textit{Contact geometry} (HGCM),
  arXiv:math/0307242v2 [math.SG]
  \url{http://arxiv.org/abs/math/0307242}. Można tu znaleźć dobre,
  jak na ten dział matematyki, wprowadzenie w~teorię rozmaitości
  kontaktowych, podstawy matematycznego opisu układów
  nieholonomicznych.


\item[] \textbf{Geometria różniczkowa poza mechaniką}

\item[--] G. Svetlichny, \textit{Preparation for Gauge Theory} (PGT),
  arXiv:math-ph/9902027v3 \url{http://arxiv.org/abs/math-ph/9902027} .
  Czasami trochę zbyt zwięzły, lecz merytorycznie bardzo dobry, wykład
  na temat zastosowania geometrii różniczkowej, i~pochodnych działów
  matematyki, do opisu \textbf{klasycznych} teorii pola z~cechowaniem,
  takich jak elektrodynamika, czy pola Yanga-Millesa.

\end{itemize}





\textbf{Ważne:}

\begin{quote}

  Dla prawdziwego matematyka, jest dużo ważniejsze by wiedzieć jakie
  problemy nie zostały wciąż rozwiązane i~gdzie znane obecnie metody
  okazały~się niewystarczające, niż pamiętać wszystkie liczby których
  iloczyny udało~się do tej pory uzyskać, czy orientować się w~
  ocenianie literatury stworzonej na przestrzeni ostatnich 20 tysięcy
  lat.

\end{quote}

W. Arnold w~przedmowie do książki „Arnold's Problems”, Springer Link, tłumaczenie swobodne.

\vspace{1em}


\begin{quote}

  Zauważyliśmy bowiem, że dla początkujących słuchaczy dużą przeszkodę
  w zdobywaniu tej nauki stanowią dzieła różnych teologów: już to
  dlatego, że są nadmiernie przeładowane bezużytecznymi zagadnieniami,
  artykułami i dowodami, już to dlatego, że zagadnienia, z jakimi owi
  początkujący winni koniecznie się zapoznać, nie są podane
  systematycznie: według uporządkowanej kolejności nauk czy traktatów,
  ale omawiane są albo w związku z komentowaniem dzieł, albo z okazji
  dysputy; już to wreszcie dlatego, że częste wałkowanie tego samego
  budziło w ich umysłach nudę i zamęt.

  Ufni w pomoc Bożą i starając się uniknąć tych i podobnych
  niedociągnięć, będziemy usiłowali krótko i jasno - o ile na to sama
  rzecz pozwoli - wyłożyć wszystko, co zakresem swoim obejmuje nauka
  święta.

\end{quote}

Św. Tomasz z~Akwinu we wstępie do \textit{Sumy Teologicznej},
\url{http://www.katedra.uksw.edu.pl/suma/suma_1.pdf}.










% #####################################################################
% #####################################################################
% Bibliografia
\bibliographystyle{plalpha}

\bibliography{PhilNaturBooks}{}





% ############################

% Koniec dokumentu
\end{document}
