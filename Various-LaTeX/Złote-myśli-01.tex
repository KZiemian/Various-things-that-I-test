% ---------------------------------------------------------------------
% Podstawowe ustawienia i pakiety
% ---------------------------------------------------------------------
\RequirePackage[l2tabu, orthodox]{nag} % Wykrywa przestarzałe i niewłaściwe
% sposoby używania LaTeXa. Więcej jest w l2tabu English version.
\documentclass[a4paper,11pt]{article}
% {rozmiar papieru, rozmiar fontu}[klasa dokumentu]
\usepackage[MeX]{polski} % Polonizacja LaTeXa, bez niej będzie pracował
% w języku angielskim.
\usepackage[utf8]{inputenc} % Włączenie kodowania UTF-8, co daje dostęp
% do polskich znaków.
\usepackage{lmodern} % Wprowadza fonty Latin Modern.
\usepackage[T1]{fontenc} % Potrzebne do używania fontów Latin Modern.



% ------------------------------
% Podstawowe pakiety (niezwiązane z ustawieniami języka)
% ------------------------------
\usepackage{microtype} % Twierdzi, że poprawi rozmiar odstępów w tekście.
\usepackage{graphicx} % Wprowadza bardzo potrzebne komendy do wstawiania
% grafiki.
\usepackage{verbatim} % Poprawia otoczenie VERBATIME.
\usepackage{textcomp} % Dodaje takie symbole jak stopnie Celsiusa,
% wprowadzane bezpośrednio w tekście.
\usepackage{vmargin} % Pozwala na prostą kontrolę rozmiaru marginesów,
% za pomocą komend poniżej. Rozmiar odstępów jest mierzony w calach.
% ------------------------------
% MARGINS
% ------------------------------
\setmarginsrb
{ 0.7in}  % left margin
{ 0.6in}  % top margin
{ 0.7in}  % right margin
{ 0.8in}  % bottom margin
{  20pt}  % head height
{0.25in}  % head sep
{   9pt}  % foot height
{ 0.3in}  % foot sep



% ------------------------------
% Często przydatne pakiety
% ------------------------------
\usepackage{csquotes} % Pozwala w prosty sposób wstawiać cytaty do tekstu.
\usepackage{xcolor} % Pozwala używać kolorowych czcionek (zapewne dużo
% więcej, ale ja nie potrafię nic o tym powiedzieć).



% ------------------------------
% Pakiety do tekstów z nauk przyrodniczych
% ------------------------------
\let\lll\undefined % Amsmath gryzie się z językiem pakietami do języka
% polskiego, bo oba definiują komendę \lll. Aby rozwiązać ten problem
% oddefiniowuję tę komendę, ale może tym samym pozbywam się dużego Ł.
\usepackage[intlimits]{amsmath} % Podstawowe wsparcie od American
% Mathematical Society (w skrócie AMS)
\usepackage{amsfonts, amssymb, amscd, amsthm} % Dalsze wsparcie od AMS
% \usepackage{siunitx} % Dla prostszego pisania jednostek fizycznych
\usepackage{upgreek} % Ładniejsze greckie litery
% Przykładowa składnia: pi = \uppi
\usepackage{slashed} % Pozwala w prosty sposób pisać slash Feynmana.
\usepackage{calrsfs} % Zmienia czcionkę kaligraficzną w \mathcal
% na ładniejszą. Może w innych miejscach robi to samo, ale o tym nic
% nie wiem.



% ##########
% Tworzenie otoczeń "Twierdzenie", "Definicja", "Lemat", etc.
\newtheorem{theorem}{Twierdzenie}  % Komenda wprowadzająca otoczenie
% „theorem” do pisania twierdzeń matematycznych
\newtheorem{definition}{Definicja}  % Analogicznie jak powyżej
\newtheorem{corollary}{Wniosek}



% ---------------------------------------
% Pakiety napisane przez użytkownika.
% Mają być w tym samym katalogu to ten plik .tex
% ---------------------------------------
\usepackage{latexgeneralcommands}
% \usepackage{mathcommands}
% \usepackage{calculuscommands}
% \usepackage{SchwartzBooksCommands}  % Pakiet napisany m.in. dla tego pliku.



% ---------------------------------------------------------------------
% Dodatkowe ustawienia dla języka polskiego
% ---------------------------------------------------------------------
\renewcommand{\thesection}{\arabic{section}.}
% Kropki po numerach rozdziału (polski zwyczaj topograficzny)
\renewcommand{\thesubsection}{\thesection\arabic{subsection}}
% Brak kropki po numerach podrozdziału



% ------------------------------
% Ustawienia różnych parametrów tekstu
% ------------------------------
\renewcommand{\arraystretch}{1.2} % Ustawienie szerokości odstępów między
% wierszami w tabelach.



% ------------------------------
% Pakiet "hyperref"
% Polecano by umieszczać go na końcu preambuły.
% ------------------------------
\usepackage{hyperref} % Pozwala tworzyć hiperlinki i zamienia odwołania
% do bibliografii na hiperlinki.










% ---------------------------------------------------------------------
% Tytuł i autor tekstu
\title{Geometria nieprzemienna \\
  Błędy i~uwagi}

\author{Kamil Ziemian}
% \date{}
% ---------------------------------------------------------------------










% ####################################################################
\begin{document}
% ####################################################################





% ######################################
\maketitle % Tytuł całego tekstu
% ######################################





% ######################################
\section{Platon}
% ######################################





% ##################
\noindent
A teraz my oto do Protagorasa pójdziemy, ja i~ty, i~będziemy
gotowi wypłacić mu pieniądze za ciebie, jeżeli tylko wystarczą nasze
kapitały i~jeśli nimi potrafimy na niego wpływać -- a~jeśli nie to
dołożymy jeszcze pieniądze przyjaciół.

% \attribA{Platon „Protagoras”, 311 D.\\ Słowa Sokratesa do
%   Hippokratesa, syna Appollodora, na temat tego \\ co zrobić by
%   Protagoras przyjął go jako ucznia.}
% ##################

\vspace{\spaceThree}



% ##################
\noindent
A~wiesz ty, na jakie niebezpieczeństwo idziesz wydać duszę? Czy
też, jeśliby ci wypadało ciało komuś powierzyć, a~zachodziła obawa,
że~ono albo~się tęgie stanie, albo liche, bardzo byś~się rozglądał
i~namyślał czy je komuś powierzyć, czy nie, i~na naradę przyjaciół byś
wołał i~krewnych, i~ci by długie dni na rozważaniach trawili; a~kiedy
chodzi o~duszę, którą wyżej cenisz niźli ciało, i~w~której całe twoje
przyszłe szczęście albo nieszczęście złożone, zależnie od tego,
czy~się tęga stanie, czy licha -- jej sprawą nie podzieliłeś się ani
z~ojcem, ani z~bratem, ani z~kimkolwiek z~nas znajomych: czy
powierzać, czy nie powierzać, czy nie powierzać duszy temu komuś, co
przyjechał z~obcych stron, tylkoś wieczorem usłyszał, jak powiadasz,
a~rano przychodzisz i~o~tym wcale nie myślisz, i~nie radzisz~się ani
słówkiem, czy należy mu siebie samego powierzyć, czy nie, i~gotów
jesteś wyłożyć kapitały własne i~przyjaciół, jak gdybyś już doszedł,
że~ze wszech miar należy pójść do Protagorasa, którego ani nie znasz,
jak mówisz, aniś z~nim nie rozmawiał nigdy, tylko sofistą go nazywasz,
a~co to jest sofista, tego, pokazuje się, nie wiesz, choć mu chcesz
oddać siebie samego.

% \attribA{Platon „Protagoras”, 313 A.}
% ##################

\vspace{\spaceThree}



% ##################
\noindent
Otóż o Ateńczykach ja, podobnie jak inni Hellenowie, mówię, że~to
mądry naród. Widzę zaś, że~kiedy się zbieramy na Zgromadzenie Ludowe
i~miasto ma czegoś dokonać z zakresu budownictwa, wtedy się posyła po
architektów, aby radzili w~sprawach budowlanych, a~kiedy chodzi
o~budowę okrętów, po cieśli okrętowych. I~w~innych sprawach tak samo,
o~których tylko sądzą, że~się ich samemu nauczyć można i~można ich
nauczyć drugiego. Jeśli zaś ktoś inny próbuje im rady udzielać,
którego oni nie uważają za fachowca, to choćby był bardzo piękny
i~bogaty, i~znakomitego rodu, mimo to go nie słuchają, tylko
wyśmiewają i~hałasują tak długo, póki albo sam nie ustąpi przed
krzykami z~mównicy, albo go służb stamtąd nie ściągnie i~wyrzuci na
rozkaz prytanów. Więc tak postępują w sprawach, które tylko uchodzą za
przedmiot fachu. Natomiast, kiedy wypadają obrady nad czymś
z~gospodarki państwowej, wtedy wstaje i~rad im udziela w~takich
rzeczach równie dobrze cieśla jak kowal, szewc, kupiec, dowódca
okrętu, bogaty, biedny, z~dobrej rodziny, czy byle kto. I~takiemu nikt
nie gani, jak tym poprzednim, że~nigdzie się nie uczył i~nie ma
żadnego nauczyciela, a~jednak sam~się ośmiela głos zabierać
w~radzie. To przecież jasna rzecz; bo nie sądzą, żeby się można było
tego nauczyć.

% \attribA{Platon „Protagoras”, 319 B, C, D.}
% ##################

\vspace{\spaceThree}



% ##################
\noindent
Gdyby tak ktoś na ten temat rozprawiał z którymkolwiek z mówców
ludowych, pewnie by też takie słowa słyszał - choćby z ust Peryklesa
albo od kogokolwiek innego z tych, co umieją mówić. Gdyby jedna potem
jeszcze zapytać takiego o cokolwiek, wtedy ten jak książka: nic nie
potrafi ani odpowiedzieć, ani sam zapytać, a jeśliby go kto i o
drobiazg jaki z tego przemówienia dodatkowo zapytał, ten, jak kocioł
miedziany, kiedy go uderzyć, długo huczy i ciągnie, póki go kto nie
chwyci; tak samo i mówcy, kiedy ich o drobiazg nawet zapytać, robią z
rozmowy wyścig na daleką metę.

% \attribA{Platon „Protagoras”, 329 A, B.}
% ##################

\vspace{\spaceThree}



% ##################
\noindent
Niesłusznie -- powiada -- mówisz Kaliaszu. Bo nasz Sokrates
przyznaje się, ze długie mowy to nie jego specjalność i ustępuje na
tym punkcie Protagorasowi, ale chciałby to widzieć, czy on komukolwiek
z ludzi ustąpi w sztuce dyskutowania, czy ktokolwiek lepiej od niego
potrafi argumentować i przyjmować argumenty. Więc jeżeli i Protagoras
zgadza się, że jest słabszy w dyskusji od Sokratesa -- to to
Sokratesowi wystarczy. A jeżeli staje do zawodu, to niech rozprawia
pytając i odpowiadając, a nie, żeby na każde pytanie długi wywód
rozpoczynał, uniemożliwiał chwyty myślowe, nie chciał dawać
argumentów, tylko ciągnął i ciągnął, że niejeden ze słuchaczy zapomni,
o co szło w zapytaniu.

% \attribA{Platon „Protagoras”, 336 B, C, D.}
% ##################

\vspace{\spaceThree}



% ##################
\noindent
Ja z~mojej strony, również uważam, Protagorasie i~Sokratesie,
że~powinniście sobie ustąpić nawzajem i~spierać się na dany temat,
ale~się nie sprzeczać. Bo spierają się i~po dobremu przyjaciel
z~przyjacielem, a~sprzeczają~się ludzie poróżnieni z~sobą
i~nieprzyjaciele.

% \attribA{Platon „Protagoras”, 337 A, B.}
% ##################

\vspace{\spaceThree}



% ##################
\noindent
Sądzę -- powiada -- Sokratesie, że~najważniejszy składnik kultury
duchowej każdego człowieka to jego orientacja w~dziedzinie utworów
poetyckich. Ona polega na tym, żeby w~tym, co poeci mówią, umieć
ocenić, co jest dobrze powiedziane, a~co nie, umieć dokonać rozbioru
dzieła i~odpowiadać na odnośne pytania.

% \attribA{Platon „Protagoras”, 339 A.}
% ##################

\vspace{\spaceThree}



% ##################
Mam wrażenie, że dyskusja o utworach poetyckich to zupełnie jak zabawa
przy kieliszku w gorszych towarzystwach mieszczańskich. Ci ludzie
także nie umieją nic sobie nawzajem dać z siebie w towarzystwie, przy
winie, ani mówić własnym głosem i własnymi myślami, bo im brak
kultury; więc cenią i przepłacają flecistki, wynajmują sobie obcy
głos, dźwięki fletów i tym dopiero głosem obcują jedni z drugimi. Ale
tam, gdzie piją ludzie pełni, o wysokiej kulturze, tam nie zobaczysz
flecistek ani tancerek, ani kitarzystek; ich stać na to, żeby sami z
sobą obcowali; bez tych błazeństw i zabawek; wystarcza im własny głos:
każdy sam potrafi raz mówić, raz słuchać drugiego -- w porządku --
choćby i bardzo dużo wina pili. Tak samo i takie zebrania, jak nasze
jeśli się zbiorą ludzie za jakich się wielu z nas uważa, ta nie
potrzeba obcego głosu ani poetów, których nawet zapytać niepodobna, o
czym mówią właściwie, a z tych, którzy się nieraz na nich powołują w
przemówieniach, jedni mówią, że poeta to miał na myśli, drudzy, że co
innego i rozmowa schodzi na teren, na którym niczego dowieść nie
potrafią.

% \attribA{Platon „Protagoras”, 347 B, C, D.}
% ##################

\vspace{\spaceThree}



% ##################
\noindent
Mam to przekonanie, że Homer ma słuszność, kiedy mówi: Dwóch
jeśli pójdzie na zwiady, jeden myśli, a~drugi dostrzega. We dwójkę
jakoś łatwiej nam ludziom idzie każda robota i~słowo, i~myśl
i~poszukiwanie, a człowiek sam, kiedy co wymyśli, zaraz chodzi
i~szuka, póki nie znajdzie, komu by to pokazać i~z~kim by~się
utwierdzić w~przekonaniu.

% \attribA{Platon „Protagoras”, 348 C, D.}
% ##################

\vspace{\spaceThree}


% \noi Gdyby zaś mój rozmówca był jednym z~tych mędrców, którzy szukają
% tylko dyskusji i~sprzeczki, powiedziałby mu co następuje: ja już
% odpowiedziałem; jeśli mówię niesłusznie, twoim obowiązkiem jest zabrać
% głos i~wykazać mi błąd.

% \attribA{Platon ,,Menon'', 75 D.}

% \noi \tb{Sokrates:} Gdy mówisz, Menonie, nawet z~zasłoniętymi oczami
% można by
% poznać, że~jesteś piękny i~że~cię kochają jeszcze. \\
% \tb{Menon:} Dlaczego? \\
% \tb{Sokrates:} Bo zawsze mówisz rozkazująco: tak czynią swawolnicy, ci
% tyrani, póki im wiek pozwala.

% \attribA{Platon ,,Menon'', 76 B, C.}

% \noi Zanim cię spotkałem, Sokratesie, słyszałem, że~i~sam pogrążony
% jesteś w~wątpliwościach i~innych doprowadzasz do zwątpienia. I~teraz,
% jak mi~się wydaje, zaczarowałeś mnie i~opętałeś jakimś magicznym
% napojem, tak że~stałem~się pełen wątpliwości. Jeśli wolno mi
% zażartować, zdajesz~się tak z wyglądu, jak i~pod innymi względami,
% zupełnie podobny do tej płaski ryby morskiej zwanej płaszczką. Sprawia
% bowiem ona, że~ten, kto się do nie zbliży i~jej dotknie, drętwieje;
% tak i~ty, jak mi się wydaje, postąpiłeś ze mną.

% \attribA{Platon ,,Menon'', 80 A, B.}

% \noi Oto co by wówczas było: gdyby ludzie byli dobrzy z~natury, byliby
% wśród nas tacy, którzy potrafiliby rozpoznać takich wśród młodzieży;
% wówczas my byśmy wzięli tych, których by w~ten sposób wskazano,
% i~trzymalibyśmy ich pod strażą na Akropolu strzegąc bardziej niż
% złota, by żaden z~nich nie uległ zepsuciu, ale by po osiągnięciu
% stosownego wieku przynieśli państwu pożytek.

% \attribA{Platon ,,Menon'', 89 B.}


% \noi Mówisz coś bardzo dziwnego: ci, którzy naprawiają stare obuwie
% i~stare płaszcze, nie~mogliby ukryć nawet przez trzydzieści dni,
% że~oddali w~gorszym stanie to, co przyjęli -- płaszcze i sandał; gdyby
% zaś tak postępowali, szybko pomarliby z~głodu; Protagoras zaś ukrywał
% przed całą Helladą od przeszło czterdziestu lat, że~psuje tych, co~się
% do niego zbliżają i~że~czyni ich gorszymi niż byli - wydaje mi~się
% bowiem, że~miał blisko siedemdziesiąt lat gdy umarł, po czterdziestu
% latach uprawiania swej sztuki.

% \attribA{Platon ,,Menon'', 91 D, E.}


% Jakieś wy, obywatele, odebrali wrażenie od moich oskarżycieli, tego
% nie wiem; bo i~ja sam przy nich omal żem~się nie zapomniał, tak
% przekonująco mówili. Chociaż znowu prawdziwego, powiem po prostu, nic
% nie powiedzieli. A~najwięcej mnie u~nich jedno zdziwiło z~tych wielu
% kłamstw; jak to mówili, że~wyście~się powinni strzec, abym ja was nie
% oszukał, bo doskonale umiem mówić. To, że~się nie wstydzili (toż ja
% zaraz czynem obalę ich twierdzenia, kiedy~się pokaże, że~ja ani trochę
% mówić nie umiem), to mi~się wydało u~nich największą bezczelnością.

% Chyba że~oni może tęgim mówcą nazywają tego, co prawdę mówi. Bo jeżeli
% tak mówią, to ja bym~się zgodził, że~tylko nie według nich, jest
% mówcą.

% Więc oni, jak ja mówię, bodaj że~i~słowa prawdy nie powiedzieli; wy
% dopiero ode mnie usłyszycie całą prawdę.

% \attribA{Platon ,,Obrona Sokratesa'', 17 A, B.}


% Bo na mnie wielu skarżyło przed wami od dawna, i już od lat całych,
% a~prawdy nic nie mówili; tych ja~się boję więcej niż tych koło
% Antyosa, chociaż i~to ludzie straszni.

% Ale tamci straszniejsi. Obywatele, oni niejednego z~was już jako
% chłopaka brali do siebie, wmawiali w~was i~skarżyli na mnie, że~jest
% taki jeden Sokrates, człowiek mądry, i~na gwiazdach~się rozumie, i~co
% pod ziemią, to on wszystko wybadał, i~ze słabszego zdania robi
% mocniejsze. Obywatele, to oni, to ci, co o~mnie takie pogłoski
% porozsiewali, to są moi straszni oskarżyciele. Bo kto słyszy, ten
% myśli, że~tacy badacze to nawet w~bogów nie wierzą. A~potem jest
% takich oskarżycieli wielu i~już długi czas skarżą, a oprócz tego
% jeszcze w~takim wieku do was mówią, w~którymeście uwierzyć mogli
% najłatwiej, dziećmi będąc, a~niejeden z~was wreszcie młodym chłopcem;
% po prostu taka zaoczna skarga, bez żadnej obrony.

% A~ze wszystkiego najgłupsze to, że~nawet nazwiska ich nie~można znać
% ani ich wymienić. Chyba że~przypadkiem który jest komediopisarzem.

Jedni z~zazdrości potwarzy w~uszy wam nakładli, drudzy uwierzyli
i~z~przekonania zrażają do mnie innych, a~ze wszystkim nieporadna
godzina. Bo ani~ich tutaj przed sąd nie może pociągnąć, ani~rozumnie
przekonać żadnego, tylko po prostu tak człowiek musi niby z~cieniami
walczyć; broni~się i~zbija zarzuty, a~nikt nie odpowiada.

% \attribA{Platon „Obrona Sokratesa”, 18 B, C, D.}
% ##################

\vspace{\spaceThree}



% ##################
A tu jest jeszcze inny taki obywatel z~Paros, mędrzec;
dowiedziałem~się niedawno, że~przyjechał, bom przypadkiem spotkał
jednego obywatela, który zapłacił sofistom więcej pieniędzy niż
wszyscy inni razem, Kalliasza syna Hipponika. Więc ja go zapytałem --
bo on ma dwóch synów. ,,Kaliaszu -- powiadam -- jak by ci~się tak byli
twoi dwaj synowie źrebakami albo cielętami porodzili, to my byśmy
umieli wyszukać im kierownika i~zgodzić go, żeby z~nich zrobiły piękne
i~dobre sztuki we właściwym im rodzaju zalet. I~to by był albo jakiś
człowiek od koni, albo od roli. No teraz, skoro~są ludźmi, to kogo om
zamyślasz wziąć na kierownika? Kto~się tak rozumie na zaletach
człowieka i~obywatela? Myślę przecież, żeś~ty~się nad tym zastanowił,
bo masz synów. Jeśli ktoś taki -- mówię mu -- czy nie?

„A~no, pewnie” -- powiada.

„Któż taki -- mówię -- i~skąd on, i~po czemu uczy?”

„Euenos -- powiada -- Sokratesie, ten z~Paros, po pięć min”.

% \attribA{Platon „Obrona Sokratesa”, 20 A, B.}
% ##################

\vspace{\spaceThree}



% ##################
Poszedłem do kogoś z~tych, którzy uchodzą za mądrych, aby jeśli gdzie,
to~tam przekonać wyrocznię, że~się~myli, i~wykazać jej, że~ten oto tu
jest mądrzejszy ode mnie, a~tyś powiedziała, że~ja.

Więc, kiedy~się tak w~nim rozglądam -- nazwiska wymieniać nie mam
potrzeby: to był ktoś spośród polityków, który na mnie takie jakieś
z~bliska zrobił wrażenia, obywatele -- otóż, kiedym tak z~nim
rozmawiał, zaczęło mi~się zdawać, że~ten obywatel wydaje~się mądrym
wielu innym ludziom, a~najwięcej sobie samemu, a~jest? Nie! A~potem
próbowałem mu wykazać, że~się tylko uważa za mądrego, a~nie jest nim
naprawdę. No i~stąd mnie nienawidził i~on, i~wielu z~tych, co przy tym
byli.

Wróciwszy do domu zacząłem miarkować, że~od tego człowieka jednak
jestem mądrzejszy. Bo z~nas dwóch żaden, zdaje~się, nie wie o~tym, co
piękne i~dobre; ale jemu~się zdaje, że~coś wie, choć nic nie wie,
a~ja, jak nic nie wiem, tak mi~się nawet i~nie zdaje. Więc może o~tę
właśnie odrobinę jestem od niego mądrzejszy, że~jak czego nie wiem,
to~i~nie myślę, że~wiem.

Stamtąd poszedłem do innego, który~się wydawał mądrzejszy niż tamten,
i~znowu takie samo odniosłem wrażenie. Tu znowu mnie ten znienawidził
i~wielu innych ludzi.

Więc potem, tom już po kolei dalej chodził, choć wiedziałem, i~bardzo
mnie to martwiło i~niepokoiło, że~mnie zaczynają nienawidzić, a~jednak
mi~się koniecznym wydawało to, co bóg powiedział, stawiać nade
wszystko.

Trzeba było iść dalej, dojść, co ma znaczyć wyrocznia, iść do
wszystkich, którzy wyglądali na to, że~coś wiedzą. I~dalipies,
obywatele -- bo przed wami potrzeba prawdę mówić -- ja, doprawdy,
odniosłem takie jakieś wrażenie: ci, którzy mieli najlepszą opinię,
wydali mi~się bodaj że~największymi nędzarzami, kiedym tak za wolą
boską robił poszukiwania, a~inni, lichsi z~pozoru, byli znacznie
przyzwoitsi, naprawdę, co do porządku w~głowie.

Muszę wam jednak moją wędrówkę opisać; jakiem ja trudy podejmował, aby
w~końcu przyznać słuszność wyroczni.

Otóż po rozmowach z~politykami poszedłem do poetów, tych, co to
tragedię piszą i~dytyramby, i~do innych, żeby~się tam na miejscu
niezbicie przekonać, żem~głupszy od nich.

Brałem tedy do ręki ich poematy, zdawało~się najbardziej opracowane
i~bywało, rozpytywało ich o~to, co chcą właściwie powiedzieć, aby~się
przy tej sposobności też i~czegoś od nich nauczyć. Wstydzę~się wam
prawdę powiedzieć, obywatele, a~jednak powiedzieć potrzeba. Więc
krótko mówiąc: nieledwie wszyscy inni, z~boku stojący, umieli lepiej
niż sami poeci mówić o~ich własnej robocie.

Więc i~o~poetach~się przekonałem niedługo, że~to, co oni robią, to nie
z~mądrości płynie, tylko z~jakiejś przyrodzonej zdolności, z~tego,
że~w~nich bóg wstępuje, jak w~wieszczków i~wróżbitów; ci także mówią
wiele pięknych rzeczy, tylko nic z~tego nie wiedzą, co mówią. Zdaje
mi~się, że~coś takiego dzieje~się i~z~poetami. A~równocześnie
zauważyłem, że~oni przez tę poezję uważają~się za najmądrzejszych
ludzi i~pod innymi względami, a~wcale takimi nie~są. Więc i~od nich
odszedłem, uważając, że~tym samym ich przewyższam, czym i~polityków.

W~końcu zwróciłem~się do rzemieślników. Bo sam zdawałem sobie
doskonale sprawę z~tego, że~nic nie wiem, a~u~tych wiedziałem,
że~znajdę wiedzę o~różnych pięknych rzeczach. I~nie pomyliłem~się. Ci
wiedzieli rzeczy, których ja nie wiedziałem, i~tym byli mądrzejsi ode
mnie. Ale znowu, obywatele, wydało mi~się, że~dobrzy rzemieślnicy
popełniają ten sam grzech, co i~poeci. Dlatego że~swoją sztukę dobrze
wykonywał, myślał każdy, że~jest bardzo mądry we wszystkich innym,
nawet w~największych rzeczach, i~ta ich wada rzucała cień na ich
mądrość.

Tak żem~się zaczął sam siebie pytać zamiast wyroczni, co bym wolał:
czy zostać tak jak jestem i~obejść~się bez ich mądrości, ale i~bez tej
ich głupoty, czy mieć jedno i~drugie, jak oni. Odpowiedziałem i~sobie,
i~wyroczni, że~mi~się lepiej opłaci zostać tak, jak jestem.

Z~tych tedy dochodzeń i~badań, obywatele, liczne~się porobiły
nieprzyjaźnie, i~to straszne, i~bardzo ciężkie, tak~że~stąd i~potwarze
poszły, i~to imię stąd, że~to mówią: mądry jest. Bo zawsze ci, co
z~boku stoją, myślą, że~ja sam jestem mądry w~tym, w~czym mi~się kogo
trafi położyć w~dyskusji.

A~to naprawdę podobno bóg jest mądry i~w~tej wyroczni to chyba mówi, że~ludzka mądrość mało co jest warta albo nic. I~zdaje~się, że~mu nie o~Sokratesa chodzi, a~tylko użył mego imienia, dając mnie na przykład, jak by mówił, że~ten z~was, ludzie, jest najmądrzejszy, który, jak Sokrates, poznał, że~nic nie jest naprawdę wart, tam gdzie chodzi o~mądrość.

Ja jeszcze i~dziś chodzę i~szukam tego, i~myszkuję, jak bóg nakazuje, i~między mieszczanami naszymi, i~między obcymi, jeżeli mi~się który mądry wydaje, a~jak mi~się który mądry wydaje, to zaraz bogu pomagam i~dowodzę takiemu, że~nie jest mądry. I~to mi tyle czasu zabiera, że~ani nie miałem kiedy w~życiu obywatelskim zrobić czegoś, o~czym by warto było mówić, ani~koło własnych interesów chodzić; ostatnią biedę klepię przez tę służbę bożą.

A~oprócz tego chodzą za mną młodzi ludzie, którzy najwięcej mają
wolnego czasu, synowie co najbogatszych obywateli; nikt im chodzić nie
każe, ale oni lubią słuchać, jak~się to ludzi bada, a~nieraz mnie
naśladują na własną rękę i~próbują takich badań na innych.

Pewnie -- znajdują mnóstwo takich, którym~się zdaje, że~coś wiedzą,
a~wiedzą mało albo wcale nic. Więc stąd ci, których oni na spytki
biorą, gniewają~się na mnie, a~nie na nich: mówią, że~to ostatni
łajdak ten jakiś Sokrates i~psuje młodzież. A~jak ktoś pyta, co on robi takiego i~czego on naucza, nie umieją nic powiedzieć, nie wiedzą; żeby zaś~pokryć zakłopotanie, mówią to, co~się na każdego miłośnika wiedzy zaraz mówi: że~tajemnice nieba odsłania i~ziemi, a~bogów nie~szanuje, a~z~gorszego zdania robi lepsze.

Bo prawdy żaden by chyba nie~powiedział: że~się ich niewiedzę odsłania i~udawanie mądrości. A~że~im widać na poważaniu zależy, a~zaciekli~są i~dużo ich jest, a~systematycznie i~przekonująco na mnie wygadują, więc macie pełne uszy ich potwarzy, rzucanych na mnie od~dawna a~zajadle. Spośród nich też wyszli na mnie Meletos, Anytos i~Lykon. Meletos~się obraził za poetów, Antytos za rzemieślników i~polityków, a~Lykon za mówców.

% \attribA{Platon „Obrona Sokratesa”, 21 B, C, D, E; 22; 23.}
% ##################

\vspace{\spaceThree}













% #####################################################################
% #####################################################################
% Bibliografia
\bibliographystyle{plalpha}

\bibliography{MathComScienceBooks}{}





% ############################

% Koniec dokumentu
\end{document}
