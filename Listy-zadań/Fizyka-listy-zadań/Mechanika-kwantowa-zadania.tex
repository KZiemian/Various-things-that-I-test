% Autor: Kamil Ziemian

% --------------------------------------------------------------------
% Podstawowe ustawienia i pakiety
% --------------------------------------------------------------------
\RequirePackage[l2tabu, orthodox]{nag} % Wykrywa przestarzałe i niewłaściwe
% sposoby używania LaTeXa. Więcej jest w l2tabu English version.
\documentclass[a4paper,11pt]{article}
% {rozmiar papieru, rozmiar fontu}[klasa dokumentu]
\usepackage[MeX]{polski} % Polonizacja LaTeXa, bez niej będzie pracował
% w języku angielskim.
\usepackage[utf8]{inputenc} % Włączenie kodowania UTF-8, co daje dostęp
% do polskich znaków.
\usepackage{lmodern} % Wprowadza fonty Latin Modern.
\usepackage[T1]{fontenc} % Potrzebne do używania fontów Latin Modern.



% ----------------------------
% Podstawowe pakiety (niezwiązane z ustawieniami języka)
% ----------------------------
\usepackage{microtype} % Twierdzi, że poprawi rozmiar odstępów w tekście.
\usepackage{graphicx} % Wprowadza bardzo potrzebne komendy do wstawiania
% grafiki.
\usepackage{verbatim} % Poprawia otoczenie VERBATIME.
\usepackage{textcomp} % Dodaje takie symbole jak stopnie Celsiusa,
% wprowadzane bezpośrednio w tekście.
\usepackage{xcolor}
\usepackage{vmargin} % Pozwala na prostą kontrolę rozmiaru marginesów,
% za pomocą komend poniżej. Rozmiar odstępów jest mierzony w calach.
% ----------------------------
% MARGINS
% ----------------------------
\setmarginsrb
{ 0.7in} % left margin
{ 0.6in} % top margin
{ 0.7in} % right margin
{ 0.8in} % bottom margin
{  20pt} % head height
{0.25in} % head sep
{   9pt} % foot height
{ 0.3in} % foot sep



% ------------------------------
% Często przydatne pakiety
% ------------------------------
% \usepackage{csquotes} % Pozwala w prosty sposób wstawiać cytaty do tekstu.
% \usepackage{xcolor} % Pozwala używać kolorowych czcionek (zapewne dużo
% więcej, ale ja nie potrafię nic o tym powiedzieć).



% ------------------------------
% Pakiety do tekstów z nauk przyrodniczych
% ------------------------------
\let\lll\undefined % Amsmath gryzie się z językiem pakietami do języka
% polskiego, bo oba definiują komendę \lll. Aby rozwiązać ten problem
% oddefiniowuję tę komendę, ale może tym samym pozbywam się dużego Ł.
\usepackage[intlimits]{amsmath} % Podstawowe wsparcie od American
% Mathematical Society (w skrócie AMS)
\usepackage{amsfonts, amssymb, amscd, amsthm} % Dalsze wsparcie od AMS
% \usepackage{siunitx} % Do prostszego pisania jednostek fizycznych
% \usepackage{upgreek} % Ładniejsze greckie litery
% Przykładowa składnia: pi = \uppi
% \usepackage{slashed} % Pozwala w prosty sposób pisać slash Feynmana.
% \usepackage{calrsfs} % Zmienia czcionkę kaligraficzną w \mathcal
% na ładniejszą. Może w innych miejscach robi to samo, ale o tym nic
% nie wiem.



% ----------
% Tworzenie otoczeń "Twierdzenie", "Definicja", "Lemat", etc.
% ----------
\newtheorem{twr}{Twierdzenie} % Komenda wprowadzająca otoczenie
% ,,twr'' do pisania twierdzeń matematycznych
\newtheorem{defin}{Definicja} % Analogicznie jak powyżej
\newtheorem{wni}{Wniosek}



% ----------------------------
% Pakiety napisane przez użytkownika.
% Mają być w tym samym katalogu to ten plik .tex
% ----------------------------
% \usepackage{} % Pakiet napisany między innymi
% dla tego pliku.
\usepackage{latexgeneralcommands}
% \usepackage{mathshortcuts}




% --------------------------------------------------------------------
% Dodatkowe ustawienia dla języka polskiego
% --------------------------------------------------------------------
\renewcommand{\thesection}{\arabic{section}.}
% Kropki po numerach rozdziału (polski zwyczaj topograficzny)
\renewcommand{\thesubsection}{\thesection\arabic{subsection}}
% Brak kropki po numerach podrozdziału



% ----------------------------
% Ustawienia różnych parametrów tekstu
% ----------------------------
\renewcommand{\arraystretch}{1.2} % Ustawienie szerokości odstępów między
% wierszami w tabelach.



% ----------------------------
% Pakiet "hyperref"
% Polecano by umieszczać go na końcu preambuły.
% ----------------------------
\usepackage{hyperref} % Pozwala tworzyć hiperlinki i zamienia odwołania
% do bibliografii na hiperlinki.










% ---------------------------------------------------------------------
% Tytuł tekstu
\title{Mechanika Newtona, zadania}

% \author{}
% \date{}
% ---------------------------------------------------------------------










% ####################################################################
\begin{document}
% ####################################################################





% ######################################
\maketitle % Tytuł całego tekstu
% ######################################





% ############################
\Work{ % Autor i tytuł dzieła
  Ramamurti Shankar \\
  „Mechanika kwantowa”, \cite{} }


\noindent
\romannumeral1) 1) 1, 2, 3, 4, 5. 3) 1, 2, 3, 4. 4) 1, 2. 6) 1, 2, 3,
4, 5, 6. 7) 1, 2. 8) 1, 2, 3, 4, 5, 6, 7, 8, 9, 10, 11, 12. 9) 1, 2,
3. 10) 1, 2, 3, 4.

\vspace{\spaceFour}



\noindent
\romannumeral2) 1) 1, 2, 3. 5) 1, 2, 3, 4. 7) 1, 2, 3, 4, 5, 6, 7, 8,
9. 8) 1, 2, 3, 4, 5, 6, 7.



\noindent
\romannumeral4) 2) 1, 2, 3.

\vspace{\spaceFour}



\noindent
\romannumeral5) 1) 1, 2, 3, 4. 2) 1, 2, 3, 4, 5, 6. 3) 1, 2, 3, 4. 4)
1, 2, 3.

\vspace{\spaceFour}



\noindent
\romannumeral7) 3) 1, 2, 3, 4, 5, 6, 7. 4) 1, 2, 3, 4, 5, 6, 7, 8, 9, 10.
5) 1, 2, 3, 4, 5.

\vspace{\spaceFour}



\noindent
\romannumeral8) 6) 1, 2, 3, 4.

\vspace{\spaceFour}



\noindent
\romannumeral9) 4) 1, 2, 3, 4.

\vspace{\spaceFour}



\noindent
\romannumeral10) 1) 1, 2. 2) 1, 2, 3. 3) 1, 2, 3, 4, 5, 6.

\vspace{\spaceFour}



\noindent
\romannumeral11) 2) 1, 2. 4) 1, 2, 3, 4.

\vspace{\spaceFour}



\noindent
\romannumeral12) 1) 1. 2) 1, 2, 3, 4. 3) 1, 2, 3, 4, 5, 6, 7, 8. 5) 1,
2, 3, 4, 5, 6, 7, 8, 9, 10, 11, 12, 13, 14. 6) 1, 2, 3, 4, 5, 6, 7, 8,
9, 10, 11.

\vspace{\spaceFour}



\noindent
\romannumeral13) 1) 1, 2, 3, 4, 5. 3) 1, 2, 3. 4) 1, 2, 3.

\vspace{\spaceFour}



\noindent
\romannumeral14) 3) 1, 2, 3, 4, 5, 6, 7, 8. 4) 1, 2, 3, 4, 5, 6. 5) 1,
2, 3, 4.

\vspace{\spaceFour}



%% \item[--] \romannumeral15. 1: 1, 2.\\
%%   2: 1, 2, 3, 4, 5, 6, 7.\\
%%   3: 1, 2, 3, 4, 5.
%% \item[--] \romannumeral16. 1: 1, 2, 3, 4, 5.\\
%%   2: 1, 2, 3, 4, 5, 6, 7, 8.
%% \item[--] \romannumeral17. 2: 1, 2, 3, 4, 5, 6, 7, 8.\\
%%   3: 1, 2, 3, 4.
%% \item[--] \romannumeral18. 2: 1, 2, 3, 4, 5, 6.\\
%%   3: 1, 2.\\
%%   4: 1, 2, 3, 4.\\
%%   5: 1,2.
%% \item[--] \romannumeral19. 3: 1, 2, 3, 4.\\
%%   5: 1, 2, 3, 4, 5, 6.\\
%%   6: 1, 2, 3.
%% \item[--] \romannumeral20. 1: 1.\\
%%   2: 1, 2.
%% \item[--] \romannumeral21. 1: 1, 2, 3, 4, 5, 6, 7, 8, 9, 10, 11,
%%   12, 13, 14, 15, 16, 17, 18, 19,
%%   20, 21.\\
%%   2: 1, 2, 3, 4, 5, 6, 7, 8, 9, 10.\\
%%   3: 1, 2, 3, 4, 5, 6, 7, 8, 9, 10, 11.
% \end{itemize}










% ####################################################################
% ####################################################################
% Bibliografia
\bibliographystyle{plalpha}

\bibliography{PhilNaturBooks}{}





% ############################

% Koniec dokumentu
\end{document}