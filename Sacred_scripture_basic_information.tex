% ---------------------------------------------------------------------
% Podstawowe ustawienia i pakiety
% ---------------------------------------------------------------------
\RequirePackage[l2tabu, orthodox]{nag} % Wykrywa przestarzałe i niewłaściwe
% sposoby używania LaTeXa. Więcej jest w l2tabu English version.
\documentclass[a4paper,11pt]{article}
% \usepackage{babel}
% {rozmiar papieru, rozmiar fontu}[klasa dokumentu]
% w języku angielskim.
% \usepackage[utf8]{inputenc} % Włączenie kodowania UTF-8, co daje dostęp
% % do polskich znaków.
\usepackage{fontspec}
\setmainfont{Latin Modern Roman}

\usepackage{polyglossia}
\setdefaultlanguage{english}
\setotherlanguage{hebrew}


\newfontfamily{\hebrewfont}{FreeSerif}[
  Extension=.otf,
  Path=/usr/local/texlive/2022/texmf-dist/fonts/opentype/public/gnu-freefont/,
  UprightFont=*,
  ItalicFont=*Italic,
  BoldFont=*Bold,
  BoldItalicFont=*BoldItalic,
]





% ------------------------------
% Podstawowe pakiety (niezwiązane z ustawieniami języka)
% ------------------------------
\usepackage{microtype} % Twierdzi, że poprawi rozmiar odstępów w tekście.
\usepackage{graphicx} % Wprowadza bardzo potrzebne komendy do wstawiania
% grafiki.
\usepackage{verbatim} % Poprawia otoczenie VERBATIME.
\usepackage{textcomp} % Dodaje takie symbole jak stopnie Celsiusa,
% wprowadzane bezpośrednio w tekście.
\usepackage{vmargin} % Pozwala na prostą kontrolę rozmiaru marginesów,
% za pomocą komend poniżej. Rozmiar odstępów jest mierzony w calach.
% ------------------------------
% MARGINS
% ------------------------------
\setmarginsrb
{ 0.7in}  % left margin
{ 0.6in}  % top margin
{ 0.7in}  % right margin
{ 0.8in}  % bottom margin
{  20pt}  % head height
{0.25in}  % head sep
{   9pt}  % foot height
{ 0.3in}  % foot sep



% ------------------------------
% Często przydatne pakiety
% ------------------------------
\usepackage{csquotes} % Pozwala w prosty sposób wstawiać cytaty do tekstu.
\usepackage{xcolor} % Pozwala używać kolorowych czcionek (zapewne dużo
% więcej, ale ja nie potrafię nic o tym powiedzieć).



% ------------------------------
% Pakiety do tekstów z nauk przyrodniczych
% ------------------------------
\let\lll\undefined % Amsmath gryzie się z językiem pakietami do języka
% polskiego, bo oba definiują komendę \lll. Aby rozwiązać ten problem
% oddefiniowuję tę komendę, ale może tym samym pozbywam się dużego Ł.
\usepackage[intlimits]{amsmath} % Podstawowe wsparcie od American
% Mathematical Society (w skrócie AMS)
\usepackage{amsfonts, amssymb, amscd, amsthm} % Dalsze wsparcie od AMS
% \usepackage{siunitx} % Dla prostszego pisania jednostek fizycznych
\usepackage{upgreek} % Ładniejsze greckie litery
% Przykładowa składnia: pi = \uppi
\usepackage{slashed} % Pozwala w prosty sposób pisać slash Feynmana.
\usepackage{calrsfs} % Zmienia czcionkę kaligraficzną w \mathcal
% na ładniejszą. Może w innych miejscach robi to samo, ale o tym nic
% nie wiem.



% ##########
% Tworzenie otoczeń "Twierdzenie", "Definicja", "Lemat", etc.
\newtheorem{theorem}{Twierdzenie}  % Komenda wprowadzająca otoczenie
% „theorem” do pisania twierdzeń matematycznych
\newtheorem{definition}{Definicja}  % Analogicznie jak powyżej
\newtheorem{corollary}{Wniosek}



% ---------------------------------------
% Pakiety napisane przez użytkownika.
% Mają być w tym samym katalogu to ten plik .tex
% ---------------------------------------
% \usepackage{latexgeneralcommands}
% \usepackage{mathcommands}
% % \usepackage{calculuscommands}
% \newcommand{\conca}{\textrm{conca}}
% \usepackage{SchwartzBooksCommands}  % Pakiet napisany m.in. dla tego pliku.



% ---------------------------------------------------------------------
% Dodatkowe ustawienia dla języka polskiego
% ---------------------------------------------------------------------
\renewcommand{\thesection}{\arabic{section}.}
% Kropki po numerach rozdziału (polski zwyczaj topograficzny)
\renewcommand{\thesubsection}{\thesection\arabic{subsection}}
% Brak kropki po numerach podrozdziału



% ------------------------------
% Ustawienia różnych parametrów tekstu
% ------------------------------
\renewcommand{\arraystretch}{1.2} % Ustawienie szerokości odstępów między
% wierszami w tabelach.



% ------------------------------
% Pakiet "hyperref"
% Polecano by umieszczać go na końcu preambuły.
% ------------------------------
\usepackage{hyperref} % Pozwala tworzyć hiperlinki i zamienia odwołania
% do bibliografii na hiperlinki.










% ---------------------------------------------------------------------
% Tytuł i autor tekstu
\title{Basic information about languages in Sacred Scipture}


% \date{}
% ---------------------------------------------------------------------










% ####################################################################
\begin{document}
% ####################################################################





% ######################################
\maketitle % Tytuł całego tekstu
% ######################################





% ######################################
\section{Languages} % Tytuł całego tekstu
% ######################################



Basic fact. We are using Latin alphabet, which was created by the
Romans\footnote{Name \textit{latin}, comes from the fact that region of the
  city of Rome was called Latium.}. Bible was written in mainly three
languages each one have its own alphabet, so they basically unreadable to
the most of us. These languages are Old Hebrew, Old Aramaic and one of the
version of Greek language known as Koine Greek or Alexandrian Greek. Bible
was written over the span that is usually count as 1000 years, so languages
changes during this time. But this is separate topic

Try to read John Milton now and he is only 400 or so years old.

Old Hebrew and Old Aramaic are similar in few ways. Both you read from
right-to-left and they alphabet consists of symbols only for consonants.
This is a bit more complicated when you encounter things like Hebrew letter
\begin{hebrew} א \end{hebrew}, but basically, there is no vowels in the
text, reader must to add it himself/herself.

Around 900 years after the time of Jesus, things called \textit{nekkudots}
that show which vowel is where, was added to the Hebrew script. But things
are far more complicated.

Basic Old Hebrew alphabet: \begin{hebrew} א \end{hebrew} (alef),
\begin{hebrew} ב‎ \end{hebrew} (bet), \begin{hebrew} ג‎ \end{hebrew} (gimel),
\begin{hebrew} ד‎ \end{hebrew} (dalet), \begin{hebrew} ה‎ \end{hebrew} (he),
\begin{hebrew} ו‎ \end{hebrew} (waw), \begin{hebrew} ז \end{hebrew} (zayin),
\begin{hebrew} ח \end{hebrew} (chet), \begin{hebrew} ט‎ \end{hebrew} (tet),
\begin{hebrew} י \end{hebrew} (yod), \begin{hebrew} כ‎, ך \end{hebrew} (kaf),
\begin{hebrew} ל‎ \end{hebrew} (lamed), \begin{hebrew} מ, ם \end{hebrew}
(mem), \begin{hebrew} נ, ן‎ \end{hebrew} (nun), \begin{hebrew} ס‎ \end{hebrew}
(samech), \begin{hebrew} ע \end{hebrew} (ayin),
\begin{hebrew} ע, ף‎ \end{hebrew} (pe), \begin{hebrew} צ, ץ‎ \end{hebrew}
\begin{hebrew} ק \end{hebrew} (qof), \begin{hebrew} ר \end{hebrew} (resh),
\begin{hebrew} ש \end{hebrew} (shin), \begin{hebrew} ת \end{hebrew} (tav).

Very crude way of reading Hebrew: for letters other than
\begin{hebrew} א, ע‎ \end{hebrew} read first consonant in the name. Again,
you basically must remember where and which vowels are in the world.

I won't show you Aramaic alphabet, because it was already hard to make my
computer accept Hebrew alphabet. You can find it e.g. on Wikipedia page
``Aramaic alphabet''.

Basic Greek alphabet: A, $\alpha$ (alpha), B, $\beta$ (beta), $\Gamma$, $\gamma$ (gamma),
$\Delta$, $\delta$ (delta), E, $\varepsilon$ (epsilon), Z, $\zeta$ (zeta), H, $\eta$ (eta), $\Theta$, $\theta$
(theta), I, $\iota$ (iota), K, $\kappa$ (kappa), $\Lambda$, $\lambda$ (lambda), M, $\mu$ (mu), N,
$\nu$ (nu), $\Xi$, $\xi$ (xi), O, o (omicron), $\Pi$, $\pi$ (pi), $\Sigma$, $\sigma$, $\xi$
(sigma), T, $\tau$ (tau), Y, $\upsilon$ (upsilon), $\Phi$, $\varphi$ (phi), X, $\chi$, $\Psi$, $\psi$
(psi), $\Omega$, $\omega$ (omega).

Crude way of reading Greek: just read first letter of the name.

There is much more to Hebrew and Greek that this letters. You can various
additional symbols, but since using them is beyond me, I will omit them.










% ######################################
\section{Names} % Tytuł całego tekstu
% ######################################



``Bible'' comes from Koine Greek $\tau \alpha$ $\beta \iota \beta \lambda \iota \alpha$, ``ta biblia'', meaning
``the books''\footnote{Note that in English we still have
  ``bibliography''.}. ``Sacred Scripture'' comes from Latin, it means
``sacred writing''\footnote{Note in English we have words as ``scribe'' and
  ``scribbling''.}.

Old Testament is called by Jews \begin{hebrew} תנך \end{hebrew} (Tanakh) or
\begin{hebrew} מקרא \end{hebrew} (Mikra). It is divided in three parts:
\begin{hebrew} תורה \end{hebrew} (Torah, en. ``Teachings''),
\begin{hebrew} נביאים \end{hebrew} (Nevi'im, en. ``Spokeperson'') and
\begin{hebrew} כתובים \end{hebrew} (Ketuvim, en. ``Writing'').

Torah \begin{hebrew} תורה \end{hebrew} contains Five Books of Moses. Nevi'im
\begin{hebrew} נביאים \end{hebrew} you can say it contains
prophets, but live is not so simple. I don't even try to characterize
Ketuvim \begin{hebrew} כתובים \end{hebrew}. Name ``Tanakh''
\begin{hebrew} תנך \end{hebrew} is quasi anagram of first letters of Tora,
Nevi'im and Ketuvim.

Five Books of Moses derives they Hebrew names from their first word.
\begin{itemize}

\item \begin{hebrew} בראשית \end{hebrew}, Bereshit, en. ``In the beginning''.
  In Greek $\Gamma \varepsilon \nu \varepsilon \sigma \iota \zeta$, Genesis, en. ``Creation''\footnote{Note words
    like ``genetics'' or ``genealogy''}.

\item \begin{hebrew} שמות \end{hebrew}, Shemot, en. ``Names''. In Greek E$\xi$o$\delta$o$\zeta$, Exodos, en. ``Exit''.

\item \begin{hebrew} וקרא \end{hebrew}, Vayikra, en. ``And He called''. In Greek $\Lambda \varepsilon \upsilon \iota \tau \iota \kappa$o$\nu$, Leutikon, en. ``Relating to the Levites''.

\item \begin{hebrew} \end{hebrew}

\end{itemize}





\begin{hebrew} א ב‎ \end{hebrew} (bet), \begin{hebrew} ג‎ \end{hebrew} (gimel),
\begin{hebrew} ד‎ \end{hebrew} (dalet), \begin{hebrew} ה‎ \end{hebrew} (he),
\begin{hebrew} ו‎ \end{hebrew} (waw), \begin{hebrew} ז \end{hebrew} (zayin),
\begin{hebrew} ח \end{hebrew} (chet), \begin{hebrew} ט‎ \end{hebrew} (tet),
\begin{hebrew} י \end{hebrew} (yod), \begin{hebrew} כ‎, ך \end{hebrew} (kaf),
\begin{hebrew} ל‎ \end{hebrew} (lamed), \begin{hebrew} מ, ם \end{hebrew},
\begin{hebrew} נ, ן‎ \end{hebrew} (nun), \begin{hebrew} ס‎ \end{hebrew}
(samech), \begin{hebrew} ע \end{hebrew} (ayin),
\begin{hebrew} ע, ף‎ \end{hebrew} (pe), \begin{hebrew} צ‎, ץ \end{hebrew}
(tsadi), \begin{hebrew} ק \end{hebrew} (Qof), \begin{hebrew} ר \end{hebrew}
(resh), \begin{hebrew} ש \end{hebrew} (shin), \begin{hebrew} ת \end{hebrew} (tav).




% ######################################
\section{Songs} % Tytuł całego tekstu
% ######################################



Depending on whom published your Bible,




% #####################################################################
% #####################################################################
% Bibliografia
\bibliographystyle{plalpha}

\bibliography{MathComScienceBooks}{}





% ############################

% Koniec dokumentu
\end{document}
