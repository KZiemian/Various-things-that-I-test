% ---------------------------------------------------------------------
% Basic configuration of Beamera and Jagiellonian
% ---------------------------------------------------------------------
\RequirePackage[l2tabu, orthodox]{nag}



\ifx\PresentationStyle\notset
\def\PresentationStyle{dark}
\fi



\documentclass[10pt,t]{beamer}
\mode<presentation>
\usetheme[style=\PresentationStyle,logoLang=Latin,logoColor=monochromaticJUwhite,JUlogotitle=yes]{jagiellonian}



% ---------------------------------------
% Configuration files of Jagiellonian loceted in catalog preambule
% ---------------------------------------
\input{./preambule/LanguageSettings/JagiellonianEnglishLanguageSettings.tex}
\input{./preambule/TextposConfiguration/TextposConfiguration.tex}

\input{./preambule/JagiellonianCustomizationGeneral.tex}
\input{./preambule/JagiellonianCustomizationCommands.tex}










% ---------------------------------------
% Packages, libraries and their configuration
% ---------------------------------------
\usepackage{mathcommands}





% ---------------------------------------
% Configuration for this particular presentation
% ---------------------------------------










% ---------------------------------------------------------------------
\title{Overview Heredegen's approach to Casimir effect}
\subtitle{Part I. Basics and two plate systems}
% Część druga nie została napisana, bo z uwagi na epidemię Covid-19
% pierwsza część nigdy nie została wygłoszona

\author{Kamil Ziemian \\
  \texttt{kziemianfvt@gmail.com}}


\institute{Uniwersytet Jagielloński w~Krakowie}

\date[14 March 2020]{Seminar~of Field Theory Department \\
  14 March 2020}
% ---------------------------------------------------------------------










% ####################################################################
% Początek dokumentu
\begin{document}
% ####################################################################





% ######################################
\maketitle % Tytuł całego tekstu
% ######################################





% ######################################
\begin{frame}
  \frametitle{Table of contents}


  \tableofcontents % Spis treści

\end{frame}
% ######################################










% ######################################
\section{Casimir formula and experiments}
% ######################################



% ##################
\begin{frame}
  \frametitle{State of Casimir effect}


  Ancient chines wisdom says: Casimir effect is independent of
  renormalization scheme.

  It is not ancient, it is not chines and maybe it is not a wisdom.

\end{frame}
% ##################





% ##################
\begin{frame}
  \frametitle{How well we empirical know Casimir force?}


  Due to my knowledge, there are many problems with finding exact
  experimentally form of Casimir force for two plates.
  \begin{itemize}

  \item Temperate is bigger that $0^{ \circ }$ K.

  \item Electromagnetic field don't vanish totally in real metal.

  \item Real metal plate is in fact system of small metallic crystals
    not one big crystal.

  \item Two plate system is hard for experiment, so we approximations
    by this system geometry of plate and half sphere.

  \item \ldots

  \end{itemize}

\end{frame}
% ##################










% ######################################
\section{Herdegen's framework for Casimir effect}
% ######################################



% ##################
\begin{frame}
  \frametitle{Herdegen framework for Casimir effect}


  Herdegen's approach to Casimir effect put it into framework of
  algebraic quantum field theory, in short AQFT (also know as local
  quantum physics). AQFT in oversimplified version can be stated as
  follow.

  Quantum system is defined by
  \begin{itemize}

  \item abstract $*$-algebra $\Acal$, e.g. algebra~of commutations
    relations between $\widehat{x}$ and $\widehat{p}$;

  \item representation $\pi$ of $\Acal$ in Hilbert space $\Hcal$.
    \begin{equation*}
      \pi: \Acal \to \Bcal( \Hcal )
    \end{equation*}
    Where $\Bcal( \Hcal )$ is algebra of \alert{bounded} operators
    on~$\Hcal$.

  \end{itemize}
  Physical situations can be compered only when they belong to the
  same algebra $\Acal$ and equivalent representation $\pi: \Acal \to \Hcal$.

\end{frame}
% ##################





% ##################
\begin{frame}
  \frametitle{Herdegen framework for Casimir effect}


  When situations belonging to different algebras $\Acal$, $\Acal'$ or to
  nonequivalent representations $\pi: \Acal \to \Bcal( \Hcal )$,
  $\pi': \Acal \to \Bcal( \Hcal' )$ are compered often infinite's of
  \textbf{physical} origin appears.

  Consider quantum state living in infinite Minkowski spacetime with
  temperature $T = 0$ and state in the same spacetime at thermal
  equilibrium for $T = 1$. To arrive from first state to the second we
  need to pump up infinite energy to Minkowski space. Thus, this is
  physical infinite difference in energy.

  We call interaction \textbf{singular} if its introduction to
  the~system change its $*$-algebra $\Acal$ or representation to
  nonequivalent to $\pi': \Acal \to \Bcal( \Hcal )$.

  Herdgen point out in ???? that quantum field with Dirichlet
  conditions at plates belong to different algebra that free field, so
  comparison in ???? is from point of AQFT it is meaningless.

\end{frame}
% ##################





% ##################
\begin{frame}
  \frametitle{Physical assumptions}


  Let $H_{ 0 }$ be Hamiltonian of free field, $a$ be set of
  macroscopic parameters of system, like separations of the plates,
  $H_{ a }$ Hamiltonian of system with parameter $a$ fixed.

  We allow only adiabatic changes in the system. This implies that
  when macroscopic parameters change with time according to function
  $a( t )$,

\end{frame}
% ##################










% ######################################
\appendix
% ######################################





% ######################################
\EndingSlide{Thank you! Questions?}
% ######################################










% % ####################################################################
% % ####################################################################
% % Bibliografia
% \bibliographystyle{alpha}

% \bibliography{}{}





% ############################

% Koniec dokumentu
\end{document}
