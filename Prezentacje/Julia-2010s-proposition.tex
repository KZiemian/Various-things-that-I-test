% ---------------------------------------------------------------------
% Basic configuration of Beamera and Jagiellonian
% ---------------------------------------------------------------------
\RequirePackage[l2tabu, orthodox]{nag}



\ifx\PresentationStyle\notset
\def\PresentationStyle{dark}
\fi



\documentclass[10pt,t]{beamer}
\mode<presentation>
\usetheme[style=\PresentationStyle,logoLang=Latin,logoColor=monochromaticJUwhite,logoShape=D,JUlogotitle=yes]{jagiellonian}



% ---------------------------------------
% Configuration files of Jagiellonian loceted in catalog preambule
% ---------------------------------------
% \input{./preambule/LanguageSettings/JagiellonianPolishLanguageSettings.tex}
\input{./preambule/TextposConfiguration/TextposConfiguration.tex}

\input{./preambule/JagiellonianCustomizationGeneral.tex}
\input{./preambule/JagiellonianCustomizationCommands.tex}










% ---------------------------------------
% Packages, libraries and their configuration
% ---------------------------------------





% ---------------------------------------
% Configuration for this particular presentation
% ---------------------------------------










% ---------------------------------------------------------------------
\title{Julia}
\subtitle{2010s proposition for scientific (and other) programming}

\author{Kamil Ziemian \\
  \texttt{kziemianfvt@gmail.com}}

% \institute{Jagiellonian University in~Cracow}
\institute{Uniwersytet Jagielloński w~Krakowie}

\date[11 December 2018]{Seminarium astrofizyczne PAU \\
  11 grudnia 2019}
% ---------------------------------------------------------------------










% ####################################################################
% Początek dokumentu
\begin{document}
% ####################################################################





% ######################################
\maketitle % Tytuł całego tekstu
% ######################################





% ######################################
\begin{frame}
  \frametitle{Table of contents}


  \tableofcontents % Spis treści

\end{frame}
% ######################################










% ######################################
\section{Basic information}
% ######################################



% ##################
\begin{frame}
  \frametitle{Julia background information}


  \begin{itemize}

  \item Julia project had started in 2009, first release 0.1 in 2012,
    version 1.0 on 8~August~2018.

  \item Current (10~December~2019) stable version is~1.3.0 (release at 25
    November 2019).

  \item It is \alert{free} and \alert{open software} (GitHub
    \colorhref{https://github.com/JuliaLang/julia}
    {https://github.com/JuliaLang/julia}).

  \item Most of ecosystem if open and free, with \alert{some}
    exception. \\
    Eg.~you must pay for Julia backend to Microsoft Exel.

  \item Created by Jeff Bezanson, Alan Edelman, Stefan Karpinski,
    and~Viral B.~Shah, most of them work at~MIT at~some stages~of
    their life.

  \item Today one~of main center~of development~of the~language
    is~Julia Lab at~MIT, with Alan Edelman as it current head.

  \item They co-founded Julia Computing \emph{to develop products that
      make Julia easy to~use, easy to~deploy and easy to~scale}.

  \end{itemize}

\end{frame}
% ##################





% ##################
\begin{frame}
  \frametitle{Julia creators}


  \begin{figure}

    \centering

    \includegraphics[scale=0.30]{./PresentationPictures/JuliaCreators.png}

  \end{figure}


  \begin{center}

    Stefan Karpinski, Viral B.~Shah, Jeff Bezanson \\
    \hspace{-12em} and Alan Edelman

  \end{center}

\end{frame}
% ##################





% ##################
\begin{frame}
  \frametitle{LLVM and all of that}


  \begin{figure}

    \includegraphics[scale=0.16]{./PresentationPictures/LLVMLogo.png}

    \caption{LLVM logo}

  \end{figure}


  By Source, Fair use,
  \colorhref{https://en.wikipedia.org/w/index.php?curid=27377220}
  {https://en.wikipedia.org/w/index.php?curid=27377220}.


  Julia as Apple's Swift (version 1.0 from 2014) and Mozilla Rust
  (version 1.0 from 2015) is build on the top of LLVM, which is a set
  of compilers and toolchains technologies.

\end{frame}
% ##################





% ##################
\begin{frame}
  \frametitle{Aims of Julia (partialy by Alan Edelman)}


  Language that is
  \begin{itemize}

  \item easy to~use (write and~read);

  \item fast to~write;

  \item open source (science need to be open source!);

  \item high performance (who says interpreted must mean low
    performance?);

  \item easy to parallelize;

  \item both compiled and dynamical (not interpreted out of box)
    in~use, by~REPL (shell) or~notebook;

  \item in most cases almost seamless incorporation code~in FORTRAN,
    C, C++, Python, R, etc.;

  \item give native support for~GPU;

  \item and~easy access to~documentation.

  \end{itemize}

\end{frame}
% ##################





% ##################
\begin{frame}
  \frametitle{Why not use Julia?}


  \begin{itemize}

  \item You are happy with tools you have now.

  \item Julia doesn't find libraries you need.

  \item You don't have time/want learn new programming language.

  \item You don't want to learn new style~of programming.

  \item There is~not enough good learning materials for~version~1.x.

  \item Still not enough user base.

  \item Packages ecosystem isn't mature.

  \item Low ``StackOverflow effect'': how likely answer for your
    question is~under the top link in~Google search results.

  \end{itemize}

\end{frame}
% ##################





% ##################
\begin{frame}
  \frametitle{Why not use Julia?}


  \begin{itemize}

  \item Since Julia is more general purpose language than MATLAB,
    Mathematica,~R, etc., makes finding some things a~little bit
    harder.

  \item No serious project/company use it. Please, \alert{wait} few
    minutes.

  \item You \alert{must} learn Julia's way to write good code, to
    achieve high speed~of it.

  \end{itemize}

\end{frame}
% ##################





% % ##################
% \begin{frame}
%   \frametitle{Why use Julia?}


%   \begin{itemize}
%   \item Allows high productivity~of writing code (compilers can make
%     more for~us, than in the 1990s).

%   \item Give you good opportunity to write readable code (Python
%     or~MATLAB like).

%   \item After you write your code, in \alert{most cases} you can
%     optimize it~to, achieve 35\%--~105\% speed~of FORTRAN, C or~C++.

%   \item Most~of the~time it allows a~seamless incorporation~of code
%     written in~FORTRAN, C, C++, Python,~R and~Java code.

%   \item Julia code is very reusable, even in strange situations.
%     And~in many situation work just as~you want.

%   \item You can write code closely to~Python way with~REPL (shell like
%     thing) and~``scripting'' or~in~browser (I~will \alert{show} this
%     in the~minute).

%   % \item Most languages we used are really old. They were created
%   %   in~1990s, 1980s, 1970s, 1960s and~1950s. They were designed for
%   %   very, very different computers, environments, to solve different
%   %   problems and~at~different stage~of computer science development.

%   \end{itemize}

% \end{frame}
% % ##################





% % ##################
% \begin{frame}
%   \frametitle{Why use Julia?}


%   \begin{itemize}
%   % \item Julia code is very reusable, even in strange situations.
%   %   And~in many situation work just as~you want.

%   % \item You can write code closely to~Python way with~REPL (shell like
%   %   thing) and~``scripting'' or~in~browser (I~will \alert{show} this
%   %   in the~minute).

%   % \item If there is a~small mistake in~MATLAB, you probably must learn
%   %   to live with that and walk around it. In Julia you can correct it
%   %   yourself on GitHub.

%   \item \emph{Is~Julia the~next big programming language? MIT thinks
%       so, as~version~1.0 lands}, see~full TechRepublic article from 2018
%     \colorhref{https://www.techrepublic.com/article/is-julia-the-next-big-programming-language-mit-thinks-so-as-version-1-0-lands/}{here}. \\
%     Updating \emph{Pan Tadeusz}: What American invent, Pole will like.

%     %   \item In Poland we often talking about modernization, so~this is
%     % good thing to watch what happened at~world level.

%   \item Community is~very welcoming and~helpful.

%   \item Very easy to use package manager.

%   \item Few more.

%   \end{itemize}

% \end{frame}
% % ##################





% % ##################
% \begin{frame}
%   \frametitle{There are lies, big lies and benchmarks}

%   \vspace{-1em}


%   \begin{figure}
%     \centering

%     \includegraphics[scale=0.29]
%     {./SeminarPictures/JuliaMicroBenchmarks.png}

%     \caption{\textbf{Warning!} Python code use numpy libraries that
%       is~written 52.8\% in C (state of GitHub repository at~4
%       January~2019).}
%   \end{figure}

% \end{frame}
% % ##################









% % % ##################
% % \begin{frame}
% %   \frametitle{Why use Julia?}


% %   \begin{itemize}
% %   \item In Poland we often talking about modernization, so~this is
% %     good thing to watch what happened at~world level.

% %   % \item New style of programming == new opportunities. You~can to some
% %   %   extant write domain specific language.

% %   % \item Community is~very welcoming and~helpful.

% %   \item Few more.

% %   \end{itemize}

% % \end{frame}
% % % ##################





% % ##################
% \begin{frame}
%   \frametitle{Bad code is~\alert{slow}, good code is~\alert{mostly}
%     fast}


%   I can't stress this point enough. Difference in~speed~of good
%   and~bad Julia code can be~in~\alert{orders~of magnitude}. And~you
%   \alert{must} learn Julia way of writing good code.

%   So~while you testing how fast is~Julia, make sure that you check
%   tips for~high performance.

%   I will give example of good and bad code in the next part of the
%   talk.

% \end{frame}
% % ##################





% % ##################
% \begin{frame}
%   \frametitle{What make Julia code good?}


%   \begin{itemize}
%   \item Write ``\alert{type-stable}'' functions.

%   \item Avoid global variables. There are exception to this.

%   \item Avoid changing the type of a variable.

%   \item Avoid containers with abstract type parameters.

%   \item Avoid fields with abstract type.

%   \item Access arrays in memory order, along columns.

%   \item More dots: Fuse vectorized operations.

%   \item \ldots
%   \end{itemize}

%   More on \colorhref{https://docs.julialang.org/en/v1.2/manual/performance-tips/}{Julia Documentation: Performance Tips}.

% \end{frame}
% % ##################





% % ##################
% \begin{frame}
%   \frametitle{Who use Julia?}


%   \begin{figure}
%     \includegraphics[scale=0.27]
%     {./SeminarPictures/BigPlayersUsingJulia.png}

%     \caption{From \colorhref{https://juliacomputing.com/}{Julia
%         Computing} page.}
%   \end{figure}


%   % List of institutions, to give some scope

%   % MIT Robot Locomotion Group, USA Federal Aviation Administration,
%   % Federal Reserve Bank of New York, The Brazilian National
%   % Institute for~Space Research, AOT Energy (energy trading), PSR
%   % (global electricity and~natural gas consulting, analytic
%   % and~technology firm), University of Auckland Electric Power
%   % Optimization Centre (The Milk Output Optimizer), Voxel8 (3D
%   % printing), University of Copenhagen, Peking University and
%   % Imperial College London collaboration on \emph{An~Anthropocene
%   % Map~of Genetic Diversity} (Science, vol 343, issue 6307, pp.
%   % 1532-1535), Path BioAnalytics (medicine).

%   % See for more cases and details here
%   % \colorlink{https://juliacomputing.com/case-studies/} and here
%   % \colorlink{https://www.forbes.com/sites/suparnadutt/2017/09/20/this-startup-created-a-new-programming-language-now-used-by-the-worlds-biggest-companies/\#7f322c907de2}.

% \end{frame}
% % ##################





% % ##################
% \begin{frame}
%   \frametitle{Celeste.jl}


%   Project written at~National Energy Research Scientific Computing
%   Center (NERSC) at Lawrence Berkeley National Laboratory (Berkeley
%   Lab). Aim: analyzing data from 35\% of visible sky gathered by~Sloan
%   Digital Sky Survey.

%   Peak performance of 1.54 petaflops ($10^{ 15 }$ flops) using 1.3
%   million threads on 9,300 Knights Landing nodes~of the Cori
%   supercomputer at~NERSC. They say that only assembler, C, C++ and
%   FORTRAN achieved previously over 1~petaflops performance.

%   Loaded approximately 178 terabytes of image data and give
%   parameters estimates for 188 millions stars and galaxies \\
%   in~14.6 minutes.

% \end{frame}
% % ##################





% % ##################
% \begin{frame}
%   \frametitle{Celeste.jl}


%   It is worth noting that this computation was performed before 2018,
%   so~it used 0.x version of the language (probably 0.4 or~0.5), which
%   had unstable syntax.

%   For more information see~article: \\
%   Jeffrey Regier, et~al.,
%   \emph{Cataloging the~Visible Universe through Bayesian Inference
%     at~Petascale},
%   \arXiv{https://arxiv.org/abs/1801.10277}{1801.10277 [cs.DC]}.

%   Celeste.jl GitHub repositor
%   \colorhref{https://github.com/jeff-regier/Celeste.jl}{https://github.com/jeff-regier/Celeste.jl}.

% \end{frame}
% % ##################





% % % ##################
% % \begin{frame}
% %   \frametitle{Gamedev in Julia is almost not existing}


% %   \begin{figure}
% %     \centering

% %     \includegraphics[scale=0.17]{./SeminarPictures/PaddleBattle.png}
% %   \end{figure}

% %   Nathan Daly write simple pong game using 96.6\% Julia and 3.5\%
% %   Shell code and SDL2 library (Simple DirectMedia Layer) to~show that
% %   if~you want, you~can do it quite easily. After that he put it on~Mac
% %   App Store with. Code of game is available
% %   \colorhref{https://github.com/NHDaly/PaddleBattleJL}{here}.

% % \end{frame}
% % % ##################










% % ######################################
% \section{Live coding in Julia}

% \addtocounter{framenumber}{1}
% % ######################################










% % ######################################
% \section{Lies, big lies and~PDE benchmarks}

% \addtocounter{framenumber}{1}
% % ######################################



% % ##################
% \begin{frame}
%   \frametitle{Solving Kuramoto-Sivashinsky PDE
%     in~$\beamermathcolorwhite{1 + 1}$ dimensions}


%   Following benchmarks were crated by~John F.~Gibson, Dept.
%   Mathematics and~Statistics, University~of New Hampshire, main
%   author~of \textbf{Channelflow}: \emph{a~set of high-level software
%     tools and~libraries for research in~turbulence in~channel
%     geometries} written in~C++, \colorlink{http://channelflow.org/}.
%   Full list~of people that contribute to~benchmarks is
%   \colorhref{https://github.com/johnfgibson/julia-pde-benchmark/blob/master/1-Kuramoto-Sivashinksy-benchmark.ipynb}{here}.

%   They were created in November 2019 using Julia 1.2.

% \end{frame}
% % ##################





% % ##################
% \begin{frame}
%   \frametitle{Solving Kuramoto-Sivashinsky PDE
%     in~$\beamermathcolorwhite{1 + 1}$ dimensions}


%   Kuramoto-Sivashinsky equation is nonlinear (more precisely:
%   semilinear) partial differential equation, which in $1 + 1$
%   dimension takes form

%   \begin{equation}
%     \label{eq:Julia-Proposition-01}
%     \pd{ }{ u( t, x ) }{ t } + \pd{ 4 }{ u( t, x ) }{ x }
%     + \pd{ 2 }{ u( t, x ) }{ x } + u( t, x ) \pd{ }{ u( t, x ) }{ x }
%     = 0.
%   \end{equation}

%   Full $1 + 3$ dimensional version~of this equation was proposed
%   to~\emph{model the~diffusion instabilities in~a~laminar flame
%     front}
%   (from
%   \colorhref{https://en.wikipedia.org/wiki/Kuramoto\%E2\%80\%93Sivashinsky_equation}{Wikipedia}).

% \end{frame}
% % ##################


















% % ##################
% \begin{frame}
%   \frametitle{Solving Kuramoto-Sivashinsky PDE
%     in~$\beamermathcolorwhite{1 + 1}$ dimensions}


%   \emph{The KS-CNAB2 benchmark algorithm is~a~simple numerical
%     integration scheme for~the~KS [Kuramoto-Sivashinsky] equation that
%     uses Fourier expansion in~space [All codes call this same Fourier
%     Transform external library.], collocation calculation~of
%     the~nonlinear term $u( t, x ) u_{ x }( t, x )$,
%     and~finite-differencing in~time, specifically 2nd-order
%     Crank-Nicolson Adams-Bashforth (CNAB2) timestepping.}

%   \flushright{John F. Gibson}

% \end{frame}
% % ##################





% % ##################
% \begin{frame}
%   \frametitle{Solving Kuramoto-Sivashinsky PDE
%     in~$\beamermathcolorwhite{1 + 1}$ dimensions}


%   \begin{figure}
%     \centering

%     \includegraphics[scale=0.22]{./SeminarPictures/KSResult.png}

%     \caption{Kuramoto-Sivashinsky heat evolution in~1~dimension.}
%   \end{figure}

% \end{frame}
% % ##################





% % ##################
% \begin{frame}
%   \frametitle{Chart with linear scale}


%   \begin{figure}
%     \centering

%     \includegraphics[scale=0.22]{./SeminarPictures/JFGLinearScale.png}

%     \caption{Results for $N_{ x }$ points on $x$ axis, CPU time in
%       seconds.}
%   \end{figure}


% \end{frame}
% % ##################





% % ##################
% \begin{frame}
%   \frametitle{Chart with logarithmic scale}


%   \begin{figure}
%     \centering

%     \includegraphics[scale=0.22]{./SeminarPictures/JFGLogarithmScale.png}

%     \caption{Results for $N_{ x }$ points on $x$ axis, CPU time in
%       seconds.}
%   \end{figure}

% \end{frame}
% % ##################





% % ##################
% \begin{frame}
%   \frametitle{Time for maximal
%     $\beamermathcolorwhite{N_{ x } = 2^{ 17 } = 131072}$ used
%     in~computation}


%   \begin{table}
%     \centering

%     \begin{tabular}{|l|r|r|}
%       \hline
%       Language & CPU time [s] & Ratio to~C \\
%       \hline
%       Fortran & 13.0 & 0.90 \\
%       \hline
%       C++ & 14.4 & 0.99 \\
%       \hline
%       C & 14.6 & 1.00 \\
%       \hline
%       Julia unrolled & 15.2 & 1.04 \\
%       \hline
%       Julia in-place & 15.5 & 1.06 \\
%       \hline
%       Julia naive & 25.9 & 1.77 \\
%       \hline
%       MATLAB & 26.8 & 1.83 \\
%       \hline
%       Python & 35.7 & 2.45 \\
%       \hline
%     \end{tabular}

%     \caption{From the~fastest to~the~slowest code.}
%   \end{table}

% \end{frame}
% % ##################





% % ##################
% \begin{frame}
%   \frametitle{Time over~line~of code}


%   \begin{figure}
%     \centering

%     \includegraphics[scale=0.22]{./SeminarPictures/JFGTimeOverCode.png}

%     \caption{Compression~of time and~lines~of code.}
%   \end{figure}

% \end{frame}
% % ##################





% % ##################
% \begin{frame}
%   \frametitle{Speed/lines~of code}


%   \begin{figure}
%     \centering

%     \includegraphics[scale=0.22]{./SeminarPictures/JFGSpeedOverCode.png}

%     \caption{Compression~of ratio~of speed to the lines~of code.}
%   \end{figure}

% \end{frame}
% % ##################










% % ######################################
% \section{Live coding with JuliaDiffEq}

% \addtocounter{framenumber}{1}
% % ######################################










% % ######################################
% \section{Closing remarks}

% \addtocounter{framenumber}{1}
% % ######################################



% % ##################
% \begin{frame}
%   \frametitle{Is someone doing\ldots}


%   \begin{itemize}
%   \item machine learning:
%     \colorhref{https://github.com/FluxML/Flux.jl}{Flux.jl};

%   \item deep learning:
%     \colorhref{https://github.com/denizyuret/Knet.jl}{Knet.jl};

%     \item dynamical systems: \colorhref{https://github.com/JuliaDynamics/DynamicalSystems.jl}{DynamicalSystems.jl};

%   \item optimization:
%     \colorhref{https://github.com/JuliaOpt/JuMP.jl}{JuMP.jl};

%   \item quantum algorithm design:
%     \colorhref{https://github.com/QuantumBFS/Yao.jl}{Yoa.jl};

%   \item quantum information:
%     \colorhref{https://github.com/iitis/QuantumInformation.jl}{QuantumInformation.jl};

%   \item biology: \colorhref{https://github.com/BioJulia}{BioJulia };

%     \item milk output optimization: \colorhref{https://github.com/odow/MOO}{MOO: the Milk Output Optimizer}.
%   \end{itemize}

%   Basically, google what you are interested and check if you find some project.

% \end{frame}
% % ##################





% % ##################
% \begin{frame}
%   \frametitle{Reflections}


%   \emph{Julia is still not great when you want import some libraries
%     and use them. It~is great when you must write new library.}

%   \flushright{Dr Chris Rackauckas, Applied Mathematics Instructor
%     at~MIT. \\ From~Julia Discourse.}


%   \flushleft{\textbf{My reflections}}

%   \begin{itemize}
%   \item Julia proved that it~can be~used to~make serious scientific
%     work in~various fields, including numerical computations relevant
%     to~field theory.

%   \item Pedagogical perspective. Learning and~using it~is comparable
%     to~Python (in my~opinion), which is~starting language for computer
%     scientists on~MIT and~at~the same time is~more deep as~computer
%     language and~have more computational power.

%   \item Have potential to~make money outside academic world.
%   \end{itemize}

% \end{frame}
% % ##################





% % ##################
% \begin{frame}
%   \frametitle{Remember}

%   \alert{If you use it in your research} creators of it ask you for
%   citing paper \textbf{Julia: A~Fresh Approach to Numerical
%     Computing}, Jeff Bezanson, Alan Edelman, Stefan Karpinski
%   and~Viral B.~Shah~(2017) SIAM Review, 59:~65--98 and to add your
%   paper to the list
%   \colorhref{https://julialang.org/publications/}{https://julialang.org/publications/}.

%   \vspace{1em}



%   \textbf{I want to thank these people for help and discusions}
%   \begin{itemize}
%   \item Tamas Papp;

%   \item John F. Gibson;

%   \item Yakir Luc Gagnon;

%   \item Antoine Levitt;

%   \item Krzysztof Musiał.
%   \end{itemize}

%   With most of them I converse on Julia Discourse,
%   \colorlink{https://discourse.julialang.org/}.


% \end{frame}
% % ##################










% % ######################################
% \appendix
% % ######################################





% % ##################
% \begin{frame}[standout]

%   {\color{jStrongWhite} Thank you. Questions?}

% \end{frame}
% % ##################











% % ######################################
% \section{Bibliography and~resources}
% % ######################################



% % ##################
% \begin{frame}
%   \frametitle{Relevant articles}


%   \begin{itemize}
%   \item Jeff Bezanson, Alan Edelman, Stefan Karpinski and~Viral
%     B.~Shah, \emph{Julia: A~Fresh Approach to Numerical Computing},
%     (2017) SIAM Review, 59:~65--98. doi:~10.1137/141000671,
%     url:~http://julialang.org/publications/julia-fresh-approach
%     -BEKS.pdf,
%     \colorlink{https://julialang.org/publications/julia-fresh-approach-BEKS.pdf}.

%   \item Chistopher Rackauckas and Qing~Nie,
%     \emph{DifferentialEquations.jl --~A~Performant and~Feature-Rich
%       Ecosystem for Solving Differential Equations in~Julia}.
%     Journal~of Open Research Software (2017), 5(1), p.~15. Doi:
%     http://doi.org/10.5334/jors.151,
%     \colorlink{https://openresearchsoftware.metajnl.com/articles/10.5334/jors.151/}.

%   % \item Keno Fischer, Elliot Saba, \emph{Automatic Full Compilation~of
%   %     Julia Programs and~ML Models to~Cloud TPUs},
%   %   \arXiv{https://arxiv.org/abs/1810.09868}{1810.09868}.

%   \item Nick Heath, \emph{Is~Julia the~next big programming
%       language? \\
%       MIT thinks so, as~version~1.0~lands}. \\
%     TechRepublic, August~29, 2018,
%     \colorlink{https://www.techrepublic.com/article/is-julia-the-next-big-programming-language-mit-thinks-so-as-version-1-0-lands/}.
%   \end{itemize}

% \end{frame}
% % ##################





% % ##################
% \begin{frame}
%   \frametitle{Netography}


%   \begin{itemize}
%   \item JuliaLang/julia -- Julia GitHub repository,
%     \colorlink{https://github.com/JuliaLang/julia}.

%   \item All code in all languages used for creating benchmarks for
%     Kuramoto-Sivashinsky PDE is~available on~GitHub
%     \colorhref{https://github.com/johnfgibson/julia-pde-benchmark}{johnfgibson/julia-pde-benchmark}.
%     Description of used numerical algorithm can be~founded
%     \colorhref{https://github.com/johnfgibson/julia-pde-benchmark/blob/master/1-Kuramoto-Sivashinksy-benchmark.ipynb}{here}.

%     % \item numpy/numpy -- \texttt{numpy} GitHub repository,
%     %   \colorlink{https://github.com/numpy/numpy}.

%   \item John F.~Gibson, \emph{Why~Julia}, GitHub
%     \colorhref{https://github.com/johnfgibson/whyjulia/blob/master/1-whyjulia.ipynb}{johnfgibson/whyjulia}.

%   \item Alan Edelman homepage
%     \colorhref{http://math.mit.edu/~edelman/}{http://math.mit.edu/$\sim$ edelman/}.


%   \end{itemize}

% \end{frame}
% % ##################





% % ##################
% \begin{frame}
%   \frametitle{Netography}


%   \begin{itemize}
%   \item Picture on slide \emph{Lies, big lies and benchmarks} taken from
%     page \emph{Julia Micro-Benchmarks},
%     \colorhref{https://julialang.org/benchmarks/}{https://julialang.org/benchmarks/}.

%   \item Picture of Julia creators taken from MIT News page: \colorhref{http://news.mit.edu/2018/julia-language-co-creators-win-james-wilkinson-prize-numerical-software-1226}{http://news.mit.edu/2018/julia-language-co-creators-win-james-wilkinson-prize-numerical-software-1226}.

%   \item Picture of companies using Julia taken from Julia Computing page \colorhref{https://juliacomputing.com/}{https://juliacomputing.com/}.
%   \end{itemize}

% \end{frame}
% % ##################





% % ##################
% \begin{frame}
%   \frametitle{Mentioned projects and articles}


%     \begin{itemize}
%     \item Channelflow (written in~C++),
%       \colorhref{http://channelflow.org/}{http://channelflow.org/}.

%     % \item DynamicalSystems.jl, GitHub
%     %   \colorhref{https://github.com/JuliaDynamics/DynamicalSystems.jl}{JuliaDynamics/DynamicalSystems.jl}.

%     % \item QuantumOptics.jl. Numerical framework for numerical
%     %   solving open quantum systems, more on this page
%     %   \colorhref{https://qojulia.org/}{https://qojulia.org/}.

%     % \item REPL (shell) for C++ (broken in Julia 1.x), GitHub
%     %   \colorhref{https://github.com/NHDaly/PaddleBattleJL}{Keno/Cxx.jl}.

%     \item Game and educational tool \emph{Paddle Battle}, GitHub
%       \colorhref{https://github.com/NHDaly/PaddleBattleJL}{NHDaly/PaddleBattleJL}.
%     \end{itemize}


% \end{frame}
% % ##################










% % ######################################
% \section{Additional information}
% % ######################################



% % ##################
% \begin{frame}
%   \frametitle{Some statistics}


%   \begin{itemize}
%   \item Google Scholar: 1043 papers about Julia or using it in research
%     (state at~October~2019). See also topic \emph{Research}
%     at~Julia Language page https://julialang.org/publications/,
%     \colorlink{https://julialang.org/publications/}.

%   \item TIOBE Index for December 2019: 43.

%   \item Used in some way in over 1,000 universities.

%   \item Ecosystem of over 3,000 packages for wide are of topics.

%   \item Over 10~millions downloads.

%   \item Over 80,000 GitHub stars for language and packages.

%   \item 130\% annual growth (based on~downloads).
%   \end{itemize}
%   Outside two first positions rest was taken from Julia
%   Computing and~dated from September~2019.

% \end{frame}
% % ##################





% % ##################
% \begin{frame}
%   \frametitle{Julia source code}


%   \begin{table}
%     \centering

%       \begin{tabular}{|l|r|r|}
%         \hline
%         Language & Percent of code \\
%         \hline
%         Julia & 68.6\% \\
%         \hline
%         C & 16.3\% \\
%         \hline
%         C++ & 9.9\% \\
%         \hline
%         Scheme & 3.2\% \\
%         \hline
%         Makefile & 0.7\% \\
%         \hline
%         LLVM & 0.4\% \\
%         \hline
%         Other & 0.9\% \\
%         \hline
%       \end{tabular}

%       \caption{Numbers from \colorhref{GitHub JuliaLang/julia}{https://github.com/JuliaLang/julia}, 3~October~2019.}

%     \end{table}

% \end{frame}
% % ##################





% % ##################
% \begin{frame}
%   \frametitle{Gamedev in Julia is almost not existing}


%   \begin{figure}
%     \centering

%     \includegraphics[scale=0.17]{./SeminarPictures/PaddleBattle.png}
%   \end{figure}

%   Nathan Daly write simple pong game using 96.6\% Julia and 3.5\%
%   Shell code and SDL2 library (Simple DirectMedia Layer) to~show that
%   if~you want, you~can do it quite easily. After that he put it on~Mac
%   App Store with. Code of game is available
%   \colorhref{https://github.com/NHDaly/PaddleBattleJL}{here}.

% \end{frame}
% % ##################





% % % ##################
% % \begin{frame}
% %   \frametitle{Few interesting things for which there is no time
% %   here}


% %   \begin{block}{Packages and projects}
% %     \begin{itemize}
% %     \item DynamicalSystems.jl, GitHub
% %       \colorhref{https://github.com/JuliaDynamics/DynamicalSystems.jl}{JuliaDynamics/DynamicalSystems.jl}.
% %       Winner~of one~of three main prizes of The~Dynamical System Web
% %       in~2018.
% %     \item QuantumOptics.jl. Numerical framework for numerical
% %       solving open quantum systems, more on this page
% %       \colorhref{https://qojulia.org/}{https://qojulia.org/}.
% %     \item Machine Learning on Google's Cloud TPUs (tensor processing
% %       units). \\
% %       \emph{Targeting TPUs using our compiler, we are able to
% %       evaluate the~VGG19 forward pass on~a~batch~of 100 images
% %       in~0.23s which compares favorably to~the~52.4s required
% %       for~the~original model on~the~CPU. Our implementation is~less
% %       than 1000
% %       lines~of Julia, \\
% %       with no~TPU specific changes made to~the~core
% %       Julia compiler \\
% %       or~any other Julia packages.} Whole article by Keno Fischer
% %       and Elliot Saba
% %       \colorhref{https://arxiv.org/abs/1810.09868}{here}.
% %     \end{itemize}
% %   \end{block}

% % \end{frame}
% % % ##################





% % % ##################
% % \begin{frame}
% %   \frametitle{Few interesting things for which there is no time
% %   here}


% %   {Things that are important or~promising}
% %   \begin{itemize}
% %   \item Type system,
% %     \colorlink{https://www.youtube.com/watch?v=Z2LtJUe1q8c}.

% %   \item Metaprogramming and~how use it to make code efficient,
% %     \colorlink{https://www.youtube.com/watch?v=SeqAQHKLNj4}.

% %   \item Multiple dispatch as rare way of Object Oriented Programming
% %     (I~don't want holy war over this statement),
% %     \colorlink{https://www.youtube.com/watch?v=gZJFHrYopxw}.

% %   \item Using GPU,
% %     \colorlink{https://www.youtube.com/watch?v=6ntJ_al4oXA}.

% %   \item Julia package manager,
% %     \colorlink{https://www.youtube.com/watch?v=HgFmiT5p0zU}.

% %   \item Symbolic computing like in~Wolfram Mathematica,
% %     \colorlink{https://www.youtube.com/watch?v=M742_73edLA}.

% %   \item Monte Carlo, very immature,
% %     \colorlink{https://www.youtube.com/watch?v=BmVd7pw6Trc}.

% %   \item Writing fast code,
% %     \colorlink{https://www.youtube.com/watch?v=szE4txAD8mk}

% %   \item Parallel computing,
% %     \colorlink{https://www.youtube.com/watch?v=euZkvgx0fG8}.

% %   \item Regulars expressions,
% %     \colorlink{https://docs.julialang.org/en/v1/manual/strings/}.

% %   \item Tensor compilers,
% %     \colorlink{https://www.youtube.com/watch?v=Rp7sTl9oPNI}.

% %   \item How Julia work inside,
% %     \colorlink{https://www.youtube.com/watch?v=7KGZ_9D_DbI}.

% %   \end{itemize}

% % \end{frame}
% % % ##################





% % % % ##################
% % % \begin{frame}
% % %   \frametitle{Lies, big lies and benchmarks from 2017}


% % %   \begin{block}{They are outdated now, but there aren't any newer}
% % %     \begin{figure}
% % %       \centering

% % %       \includegraphics[scale=0.7]{benchmarks_QO.pdf}
% % %       \caption{QuTiP --~Quantum Toolbox in Python.}
% % %     \end{figure}
% % %     More benchmarks on page ?????
% % %   \end{block}

% % %   \begin{block}{Article to cite}

% % %   \end{block}

% % % \end{frame}
% % % % ##################















% % % ##################
% % \begin{frame}
% %   \frametitle{Learning materials}


% %   \begin{block}{Basics materials}
% %     Most~of them are more or~less outdated, since they were created
% %     before release~of Julia 1.x.
% %     \begin{itemize}
% %     \item JuliaBoxTutorials, version 1.x, GitHub
% %       \colorhref{https://github.com/JuliaComputing/JuliaBoxTutorials}{JuliaComputing/JuliaBoxTutorials}.
% %       Good starting point.
% %     \item Julia 1.x Documentation. Always~up to~date, really good
% %       written in~comparison to~others manuals,
% %       \colorhref{https://docs.julialang.org/en/v1/}{https://docs.julialang.org/en/v1/}.
% %     \item David P.~Sanders, \emph{Introduction to~Julia
% %       for~scientific Computing (Workshop)}, 2015. Outdated, but very
% %       good introduction to~language,
% %       \YouTube{https://www.youtube.com/watch?v=gQ1y5NUD_RI}.
% %     \item The Julia Language channel
% %       on~\colorhref{https://www.youtube.com/user/JuliaLanguage}{YouTube}.
% %       Contains dozens videos from JuliaCons and hold regulars
% %       \emph{Intro to~Julia}, keeping it up to date. You~can find
% %       next \emph{Intro to~Julia} and other introductions
% %       on~\colorhref{https://www.facebook.com/Learn-Julia-529467964069525/?eid=ARDnSanRh6e3_6NQanjfziNeMYt3-hRUqje-5xOvRidGM5bmm6ZGCZ49fy5CZ1AYX9T503OEdMUMRlOe}{Facebook}.
% %     \item Ben Lauwens, Allen Downey, \emph{Think Julia: How to~Think
% %       Like a~Computer Scientist},
% %       \colorlink{https://benlauwens.github.io/ThinkJulia.jl/latest/book.html}.
% %     \end{itemize}
% %   \end{block}

% % \end{frame}
% % % ##################





% % % ##################
% % \begin{frame}
% %   \frametitle{Learning materials}


% %   \begin{block}{Practical introductions to many different topics}
% %     JuliaCon is annual conference on~Julia language which start
% %     at~2014.
% %     \begin{itemize}
% %     \item Stefan Karpinski and~Kristoffer Carlsson,
% %       \emph{Pkg3:~The~new Julia package manager}, JuliaCon 2018,
% %       \YouTube{https://www.youtube.com/watch?v=HgFmiT5p0zU}.

% %     \item Arch D. Robison, \emph{Introduction to~Writing High
% %       Performance Julia (Workshop)}, JuliaCon 2016,
% %       \YouTube{https://www.youtube.com/watch?v=szE4txAD8mk}.

% %     \item Andy Ferris, \emph{A~practical introduction
% %       to~metaprogramming in~Julia}, JuliaCon 2018,
% %       \YouTube{https://www.youtube.com/watch?v=SeqAQHKLNj4}.

% %     \item Chris Rackauckas, \emph{Intro to solving differential
% %       equations in Julia},
% %       \YouTube{https://www.youtube.com/watch?v=KPEqYtEd-zY}.

% %     \item DiffEqTutorials.jl. Tutorials~of JuliaDiffEq project,
% %       GitHub
% %       \colorhref{https://github.com/JuliaDiffEq/DiffEqTutorials.jl}{JuliaDiffEq/DiffEqTutorials.jl}.
% %     \end{itemize}
% %   \end{block}

% % \end{frame}
% % % ##################





% % % ##################
% % \begin{frame}
% %   \frametitle{Learning materials}


% %   \begin{block}{More theoretical, less practical materials}
% %     \begin{itemize}
% %     \item Jeff Bezanson, \emph{Why is~Julia fast?}, 2015,
% %       \YouTube{https://www.youtube.com/watch?v=cjzcYM9YhwA}.

% %     \item Jeff Bezanson, \emph{The~State~of the~Type System},
% %       JuliaCon 2017,
% %       \YouTube{https://www.youtube.com/watch?v=Z2LtJUe1q8c}.

% %     \item Jiahao Chen, \emph{Why language matters: Julia
% %       and~multiple dispatch}, 2016,
% %       \YouTube{https://www.youtube.com/watch?v=gZJFHrYopxw}.

% %     \item Jameson Nash, \emph{AoT or~JIT: How Does Julia Work?},
% %       (AoT --~Ahead~of Time compilation, JIT --~Just In~Time
% %       compilation), JuliaCon 2017,
% %       \YouTube{https://www.youtube.com/watch?v=7KGZ_9D_DbI}.

% %     \item John Lapyre, \emph{Symbolic Mathematics in Julia},
% %       JuliaCon 2018,
% %       \YouTube{https://www.youtube.com/watch?v=M742_73edLA}.

% %     \item Julia Lab at~MIT, \emph{Parallel Computing (Workshop)},
% %       JuliaCon 2016,
% %       \YouTube{https://www.youtube.com/watch?v=euZkvgx0fG8}.

% %     \item Tim Besard, Valentin Churavy and Simon Danisch,
% %       \emph{GPU~Programming with~Julia}, JuliaCon 2017,
% %       \YouTube{https://www.youtube.com/watch?v=6ntJ_al4oXA}.

% %     \end{itemize}
% %   \end{block}

% % \end{frame}
% % % ##################





% % % ##################
% % \begin{frame}
% %   \frametitle{Learning materials}


% %   \begin{block}{More theoretical, less practical materials}
% %     \begin{itemize}
% %     \item Peter Ahrens, \emph{For~Loops~2.0: Index Notation
% %       And~The~Future~Of Tensor Compilers}, JuliaCon 2018,
% %       \YouTube{https://www.youtube.com/watch?v=Rp7sTl9oPNI}.

% %     \item Carsten Bauer, \emph{Julia for Physics: Quantum Monte
% %       Carlo}, JuliaCon 2018,
% %       \YouTube{https://www.youtube.com/watch?v=BmVd7pw6Trc}.

% %     \item George Datseris, \emph{Why Julia is the most suitable
% %       language for science}, case study~of project JuliaDynamics,
% %       \colorlink{https://www.youtube.com/watch?v=7y-ahkUsIrY}.

% %     \item Nick Higham, \emph{Tricks and~Tips in~Numerical
% %       Computing}, JuliaCon 2018,
% %       \YouTube{https://www.youtube.com/watch?v=Q9OLOqEhc64}.
% %     \end{itemize}
% %   \end{block}

% % \end{frame}
% % % ##################










% % % ######################################
% % \section[]{Additional information and topics}
% % % ######################################










% % ##################
% % \begin{frame}
% %   \frametitle{Cxx.jl (probably still broken in~1.x)}

% %   \begin{block}{RELP for C++}
% %     \begin{figure}
% %       \centering

% %       \includegraphics[scale=0.29]{Cxx-jl.png}
% %       \caption{Compression~of ratio~of speed to the lines~of code.}
% %     \end{figure}
% %   \end{block}

% % \end{frame}
% % ##################









% % % ##################
% % \begin{frame}
% %   \frametitle{Mentioned projects and articles}


% %   \begin{block}{Written Julia if not mentioned otherwise}
% %     Still many~of them don't work in Julia 1.x.
% %     \begin{itemize}
% %     \item[--] Channelflow (written in~C++),
% %       \colorhref{http://channelflow.org/}{http://channelflow.org/}.
% %     \item[--] DynamicalSystems.jl, GitHub
% %       \colorhref{https://github.com/JuliaDynamics/DynamicalSystems.jl}{JuliaDynamics/DynamicalSystems.jl}.
% %     \item[--] QuantumOptics.jl. Numerical framework for numerical
% %       solving open quantum systems, more on this page
% %       \colorhref{https://qojulia.org/}{https://qojulia.org/}.
% %     \item[--] REPL (shell) for C++ (broken in Julia 1.x), GitHub
% %       \colorhref{https://github.com/NHDaly/PaddleBattleJL}{Keno/Cxx.jl}.
% %     \item[--] Game and educational tool \emph{Paddle Battle}, GitHub
% %       \colorhref{https://github.com/NHDaly/PaddleBattleJL}{NHDaly/PaddleBattleJL}.
% %     \end{itemize}
% %   \end{block}

% % \end{frame}
% % % ##################






% ####################################################################
% ####################################################################
% Bibliography
% \bibliographystyle{alpha}

% \bibliography{Bibliography}{}





% ############################

% Koniec dokumentu
\end{document}
