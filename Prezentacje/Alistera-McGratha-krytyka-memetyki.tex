% ---------------------------------------------------------------------
% Basic configuration of Beamera and Jagiellonian
% ---------------------------------------------------------------------
\RequirePackage[l2tabu, orthodox]{nag}



\ifx\PresentationStyle\notset
\def\PresentationStyle{dark}
\fi



\documentclass[10pt,t]{beamer}
\mode<presentation>
\usetheme[style=\PresentationStyle,logoLang=Latin,logoColor=monochromaticJUwhite,logoShape=D,JUlogotitle=yes]{jagiellonian}



% ---------------------------------------
% Configuration files of Jagiellonian loceted in catalog preambule
% ---------------------------------------
\input{./preambule/LanguageSettings/JagiellonianPolishLanguageSettings.tex}
\input{./preambule/TextposConfiguration/TextposConfiguration.tex}

\input{./preambule/JagiellonianCustomizationGeneral.tex}
\input{./preambule/JagiellonianCustomizationCommands.tex}










% ---------------------------------------
% Packages, libraries and their configuration
% ---------------------------------------





% ---------------------------------------
% Configuration for this particular presentation
% ---------------------------------------










% ---------------------------------------------------------------------
\title{Alistera McGratha krytyka memetyki}

\author{Kamil Ziemian \\
  \texttt{kziemianfvt@gmail.com}}

% \institute{Uniwersytet Jagielloński w~Krakowie}

\date[19 XII 2016]{Seminarium Międzywydziałowego Koła Naukoznawstwa $\Omega$, \\
  19 grudnia 2016}
% ---------------------------------------------------------------------





% ####################################################################
% Początek dokumentu
\begin{document}
% ####################################################################





% ######################################
\maketitle % Tytuł całego tekstu
% ######################################





% ######################################
\begin{frame}
  \frametitle{Plan prezentacji}


  \tableofcontents % Spis treści

\end{frame}
% ######################################




% ##################
\begin{frame}
  \frametitle{Podobno}


  Amerykanie zaczynają zawsze od~żartu, a~Japończycy od~przeprosiny
  (usłyszałem to od Amerykanina). Ja~pójdę japońską drogą.


  Za co mam przeprosić?
  \begin{enumerate}

  \item Nie mam formalnego wykształcenia~z:
    \begin{itemize}

    \item[--] biologi;

    \item[--] kulturoznawstwa;

    \item[--] socjologi;

    \item[--] filozofii;

    \item[--] informatyki.

    \end{itemize}

  \item Nie znam specjalnie dobrze współczesnej teorii ewolucji.

  \item Pełna lektura \textit{Samolubnego genu} jest wciąż przed mną.

  \item Wiele rzeczy, które chciałbym wam przedstawić, wciąż jest
    w~fazie intensywnych przemyśleń, tak~że~sam nie jestem
    przekonany, czy~są poprawne.

  \end{enumerate}

  Wartość tego co powiem, będziecie musieli sami ocenić

\end{frame}
% ##################





% ##################
\begin{frame}
  \frametitle{Z tego względu}


  Wartość tego co powiem, będziecie musieli ocenić tylko
  na~argumentów, które przedstawię.

  Wszelkie uwagi czy wskazanie błędów, będzie mile widziane :).

\end{frame}
% ##################










% ######################################
\section{Kim jest McGrath?}
% ######################################



% ##################
\begin{frame}
  \frametitle{Nasi bohaterowie}


  Zanim przejdziemy do~głównego tematu, poznajmy ich



  \begin{figure}

    \includegraphics[scale=0.6]{./PresentationPictures/Richard_Dawkins.jpg}
    \includegraphics[scale=0.555]
    {./PresentationPictures/Alister_McGrath_01.jpg}


    \caption{Clinton Richard Dawkins (1941~--), Alister McGrath
      (1953~--)}

  \end{figure}

\end{frame}
% ##################





% ##################
\begin{frame}
  \frametitle{Nasi bohaterowie}


  Richard Dawkins mam nadzieję, że~nie potrzebuje przedstawienia,
  skupmy~się na drugiej osobie.



  \begin{figure}

    \includegraphics[scale=0.5]
    {./PresentationPictures/Alister_McGrath_02.jpg}


    \caption{Alister McGrath (1953~--)}

  \end{figure}

  \vspace{-1em}



  Krótki opis z~Wikipedii: teolog, duchowny anglikański, historyk
  idei, naukowiec i~apologeta chrześcijański. Związany z~nurtem
  ewangelikalnym Kościoła Anglii.

\end{frame}
% ##################





% ##################
\begin{frame}
  \frametitle{McGrath sam o~sobie}


  \begin{center}

    \includegraphics[scale=0.4]
    {./PresentationPictures/Alister_McGrath_03.jpg}

  \end{center}



  Jako trzynastolatek dałem~się uwieść przyrodzie. [\ldots] W~latach
  szkolnych uważałem --~jak Dawkins --~że~nauki przyrodnicze
  wymagają światopoglądu ateistycznego. Teraz już tak nie myślę.
  [\ldots] Nauki przyrodnicze pokazywały, że~Bóg nie jest potrzebny
  do~wyjaśnienia jakiegokolwiek aspektu świata. Jak wielu w~owych
  dniach [koniec lat 60~XX~w.] w~owych gorących dniach optymizmu
  i~rewolucyjnego zapału czerpałem obficie ze~źródeł marksizmu
  i~uznałem religię za~niebezpieczne złudzenie. \\
  \textbf{A. McGrath \emph{Bóg Dawkinsa. Geny, memy i~sens życia} (dalej
    \cite{McGrathBogDawkinsa2008}), str.~7, 8.}

\end{frame}
% ##################





% ##################
\begin{frame}
  \frametitle{McGrath sam o~sobie}


  \begin{center}

    \includegraphics[scale = 0.5]
    {./PresentationPictures/Alister_McGrath_04.jpg}

  \end{center}


  Podczas nauki odkrył coś w bibliotece. \\
  „Nazywał~się [dział w~szkolnej bibliotece] <<Historia i~Filozofia
  Nauki>> \\
  i~był pokryty grubą warstwą kurzu. Nie miałem zbyt wiele czasu
  na~tego typu rozważania, zresztą uważałem je~za~dyletancką krytykę
  oczywistości i~pewników nauk przyrodniczych, uprawianą przez
  ludzi, dla~których wiedza stanowiła zagrożenie, a~których
  działalność Dawkins nazwał później <<zagłuszaniem prawdy>>.”
  \textbf{Str.~9, \cite{McGrathBogDawkinsa2008}}.

\end{frame}
% ##################





% ##################
\begin{frame}
  \frametitle{McGrath sam o~sobie}


  \begin{center}

    \includegraphics[scale=0.35]
    {./PresentationPictures/Alister_McGrath_05.jpg}

  \end{center}

  Była tam ksiązka
  L.~W.~Hulla \textit{History and~Philosophy~of Science:
    An~Introduction}~(1959). Po latach McGrath ocenia ją jako relatywnie
  słabą, pisaną z~przestarzałej perspektywy.

  „Myślę, że~w~głębi duszy wolałbym nigdy nie natrafić na~tę
  książkę, nigdy nie zadać sobie tylu kłopotliwych pytań i~nigdy nie
  podważać prostoty moje naukowej młodości. Lecz nie było już
  powrotu. Przeszedłem przez bramę i~nie mogłem uciec od~nowego
  świata, jaki~się przede mną jawił.” \\
  \textbf{Str.~10, \cite{McGrathBogDawkinsa2008}.}

\end{frame}
% ##################





% ##################
\begin{frame}
  \frametitle{McGrath sam o~sobie}


  \begin{center}

    \includegraphics[scale=0.37]
    {./PresentationPictures/Alister_McGrath_06.jpg}

  \end{center}


  „We~wrześniu 1974~roku znalazłem~się w~zespole profesora George’a
  Raddy na~kierunku biochemii Uniwersytetu Oksfordzkiego.” \\
  „Główny przedmiotem moich zainteresowań były innowacyjne metody fizyczne,
  umożliwiające poznawanie właściwości błon biologicznych, między
  innymi zastosowanie sond fluorescencyjnych i~anihilacji pozytronów
  w~badaniach nad~zależnymi od~temperatury zmianami w~systemach
  biologicznych i~ich modelach. (A.~E.~McGrath, Ch.~G.~Morgan,
  G.~K.~Radda, \textit{Photobleaching: A~Novel Fluorescence Method
      for~Diffusion Studies in~Lipid Systems}. Biochimica
    et~Biophisica Acta, 426 (1976), s.~173--185. \textbf{Str.~11,
      \cite{McGrathBogDawkinsa2008}.}

\end{frame}
% ##################





% ##################
\begin{frame}
  \frametitle{Kilka dat}



  \begin{center}

    \includegraphics[scale=0.44]
    {./PresentationPictures/Alister_McGrath_07.jpg}

  \end{center}


  \begin{enumerate}

  \item \textbf{1977~r.} Pierwszy raz spotyka~spotyka~się z~poglądami
    Richarda Dawkinsa poprzez lekturę \textit{Samolubnego genu}. Jak
    sam stwierdza, minęło parę lat zanim książka ta stanie~się
    pozycją kultową, którą „pozostaje do dzisiaj” \textbf{(str.~7,
      \cite{McGrathBogDawkinsa2008})}. Czy coś się przez te ponad
    10~lat zmieniło?

  \item \textbf{1978~r.} Uzyskuje doktorat z~biofizyki i~dyplom studiów
    teologicznych na Oksfordzie.

  \item \textbf{2004~r.} Publikuje książkę \textit{Bóg Dawkinsa. Geny,
      memy i~sens życia}, na~której bazuje to wystąpienie.

  \end{enumerate}

\end{frame}
% ##################





% ######################################
\section{Ogólne rozważania o~nauce we~współczesnym świecie}
% ######################################



% % ##########
% \begin{frame}
%   \frametitle{Przechodzimy do rzeczy}

%   \begin{block}{Ważne pytanie}
%     Skąd ludzie czerpią wiedzę o~współczesnych teoriach, zarówno
%     w~naukach ścisłych, jak i~humanistycznych? Według mnie:
%     \begin{itemize}
%     \item[--] z~filmów, seriali, komiksów, książek fabularnych, etc.;
%     \item[--] z~artykułów, filmów, wykładów popularnonaukowych
%       w~internecie, gazetach, etc.;
%     \item[--] od~znajomych;
%     \item[--] z~książek popularnonaukowych;
%     \item[--] z~krótkich uwag rzucanych przez wykładowców, na temat
%       myśli, \\
%       bądź prac wielkich uczonych;
%     \item[--] z~wielotomowych opracowań, 900 stron każdy, napisanych
%       przez kompetentnych ludzi, które w~sposób wyczerpujący
%       i~dokładny rozważają wszystkie dowody, argumenty przemawiające
%       za tymi teoriami, jak i~ich konsekwencje.
%     \end{itemize}
%   \end{block}

% \end{frame}
% % ##########



% % ##########
% \begin{frame}
%   \frametitle{Przechodzimy do rzeczy}

%   \begin{block}{Skąd ludzie czerpią wiedzę o~współczesnych teoriach?}
%     \begin{itemize}
%     \item[--] Od~znajomych;
%     \item[--] z~książek popularnonaukowych;
%     \item[--] z~krótkich uwag rzucanych przez wykładowców, na temat
%       myśli, \\
%       bądź prac wielkich uczonych;
%     \item[--] z~wielotomowych opracowań, 900 stron każdy, napisanych
%       przez kompetentnych ludzi, które w~sposób wyczerpujący
%       i~dokładny rozważają wszystkie dowody, argumenty przemawiające
%       za tymi teoriami, jak i~ich konsekwencje;
%     \item[--] dwu semestralnych wykładów monograficznych, prowadzonych
%       przez autorów książek z~poprzedniego punktu.
%     \end{itemize}
%   \end{block}

%   \begin{block}{Gorzka obserwacja}
%     Dwa ostatnie źródła~są mają znikomy wpływ na~ludzi których znam \\
%     i~na mnie. Ale próbuję~się podszkolić.
%   \end{block}

% \end{frame}
% % ##########



% % ##########
% \begin{frame}
%   \frametitle{Memetyka}

%   \begin{block}{Moje podejrzenie}
%     Memetykę większość ludzi poznała głównie z~tych źródłem, które
%     przekazują wiedzę w~sposób skrótowy, bezkrytyczny, apelujący
%     bardziej do~emocji i~pragnień, niż~do~argumentów i~dowodów. Nie
%     wydaje mi~się też, by~osoby które spotkałem, poświęciły wiele
%     czasu na jej zrozumienie.
%   \end{block}

%   \begin{block}{Jeśli tak jest, to}
%     Należy ustalić prawdziwą wartość memetyki i~w~świetle tej wiedzy
%     spróbować zrozumieć, co nam mówi o~świecie jej popularność. \\
%     \tb{To jest główny cel tego wystąpienia.}
%   \end{block}

%   \begin{block}{Uwaga chronologiczna}
%     Wśród moich rówieśników, czyli osób urodzonych w~drugiej połowie \\
%     lat 80 XX~w., Dawkins stał~się powszechnie znaną postacią, wraz
%     z~publikacją \emph{Boga urojonego} (oryginał 2006, polskie wydanie
%     2007). \\
%     \emph{Bóg Dawkinsa} ukazał~się 2004~r., więc \tb{NIE} jest
%     odpowiedzią na tę~książkę.
%   \end{block}

% \end{frame}
% % ##########



% % ##########
% \begin{frame}
%   \frametitle{Dlaczego powstała memetyka?}

%   \begin{block}{}
%     \hspace{1em} Darwinizm jest zbyt wielką teorią, aby~sprowadzać
%     go~tylko do~biologii. Dlaczego ograniczać darwinizm do~świata
%     genów, skoro jego treści dotyczą wszystkich aspektów ludzkiego
%     życia i~myślenia? W~\emph{Samolubnym genie}~(1976) Dawkins
%     wyjaśnia, że~od~dawna interesowała go~analogia pomiędzy informacją
%     kulturową a~genetyczną. Czy teoria darwinowska nie~może
%     stosować~się do~ludzkiej kultury tak samo jak do~świata biologii?
%     Ten pomysł leży u~podstaw przekształcenia darwinizmu z~teorii
%     naukowej w~światopogląd, metanarrację, całościowy obraz
%     rzeczywistości.

%     \hspace{1em} Wymaga to jednak znalezienia kulturowego odpowiednika
%     genu --~,,replikatora kulturowego'', który zapewniałby
%     przenoszenie informacji w~czasie i~przestrzeni. Gdyby pojęcie
%     replikatora udało~się solidnie umiejscowić w~kategoriach
%     naukowych, darwinizm stałby~się metodą uniwersalną, wykraczającą
%     poza wąską dziedzinę ewolucji biologicznej i~obejmującą świat
%     kultury. (J.~Poulshock, \textbf{Universal Darwinism
%       and~the~Potential~of Memetics}. ,,Quarterly Review~of Biology''
%     77 (2002), s.~174--175.) \textbf{\BogDaw{117}}.
%   \end{block}

% \end{frame}
% % ##########



% % ##########
% \begin{frame}
%   \frametitle{Dlaczego powstała memetyka?}

%   \begin{block}{Z~tego wynikałoby,~że}
%     U~podstaw jej powstania leży częsta u~naukowców wiara, że~to
%     czym~się zajmują jest najważniejszą, ostateczną teorią
%     \tb{wszystkiego} (,,Universal Darwinism''), bo niby kto
%     zajmowałby~się teorią, która nie jest najlepsza i~najgłębsza?
%     (Oprócz tych którzy patrzą, gdzie są największe granty;).
%     Przykładów tego we~współczesnej nauce jest niemało, \\
%     np.~,,spór'' Stevena Weinberga i~Philipa Andersona pod tytułem:
%     ,,Redukcjonizm vs~emergencjonizm w~fizyce''.
%   \end{block}

%   \begin{block}{Zatrzymajmy~się na chwilę}
%     Czynnik typu ,,Wreszcie mamy teorię, która wyjaśni wszystko!'',
%     na~pewno odgrywa tu~ważną rolę, bo naukowcy też są~ludźmi i~mają
%     ludzkie słabości. W~tym słabość do szybkich, łatwych rozwiązań,
%     które nie wymagają zbyt dużo pracy. Takie postawienie sprawy
%     jednak nie jest satysfakcjonujące, spróbuję to bardziej rozwinąć.
%   \end{block}

% \end{frame}
% % ##########



% % ##########
% \begin{frame}
%   \frametitle{,,Darwinizm'' jako teoria wszystkiego}

%   \begin{block}{Dlaczego właśnie on?}
%     \begin{enumerate}
%     \item Jest centralną teorią współczesnej biologii, unifikującą jej
%       wszystkie działy, dla biologa to~naturalny kandydat na~teorię
%       wszystkiego.
%     \item Istnieje ogólny prąd w~szeroko pojętej kulturze, który mówi,
%       że~to biologia \tb{będzie} najważniejszą teorią nowej epoki: XXI
%       wieku (bardzo mnie to dziwi, że~wciąż mówimy o~XXI wieku jako
%       o~przyszłości, żyjemy w~nim przecież od~16 lat). Więc to ona
%       powinna nam dostarczyć teorii wszystkiego, co~czyni teorię
%       ewolucji atrakcyjną dla nie\dywiz biologów.
%     \item Spektakularny sukces teorii na~jednym polu, często wywołuje
%       u~ludzi, naukowców i~nie\dywiz naukowców, nieuzasadniony niczym
%       entuzjazm, że~teoria ta rozwiąże wszystko. Patrz np. newtonizm
%       w~XVIII~wieku, Wolter, baron d'Holbach, La Mettrie, Laplace
%       i~inni.
%     \item Syntetyczna teoria ewolucji jest gotowa, nie trzeba nic
%       wymyślać.
%     \item \textbf{Uniwersalny darwinizm} jest \textbf{samotłumaczącą~się
%       teorią} (słowa jednej osoby z~którą o~tym dyskutowałem).
%     \end{enumerate}
%   \end{block}

% \end{frame}
% % ##########


% % % ##########
% % \begin{frame}
% %   \frametitle{,,Darwinizm'' jako teoria wszystkiego}

% %   \begin{block}{Dlaczego właśnie on?}
% %     \begin{enumerate}
% %     \item Jest centralną teorią współczesnej biologii, unifikującą
% %       jej wszystkie działy, dla biologa to~naturalny kandydat
% %       na~teorię wszystkiego.
% %     \item Istnieje ogólny prąd w~szeroko pojętej kulturze, który
% %       mówi, że~to biologia \tb{będzie} najważniejszą teorią nowej
% %       epoki: XXI wieku (bardzo mnie to dziwi, że~wciąż mówimy o~XXI
% %       wieku jako o~przyszłości, żyjemy w~nim przecież od~16 lat).
% %       Więc to ona powinna nam dostarczyć teorii wszystkiego,
% %       co~czyni teorię ewolucji atrakcyjną dla nie\dywiz biologów.
% %     \item Spektakularny sukces teorii na~jednym polu, często
% %       wywołuje u~ludzi, naukowców i~nie\dywiz naukowców,
% %       nieuzasadniony niczym entuzjazm, że~teoria ta rozwiąże
% %       wszystko. Patrz np. newtonizm w~XVIII~wieku, Wolter, baron
% %       d'Holbach, La Mettrie, Laplace i~inni.
% %     \item Syntetyczna teoria ewolucji jest gotowa, nie trzeba nic
% %       wymyślać.
% %     \item ,,Uniwersalny Darwinizm'' jest ,,samotłumaczącą~się
% %       teorią''(słowa jednej osoby z~którą o~tym dyskutowałem).
% %     \end{enumerate}
% %   \end{block}

% % \end{frame}
% % % ##########



% % ##########
% \begin{frame}
%   \frametitle{,,Darwinizm'' jako teoria wszystkiego}

%   \begin{block}{Jeszcze jedna myśl}
%     Wydaje mi~się, że~nie zrozumiemy tego wszystkiego, jeśli nie
%     zwrócimy uwagi, iż~w~tle tych rozważań znajduje~się pewien ciąg
%     myśli, prawie nigdy nie formułowany jawnie. Brzmi mniej więcej
%     tak. Skoro powstanie człowieka tłumaczy teoria ewolucji
%     i~wszystkie jego cechy~są przez ten proces zdeterminowane, to musi
%     ona też tłumaczyć wszystko co on robi lub~myśli. W~szczególności,
%     skoro wszystkie idee i~teorie naukowe, odkrył bądź wymyślił jakiś
%     człowiek, to~musi istnieć ewolucyjne uzasadnienie dla tych
%     wszystkich teorii.
%   \end{block}

%   \begin{block}{}
%     Do~tego zagadnienia jeszcze wrócimy.
%   \end{block}

% \end{frame}
% % ##########





% % ######################################
% \section{Główne zarzuty McGratha}
% % ######################################


% % ##########
% \begin{frame}
%   \frametitle{McGrath jest również historykiem idei, więc}

%   \begin{block}{Prominentni ,,darwiniści kulturowi''}
%     \begin{center}
%       \begin{figure}
%         \includegraphics[scale = 0.449]{HS.jpg} \includegraphics[scale
%         = 0.7]{EOW.jpg}
%         \caption{Herbert Spencer (1820--1903), Edward Osborne Wilson
%           (1929--)}
%       \end{figure}
%     \end{center}
%   \end{block}

%   \begin{block}{,,Replikator kulturowy''}
%     Jako pojęcie wprowadził w~1960 roku psycholog ewolucyjny Donald
%     T.~Campbell (1916--1996), nadał mu nazwę ,,mnenomu''. % Wśród
%     % socjobiologów amerykańskich zyskał popularność pokrewna teoria
%     % ,,kuturogenu''.
%     {Str.~118, \cite{AMG08}}.
%   \end{block}

% \end{frame}
% % ##########



% % ##########
% \begin{frame}
%   \frametitle{McGrath jest również historykiem idei, więc}

%   \begin{block}{Teoria ,,i-kultury'' i~,,m-kultury''}
%     W~artykule z~1968 roku, wersja poszerzona z~1975~r., antropolog
%     F.~T.~Cloak ,,sformułował tezę, że~rozwój kultury podlega
%     zasadniczo mechanizmom drawinowskim i~zaproponował stosowanie
%     metod etologicznych do~wyjaśniania zachowań kulturowych.''
%     Wprowadził też rozróżnienie na ,,i\dywiz kulturę'', czyli ,,zbioru
%     instrukcji kulturowych zawartych w~układzie nerwowym'' oraz
%     ,,m\dywiz kulturę'' czyli ,,relacje w~strukturach materialnych
%     podtrzymywane przez owe instrukcje lub zmiany w~strukturach
%     zachodzące pod wpływem instrukcji''. (F.~T.~Cloak,
%     \emph{Is~a~Culture Ethology Possible?}. ,,Human Ecology'' 3
%     (1975), s.~161--181.) \\
%     \tb{\BogDaw{119}}.
%   \end{block}

%   \begin{block}{Trochę to skomplikowane}
%     Spróbujmy znaleźć jakiś przykład.
%   \end{block}

% \end{frame}
% % ##########



% % ##########
% \begin{frame}
%   \frametitle{McGrath jest również historykiem idei, więc}

%   \begin{block}{Taniec}
%     \begin{figure}
%       \includegraphics[scale = 0.3]{NS.jpg} \includegraphics[scale =
%       0.237]{Taniec.jpg}
%     \end{figure}
%     I\dywiz kultura, to instrukcja jak tańczyć zawarta gdzieś
%     w~ludzkim układzie nerwowym (,,i'' pochodzi zapewne od
%     ,,intellectual'' lub podobnego słowa), zaś m\dywiz kultura
%     to~konkretny taniec wykonany przez człowieka, bo ta czynność
%     podtrzymuje relacje przestrzenne między ludzkimi ciałami,
%     kostiumami, etc., czyli relacje w~strukturach materialnych (,,m''
%     pochodzi zapewne od~,,material'').
%   \end{block}

% \end{frame}
% % ##########



% % ##########
% \begin{frame}
%   \frametitle{McGrath jest również historykiem idei, więc}

%   \begin{block}{Czy nie jest tak,~że}
%     Samo umieszczenie memetyki w~perspektywie historycznej jest już
%     formą krytyki? Bowiem, tym samym:
%     \begin{enumerate}
%     \item Memetyka przestaje~się jawić jako zupełnie oryginalna
%       teoria, a~kultura Zachodu bardzo cenni oryginalność. Ale~to
%       temat na osobną dyskusję.
%     \item Skoro istnieje wiele teorii kulturowego darwinizmu, to która
%       jest poprawna?
%     \end{enumerate}
%   \end{block}

% \end{frame}
% % ##########



% % ##########
% \begin{frame}
%   \frametitle{Główne zarzuty McGratha}

%   \begin{block}{\emph{Bóg Dawkinsa}, str.~119}
%     \begin{enumerate}
%     \item Nie ma powodu, by~zakładać, że~ewolucja kulturowa
%       ma~charakter darwinowski albo~że~biologia ewolucyjna może~się
%       w~istotny sposób przyczynić do~wyjaśnienia rozwoju idei.
%     \item Nie ma bezpośrednich dowodów na istnienie ,,memów''.
%     \item Przekonanie o~istnieniu ,,memów'' wynika z~problematycznego
%       założenia bezpośredniej analogii do~genów, jednak okazuje~się,
%       że~nie jest ono w~stanie sprostać wymaganiom teorii.
%     \item Wskazywanie na~istnienie ,,memu'' jako konstruktu
%       wyjaśniającego rzeczywistość nie jest konieczne. Dane
%       z~obserwacji można równie dobrze opisać za~pomocą innych modeli
%       i~mechanizmów.
%     \end{enumerate}
%   \end{block}

% \end{frame}
% % ##########



% % ##########
% \begin{frame}

%   \begin{block}{Dodatkowe pytania}
%     Dalsza dyskusja wymaga postawienia w~sposób jawny i~prosty, kilku
%     problemów na~jakie natrafiłem rozmyślając nad ,,darwinizmem
%     kulturowym''.
%     \begin{enumerate}
%     \item Nie jest dla mnie jasne, czego ludzie zajmujący~się memetyką
%       od~niej oczekują bądź oczekiwali. Czy ma być ona ,,klonem''
%       darwinowskiej teorii ewolucji, czy też teorią inspirowaną przez
%       nią, a~mogącą~się od~niej nawet fundamentalnie różnić?
%     \item Jaki jest zakres stosowania memetyki? Czy ma ona wyjaśniać
%       całą kulturę i~myśl ludzką, czy tylko jej część? Np. taniec tak,
%       ale techniki siania zboża już nie? W~szczególności, czy teorie
%       naukowe rządzone są prawami memetyki? (Na to ostatnie jakieś
%       odpowiedzi~są udzielane, ale~mało satysfakcjonujące.)
%     \end{enumerate}
%   \end{block}

% \end{frame}
% % ##########



% % ##########
% \begin{frame}
%   \frametitle{Zarzut 1: ewolucja kultury}

%   \begin{block}{}
%     Wszystkie argumenty pochodzą z~4~rozdziału \emph{Boga Dawkinsa}.
%   \end{block}

%   \begin{block}{Historia kultury}
%     Pokazuje, że~ogromną rolę odgrywają w~niej działania \tb{celowe,
%       zamierzone i~zaplanowane}, takie jak chęć wskrzeszenia potęgi
%     Rzymu, dawnych nurtów filozoficznych, etc. Ortodoksyjny darwinizm
%     raczej nie pozwala na rozumienie czegoś w~tych
%     kategoriach. % Do~tego zagadnienia jeszcze wrócę.
%   \end{block}

%   \begin{block}{Co jest dziedziczone?}
%     W~\emph{Samolubnym genie} Dawkins jako przykłady memów wymienia
%     melodie, poglądy, obiegowe zwroty, mody, detale architektoniczne,
%     pieśni i~ideę Boga. Problem w~tym, że~jeśli mem ma być
%     odpowiednikiem genu, a~tak rozumie go Dawkins, to podane przykłady
%     nie~są elementami kulturowego \tb{genotypu}, lecz~\tb{fenotypu}.
%     Mówiąc inaczej mem piosenki, ma tak~się do śpiewanej piosenki, jak
%     ciągu zasad azotowych w~DNA do niebieskiego koloru oczu.
%   \end{block}

% \end{frame}
% % ##########



% % ##########
% \begin{frame}
%   \frametitle{Zarzut 1: ewolucja kultury}

%   \begin{block}{}
%     Wszystkie argumenty pochodzą z~4~rozdziału \emph{Boga Dawkinsa}.
%   \end{block}

%   \begin{block}{Zdając sobie sprawę z~tego problemu}
%     Dawkins w~\emph{Fenotypie rozszerzonym} (1982), przedstawił
%     następujące wyjaśnienie. \tb{(Za \BogDaw{120--121}.)} \\
%     \hspace{1em} ,,Nie wprowadziłem dostatecznie wyraźnego
%     rozróżnienia między memem jako replikatorem a~jego <<efektem
%     fenotypowym>> w~postaci <<produktów memu>>. Mem powinien być
%     postrzegany jako jednostka informacji mieszcząca~się w~mózgu
%     (<<i\dywiz kultura>> według Cloaka). Ma on określoną strukturę
%     w~tym rodzaju fizycznego medium, w~jakim mózg magazynuje
%     informacje. [...] W~ten sposób da~się go~odróżnić od jego efektów
%     fenotypowych, które istnieją
%     w~otaczającym świecie (<<m\dywiz kultura>> według Cloaka).'' \\
%     \hspace{1em} McGrath zapewne słusznie konkluduje, że~memetyka
%     zyskała popularność w~swej wersji z~\emph{Samolubnego genu}, zaś
%     wersja zrewidowana z~\emph{Fenotypu rozszerzonego} nie odbiła~się
%     większym echem.
%   \end{block}

% \end{frame}
% % ##########



% % ##########
% \begin{frame}
%   \frametitle{Zarzut 1: ewolucja kultury}

%   \begin{block}{Czy wszystko podlega memetyce?}
%     Dawkins w~\emph{Kapłanie diabła} \tb{(Za \BogDaw{122}.)} \\
%     \hspace{1em} \textbf{Idee naukowe, jak wszystkie memy,
%       są~przedmiotem doboru naturalnego i~na~pozór mogą przypominać
%       wirusy. Jednak czynniki doboru, które weryfikują idee naukowe,
%       nie~są arbitralne ani~kapryśne. Są~to ścisłe, precyzyjne reguły,
%       które nie faworyzują niepotrzebnych zachowań obliczonych
%       na~własną korzyść.}
%   \end{block}

%   \begin{block}{Podstawowe problem}
%     Co decyduje, że~w~pewnych przypadkach czynniki doboru~są
%     arbitralne i~kapryśne, a~w~innych nie? I~jakie argumenty
%     przemawiają za~tym, że~dla nauki takie nie~są? Można, w~tym
%     kontekście prowokacyjnie, zapytać, czy czyniki doboru religii też
%     nie~są arbitralne i~kapryśne (David Sloan Wilson, \emph{Darwin's
%       Cathedral})? Dobrej odpowiedzi na~te pytania nie znam.
%   \end{block}

% \end{frame}
% % ##########



% % ##########
% \begin{frame}
%   \frametitle{Zarzut 4: memetyka jest niepotrzebna}

%   \begin{block}{}
%     Uwagi McGratha dotyczą stanu nauki na 2004~r., jednak nie~widzę,
%     żeby wiele od tego czasu~się zmieniło.
%   \end{block}

%   \begin{block}{,,Przyrodnicza'' nauka o~kulturze}
%     \begin{enumerate}
%     \item W~badaniu rozwoju kulturalnego i~intelektualnego bardzo
%       użyteczne~są modele fizyczne i~ekonomiczne.
%     \item Zamiast szukać ,,analogi biologicznej'', można używać
%       ,,analogii fizycznych'' i~tak otrzymana teoria okazuje~się
%       wartościowa.
%     \end{enumerate}
%   \end{block}

%   \begin{block}{Przykłady (brr, co za~groźne nazwy)}
%     \begin{itemize}
%     \item[--] Transmisja informacji na sieciach losowych.
%     \item[--] Idee jako kaskady informacji lub trwałe artykuły
%       konsumpcyjne.
%     \item[--] Teoria chwilowych mód (\emph{theory~of fads}).
%     \item[--] Współcześnie --~automaty komórkowe.
%     \end{itemize}
%     Jeśli jest potrzeba, bądź chęć, mogę potem powiedzieć o~tym coś
%     więcej.
%   \end{block}

% \end{frame}
% % ##########



% % ##########
% \begin{frame}
%   \frametitle{Zarzut 4: memetyka jest niepotrzebna}

%   \begin{block}{McGrath wskazuje}
%     Że~część z~wymienionych modeli, może wszystkie, ,,zawierają
%     darwinowskie motywy <<rywalizacji>> i~<<wygasania>>, choć nie
%     zawsze przyjmują związane z~nimi poglądy na temat źródeł
%     innowacji''. \tb{(Str.~131.)} Jednak z~własnego doświadczenia
%     wiemy, że~mody przychodzą, słabną, rosną, przemijają, powracają,
%     więc każda teoria kultury musi wyjaśnić to zjawisko (Kino Nowej
%     Nowej Przygody, ,,Gwiezdne Wojny''). Czy to jednak wystarczy, żeby
%     teoria była darwinowska?
%   \end{block}

%   \begin{block}{Może}
%     Trzeba na to spojrzeć na odwrót: to darwinizm jest szczególnym
%     przypadkiem teorii \tb{,,powielania i~rozprzestrzeniania''}? Ważne
%     pytanie, ale~nie ma nic wartościowego do powiedzenia w~tej
%     kwestii.
%   \end{block}

% \end{frame}
% % ##########



% % ##########
% \begin{frame}
%   \frametitle{Zarzut 4: memetyka jest niepotrzebna}

%   \begin{block}{Czy jednak memetyka nie tłumaczy wielu rzecz?}
%     Wytłumaczenie memetyczne zjawisk jakie ja znam, wszystkie
%     sprawiają mi~spore problemy.
%   \end{block}

%   \begin{block}{M.~Gardner, \emph{Kilroy Was Here},
%       ,,Los Angeles Times'', 5~marca 2000~r.}
%     Mem został zdefiniowany przez jego zwolenników tak szeroko,
%     że~stał~się pojęciem bezużytecznym, które raczej wprowadza zamęt
%     niż cokolwiek rozjaśnia. Uważam, że~wkrótce zostanie zapomniany
%     jako dziwactwo językowe bez~większej wartości. Dla krytyków,
%     którzy obecnie znacznie przewyższają liczą wyznawców, memetyka nie
%     jest niczym więcej niż niezręczną terminologią, mającą wyrażać
%     coś, co~i~tak wszyscy wiedzą \\
%     i~co można powiedzieć trafniej, odwołując~się do mnie efektownej
%     terminologii transferu informacji.
%   \end{block}

%   \begin{block}{}
%     Definicja memu jest tak szeroka, że~chyba NIE jest możliwe
%     znalezienie czegoś, czego ona nie tłumaczy. A~jakie dobre jest to
%     wytłumaczenie?
%   \end{block}

% \end{frame}
% % ##########



% % ##########
% \begin{frame}
%   \frametitle{Zarzut 4: memetyka jest niepotrzebna}

%   \begin{block}{Bardziej szczegółowe problemy}
%     \begin{enumerate}
%     \item Jeśli zauważę, że~jedna osoba przejęła jakieś myśli,
%       upodobania, idee od~drugiej, to jest to dowód na prawdziwość
%       memetyki, jeśli zinterpretuje to zjawisko, jako ,,kopiowanie
%       memów''. Ale~dlaczego mam przyjąć taką interpretację, a~nie
%       inną? Dlaczego np.~nie przyjąć interpretacji psychoanalizy
%       w~jednej z~jej miliarda wersji, że~to co zostało przekazane,
%       zostało przyjęte, aby~zaspokoić wewnętrzne potrzeby
%       ,,kopiującego'', nie troszcząc~się o~sam mechanizm przekazu?
%     \item Argument za tym, że~coś jest w~danym momencie popularne \\
%       np.~PSY \emph{Gangnam Style}, jest taki, że~jest dobrze
%       przystosowane do~obecnych warunków, dlatego jest popularne.
%       Jeśli~się jednak nie wskaże na~czym konkretnie to dostosowanie
%       polega, to nic tak na prawdę nie mówi, a~tym bardziej nie
%       tłumaczy. Znowu weźmy przykład \emph{Gangnam Style}.
%     \item Ponownie pojawia~się pytanie: czy memetyka jest popularna
%       w~pewnych kręgach, bo jest do nich dobrze dostosowana?
%     \end{enumerate}
%   \end{block}

% \end{frame}
% % ##########



% % ##########
% \begin{frame}
%   \frametitle{Zarzuty 2 i~3: geny, memy, obserwacja}

%   \begin{block}{}
%     Ponownie, nic mi nie wiadomo o~jakimś fundamentalnym postępie
%     w~tej dziedzinie. Chętnie dowiem~się o~tym, że~się mylę.
%   \end{block}

%   \begin{block}{Genetyka}
%     Prace badawcze nad~dziedziczeniem z~okolic przełomu XIX i~XX
%     wieku, pokazywały potrzebę istnienia jednostki przekazywania cech.
%     Był więc dobry dowód pośredni jej przyjęcia, nawet jeśli on sama
%     nie była znana.
%   \end{block}

%   \begin{block}{Skoro}
%     \begin{enumerate}
%     \item Ewolucja kultury, nie musi być darwinowska, więc nie jest
%       wymagana obecności ,,genu kultury'', tym samym argument przez
%       analogię teorii odpada.
%     \item W~,,przyrodniczych'' teoriach kultur potrafimy sobie
%       poradzić bez memu, więc nie musimy go~postulować, aby wyjaśnić
%       zachowanie świat.
%     \item Może więc należy przyjąć istnienie memu, bo go widzieliśmy?
%     \end{enumerate}
%   \end{block}

% \end{frame}
% % ##########



% % ##########
% \begin{frame}
%   \frametitle{Zarzuty 2 i~3: geny, memy, obserwacja}

%   \begin{block}{Mem internetowy}
%     \begin{figure}
%       \centering \includegraphics[scale = 0.4]{Mem-structure.png}
%       \caption{Jaki mem może być bardziej widoczny od~tego?}
%     \end{figure}
%   \end{block}

%   \begin{block}{Otwarte pytania, pewnie nietrudne}
%     Konkretny mem internetowy, to ,,fenotyp memu'', czy więc ta
%     struktura na~obrazku, to jego ,,genotyp''? A~może za genotyp
%     należy uważać strukturę plików graficznego w~którejś z~pamięci
%     komputera? Jak w~takim wypadku wyglądałby proces kopiowania? Czy
%     z~tego wynika, że~memy są kopiowane ,,fenotypowo''?
%   \end{block}

% \end{frame}
% % ##########



% % ##########
% \begin{frame}
%   \frametitle{Zarzuty 2 i~3: geny, memy, obserwacja}

%   \begin{block}{Neuronauka jest na topie, więc powstaje pytanie}
%     Czy udało~się zaobserwować w~mózgu analog helisy DNA dla~memów? \\
%     Ani McGrath w~2004, ani ja dziś nic nie wiemy o~udanej próbie
%     znalezienia~go. Są~pewne sugestie, hipotezy i~teorie, ale~ktoś kto
%     zna dobrze naukę, że~one zawsze~są, jaki problem by nie był.
%     Pytanie co~one mają wspólnego z~rzeczywistością?
%   \end{block}

%   \begin{block}{Mechanizm kopiowania}
%     Nie znam żadnego opisu tego procesu, oprócz ogólnika
%     ,,że~przeskakują z~jednego mózgu do~drugiego, w~procesie szeroko
%     rozumianego naśladownictwa.'' (R. Dawkins, \emph{Samolubny gen},
%     \tb{za~str.120, \cite{AMG08}}). \\
%     To jednak za mało, by~na~podstawie tego dało~się coś ważkiego
%     stwierdzić. Skoro potrzebujemy~się odwołać, do~,,szeroko
%     rozumianego naśladownictwa'', to~czemu w~ogóle wprowadzamy memy,
%     zamiast na tym poprzestać?
%   \end{block}

% \end{frame}
% % ##########



% % ##########
% \begin{frame}
%   \frametitle{Zarzuty 2 i~3: geny, memy, obserwacja}

%   \begin{block}{Doświadczalne stwierdzenie mechanizmu kopiowania}
%     Skoro brak wiadomości o~jego kształcie jest tak dotkliwy,
%     to~potwierdzenie musi byś w~jeszcze gorszym stanie niż samego
%     memu. Zwróćmy też uwagę, że~mechanizm kopiowania genu dużo
%     tłumaczy, memetyka jest tym samym w~słabszej pozycji.
%   \end{block}

%   \begin{block}{Refleksje Dawkinsa}
%     ,,Memy mogą~się zlewać ze~sobą, co~się nie zdarza genom. Nowe
%     ,,mutacje'' mogą być częściej <<kierunkowe>> niż <<losowe>>.
%     Odpowiednik doktryny Weismanna (weismanizmu) w~odniesieniu
%     do~memów byłby znacznie mniej rygorystyczny, <<lamarkowskie>>
%     strzałki przyczynowo\dywiz skutkowe mogą bowiem prowadzić zarówno
%     od~replikatora do~fenotypu, jak i~w~drugą stronę. Te~różnice mogą
%     sprawić, że~cała analogia do~genowego doboru naturalnego okaże~się
%     w~rzeczywistości całkowicie pozorna, a~nawet, co~gorsza, błędna.''
%     \tb{(\emph{Fenotyp rozszerzony}, za~str.~125, \cite{AMG08}).}
%   \end{block}

% \end{frame}
% % ##########



% % ##########
% \begin{frame}
%   \frametitle{Weźmy jeszcze jeden przykład}

%   \begin{block}{Czy ,,Gwiezdne Wojny''~są memem?}
%     \begin{figure}
%       \centering \includegraphics[scale = 0.464]{JC.jpg}
%       \includegraphics[scale = 0.25]{DV.jpg}
%       \caption{Joseph Campbell (1904--1987), intelektualny ojciec SW.
%         Darth Vader, chyba wiecznie żywy;).}
%     \end{figure}
%     Jeśli tak, to byłby to bardzo skomplikowany mem, na~którego
%     zlały~się memy psychoanalizy, s\dywiz f, filmy przygodowe,
%     nazistowskie mundury \\
%     od~Hugo Bossa, etc. (Temat na osobną dyskusję.) Jeżeli już, to
%     byłby to raczej ,,organizm memowy'', tak jak słowik jest
%     ,,organizmem genowym''.
%   \end{block}

% \end{frame}
% % ##########





% % ######################################
% \section{Podsumowanie, dodatkowe uwagi i~pytania na przyszłość}
% % ######################################


% % ##########
% \begin{frame}
%   \frametitle{Podsumowanie, uwagii, pytania}

%   \begin{block}{Ogólne wnioski pod dyskusję}
%     \begin{enumerate}
%     \item Rozważania nad nauką, to nie ,,zagłuszanie prawdy''.
%     \item Należy sobie zadać pytaniem, czy w~danej sytuacji,
%       należy~się zajmować teorią w~jej prawdziwym kształcie, czy~jej
%       powszechną recepcją?
%     \item Nauka nie staje~się mniej wartościowa, gdy~zrozumiemy,
%       że~ludzkie preferencje i~słabości wpływają na jej rozwój, często
%       w~sposób zasadniczy.
%     \item Przekonanie, że~jesteśmy już u~wrót ostatecznej teorii
%       czegoś\dywiz tam, choć zwykle jest to zupełnie nieuzasadnione,
%       jest bardzo powszechną i~groźną przypadłością.
%     \item Hype nad czymś, może być ważniejszy od~samego przedmiotu
%       hypu.
%     \item Wąskie zainteresowania, jeśli czemuś sprzyjają,
%       to~ignorancji na~własną ignorancję.
%     \item Skoro o~biologii człowieka zadecydowała ewolucja, to~czy
%       gdzieś jej wpływ~się kończy?
%     \end{enumerate}
%   \end{block}

% \end{frame}
% % ##########



% % ##########
% \begin{frame}
%   \frametitle{Podsumowanie, uwagii, pytania}

%   \begin{block}{Bardziej szczególne wnioski pod dyskusję}
%     \begin{enumerate}
%     \item Memetyka pojęta jako ,,klon'' syntetycznej teorii ewolucji
%       jest prawie niemożliwa (celowość rozwoju, zlewanie~się memów,
%       etc.)
%     \item Nie znalazła dobrego wyjścia z~pułapki samotłumaczenia~się.
%       Rozwiązanie przez arbitralne wyjątki, jest naprawdę słabe.
%     \item W~wersjach które ja znam albo stosuje~się do~wszystkiego, \\
%       ale ,,nijako'', albo brak jej aparatu by~uchwycić konkretne
%       dzieła kultury (czym~są ,,Gwiezdne Wojny''?).
%     \item Choć uwzględnia realnie występujące zjawiska, jak
%       przenoszenie i~kopiowanie informacji, ,,stopień
%       przystosowania'', to nie potrafiła na~tej podstawie dokonać
%       prawdziwych, wartościowych odkryć.
%     \item Na~dzisiejszym poziomie wiedzy, zupełnie inne
%       ,,przyrodnicze'' drogi badania kultury wydają~się właściwe. Jak
%       będzie zobaczymy, w~każdym razie, już w~tej okolicy słyszałem
%       niezły hype.
%     \end{enumerate}
%   \end{block}

% \end{frame}
% % ##########



% % ##########
% \begin{frame}
%   \frametitle{Podsumowanie, uwagii, pytania}

%   \begin{block}{Skąd jej popularność?}
%     \begin{enumerate}
%     \item Apeluje do~pewnych potrzeb i~pragnień, takich jak szybkie,
%       łatwe zrozumienia świata.
%     \item Pewien typ osób, ze~względu na~swoje zainteresowania jak
%       i~wykształcenie, ma do niej naturalną słabość.
%     \item Pewna płytkość rozmów o~nauce jaka obecnie panuje (mówię to
%       na~podstawie własnego doświadczenia z~teoriami fizycznymi),
%       wpiera bezkrytyczne podejście do~niej.
%     \item ,,Czynnik doboru''. Jak zwrócił uwagę McGrath, Dawkinsowi
%       udało~się to, czego jego poprzednicy nie byli w~stanie zrobić.
%       Stworzył zręczniejszą i~prostszą w~zapamiętaniu terminologię,
%       popierając przy tym swe rozważania, bardzo przemawiającymi
%       do~wyobraźni przykładami.
%     \item ,,Zrozumienie człowieka = zrozumienie wszystkiego''.
%     \end{enumerate}
%   \end{block}

% \end{frame}
% % ##########



% % ##########
% \begin{frame}
%   \frametitle{Może zwróciliście uwagę}

%   \begin{block}{}
%     Że pisałem o~,,przyrodniczych'' teoriach kultury. Są~bowiem też
%     ,,humanistyczne'' sposoby podejścia. To też temat na~osobną
%     dyskusję.
%   \end{block}

%   \begin{block}{Garść przykładów}
%     \begin{itemize}
%     \item[--] Psychoanaliza, miliard odmian. Freud, Jung, Lacan,
%       \v{Z}i\v{z}ek, etc.
%     \item[--] Teoria bajki magicznej Władimir Proppa.
%     \item[--] Antropologia strukturalistyczna C. Leviego\dywiz
%       Straussa. Np.~w~książce \textbf{Surowe i~gotowane} (1964), próbuje
%       pokazać, że~wszystkie mity Indian Południowej Ameryki (i~pewne
%       zachowania francuzów, ale to jest dopiero na samym końcu), można
%       wywieść z~jednego mitu, za pomocą serii algorytmów (choć nie
%       przedstawia swojego celu jawnie).
%     \item[--] Ogólnie, ruch strukturalistów. Obecnie zastąpił
%       go~poststrukturalizm.
%     \item[--] Praca W.~Stoczkowskiego, pokazująca jak teorie,
%       że~ludzkość stworzyli kosmici,~są materialistyczną adaptacją XIX
%       w. myśli magicznej/okultystycznej H. Blawackiej i~innych.
%     \end{itemize}
%   \end{block}

% \end{frame}
% % ##########



% % ##########
% \begin{frame}
%   \frametitle{Może zwróciliście uwagę}

%   \begin{block}{Między memetyką a~nimi jest wiele ciekawych
%       podobieńst}
%     \begin{enumerate}
%     \item ,,Zrozumienie człowieka = zrozumienie wszystkiego''.
%       Wszak~są to nauki \tb{humanistyczne}, więc mają zrozumieć
%       człowieka.
%     \item Podobny hype: teoria wszystkiego, jest tuż za rogiem.
%     \item Niezwykła popularność psychoanalizy.
%     \item Przynajmniej w~przypadku psychoanalizy, te same problemy
%       z~zastosowanie do konkretnego dzieła. Zob.~np. K.~Loska,
%       \emph{Hitchcock --~autor wśród gatunków}.
%     \end{enumerate}
%   \end{block}

% \end{frame}
% % ##########



% % ##########
% \begin{frame}
%   \frametitle{Bibliografia}

%   \begin{block}{Podstawowa (dzieł Dawkinsa, nie trzeba chyba
%       przypominać)}
%     \begin{itemize}
%     \item[--] Alister McGrath, \emph{Bóg Dawkinsa. Geny, memy i~sens
%         życia}. Wydawnictwo Uniwersytetu Jagiellońskiego, Kraków,
%       wydanie~I, 2008. \tb{W~niej jest jeszcze dużo wartościowych
%         rzeczy.}
%     \item[--] A.~E.~McGrath, Ch.~G.~Morgan, G.~K.~Radda,
%       \emph{Photobleaching: A~Novel Fluorescence Method for~Diffusion
%         Studies in~Lipid Systems}. ,,Biochimica et~Biophisica Acta''
%       426 (1976), s.~173--185.
%     \item[--] J.~Poulshock, \emph{Universal Darwinism
%         and~the~Potential~of Memetics}. ,,Quarterly Review~of
%       Biology'' 77 (2002), s.~174--175.
%     \item[--] F.~T.~Cloak, \emph{Is~a~Culture Ethology Possible?}.
%       ,,Human Ecology'' 3 (1975), s.~161--181. J.~Poulshock,
%       \emph{Universal Darwinism and~the~Potential~of Memetics},
%       ,,Quarterly Review~of Biology'' 77 (2002), s.~174--175.
%     \end{itemize}
%   \end{block}

% \end{frame}
% % ##########



% % ##########
% \begin{frame}
%   \frametitle{Literatura}

%   \begin{block}{Podstawowa}
%     \begin{itemize}
%     \item[--] Sushil Bikhchandani, David Hirshleifer, Ivo Welch,
%       \emph{A~Theory~of Fads, Fashion, Custom, and~Cultural Change
%         as~Informational Cascades}. ,,Journal of Political Economy'',
%       100, (1992), s.~992--1026.
%     \item[--] M.~Gardner, \emph{Kilroy Was Here}, ,,Los Angeles
%       Times'', 5~marca 2000~r.
%     \end{itemize}
%   \end{block}

%   \begin{block}{Dodatkowa (nie znam jej całej)}
%     \begin{itemize}
%     \item[--] David Sloan Wilson, \emph{Darwin's Cathedral: Evolution,
%         Religion, and~the~Nature~of Society}.
%     \item[--] D. Oramus, \emph{Darwinowskie paradygmant. Mit teorii
%         ewolucji w~kulturze współczesnej}. Copernicus Ceter Press,
%       Kraków, wydanie~I, 2015.
%     \item[--] W. Stoczkowski, \emph{Ludzie, bogowie i~przybysze
%         z~kosmosu} (irytuje mnie arogancja autora, ale~pozycja warta
%       poznania).
%     \end{itemize}
%   \end{block}

% \end{frame}
% % ##########



% % ##########
% \begin{frame}

%   \begin{center}
%     {\LARGE Dziękuję bardzo!}
%   \end{center}

% \end{frame}
% % ##########





% ####################################################################
% ####################################################################
% Bibliografia
\bibliographystyle{plalpha}

\bibliography{VariousFieldsBooks}{}





% ############################

% Koniec dokumentu
\end{document}
