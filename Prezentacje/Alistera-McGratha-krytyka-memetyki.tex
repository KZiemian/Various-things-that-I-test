% ---------------------------------------------------------------------
% Basic configuration of Beamera and Jagiellonian
% ---------------------------------------------------------------------
\RequirePackage[l2tabu, orthodox]{nag}



\ifx\PresentationStyle\notset
\def\PresentationStyle{dark}
\fi



\documentclass[10pt,t]{beamer}
\mode<presentation>
\usetheme[style=\PresentationStyle,logoLang=Latin,logoColor=monochromaticJUwhite,logoShape=D,JUlogotitle=yes]{jagiellonian}



% ---------------------------------------
% Configuration files of Jagiellonian loceted in catalog preambule
% ---------------------------------------
\input{./preambule/LanguageSettings/JagiellonianPolishLanguageSettings.tex}
\input{./preambule/TextposConfiguration/TextposConfiguration.tex}

\input{./preambule/JagiellonianCustomizationGeneral.tex}
\input{./preambule/JagiellonianCustomizationCommands.tex}










% ---------------------------------------
% Packages, libraries and their configuration
% ---------------------------------------





% ---------------------------------------
% Configuration for this particular presentation
% ---------------------------------------










% ---------------------------------------------------------------------
\title{Alistera McGratha krytyka memetyki}

\author{Kamil Ziemian \\
  \texttt{kziemianfvt@gmail.com}}

% \institute{Uniwersytet Jagielloński w~Krakowie}

\date[19 XII 2016]{Seminarium Międzywydziałowego Koła Naukoznawstwa $\Omega$, \\
  19 grudnia 2016}
% ---------------------------------------------------------------------





% ####################################################################
% Początek dokumentu
\begin{document}
% ####################################################################





% ######################################
\maketitle % Tytuł całego tekstu
% ######################################





% ######################################
\begin{frame}
  \frametitle{Plan prezentacji}


  \tableofcontents % Spis treści

\end{frame}
% ######################################




% ##################
\begin{frame}
  \frametitle{Podobno}


  Amerykanie zawsze zaczynają od~żartu, a~Japończycy od~przeprosiny
  (usłyszałem to od Amerykanina). Ja~pójdę japońską drogą.


  Za co mam przeprosić?
  \begin{enumerate}

  \item Nie mam formalnego wykształcenia~z:
    \begin{itemize}

    \item[--] biologi;

    \item[--] kulturoznawstwa;

    \item[--] socjologi;

    \item[--] filozofii;

    \item[--] informatyki.

    \end{itemize}

  \item Nie znam specjalnie dobrze współczesnej teorii ewolucji.

  \item Pełna lektura \textit{Samolubnego genu} jest wciąż przed mną.

  \item Wiele rzeczy, które chciałbym wam przedstawić, wciąż jest
    w~fazie intensywnych przemyśleń, tak~że~sam nie jestem
    przekonany, czy~wnioski są poprawne.

  \end{enumerate}

\end{frame}
% ##################





% ##################
\begin{frame}
  \frametitle{Z tego względu}


  Wartość tego co powiem, będziecie musieli ocenić sami na podstawie
  argumentów, które przedstawię.

  Wszelkie uwagi czy wskazanie błędów, będzie mile widziane :).

\end{frame}
% ##################










% ######################################
\section{Kim jest McGrath?}
% ######################################



% ##################
\begin{frame}
  \frametitle{Poznajmy naszych bohaterów}


  \begin{figure}

    \includegraphics[scale=0.6]{./PresentationPictures/Richard_Dawkins.jpg}
    \includegraphics[scale=0.555]
    {./PresentationPictures/Alister_McGrath_01.jpg}


    \caption{Clinton Richard Dawkins (1941~--), Alister McGrath
      (1953~--)}

  \end{figure}

\end{frame}
% ##################





% ##################
\begin{frame}
  \frametitle{Nasi bohaterowie}


  Richard Dawkins nie muszę chyba przedstawiać,
  skupmy~się więc na drugiej osobie.



  \begin{figure}

    \includegraphics[scale=0.5]
    {./PresentationPictures/Alister_McGrath_02.jpg}


    \caption{Alister McGrath (1953~--)}

  \end{figure}

  \vspace{-1em}



  Krótki opis z~Wikipedii: teolog, duchowny anglikański, historyk
  idei, naukowiec i~apologeta chrześcijański. Związany z~nurtem
  ewangelikalnym Kościoła Anglii.

\end{frame}
% ##################





% ##################
\begin{frame}
  \frametitle{McGrath sam o~sobie}


  \begin{center}

    \includegraphics[scale=0.4]
    {./PresentationPictures/Alister_McGrath_03.jpg}

  \end{center}



  Jako trzynastolatek dałem~się uwieść przyrodzie. [\ldots] W~latach
  szkolnych uważałem --~jak Dawkins --~że~nauki przyrodnicze
  wymagają światopoglądu ateistycznego. Teraz już tak nie myślę.
  [\ldots] Nauki przyrodnicze pokazywały, że~Bóg nie jest potrzebny
  do~wyjaśnienia jakiegokolwiek aspektu świata. Jak wielu w~owych
  dniach [koniec lat 60~XX~w.] w~owych gorących dniach optymizmu
  i~rewolucyjnego zapału czerpałem obficie ze~źródeł marksizmu
  i~uznałem religię za~niebezpieczne złudzenie. \\
  \textbf{A. McGrath \emph{Bóg Dawkinsa. Geny, memy i~sens życia} (dalej
    \cite{McGrathBogDawkinsa2008}), str.~7, 8.}

\end{frame}
% ##################





% ##################
\begin{frame}
  \frametitle{McGrath sam o~sobie}


  \begin{center}

    \includegraphics[scale = 0.5]
    {./PresentationPictures/Alister_McGrath_04.jpg}

  \end{center}


  Podczas nauki do przyszłych studiów odkrył coś w bibliotece. \\
  „Nazywał~się [dział w~szkolnej bibliotece] <<Historia i~Filozofia
  Nauki>> \\
  i~był pokryty grubą warstwą kurzu. Nie miałem zbyt wiele czasu
  na~tego typu rozważania, zresztą uważałem je~za~dyletancką krytykę
  oczywistości i~pewników nauk przyrodniczych, uprawianą przez
  ludzi, dla~których wiedza stanowiła zagrożenie, a~których
  działalność Dawkins nazwał później <<zagłuszaniem prawdy>>.”
  \textbf{Str.~9, \cite{McGrathBogDawkinsa2008}}.

\end{frame}
% ##################





% ##################
\begin{frame}
  \frametitle{McGrath sam o~sobie}


  \begin{center}

    \includegraphics[scale=0.35]
    {./PresentationPictures/Alister_McGrath_05.jpg}

  \end{center}

  Była tam ksiązka
  L.~W.~Hulla \textit{History and~Philosophy~of Science:
    An~Introduction}~(1959). Po latach McGrath ocenia ją jako relatywnie
  słabą, pisaną z~przestarzałej już wtedy perspektywy.

  „Myślę, że~w~głębi duszy wolałbym nigdy nie natrafić na~tę
  książkę, nigdy nie zadać sobie tylu kłopotliwych pytań i~nigdy nie
  podważać prostoty moje naukowej młodości. Lecz nie było już
  powrotu. Przeszedłem przez bramę i~nie mogłem uciec od~nowego
  świata, jaki~się przede mną jawił.” \\
  \textbf{Str.~10, \cite{McGrathBogDawkinsa2008}.}

\end{frame}
% ##################





% ##################
\begin{frame}
  \frametitle{McGrath sam o~sobie}


  \begin{center}

    \includegraphics[scale=0.37]
    {./PresentationPictures/Alister_McGrath_06.jpg}

  \end{center}


  „We~wrześniu 1974~roku znalazłem~się w~zespole profesora George’a
  Raddy na~kierunku biochemii Uniwersytetu Oksfordzkiego.” \\
  „Główny przedmiotem moich zainteresowań były innowacyjne metody fizyczne,
  umożliwiające poznawanie właściwości błon biologicznych, między
  innymi zastosowanie sond fluorescencyjnych i~anihilacji pozytronów
  w~badaniach nad~zależnymi od~temperatury zmianami w~systemach
  biologicznych i~ich modelach. (A.~E.~McGrath, Ch.~G.~Morgan,
  G.~K.~Radda, \textit{Photobleaching: A~Novel Fluorescence Method
      for~Diffusion Studies in~Lipid Systems}. Biochimica
    et~Biophisica Acta, 426 (1976), s.~173--185. \textbf{Str.~11,
      \cite{McGrathBogDawkinsa2008}.}

\end{frame}
% ##################





% ##################
\begin{frame}
  \frametitle{Kilka dat}



  \begin{center}

    \includegraphics[scale=0.44]
    {./PresentationPictures/Alister_McGrath_07.jpg}

  \end{center}


  \begin{enumerate}

  \item \textbf{1977~r.} Pierwszy raz spotyka~spotyka~się z~poglądami
    Richarda Dawkinsa poprzez lekturę \textit{Samolubnego genu}. Jak
    sam stwierdza, minęło parę lat zanim książka ta stanie~się
    pozycją kultową, którą „pozostaje do dzisiaj” \textbf{(str.~7,
      \cite{McGrathBogDawkinsa2008})}. Czy w 2017 roku dalej jest to prawdą?

  \item \textbf{1978~r.} Uzyskuje doktorat z~biofizyki i~dyplom studiów
    teologicznych na Oksfordzie.

  \item \textbf{2004~r.} Publikuje książkę \textit{Bóg Dawkinsa. Geny,
      memy i~sens życia}, na~której bazuje to wystąpienie.

  \end{enumerate}

\end{frame}
% ##################





% ######################################
\section{Ogólne rozważania o~nauce we~współczesnym świecie}
% ######################################



% ##################
\begin{frame}
  \frametitle{Przechodzimy do rzeczy}

  Ważne pytanie: skąd ludzie czerpią wiedzę o~współczesnych teoriach,
  zarówno w~naukach ścisłych, jak i~humanistycznych? Według mnie:
  \begin{itemize}

  \item[--] z~filmów, seriali, komiksów, książek fabularnych, etc.;

  \item[--] z~artykułów, filmów, wykładów popularnonaukowych
    w~internecie, gazetach, etc.;

  \item[--] od~znajomych;

  \item[--] z~książek popularnonaukowych;

  \item[--] z~krótkich uwag rzucanych przez wykładowców, na temat
    myśli, \\
    bądź prac wielkich uczonych;

  \item[--] z~wielotomowych opracowań, 900 stron każdy, napisanych
    przez kompetentnych ludzi, które w~sposób wyczerpujący
    i~dokładny rozważają wszystkie dowody, argumenty przemawiające
    za tymi teoriami, jak i~ich konsekwencje.

  \end{itemize}

\end{frame}
% ##################





% ##################
\begin{frame}
  \frametitle{Przechodzimy do rzeczy}


  Dwa ostatnie źródła~są mają znikomy wpływ na~ludzi których znam \\
  i~na mnie. Ale próbuję~się podszkolić.

\end{frame}
% ##################





% ##################
\begin{frame}
  \frametitle{Memetyka}


  Moje podejrzenie jest takie, że większość ludzi poznało
  memetykę z~tych źródłem, które przekazują wiedzę w~sposób skrótowy,
  bezkrytyczny, apelujący bardziej do~emocji i~pragnień, niż~do~argumentów
  i~dowodów. Nie wydaje mi~się też, by~osoby które spotkałem, poświęciły
  wiele czasu na jej zrozumienie.


  Jeśli tak jest, to należy ustalić prawdziwą wartość memetyki i~w~świetle
  tej wiedzy spróbować zrozumieć, co nam mówi o~świecie jej popularność. \\
  \textbf{To jest główny cel tego wystąpienia.}

  Wśród moich rówieśników, czyli osób urodzonych w~drugiej połowie \\
  lat 80 XX~w., Dawkins stał~się powszechnie znaną postacią, wraz
  z~publikacją \textit{Boga urojonego} (oryginał 2006, polskie wydanie
  2007). \\
  \textit{Bóg Dawkinsa} ukazał~się 2004~r., więc \textbf{NIE} jest
  odpowiedzią na \textit{Boga urojonego}.

\end{frame}
% ##################





% ##########
\begin{frame}
  \frametitle{Dlaczego powstała memetyka?}


  Darwinizm jest zbyt wielką teorią, aby~sprowadzać
  go~tylko do~biologii. Dlaczego ograniczać darwinizm do~świata
  genów, skoro jego treści dotyczą wszystkich aspektów ludzkiego
  życia i~myślenia? W~\textit{Samolubnym genie}~(1976) Dawkins
  wyjaśnia, że~od~dawna interesowała go~analogia pomiędzy informacją
  kulturową a~genetyczną. Czy teoria darwinowska nie~może
  stosować~się do~ludzkiej kultury tak samo jak do~świata biologii?
  Ten pomysł leży u~podstaw przekształcenia darwinizmu z~teorii
  naukowej w~światopogląd, metanarrację, całościowy obraz
  rzeczywistości.

  Wymaga to jednak znalezienia kulturowego odpowiednika
  genu~-- „replikatora kulturowego”, który zapewniałby
  przenoszenie informacji w~czasie i~przestrzeni. Gdyby pojęcie
  replikatora udało~się solidnie umiejscowić w~kategoriach
  naukowych, darwinizm stałby~się metodą uniwersalną, wykraczającą
  poza wąską dziedzinę ewolucji biologicznej i~obejmującą świat
  kultury.

  J.~Poulshock, \textit{Universal Darwinism
    and~the~Potential~of Memetics}. „Quarterly Review~of Biology”
  77 (2002), s.~174--175. Str. 117, \cite{McGrathBogDawkinsa2008}.

\end{frame}
% ##################





% ##################
\begin{frame}
  \frametitle{Dlaczego powstała memetyka?}


  Z~tego wynikałoby,~że
  u~podstaw jej powstania leży częsta u~naukowców wiara, że~to
  czym~się zajmują jest najważniejszą, ostateczną teorią
  \textbf{wszystkiego} („Universal Darwinism”). Bo niby kto
  zajmowałby~się teorią, która nie jest najlepsza i~najgłębsza?
  (Oprócz tych którzy patrzą, gdzie są największe granty.)
  Przykładów tego we~współczesnej nauce jest niemało, \\
  np.~„spór” Stevena Weinberga i~Philipa Andersona pod tytułem:
  „Redukcjonizm vs~emergencjonizm w~fizyce”.

  Zatrzymajmy~się na tym na chwilę. Czynnik typu „Wreszcie mamy teorię,
  która wyjaśni wszystko!”,
  na~pewno odgrywa tu~ważną rolę, bo naukowcy też są~ludźmi i~mają
  ludzkie słabości. W~tym słabość do szybkich, łatwych rozwiązań,
  które nie wymagają zbyt dużo pracy. Takie postawienie sprawy
  jednak nie jest satysfakcjonujące, spróbuję to bardziej rozwinąć.

\end{frame}
% ##################





% ##################
\begin{frame}
  \frametitle{„Darwinizm” jako teoria wszystkiego}


  Dlaczego właśnie Charles Darwin jest taki ważny?

  \begin{enumerate}

  \item Jest centralną teorią współczesnej biologii, unifikującą jej
    wszystkie działy, dla biologa to~naturalny kandydat na~teorię
    wszystkiego.

  \item Istnieje ogólny prąd w~szeroko pojętej kulturze, który mówi,
    że~to biologia \textbf{będzie} najważniejszą teorią nowej epoki: XXI
    wieku (bardzo mnie to dziwi, że~wciąż mówimy o~XXI wieku jako
    o~przyszłości, żyjemy w~nim przecież od~16 lat). Więc to ona
    powinna nam dostarczyć teorii wszystkiego, co~czyni teorię
    ewolucji atrakcyjną dla nie-biologów.

  \item Spektakularny sukces teorii na~jednym polu, często wywołuje
    u~ludzi, naukowców i~nie-naukowców, nieuzasadniony niczym
    entuzjazm, że~teoria ta rozwiąże wszystko. Patrz np. newtonizm
    w~XVIII~wieku, Wolter, baron d'Holbach, La Mettrie, Laplace
    i~inni.

  \end{enumerate}

\end{frame}
% ##################





% ##################
\begin{frame}
  \frametitle{„Darwinizm” jako teoria wszystkiego}


  \begin{enumerate}

  \item[4.] Syntetyczna teoria ewolucji jest gotowa, nie trzeba nic
    wymyślać.

  \item[5.] \textbf{Uniwersalny darwinizm} jest \textbf{samotłumaczącą~się
      teorią} (słowa jednej osoby z~którą o~tym dyskutowałem).

  \end{enumerate}

\end{frame}
% ##################





% ##################
\begin{frame}
  \frametitle{„Darwinizm” jako teoria wszystkiego}


  Wydaje mi~się, że~nie zrozumiemy tego wszystkiego, jeśli nie
  zwrócimy uwagi, iż~w~tle tych rozważań znajduje~się pewien ciąg
  myśli, prawie nigdy nie formułowany jawnie. Brzmi mniej więcej
  tak. Skoro powstanie człowieka tłumaczy teoria ewolucji
  i~wszystkie jego cechy~są przez ten proces zdeterminowane, to musi
  ona też tłumaczyć wszystko co on robi lub~myśli. W~szczególności,
  skoro wszystkie idee i~teorie naukowe, odkrył bądź wymyślił jakiś
  człowiek, to~musi istnieć ewolucyjne uzasadnienie dla tych
  wszystkich teorii.

  Do~tego zagadnienia jeszcze wrócimy.

\end{frame}
% ##################










% ######################################
\section{Główne zarzuty McGratha}
% ######################################



% ##################
\begin{frame}
  \frametitle{McGrath jest również historykiem idei, więc}


  \begin{figure}

    \centering

    \includegraphics[scale=0.449]
    {./PresentationPictures/Herbert_Spencer.jpg}
    \includegraphics[scale=0.7]
    {./PresentationPictures/Edward_Osborne_Wilson.jpg}


    \caption{Herbert Spencer (1820--1903), Edward Osborne Wilson
      (1929--2021), prominentni „kulturowi darwiniści”}

  \end{figure}


  Pojęcie \textbf{replikatora kulturowego} wprowadził w~1960 roku
  psycholog ewolucyjny Donald T.~Campbell (1916--1996), nadał mu nazwę
  „mnenomu”. Wśród socjobiologów amerykańskich zyskał popularność
  pokrewna teoria „kuturogenu”. Str.~118, \cite{McGrathBogDawkinsa2008}.

\end{frame}
% ##################





% ##################
\begin{frame}
  \frametitle{Teoria „i-kultury” i~„m-kultury”}


  W~artykule z~1968 roku, wersja poszerzona z~1975~r., antropolog
  F.~T.~Cloak „sformułował tezę, że~rozwój kultury podlega
  zasadniczo mechanizmom drawinowskim i~zaproponował stosowanie
  metod etologicznych do~wyjaśniania zachowań kulturowych.”
  Wprowadził też rozróżnienie na „i-kulturę”, czyli „zbioru
  instrukcji kulturowych zawartych w~układzie nerwowym” oraz
  „m-kulturę” czyli „relacje w~strukturach materialnych
  podtrzymywane przez owe instrukcje lub zmiany w~strukturach
  zachodzące pod wpływem instrukcji”. (F.~T.~Cloak,
  \textit{Is~a~Culture Ethology Possible?}. Human Ecology, 3
    (1975), s.~161--181.) Str. 119, \cite{McGrathBogDawkinsa2008}.

  Trochę to skomplikowane, spróbujmy znaleźć jakiś przykład.

\end{frame}
% ##################





% ##################
\begin{frame}
  \frametitle{Teoria „i-kultury” i~„m-kultury”}


  \begin{figure}

    \centering

    \includegraphics[scale=0.25]{./PresentationPictures/Nervous_system.jpg}
    \includegraphics[scale=0.2]{./PresentationPictures/Taniec.jpg}


    \caption{Taniec}

  \end{figure}


  I-kultura, to instrukcja jak tańczyć zawarta gdzieś
  w~ludzkim układzie nerwowym („i” pochodzi zapewne od
  „intellectual” lub podobnego słowa), zaś m-kultura
  to~konkretny taniec wykonany przez człowieka, bo ta czynność
  podtrzymuje relacje przestrzenne między ludzkimi ciałami,
  kostiumami, etc., czyli relacje w~strukturach materialnych („m”
  pochodzi zapewne od~„material”).

\end{frame}
% ##################





% ##################
\begin{frame}
  \frametitle{Historia jako krytyka}


  Czy nie jest tak, że~sam fakt
  umieszczenie memetyki w~konkretnym miejscu historii myśli jest już
  formą krytyki, podważenia jej prawdziwości? Bowiem, tym samym:

  \begin{enumerate}

  \item Memetyka przestaje~się jawić jako zupełnie oryginalna
    teoria, a~kultura Zachodu bardzo cenni oryginalność. Ale~to
    temat na osobną dyskusję.

  \item Skoro istnieje wiele teorii kulturowego darwinizmu, to która
    jest poprawna?

  \end{enumerate}

\end{frame}
% ##################





% ##################
\begin{frame}
  \frametitle{Główne zarzuty McGratha}


  \begin{enumerate}

  \item Nie ma powodu, by~zakładać, że~ewolucja kulturowa
    ma~charakter darwinowski albo~że~biologia ewolucyjna może~się
    w~istotny sposób przyczynić do~wyjaśnienia rozwoju idei.

  \item Nie ma bezpośrednich dowodów na istnienie „memów”.

  \item Przekonanie o~istnieniu „memów” wynika z~problematycznego
    założenia bezpośredniej analogii do~genów, jednak okazuje~się,
    że~nie jest ono w~stanie sprostać rygorom teorii naukowych.

  \item Wskazywanie na~istnienie „memu” jako konstruktu
    wyjaśniającego rzeczywistość nie jest konieczne. Dane
    z~obserwacji można równie dobrze opisać za~pomocą innych modeli
    i~mechanizmów.

  \end{enumerate}

  Str. 119, \cite{McGrathBogDawkinsa2008}.

\end{frame}
% ##################





% ##################
\begin{frame}
  \frametitle{Dodatkowe problemy z~memetyką}


  Dalsza dyskusja wymaga postawienia w~sposób jawny i~prosty, kilku
  problemów na~jakie natrafiłem próbując zrozumieć „darwinizmem
  kulturowym”.

  \begin{enumerate}

  \item Nie jest dla mnie jasne, czego ludzie zajmujący~się memetyką
    od~niej oczekują bądź oczekiwali w~przeszłości. Czy ma być ona „klonem”
    darwinowskiej teorii ewolucji, czy też teorią inspirowaną przez
    nią, a~mogącą~się od~niej nawet fundamentalnie różnić?

  \item Jaki jest zakres stosowania memetyki? Czy ma ona wyjaśniać
    całą kulturę i~myśl ludzką, czy tylko jej część? Np. taniec tak,
    ale techniki siania zboża już nie? W~szczególności, czy teorie
    naukowe rządzone są prawami memetyki? (Na to ostatnie jakieś
    odpowiedzi~są udzielane, ale~mało satysfakcjonujące.)

  \end{enumerate}

\end{frame}
% ##################





% ##################
\begin{frame}
  \frametitle{Zarzut 1: ewolucja kultury}

  Wszystkie argumenty pochodzą z~4~rozdziału \cite{McGrathBogDawkinsa2008}.

  Historia kultury pokazuje, że~ogromną rolę odgrywają w~niej działania
  \textbf{celowe, zamierzone i~zaplanowane}, takie jak chęć wskrzeszenia
  potęgi Rzymu, dawnych nurtów filozoficznych, etc. Ortodoksyjny darwinizm
  raczej nie pozwala na rozumienie czegoś w~tych kategoriach.
  % Do~tego zagadnienia jeszcze wrócę.


  Drugi problem dotyczy tego co jest dziedziczone?
  W~\textit{Samolubnym genie} Dawkins jako przykłady memów wymienia
  melodie, poglądy, obiegowe zwroty, mody, detale architektoniczne,
  pieśni i~ideę Boga. Problem w~tym, że~jeśli mem ma być
  odpowiednikiem genu, a~tak rozumie go Dawkins, to podane przykłady
  nie~są elementami kulturowego \textit{genotypu}, lecz~\textit{fenotypu}.
  Mówiąc inaczej mem piosenki, ma tak~się do śpiewanej piosenki, jak
  ciągu zasad azotowych w~DNA do niebieskiego koloru oczu.

\end{frame}
% ##################





% ##################
\begin{frame}
  \frametitle{Zarzut 1: ewolucja kultury}


  Zdając sobie sprawę z~tego problemu Dawkins w~\textit{Fenotypie
    rozszerzonym} (1982), przedstawił
  następujące wyjaśnienie.

  „Nie wprowadziłem dostatecznie wyraźnego rozróżnienia między memem
  jako replikatorem a~jego <<efektem fenotypowym>> w~postaci
  <<produktów memu>>. Mem powinien być postrzegany jako jednostka
  informacji mieszcząca~się w~mózgu (<<i-kultura>> według Cloaka).
  Ma on określoną strukturę w~tym rodzaju fizycznego medium, w~jakim
  mózg magazynuje informacje. [\ldots] W~ten sposób da~się go~odróżnić od
  jego efektów fenotypowych, które istnieją w~otaczającym świecie
  (<<m-kultura>> według Cloaka).” Str. 120--121,
  \cite{McGrathBogDawkinsa2008}.

  McGrath stwierdza, zapewne słusznie, że~memetyka
  zyskała popularność w~swej wersji z~\textit{Samolubnego genu}, zaś
  wersja zrewidowana z~\textit{Fenotypu rozszerzonego} nie odbiła~się
  większym echem.

\end{frame}
% ##################





% ##########
\begin{frame}
  \frametitle{Zarzut 1: ewolucja kultury}


  „Idee naukowe, jak wszystkie memy, są~przedmiotem doboru naturalnego
  i~na~pozór mogą przypominać wirusy. Jednak czynniki doboru, które
  weryfikują idee naukowe, nie~są arbitralne ani~kapryśne. Są~to ścisłe,
  precyzyjne reguły, które nie faworyzują niepotrzebnych zachowań
  obliczonych na~własną korzyść.” Fragment dzieła Dawkinsa
  \textit{Kapłan diabła}, str. 122, \cite{McGrathBogDawkinsa2008}.

  To jednak rodzi więcej pytań, niż odpowiedzi Co decyduje, że~w~pewnych
  przypadkach czynniki doboru~są
  arbitralne i~kapryśne, a~w~innych nie? I~jakie argumenty
  przemawiają za~tym, że~dla nauki takie nie~są? Można, w~tym
  kontekście prowokacyjnie, zapytać, czy czynniki doboru religii też
  nie~są arbitralne i~kapryśne? Coś takiego zdaje się sugerować David Sloan
  Wilson w~książce \textit{Darwin's Cathedral}.

  Nie znam tekstu Dawkinsa w którym udziela dobrej odpowiedzi na te pytania.

\end{frame}
% ##################





% ##################
\begin{frame}
  \frametitle{Zarzut 4: memetyka jest niepotrzebna}


  Poniższe uwagi odnoszą się do stanu nauki na 2004~r., jednak według mnie
  wciąż są aktualne.

  Jeśli chcemy tłumaczyć kulturę przez odniesienie do nauk przyrodniczych,
  to użyteczne okazują się modele wzorowane na tych spotykanych w~fizyce,
  czy ekonomii. Zamiast więc „biologicznej teorii kulturowej” możemy podać
  „fizyczną teorię kulturową” albo „ekonomiczną teorię kulturową”.

  Możemy podać nazwy kilku takich modeli.
  \begin{itemize}

  \item[--] Transmisja informacji na sieciach losowych.

  \item[--] Idee jako kaskady informacji lub trwałe artykuły
    konsumpcyjne.

  \item[--] Teoria chwilowych mód (ang. \textit{theory~of fads}).

  \item[--] Automaty komórkowe

  \end{itemize}

  Jeśli jest potrzeba, bądź chęć, mogę potem powiedzieć o~tym coś
  więcej.

\end{frame}
% ##################





% ##################
\begin{frame}
  \frametitle{Zarzut 4: memetyka jest niepotrzebna}


  McGrath zauważa Że~część z~wymienionych modeli, może wszystkie,
  „zawierają darwinowskie motywy <<rywalizacji>> i~<<wygasania>>, choć
  nie zawsze przyjmują związane z~nimi poglądy na temat źródeł
  innowacji”. (Str.~131 \cite{McGrathBogDawkinsa2008} Jednak z~własnego
  doświadczenia wiemy, że~mody przychodzą, słabną, rosną, przemijają,
  powracają, więc każda teoria kultury musi wyjaśnić to zjawisko (Kino Nowej
  Nowej Przygody, Kino Nowej Nowej Przygody). Czy to jednak uzasadnia
  nazwanie danej teorii „darwinowską”?


  Może właściwe jest spojrzenie odwrotne: to darwinizm jest szczególnym
  przypadkiem teorii \textit{„powielania i~rozprzestrzeniania”}? To ważne
  pytanie, ale~nie ma nic wartościowego do powiedzenia w~tej kwestii.

\end{frame}
% ##################





% ##################
\begin{frame}
  \frametitle{Zarzut 4: memetyka jest niepotrzebna}


  Kiedy poda się wyjaśnienie jakiegoś zjawiska w~terminach memetyki, to
  w~zasadzie zawsze mam z nim pewne problemy.

  „Mem został zdefiniowany przez jego zwolenników tak szeroko,
  że~stał~się pojęciem bezużytecznym, które raczej wprowadza zamęt
  niż cokolwiek rozjaśnia. Uważam, że~wkrótce zostanie zapomniany
  jako dziwactwo językowe bez~większej wartości. Dla krytyków,
  którzy obecnie znacznie przewyższają liczą wyznawców, memetyka nie
  jest niczym więcej niż niezręczną terminologią, mającą wyrażać
  coś, co~i~tak wszyscy wiedzą i~co można powiedzieć trafniej,
  odwołując~się do mnie efektownej terminologii transferu informacji.” \\
  M.~Gardner, \textit{Kilroy Was Here}, „Los Angeles Times”, 5~marca 2000~r.

  Definicja memu jest tak szeroka, że~chyba NIE jest możliwe
  znalezienie czegoś, czego ona nie tłumaczy. Pytanie czy udzielone
  wyjaśnienie ma jakąś wartość.

\end{frame}
% ##################





% ##################
\begin{frame}
  \frametitle{Zarzut 4: memetyka jest niepotrzebna}


  Jeśli zauważę, że~jedna osoba przejęła jakieś myśli,
  upodobania, idee od~drugiej, to jest to dowód na prawdziwość
  memetyki, jeśli zinterpretuje to zjawisko, jako „kopiowanie
  memów”. Ale~dlaczego mam przyjąć taką interpretację, a~nie
  inną? Dlaczego np.~nie przyjąć interpretacji psychoanalizy
  w~jednej z~jej miliarda wersji, że~to co zostało przekazane,
  zostało przyjęte, aby~zaspokoić wewnętrzne potrzeby
  „kopiującego”, nie troszcząc~się o~sam mechanizm przekazu?

  % \item Argument za tym, że~coś jest w~danym momencie popularne
  %     np.~PSY \textit{Gangnam Style} w~2012 roku, jest taki, że~jest dobrze
  %     przystosowane do~obecnych warunków. Z~dobrego przystosowania wynika więc popularność danego zjawiska.
  %     Jeśli~się jednak nie wskaże na~czym konkretnie to dostosowanie
  %     polega, to nic tak na prawdę nie mówi, a~tym bardziej nie
  %     tłumaczy. Znowu weźmy przykład \emph{Gangnam Style}.
  %   \item Ponownie pojawia~się pytanie: czy memetyka jest popularna
  %     w~pewnych kręgach, bo jest do nich dobrze dostosowana?
  %   \end{enumerate}
  % \end{block}

\end{frame}
% ##################





% ##################
\begin{frame}
  \frametitle{Zarzut 4: memetyka jest niepotrzebna}


  Argument za tym, że~coś jest w~danym momencie popularne
  np.~PSY \textit{Gangnam Style} w~2012 roku, jest taki, że~jest dobrze
  przystosowane do~obecnych warunków. Z~dobrego przystosowania wynika więc
  popularność danego zjawiska. Jeśli~się jednak nie wskaże na~czym
  konkretnie to dostosowanie polega, to nic tak na prawdę nie mówi, a~tym
  bardziej nie tłumaczy.

  Znowu weźmy przykład \textit{Gangnam Style}. Był popularny w~2012 roku bo
  był dobrze przystosowany do warunków jakie wtedy panowały. Co to tak
  naprawdę znaczy?

  Ponownie pojawia~się pytanie: czy memetyka jest popularna
  w~pewnych kręgach, bo jest do nich dobrze dostosowana?

\end{frame}
% ##################





% ##################
\begin{frame}
  \frametitle{Zarzuty 2 i~3: geny, memy, obserwacja}


  Ponownie, nic mi nie wiadomo o~jakimś fundamentalnym postępie
  w~zakresie tych dwóch problemów. Chętnie dowiem~się o~tym, że~się mylę.

  W przypadku genetyki prace badawcze nad~dziedziczeniem z~okolic przełomu
  XIX i~XX wieku, pokazywały potrzebę istnienia jednostki przekazywania
  cech. Był więc dobry dowód pośredni jej przyjęcia, nawet jeśli on sama
  nie była znana.

  Ponieważ możemy opisać ewolucję kultury jako ewolucję „niedarwinowską”,
  to nie możemy użyć argumentu na rzeczy istnienia memu na zasadzie
  analogi. „Skoro ewolucja biologiczna wymaga genu, to ewolucja kulturowa
  też wymaga czegoś takiego.”

  Istnieją całkiem obiecujące teorie działania kultury, w którym pojęcia
  memu zwyczajnie nie ma, nie jest więc on bytem koniecznym.

  Może więc należy przyjąć istnienie memu, bo go widzieliśmy?

\end{frame}
% ##################





% ##################
\begin{frame}
  \frametitle{Zarzuty 2 i~3: geny, memy, obserwacja}


  \begin{figure}

    \centering
    \includegraphics[scale=0.4]{./PresentationPictures/Mem_structure.png}

    \caption{Jaki mem może być bardziej widoczny od~tego?}

  \end{figure}


  Konkretny mem internetowy, to „fenotyp memu”, czy więc ta
  struktura na~obrazku, to jego „genotyp”? A~może za genotyp
  należy uważać strukturę plików graficznego w~którejś z~pamięci
  komputera? Albo obraz oglądany na ekranie komputera? Jak w~takim wypadku
  wyglądałby proces kopiowania? Czy z~tego wynika, że~memy są kopiowane
  „fenotypowo”?

\end{frame}
% ##################





% ##################
\begin{frame}
  \frametitle{Zarzuty 2 i~3: geny, memy, obserwacja}


  Neuronauka jest na topie, więc nasuwa się pytanie
  czy udało~się zaobserwować w~mózgu analog helisy DNA dla~memów?

  Ani McGrath w~2004, ani ja dziś nic nie wiemy o~udanej próbie
  znalezienia~go. Są~pewne sugestie, hipotezy i~teorie, ale~ktoś kto
  zna dobrze naukę, że~one zawsze~są, jaki problem by nie był.
  Pytanie co~one mają wspólnego z~rzeczywistością?

  Nie znam żadnego opisu procesu kopiowania memów, oprócz ogólnika
  „że~przeskakują z~jednego mózgu do~drugiego, w~procesie szeroko
  rozumianego naśladownictwa”. (Richard Dawkins, \textit{Samolubny gen},
  str.120, \cite{McGrathBogDawkinsa2008}.

  To jednak za mało, by~na~podstawie tego dało~się coś wartościowego
  stwierdzić. Skoro potrzebujemy~się odwołać, do~„szeroko
  rozumianego naśladownictwa”, to~czemu w~ogóle wprowadzamy memy,
  zamiast na tym poprzestać?

\end{frame}
% ##################





% ##################
\begin{frame}
  \frametitle{Zarzuty 2 i~3: geny, memy, obserwacja}


  Skoro brak wiadomości o~tym jak wygląda mechanizm kopiowania,
  to~ilość danych doświadczalnych potwierdzających jego Stenie musi być
  jeszcze mniejsza, niż te które wskazują na istnienie memu. Zwróćmy też
  uwagę, że~mechanizm kopiowania genu dużo tłumaczy, więc brak
  analogicznego mechanizmu dla memu mocno obciąża teorię.

  „Memy mogą~się zlewać ze~sobą, co~się nie zdarza genom. Nowe
  <<mutacje>> mogą być częściej <<kierunkowe>> niż <<losowe>>.
  Odpowiednik doktryny Weismanna (weismanizmu) w~odniesieniu
  do~memów byłby znacznie mniej rygorystyczny, <<lamarkowskie>>
  strzałki przyczynowo-skutkowe mogą bowiem prowadzić zarówno
  od~replikatora do~fenotypu, jak i~w~drugą stronę. Te~różnice mogą
  sprawić, że~cała analogia do~genowego doboru naturalnego okaże~się
  w~rzeczywistości całkowicie pozorna, a~nawet, co~gorsza, błędna.”
  Richard Dawkins, \textit{Fenotyp rozszerzony}, str.~125,
  \cite{McGrathBogDawkinsa2008}.

\end{frame}
% ##################





% ##################
\begin{frame}
  \frametitle{Weźmy jeszcze jeden przykład: „Gwiezdne wojny” jako mem}


  \begin{figure}

    \centering

    \includegraphics[scale=0.464]
    {./PresentationPictures/Joseph_Campbell.jpg}
    \includegraphics[scale=0.25]{./PresentationPictures/Darth_Vader.jpg}


    \caption{Joseph Campbell (1904--1987), intelektualny ojciec
      „Gwiezdnych wojen”,
      Darth Vader, ojciec chyba wiecznie żywy ;)}

  \end{figure}



  Jeśli tak, to byłby to bardzo skomplikowany mem, na~którego
  zlały~się memy psychoanalizy, s-f, filmy przygodowe,
  nazistowskie mundury od~Hugo Bossa, etc. (Temat na osobną dyskusję.)
  Jeżeli już, to byłby to raczej „organizm memowy”, tak jak słowik jest
  „organizmem genowym”.

\end{frame}
% ##################










% ######################################
\section{Podsumowanie i~otwarte pytania}
% ######################################



% ##################
\begin{frame}
  \frametitle{Mnie lub bardziej subiektywne wnioski}


  \begin{enumerate}

  \item Rozważania natury nauki, to nie jest „zagłuszanie prawdy”.

  \item Należy sobie zadać pytaniem, czy w~danej sytuacji,
    należy~się zajmować teorią w~jej prawdziwym kształcie, czy~jej
    powszechną recepcją?

  \item Nauka nie staje~się mniej wartościowa, gdy~zrozumiemy,
    że~ludzkie preferencje i~słabości wpływają na jej rozwój, często
    w~sposób zasadniczy.

  \item Przekonanie, że~jesteśmy już u~wrót ostatecznej teorii
    czegoś-tam, jest bardzo powszechną i~groźną przypadłością. Do tego do
    tej pory okazywało się zupełnie nieprawdziwe.

  \item Hype wokół jakiegoś, może być ważniejszy od~samego przedmiotu
    hypu.

  \item Wąskie zainteresowania, jeśli czemuś sprzyjają,
    to~ignorancji wobec własnej ignorancję.

  \item Skoro o~biologii człowieka zadecydowała ewolucja, to~czy
    gdzieś jej wpływ~się kończy?

    \end{enumerate}

\end{frame}
% ##################





% ##################
\begin{frame}
  \frametitle{Uwagi o~memetyce}

  \begin{enumerate}

  \item Memetyka pojęta jako „klon” syntetycznej teorii ewolucji
    wydaje się niemożliwa (celowość rozwoju, zlewanie~się memów, etc.)

  \item Nie znalazła dobrego wyjścia z~pułapki samotłumaczenia~się.
    Rozwiązanie przez tworzenie arbitralnych wyjątków, jest naprawdę słabe.

  \item W~wersjach które ja znam memetyka albo stosuje~się do~wszystkiego,
    za to w sposób nijaki, albo brak jej aparatu by~uchwycić
    konkretne dzieła kultury. (Czym~są „Gwiezdne Wojny”?)

  \item Choć uwzględnia realnie występujące zjawiska, jak
    przenoszenie i~kopiowanie informacji, „stopień
    przystosowania”, to nie potrafiła na~tej podstawie dokonać
    prawdziwych, wartościowych odkryć.

  \item Na~dzisiejszym poziomie wiedzy, zupełnie inne
    „przyrodnicze” drogi badania kultury wydają~się właściwe. Jak
    będzie zobaczymy, w~każdym razie, już w~tej okolicy widać było duży
    hype.

  \end{enumerate}

\end{frame}
% ##################





% ##################
\begin{frame}
  \frametitle{Skąd się wzięła popularność memetyki?}


  Kilka możliwych wyjaśnień.
  \begin{enumerate}
  \item Trafia do~pewnych potrzeb i~pragnień, takich jak szybkie,
    łatwe zrozumienia świata.

  \item Pewien typ osób, ze~względu na~swoje zainteresowania jak
    i~wykształcenie, ma do niej naturalną słabość.

  \item Pewna płytkość rozmów o~nauce jaka obecnie panuje (mówię to
    na~podstawie własnego doświadczenia z~teoriami fizycznymi),
    wpiera bezkrytyczne podejście do~niej.

  \item„Czynnik doboru”. Jak zwrócił uwagę McGrath, Dawkinsowi
    udało~się to, czego jego poprzednicy nie byli w~stanie zrobić.
    Stworzył zręczniejszą i~prostszą w~zapamiętaniu terminologię,
    popierając przy tym swe rozważania, bardzo przemawiającymi
    do~wyobraźni przykładami.

  \item „Zrozumienie człowieka = zrozumienie wszystkiego”.

  \end{enumerate}

\end{frame}
% ##################





% ##################
\begin{frame}
  \frametitle{Może zwróciliście uwagę}

  Że omawiałem tylko~„przyrodnicze teorie kultury”, a~przecież są jeszcze
  nauki humanistyczne.


  Kilka humanistycznych teorii kultury.
  \begin{itemize}

  \item[--] Psychoanaliza, miliard odmian. Freud, Jung, Lacan,
    \v{Z}i\v{z}ek, etc.

  \item[--] Teoria „bajki magicznej” Władimir Proppa i~jej pochodne.

  \item[--] Antropologia strukturalistyczna C. Leviego-Straussa.
    Np.~w~książce \textit{Surowe i~gotowane} (1964), próbuje
    pokazać, że~wszystkie mity Indian Południowej Ameryki (i~pewne
    zachowania francuzów, ale to jest dopiero na samym końcu), można
    wywieść z~jednego mitu, za pomocą serii algorytmów, choć nie stwierdza
    tego jawnie.

  \item[--] Ogólnie, ruch strukturalistów i~poststrukturalistów.

  \end{itemize}

\end{frame}
% ##################





% ##################
\begin{frame}
  \frametitle{Podobieństwa}


  Między memetyką a~humanistycznymi teoria kultury jest wiele
  zastanawiających podobieństw.
  \begin{enumerate}

  \item „Zrozumienie człowieka = zrozumienie wszystkiego”.
    Wszak~są to nauki \textit{humanistyczne}, więc mają zrozumieć
    człowieka.

  \item Podobny hype: teoria wszystkiego jest tuż za rogiem.

  \item Niezwykła popularność psychoanalizy.

  \item Przynajmniej w~przypadku psychoanalizy, te same problemy
    z~zastosowanie do konkretnego dzieła. Jeśli ktoś jest w stanie
    przeczytać książkę K.~Loski \textit{Hitchcock~-- autor wśród gatunków},
    to powinien zrozumieć o~co chodzi.

  \end{enumerate}

\end{frame}
% ##################





% ##################
\EndingSlide{Dziękuję! Pytania?}
% ##################










% ##################
\begin{frame}
  \frametitle{Bibliografia}


  \begin{itemize}

  \item[--] Alister McGrath, \textit{Bóg Dawkinsa. Geny, memy i~sens
      życia}. Wydawnictwo Uniwersytetu Jagiellońskiego, Kraków,
    wydanie~I, 2008. Poza tym co powiedziałem, zawiera wiele innych
    wartościowych rzeczy.

  \item[--] A.~E.~McGrath, Ch.~G.~Morgan, G.~K.~Radda,
    \textit{Photobleaching: A~Novel Fluorescence Method for~Diffusion
      Studies in~Lipid Systems}. Biochimica et~Biophisica Acta,
    426 (1976), s.~173--185.

  \item[--] J.~Poulshock, \textit{Universal Darwinism
      and~the~Potential~of Memetics}. Quarterly Review~of
    Biology, 77 (2002), s.~174--175.

  \item[--] F.~T.~Cloak, \textit{Is~a~Culture Ethology Possible?}
    Human Ecology, 3 (1975), s.~161--181.

  \item[--] J.~Poulshock, \textit{Universal Darwinism and~the~Potential~of
      Memetics}, Quarterly Review~of Biology, 77 (2002), s.~174--175.
  \end{itemize}

\end{frame}
% ##################





% ##################
\begin{frame}
  \frametitle{Bibliografia}


  \begin{itemize}

  \item[--] Sushil Bikhchandani, David Hirshleifer, Ivo Welch,
    \textit{A~Theory~of Fads, Fashion, Custom, and~Cultural Change
      as~Informational Cascades}. Journal of Political Economy,
    100, (1992), s.~992--1026.

  \item[--] M.~Gardner, \textit{Kilroy Was Here}, „Los Angeles
    Times, 5~marca 2000~r.

  \item[--] David Sloan Wilson, \textit{Darwin's Cathedral: Evolution,
      Religion, and~the~Nature~of Society}.

  \item[--] D. Oramus, \textit{Darwinowskie paradygmant. Mit teorii
      ewolucji w~kulturze współczesnej}. Copernicus Ceter Press,
    Kraków, wydanie~I, 2015.

  \item[--] W. Stoczkowski, \emph{Ludzie, bogowie i~przybysze
      z~kosmosu} (irytuje mnie arogancja autora, ale~pozycja warta
    poznania).

  \end{itemize}

\end{frame}
% ##################










% ##################
\begin{frame}
  \frametitle{Bibliografia}


  \bibliographystyle{plalpha}

  \bibliography{VariousFieldsBooks}{}

\end{frame}
% ##################





% ############################

% Koniec dokumentu
\end{document}
