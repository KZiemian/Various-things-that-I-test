% ---------------------------------------------------------------------
% Basic configuration of Beamera and Jagiellonian
% ---------------------------------------------------------------------
\RequirePackage[l2tabu, orthodox]{nag}



\ifx\PresentationStyle\notset
\def\PresentationStyle{dark}
\fi



\documentclass[10pt,t]{beamer}
\mode<presentation>
\usetheme[style=\PresentationStyle,logoLang=Latin,logoColor=monochromaticJUwhite,JUlogotitle=yes]{jagiellonian}



% ---------------------------------------
% Configuration files of Jagiellonian loceted in catalog preambule
% ---------------------------------------
\input{./preambule/LanguageSettings/JagiellonianPolishLanguageSettings.tex}
\input{./preambule/TextposConfiguration/TextposConfiguration.tex}

\input{./preambule/JagiellonianCustomizationGeneral.tex}
\input{./preambule/JagiellonianCustomizationCommands.tex}










% ---------------------------------------
% Packages, libraries and their configuration
% ---------------------------------------
\usepackage{mathcommands}





% ---------------------------------------
% Configuration for this particular presentation
% ---------------------------------------










% ---------------------------------------------------------------------
\title{Ogólna współzmienniczość i~działanie Cherna-Simonsa}

\author{Kamil Ziemian,
  \texttt{kziemianfvt@gmail.com}}


\institute{II rok, fizyka teoretyczna, studia magisterskie}

\date[25 IV 2013]{25 kwietnia 2013 r.}
% --------------------------------------------------------------------










% ####################################################################
% Początek dokumentu
\begin{document}
% ####################################################################





% Wyrównanie do lewej z łamaniem wyrazów

\RaggedRight





% ######################################
\maketitle
% ######################################





% ##################
\begin{frame}
  \frametitle{Czym jest ogólna współzmienniczość?}


  Matematyczna strona teorii pola.
  Rozpatrzmy teorię pola od jej strony matematycznej. Możemy wtedy
  wyróżnić w niej ciąg warunkujących się poziomów struktur:
  \begin{itemize}

  \item Topologiczna.

  \item Różniczkowa.

  \item Pseudoriemannowska, zdana przez tensor metryczny $g_{ \mu \nu }$.

  \item Algebraiczna.

  \end{itemize}



  Zazwyczaj. Pracujemy na dwóch najwyższych poziomach, zaś teorie nie
  przejawiają oczywistych własności odbijających ich strukturę
  topologiczną i~różniczkową.

\end{frame}
% ##################





% ##################
\begin{frame}
  \frametitle{Czym jest ogólna współzmienniczość?}


  Ładunek topologiczny. Niemniej znane są przykłady wielkość które są
  zachowywane przez dowolne odwzorowania zachowujące topologię. Wielkość
  te są znane w~fizyce jako ładunki topologiczne.

  Wielkość współzmiennicze jest to wielkości którą można wyliczyć bez
  korzystania z metryki $g_{ \mu \nu }$ \cite{WittenQFTAndJonesPolynomial1989}.

\end{frame}
% ###################





% ##################
\begin{frame}
  \frametitle{Jak otrzymać teorię współzmienniczą?}


  Metoda tradycyjna wzorując się na ogólnej teorii względności napiszmy
  działanie jako:
  \begin{equation}
    \label{eq:Ogolna-wspolzmienniczosc-01}
    S_{ \mathrm{I} } =
    \int_{ M } d^{ D } x \, \sqrt{ g } ( R + \Lcal_{ M } ).
  \end{equation}
  Taka teoria zależy od metryki, jednak traktuje ją jako zmienną
  dynamiczną. Co należy zrobić aby teoria była w pełni od niej
  niezależna?

  Odpowiedź
  \begin{equation}
    \label{eq:Ogolna-wspolzmienniczosc-02}
    S =
    \int \Dcal g \int_{ M } d^{ D } x \, \sqrt{ g } ( R + \Lcal_{ M } ).
  \end{equation}

\end{frame}
% ##################





% ##################
\begin{frame}
  \frametitle{Podejście topologiczno-różniczkowe}


  Główny problem.
  Tensor metryczny definiuje nam, wraz z orientacją, niezmienniczy
  element całkowania, który jest potrzebny do dobrego zdefiniowania
  działania.

  Jednak w szczególnych przypadkach, może się okazać że wystarczy
  uboższa struktura matematyczna. Tak jest na przykład dla
  trójwymiarowej zorientowanej rozmaitości z polami typu Yanga-Millsa.

\end{frame}
% ##################





% ##################
\begin{frame}
  \frametitle{Teoria Cherna-Simonsa}


  Oznaczenia
  \begin{itemize}

  \item $A_{ i }$ -- pola cechowania.

  \item $[ A_{ i }, A_{ j } ]$ -- odpowiednik wyrażenia
    $\Gamma^{ i }_{ k l } v^{ k } w^{ l }$ z~ogólnej teorii
    względności.

  \end{itemize}


  Lagranżjan
  \begin{equation}
    \label{eq:Ogolna-wspolzmienniczosc-03}
    \begin{split}
      \Lcal
      &= \frac{ k }{ 4 \pi } \int_{ M } \Tr ( A \wedge dA
        + \frac{ 2 }{ 3 } A \wedge A \wedge A ) \\
      &= \frac{ k }{ 8 \pi } \int_{ M } \varepsilon^{ i j k } \Tr \big( A_{ i } ( \partial_{ j } A_{ k }
        - \partial_{ k } A_{ j } ) + \frac{ 2 }{ 3 } A_{ i } [ A_{ j }, A_{ k } ] \big).
      \end{split}
    \end{equation}

\end{frame}
% ##################










% ######################################
\appendix
% ######################################





% ######################################
\EndingSlide{Dziękuję! Pytania?}
% ######################################










% ##################
\begin{frame}


  \bibliographystyle{plalpha}

  \bibliography{PhilNaturArticles}{}

\end{frame}
% ##################










% ############################

% Koniec dokumentu
\end{document}
