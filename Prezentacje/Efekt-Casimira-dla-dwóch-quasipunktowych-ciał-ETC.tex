% ---------------------------------------------------------------------
% Basic configuration of Beamera and Jagiellonian
% ---------------------------------------------------------------------
\RequirePackage[l2tabu, orthodox]{nag}



\ifx\PresentationStyle\notset
\def\PresentationStyle{dark}
\fi



\documentclass[10pt,t]{beamer}
\mode<presentation>
\usetheme[style=\PresentationStyle,logoLang=Latin,logoColor=monochromaticJUwhite,JUlogotitle=yes]{jagiellonian}



% ---------------------------------------
% Configuration files of Jagiellonian loceted in catalog preambule
% ---------------------------------------
\input{./preambule/LanguageSettings/JagiellonianPolishLanguageSettings.tex}
\input{./preambule/TextposConfiguration/TextposConfiguration.tex}

\input{./preambule/JagiellonianCustomizationGeneral.tex}
\input{./preambule/JagiellonianCustomizationCommands.tex}










% ---------------------------------------
% Packages, libraries and their configuration
% ---------------------------------------
\usepackage{mathcommands}





% ---------------------------------------
% Configuration for this particular presentation
% ---------------------------------------










% ---------------------------------------------------------------------
\title{Efekt Casimira dla dwóch quasipunktów}
\subtitle{Pytania, problemy, wątpliwości}

\author{Kamil Ziemian \\
  \textit{kziemianfvt@gmail.com}}


%\institute{Uniwersytet Jagielloński w Krakowie}

\date[3 VI 2016]{3 czerwca 2016}
% ---------------------------------------------------------------------










% ####################################################################
% Początek dokumentu
\begin{document}
% ####################################################################





% Wyrównanie do lewej z łamaniem wyrazów

\RaggedRight





% ######################################
\maketitle % Tytuł całego tekstu
% ######################################





% ######################################
\begin{frame}
  \frametitle{Plan prezentacji}


  \tableofcontents % Spis treści

\end{frame}
% ######################################










% ######################################
\section{Czym jest efekt Casimira?}
% ######################################



% ##################
\begin{frame}
  \frametitle{Popularny obraz efektu Casimira}


  \begin{figure}

    \centering

    \includegraphics[scale=0.6]{./PresentationPictures/CasimirForce.jpg}


    \caption{Pobrany ze strony
        \colorhref{https://www.ameslab.gov/dmse/research-takes-frictionless-machines-one-step-closer}
      {https://www.ameslab.gov/dmse/research-takes-frictionless-machines-one-step-closer}}

  \end{figure}

\end{frame}
% ##################





% ##################
\begin{frame}
  \frametitle{Jak to obliczyć?}


  Kantując pole swobodne zwykle dostajemy
  \begin{equation}
    \label{eq:Efekt-Casimira-01}
    \widehat{H} =
    \frac{ 1 }{ 2 } \int dk \, dk' \,
    \sqrt{ \omega_{ k } \, \omega_{ k' } } \left( a_{ k' }^{ \dagger } a_{ k }
    + a_{ k' } a_{ k }^{ \dagger } \right).
  \end{equation}

  Aby sobie z~tym poradzić przekomutowujemy $a^{ \dagger }$ oraz~$a$.
  W~konsekwencji czego:
  \begin{equation}
    \label{eq:Efekt-Casimira-02}
    \langle 0 | \widehat{H} | 0 \rangle =
    \frac{ 1 }{ 2 } \int\limits_{ 0 }^{ +\infty } dk \, \omega_{ k }.
  \end{equation}

\end{frame}
% ##################





% ##################
\begin{frame}
  \frametitle{Pochodzenie efektu}


  Efekt Casimira kryje~się w~„nieskończonej stałej” mającej sens
  energii próżni:
  \begin{equation}
    \label{eq:Efekt-Casimira-03}
    E_{ 0 } =
    \frac{ 1 }{ 2 } \int\limits_{ 0 }^{ +\infty } dk \, \omega_{ k },
  \end{equation}
  która w~standardowym rachunku zaburzeń jest do zaniedbania. \\
  Np. Peskin i~Schroeder dyskutują jej możliwy sens fizyczny dopiero
  w~\textit{Epilogu}, czyli tak gdzieś po 800 stronie, swojej książki.

  Łatwiej chyba dotrzeć do Bieguna Południowego niż do tej strony
  (mnie~się to chyba nigdy nie uda).

\end{frame}
% ##################





% ##################
\begin{frame}
  \frametitle{Dwie nieskończone płyty}


  Makroskopowe płyty wprowadzają warunki brzegowe w~$x = 0$ i~$x = a$:
  \begin{subequations}
    \begin{align}
      \label{eq:Efekt-Casimira-04-A}
      E'_{ 0 }
      &= \frac{ 1 }{ 2 } \sum_{ n } \omega_{ n, a }, \\
      \label{eq:Efekt-Casimira-04-B}
      E_{ a }
      &=
        \frac{ 1 }{ 2 } \sum_{ n } \omega_{ n, a }
        - \frac{ 1 }{ 2 } \int dk \, \omega_{ k }.
    \end{align}
  \end{subequations}

  Siła Casimira
  \begin{equation}
    \label{eq:Efekt-Casimira-05}
    F_{ a } = -\frac{ d }{ da } E_{ a }
  \end{equation}

\end{frame}
% ##################





% ##################
\begin{frame}
  \frametitle{Siła Casimira}


  \begin{equation}
    \label{eq:Efekt-Casimira-06}
    F_{ a } = -\frac{ d }{ da } E_{ a }
  \end{equation}



  To wyrażenie można skrytykować z~kilku różnych powodów.

  Chętnie usłyszę Państwa opinię na temat tych zarzutów.
  \begin{itemize}
    \RaggedRight

  \item To postępowanie sugeruje fizyczny sens nieskończonej
    energii próżni, brak jednak interpretacji fizycznej.

  \item Zaniedbanie „nieskończonej stałej” dla pól swobodnych
    i~problemów rozpraszania staje~się bardziej dyskusyjne.

  \item „Renormalizowane” są~\textbf{liniowe} równania pola.

  \item W~rachunku zaburzeń znane są matematyczne podstawy
    rozbieżności w~UV i~IR. Bogoliubow, Epstein i~Glaser: mnożenie
    dystrybucji i~obcięcie adiabatyczne; tu brak takich podstaw.

  \end{itemize}

\end{frame}
% ##################





% ##################
\begin{frame}
  \frametitle{Dalszy ciąg krytyki}


  \begin{equation}
    \label{eq:Efekt-Casimira-07}
    F_{ a } = -\frac{ d }{ da } E_{ a }.
  \end{equation}



  \begin{itemize}
    \RaggedRight

  \item Zdarza~się, że~po skwantowaniu pojawiają~się niefizyczne
    efekty w~skutek „użycia nie najlepszych wyrażeń klasycznych”,
    co stawia całą procedurę w~złym świetle.

  \item Nie istnieje chyba operator $\widehat{ H }_{ a }$, by
    zachodziło
    \begin{equation}
      \label{eq:Efekt-Casimira-08}
      E_{ a } = \langle 0 | \widehat{ H }_{ a } - \widehat{ H }_{ 0 } | 0 \rangle,
    \end{equation}
    choć niektóre prace zdają~się to sugerować.

  \end{itemize}

\end{frame}
% ##################





% ##################
\begin{frame}
  \frametitle{Podejście Herdegena, podstawy fizyczne i~jego cele}


  \begin{enumerate}
    \RaggedRight

  \item Siła Casimira jest odpowiedzią na \textbf{adiabatyczną} zmianę
    położenia ciał makroskopowych.

  \item Pole swobodne i~z~warunkami Dirichleta, to dwa różne systemy
    których nie~można fizycznie porównać.

  \item Ponieważ obliczenia dla warunków Dirichleta dają wynik dobrze
    zgodny z~rzeczywistością, należy jakoś wyjaśnić ten fenomen.

  \item $E_{ a }$ powinna być wartością oczekiwaną konkretnego,
    niezmiennego operatora.

  \end{enumerate}

\end{frame}
% ##################





% ##################
\begin{frame}
  \frametitle{Podejście Herdegena, podstawy fizyczne}


  \begin{itemize}
    \RaggedRight

  \item Warunki brzegowe, np. typu Dirichleta, są niefizyczne,
    należy je odpowiednio „rozmyć”.

  \item Ostre warunki brzegowe powinny być odtworzone w~sensie
    granicy rezolwentowej.

  \item Energia Casmira, jest dana jako wartość oczekiwana operatora
    \textbf{swobodnego} pola, w~stanie podstawowym zależny od
    rozmieszczenia ciał makroskopowych:
    \begin{equation}
      \label{eq:Efekt-Casimira-09}
      E_{ a } = \langle \Omega_{ a } | H_{ 0 } | \Omega_{ a } \rangle.
    \end{equation}

  \end{itemize}



  Jeden z~rezultatów. Otrzymane opis efektu Casimir przewiduje
  pojawienie~się odpowiedniej siły również w~problemie o~skończonej
  liczbie stopni swobody. Ilość stopni swobody tutaj, to wymiar
  przestrzeni układu klasycznego.

\end{frame}
% ##################





% ##################
\begin{frame}
  \frametitle{Formalizm algebraicznej QFT}


  Formalizm algebraiczny. Układ kwantowy, o~skończonej bądź nie ilości
  stopni swobody, definiuje algebra $\Acal$, przestrzeń Hilberta $\Hcal$
  oraz reprezentacja:
  \begin{equation}
    \label{eq:Efekt-Casimira-10}
    \pi: \Acal \to \Hcal.
  \end{equation}
  Dwa układy możemy porównać wtedy, gdy ich algebra $\Acal$ jest
  identyczna, zaś reprezentacje są równoważne.



  A. Herdegen, \textit{Quantum backreaction (Casimir) effect~I.
    What are admissible idealizations?} arXiv:hep-th/0412132 (2004).
  Podana tam została złożona konstrukcja odpowiedniej algebry,
  przestrzeni Hilberta i~reprezentacji $\pi$ dla efektu Casimira,
  w~oparciu o~\textbf{rzeczywistą} przestrzeń Hilberta $\Kcal$ i~dodatni
  operator~$h$.

\end{frame}
% ##################





% ##################
\begin{frame}
  \frametitle{Jak to obliczyć?}


  Do konstrukcji teorii swobodnego pola skalarnego $( \Acal, \Hcal, \pi )$
  potrzebujemy operator:
  \begin{equation}
    \label{eq:Efekt-Casimira-11}
    h^{ 2 } = -\Delta, \quad
    h = \sqrt{ -\Delta }.
  \end{equation}

  Wprowadzamy obiekty makroskopowe
  \begin{equation}
    \label{eq:Efekt-Casimira-12}
    h_{ a }^{ 2 } = -\Delta + V_{ a }, \quad
    h_{ a } = \sqrt{ h_{ a }^{ 2 } }.
  \end{equation}
  Operator $V_{ a }$ jest konsekwencją wprowadzenia ciał
  makroskopowych do układu. To zamiana prowadzi do konstrukcji
  $( \Acal, \Hcal, \pi_{ a } )$. W~trakcie jej przeprowadzanie potrzebny
  jest m.~in. operator $\sqrt{ h_{ a } }$.

  Uwaga. Sytuacja staje~się subtelniejsza, bo musimy zagwarantować,
  że~$h_{ a }^{ 2 }$ nie ma stanów związanych o~ujemnej energii.
  Stany o~energii 0, nie stanowią problemu.

\end{frame}
% ##################





% ##################
\begin{frame}
  \frametitle{????}


  Warunek równoważności reprezentacji. Wprowadzenie ciał makroskopowych,
  może wygenerować tylko skończoną liczbę cząstek swobodnych. Nie całkiem
  banalna analiza prowadzi do wzoru:
  \begin{equation}
    \label{eq:Efekt-Casimira-13}
    \Ncal_{ a } =
    \frac{ 1 }{ 4 } \Tr\left[ h^{ -1/2 } ( h_{ a } - h ) h_{ a }^{ -1 }
      ( h_{ a } - h ) h^{ -1/2 } \right]
    < +\infty.
  \end{equation}



  Energia Casimira. Korzystając z~założenia o~\textbf{adiabatyczności}
  ruchu ciał makroskopowych, po~nie całkiem elementarnej analizie
  otrzymujemy:
  \begin{equation}
    \label{eq:Efekt-Casimira-14}
    \Ecal_{ a } =
    \langle \Omega_{ a } | h^{ 2 } | \Omega_{ a } \rangle =
    \frac{ 1 }{ 4 } \Tr\left[ ( h_{ a } - h ) h_{ a }^{ -1 }
      ( h_{ a } - h ) \right].
  \end{equation}
  Skończona wartość tego wyrażenia, jest dodatkowym warunkiem na
  fizyczną porównywalność obu układów.

\end{frame}
% ##################





% ##################
\begin{frame}
  \frametitle{Problem dwóch punktów}


  Pomysł zajęcia~się tym problemem wyszedł od dr. Bogdana Damskiego.

  Rozwiązanie z~literatury. A. Scardicchio, \textit{Casimir dynamic:
    Interactions~of surface with codimension $> 1$ due to quantum
    fluctuations}, (2005), na pracę wskazał dr Damski.

  Korzystając ze~wzoru:
  \begin{equation}
    \label{eq:Efekt-Casimira-15}
    E_{ a } =
    \frac{ 1 }{ 2 } \int\limits_{ 0 }^{ \Lambda } dE \, \sqrt{E} \rho( E ),
  \end{equation}
  odejmując od niego odpowiednie człony
  i~$\Lambda \to +\infty$ otrzymał on wzór:
  \begin{equation}
    \label{eq:Efekt-Casimira-16}
    \Fcal_{ a } =
    -\frac{ 5 }{ 4 } \frac{ L_{ c }^{ 4 } }{ \pi a^{ 6 } }.
  \end{equation}
  Jeżeli dwa punkty są w~odległości $L_{ c }$ wówczas pojawia~się
  stan związany o~$E < 0$ i~„próżnia staje~się niestabilna”.

\end{frame}
% ##################





% ##################
\begin{frame}
  \frametitle{Problem dwóch płyt}


  Wyniki. W~ramach tego podejścia problem został obliczony, dla dwóch
  różnych sposobów modyfikowania operatora $h^{ 2 } = -\Delta$.
  \begin{itemize}
    \RaggedRight

  \item[1.)] Poprzez modyfikację zachowanie $h^{ 2 } = -\Delta$
    dla mały $k$ w~przestrzeni pędów. A. Herdegen (2005).

  \item[2.)] Poprzez dodanie odpowiedniego pseudopotencjału:
    \begin{equation}
      \label{eq:Efekt-Casimira-17}
      h_{ a }^{ 2 } = -\Delta + V_{ a }.
    \end{equation}
    A. Herdegen, M. Stop, \textit{Global vs local Casimir effect}, \\
    arXiv:1007.2139, (2010).

  \end{itemize}

\end{frame}
% ##################





% ##################
\begin{frame}
  \frametitle{Problem dwóch płyt}


  Wynik z~pracy Herdegena i~Stopy.
  \begin{equation}
    \label{eq:Efekt-Casimira-18}
    \Fcal_{ a } =
    \frac{ 2 c }{ \lambda } - \frac{ \pi^{ 2 } }{ 240 a^{ 4 } }
    + ( \textrm{wyrazy} \xrightarrow[ \lambda \to 0 ]{} 0 ).
  \end{equation}
  Symbole $c$ i~$\lambda$, to parametry zależne od kształtu „rozmytych” warunków
  brzegowych.

\end{frame}
% ##################





% ##################
\begin{frame}
  \frametitle{Skalowanie modelu}


  Ta procedura powinna odtwarzać siłę Casimira między dwoma punktami,
  czyli gdy $V_{ a }$ jest quasipotencjałem typu delty Diraca. Wprowadźmy
  parametr $\lambda$, tak by zachodziło:
  \begin{equation}
    \label{eq:Efekt-Casimira-19}
    h_{ a,\, \lambda }^{ 2 } \xrightarrow[ \lambda \to 0 ]{}
    h_{ a,\, \textrm{DP} }^{ 2 }.
  \end{equation}
  Ponieważ $h_{ a,\, \lambda }^{ 2 }$, $h_{ a,\, \textrm{DP} }^{ 2 }$ są
  nieograniczone, odpowiedź na pytanie o~to co znaczy taka granica
  nie jest oczywiste.

  Granica rezolwentowa. Dla $w^{ 2 }$ spoza widma $h_{ a,\, \lambda }^{ 2 }$
  i~$h_{ a,\, \textrm{DP} }^{ 2 }$ ich rezolwenty są operatorami
  ograniczonymi, można więc postawić pytanie czy punktowo zachodzi:
  \begin{equation}
    \label{eq:Efekt-Casimira-20}
    ( w^{ 2 } - h_{ a,\, \lambda }^{ 2 } )^{ -1 }
    \xrightarrow[ \lambda \to 0 ]{}
    ( w^{ 2 } - h_{ a,\, \textrm{DP} }^{ 2 } )^{ -1 }.
  \end{equation}

\end{frame}
% ##################





% ##################
\begin{frame}
  \frametitle{Model dla dwóch punktów, oznaczenia}


  \begin{itemize}
    \RaggedRight

  \item $g( \vecx )$ funkcja zespolona, gładka o~nośniku
    zawartym w~kuli o~promieniu $d$ i~środku w~$0 \in \Rbb^{ 3 }$;

  \item $g_{ \vec{ v } }( x ) = g( \vec{ x } - \vec{ v } )$;

  \item $g_{ \lambda }( \vecx ) = \lambda^{ -\gamma } g\left( \frac{ \vecx }{ \lambda } \right)$,
    $\gamma > 0$;

  \item $\veca$ -- odległość między punktami materialnymi,
    $a = \absOne{ \veca }$;

  \item $\vecb = \frac{ 1 }{ 2 } \veca$, $b = \absOne{ \vecb }$.

  \end{itemize}



  Pseudopotencjał
  \begin{subequations}
    \begin{align}
      \label{eq:Efekt-Casimira-21-A}
      | g \rangle \langle g | \psi \rangle
      &= g( x ) \int\limits_{ \Rbb^{ 3 } } d^{ 3 } x \; \overline{ g( x ) } \;
        \psi( x ) \\
      \label{eq:Efekt-Casimira-21-B}
      V_{ a,\, \lambda }
      &=
        \sigma( \lambda ) \left( | g_{ \vecb,\, \lambda } \rangle \langle g_{ \vecb,\, \lambda } |
        + | g_{ -\vecb,\, \lambda } \rangle \langle g_{ -\vecb,\, \lambda } | \right)
    \end{align}
  \end{subequations}

\end{frame}
% ##################





% ##################
\begin{frame}
  \frametitle{Model dla dwóch punktów}


  Podstawowe obiekty modelu
  \begin{subequations}
    \begin{align}
      \label{eq:Efekt-Casimira-22-A}
      h^{ 2 } &= -\Delta, \\
      \label{eq:Efekt-Casimira-22-B}
      g_{ \lambda }( \vecx )
              &= \lambda^{ -\gamma } g\left( \frac{ \vecx }{ \lambda } \right), \quad
                \gamma > 0, \\
      \label{eq:Efekt-Casimira-22-C}
      V_{ a, \lambda } =
      \sigma( \lambda )
              &\left( | g_{ \vecb, \lambda } \rangle
                \langle g_{ \vecb, \lambda } | + | g_{ -\vecb, \lambda } \rangle
                \langle g_{ -\vecb, \lambda } | \right), \\
      \label{eq:Efekt-Casimira-22-D}
      h_{ a }^{ 2 } &= h^{ 2 } + V_{ a }.
    \end{align}
  \end{subequations}

  Komentarz. $V_{ a }$ jest nielokalnym pseudopotencjałem. To postępowanie
  zostało zaproponowane w~pracy A.~Herdegena i~M.~Stopy, \textit{Global
    vs~local Casimir effect}, (2010) arXiv: 1007.2139v1, gdyż
  wprowadzenie potencjału lokalnego prowadziło do~rozbieżnych
  wyrażeń na ślady odpowiednich operatorów.

\end{frame}
% ##################





% ##################
\begin{frame}
  \frametitle{Wreszcie coś liczymy}


  Wynik. Niech dane będą dane stany własne $h$ i~$h_{ a }$:
  \begin{equation}
    \label{eq:Efekt-Casimira-23}
    h | p \rangle = p | p \rangle, \quad
    h_{ a } | k+ \rangle = k | k+ \rangle.
  \end{equation}
    Za pomocą tych stanów możemy obliczyć energię Casimira
    w~następujący sposób:
    \begin{equation}
      \label{eq:Efekt-Casimira-24}
      \Ecal_{ a } =
      \frac{ 1 }{ 4 } \Tr\left[ ( h_{ a } - h ) h_{ a }^{ -1 }
        ( h_{ a } - h ) \right]
      =
      \frac{ 1 }{ 4 } \int\limits d^{ 3 } k \, d^{ 3 } p \,
      \frac{ ( \absOne{ k } - \absOne{ p } )^{ 2 } }{ \absOne{ k } }
      \absOne{ \langle p | k+ \rangle }^{ 2 }.
    \end{equation}

    Teraz potrzebujemy znaleźć $\langle p | k+ \rangle$. W~dalszym
    ciągu jest $k$ zamiast $\absOne{ k }$.

    Z~pomocą przychodzi teoria rozpraszania
    \begin{equation}
      \label{eq:Efekt-Casimira-25}
      \langle p | k + \rangle =
      \delta(p - k)
      + \frac{ \langle p | T( k^{ 2 } + i0 ) | k \rangle}{ k^{ 2 } - p^{ 2 } + i0 }.
    \end{equation}

\end{frame}
% ##################





% ##################
\begin{frame}
  \frametitle{Wreszcie coś liczymy}


  Oznaczenia
  \begin{equation}
    \label{eq:Efekt-Casimira-26}
    G_{ 0 }( w^{ 2 } ) = ( w^{ 2 } - h^{ 2 } )^{ -1 }, \quad
    G_{ \lambda }( w^{ 2 } ) = ( w^{ 2 } - h_{ \lambda }^{ 2 } ).
  \end{equation}
  Tutaj pomijam indeks $a$.


  Wzory z~teorii rozpraszania
  \begin{align}
    \label{eq:Efekt-Casimira-27-A}
    G( w^{ 2 } )
    &=
      G_{ 0 }( w^{ 2 } ) + G_{ 0 }( w^{ 2 } ) T( w^{ 2 } ) G_{ 0 }( w^{ 2 } ), \\
    \label{eq:Efekt-Casimira-27-B}
    T( w^{2 } ) &= V + V G_{ 0 }( w^{ 2 } ) T( w^{ 2 } ).
  \end{align}
  $G_{ 0 }( w^{ 2 } )$ to rezolwenta laplasjanu, więc znamy ten operator
  bardzo dobrze.

  Postulujemy
  \begin{equation}
    \label{eq:Efekt-Casimira-28}
    T_{ \lambda }( w^{ 2 } ) =
    \left( | g_{ \vecb,\, \lambda } \rangle | g_{ -\vecb,\, \lambda } \rangle \right)
    \Tcal_{ \lambda }( w^{ 2 } )
    \begin{pmatrix}
      \langle g_{ \vecb, \lambda } | \\
      \langle g_{ -\vecb,\, \lambda } |
    \end{pmatrix}
  \end{equation}

\end{frame}
% ##################





% ##################
\begin{frame}
  \frametitle{Wreszcie coś liczymy}


  Postulujemy
  \begin{equation}
    \label{eq:Efekt-Casimira-29}
    T_{ \lambda }( w^{ 2 } ) =
    \left( | g_{ \vecb,\, \lambda } \rangle | g_{ -\vecb,\, \lambda } \rangle \right)
    \Tcal_{ \lambda }( w^{ 2 } )
    \begin{pmatrix}
      \langle g_{ \vecb, \lambda } | \\
      \langle g_{ -\vecb, \lambda } |
    \end{pmatrix}.
  \end{equation}

  Rozwiązanie
  \begin{subequations}
    \begin{align}
      \label{eq:Efekt-Casimira-30-A}
      &\Tcal_{ \lambda }( w^{ 2 } ) =
      \begin{bmatrix}
        \sigma^{ -1 }( \lambda )
        - ( g_{ \lambda }, G_{ 0 }( w^{ 2 } ) g_{ \lambda } )
        & -( U_{ \veca } g_{ \lambda }, G_{ 0 }( w^{ 2 } ) g_{ \lambda } ) \\
        -( U_{ \veca } g_{ \lambda }, G_{ 0 }( w^{ 2 } ) g_{ \lambda } )
        & \sigma^{ -1 }( \lambda )
          - ( g_{ \lambda }, G_{ 0 }( w^{ 2 } ) g_{ \lambda } )
      \end{bmatrix}^{ -1 }, \\
      \label{eq:Efekt-Casimira-30-B}
      &g_{ \lambda }( \vecx ) =
      \lambda^{ -\gamma } g\left( \tfrac{ \vecx }{ \lambda } \right).
    \end{align}
  \end{subequations}

  Możemy teraz obliczyć
  \begin{equation}
    \label{eq:Efekt-Casimira-31}
    \langle \vecp | G_{ \lambda }( w^{ 2 } ) - G_{ 0 }( w^{ 2 } ) | \vecq \rangle.
  \end{equation}

\end{frame}
% ##################





% ##################
\begin{frame}
  \frametitle{Funkcja $\sigma( \lambda )$}


  Wzory
  \begin{subequations}
    \begin{align}
      \label{eq:Efekt-Casimira-32-A}
      \widehat{g}( \vecp ) = \widehat{g}( | \vecp | ) = \widehat{g}( p )
      &=
        \frac{ 1 }{ \sqrt{ 2 \pi }^{ 3 } }
        \int\limits_{ \Rbb^{ 3 } } d^{ 3 } x \, e^{ -i \vecp \cdot \vecx } g( \vecx ), \\
      \label{eq:Efekt-Casimira-32-B}
      M_{ p } &= \absOne{ \widehat{g}( p ) }^{ 2 }, \\
      \label{eq:Efekt-Casimira-32-C}
      \beta &= 2( 3 - \gamma ).
    \end{align}
  \end{subequations}


  Zagadnienie funkcji $\sigma( \lambda )$. Chcieliśmy granicę rezolwentową, która
  jest zupełnie niezależne od funkcji $g( \vecx )$. Udało mi~się znaleźć
  tylko jedną \textbf{nietrywialną} granicę która spełnia ten warunek.
  Aby~ją otrzymać należy dodatkowo przyjąć:
  \begin{equation}
    \label{eq:Efekt-Casimira-33}
    \sigma^{ -1 }( \lambda ) =
    \alpha M_{ 0 } \lambda^{ \beta } - \lambda^{ \beta - 1 } \int\limits_{ -\infty }^{ +\infty } M_{ q } \, dq.
  \end{equation}

\end{frame}
% ##################





% ##################
\begin{frame}
  \frametitle{Granica rezolwentowa}


  \begin{equation}
    \label{eq:Efekt-Casimira-34}
    \begin{split}
      &\lim_{ \lambda \searrow 0 }\langle \vecp | G_{ \lambda }( w^{ 2 } ) - G_{ 0 }( w^{ 2 } )
        | \vecq \rangle
        = \\
      &= \frac{ 1 }{ ( -\tfrac{ \alpha } { 2 \pi^{ 2 } } + iw )^{ 2 }
        - \left( \tfrac{ e^{ i aw } }{ a } \right)^{ 2 } }
        \frac{ 1 }{ w^{ 2 } - \vecp^{ 2 } }
        \frac{ 1 }{ w^{ 2 } - \vecq^{ 2 } } \times \\
      &\hphantom{=}
        \times \Big[ ( -\tfrac{ \alpha }{ 2 \pi^{ 2 } } + i w ) e^{ -i \vecb \cdot \vecp }
        e^{ i \vecb \cdot \vecq }
        - \frac{ e^{ i aw } }{ a } e^{ i \vecb \cdot \vecp }
        e^{ i \vecb \cdot \vecq } -\frac{ e^{ i aw } }{ a } e^{ -i \vecb \cdot \vecp }
        e^{ -i \vecb \cdot \vecq } \; + \\
      &\hphantom{= \times}
        + \, ( -\tfrac{ \alpha }{ 2 \pi^{ 2 } } + i w ) e^{ i \vecb \cdot \vecp }
        e^{ -i \vecb \cdot \vecq } \Big].
    \end{split}
  \end{equation}

  Sens granicy. Jest to rezolwenta potencjału dwóch delt Diraca
  znajdujących~się w~punktach $\vec{ b }$ i~$-\vec{ b }$, $\alpha$ mierzy siłę
  oddziaływania tego potencjału.

\end{frame}
% ##################





% ##################
\begin{frame}
  \frametitle{Granica rezolwentowa}


  \begin{equation}
    \label{eq:Efekt-Casimira-35}
    \begin{split}
      &\lim_{ \lambda \searrow 0 }\langle \vecp | G_{ \lambda }( w^{ 2 } ) - G_{ 0 }( w^{ 2 } )
        | \vecq \rangle
        = \\
      &= \frac{ 1 }{ ( -\tfrac{ \alpha } { 2 \pi^{ 2 } } + iw )^{ 2 }
        - \left( \tfrac{ e^{ i aw } }{ a } \right)^{ 2 } }
        \frac{ 1 }{ w^{ 2 } - \vecp^{ 2 } }
        \frac{ 1 }{ w^{ 2 } - \vecq^{ 2 } } \times \\
      &\hphantom{=}
        \times \Big[ ( -\tfrac{ \alpha }{ 2 \pi^{ 2 } } + i w ) e^{ -i \vecb \cdot \vecp }
        e^{ i \vecb \cdot \vecq }
        - \frac{ e^{ i aw } }{ a } e^{ i \vecb \cdot \vecp }
        e^{ i \vecb \cdot \vecq } -\frac{ e^{ i aw } }{ a } e^{ -i \vecb \cdot \vecp }
        e^{ -i \vecb \cdot \vecq } \, + \\
      &\hphantom{=}
        + \, ( -\tfrac{ \alpha }{ 2 \pi^{ 2 } } + i w ) e^{ i \vecb \cdot \vecp }
        e^{ -i \vecb \cdot \vecq } \Big].
    \end{split}
  \end{equation}

  Uwaga matematyczna. Zdefiniowanie w~sposób poprawny matematycznie
  potencjałów będących sumą $\delta( x )$ wymaga zastanowienia. Do tej pory
  ostatecznym słowem nauki jakie znam zostały przedstawione przez grupę
  Sergia Albeverio. Dr.~Damski zwrócił nam uwagę na książkę S. Albeverio

\end{frame}
% ##################





% ##################
\begin{frame}
  \frametitle{Zagadnienie stanów związanych}


  Problem dla formalizmu. Jeżeli zachodzi związek
  \begin{equation}
    \label{eq:Efekt-Casimira-36}
    \alpha a \leq 2 \pi^{ 2 },
  \end{equation}
  to dla potencjału dwóch $\delta( x )$ pojawia~się stan związany dla
  ujemnej energii. Ten fenomen zidentyfikowali już Albeverio et. al.,
  w~kontekście problemu Casimira dyskutował go Scardicchio
  („destabilizacja próżni przez tachiony”).

  Przypuszczenie. Jeżeli $\alpha a \leq 2 \pi^{ 2 }$, to istnieje ryzyko, że~dla
  $\lambda \in ( 0, \eta )$, również pojawia~się stan związany do ujemnej energii.
  Wtedy granica $\lambda \searrow 0$ nie ma sensu. W~tej chwili nie wiemy, czy tak
  rzeczywiście jest.

\end{frame}
% ##################





% ##################
\begin{frame}
  \frametitle{Energia Casimira}


  Stan obecny. Korzystając z~wzoru na operator $T_{ \lambda }( w^{ 2 } )$, chcemy
  obliczyć graniczną postać $E_{ a, \lambda }$. Ostatnie wyrażenie jakie do tej
  pory uzyskaliśmy ma postać:
  \begin{equation}
    \label{eq:Efekt-Casimira-37}
    E_{ a, \lambda } = 4 \lambda^{ -1 } f( \lambda, a ).
  \end{equation}

  Wynik Herdegena, Stopy
  \begin{equation}
    \label{eq:Efekt-Casimira-38}
    \Fcal_{ a, \lambda } =
    \frac{ 2 c }{ \lambda } - \frac{ \pi^{ 2 } }{ 240 a^{ 4 } }
    + ( \textrm{wyrazy} \xrightarrow[ \lambda \searrow 0 ]{} 0 ).
  \end{equation}
  $c$ może być różne od 0, w~ogólności zależy od funkcji
  $g( \vecx )$.

\end{frame}
% ##################





% ##################
\begin{frame}
  \frametitle{Obecny stan pracy}


  Energia Casimira. Korzystając z~wzoru na operator $T_{ \lambda }( w^{ 2 } )$,
  chcemy obliczyć graniczną postać $E_{ a, \lambda }$. Ostatnie wyrażenie jakie
  do tej pory uzyskaliśmy ma postać:
  \begin{equation}
    \label{eq:Efekt-Casimira-39}
    E_{ a, \lambda } = 4 \lambda^{ -1 } f( \lambda, a ).
  \end{equation}


  Postać $f( \lambda, a )$
  \begin{equation}
    \label{eq:Efekt-Casimira-40}
    \begin{split}
      f( \lambda, a )
      &=
        f( \lambda )
        - \frac{ 1 }{ M_{ 0 } }
        \left( \int\limits_{ 0 }^{ +\infty } dp M_{ p } \right)
        \left( \frac{ 2 \pi^{ 2 } }{ a } \right)^{ 2 n } \\
      &\times \Imag \sum\limits_{ n = 1 }^{ \infty } \int\limits_{ 0 }^{ \infty }
        \frac{ 1 }{ ( -\alpha + i 2 \pi^{ 2 } \rho )^{ 2n + 1 } }
        e^{ i 2 na \rho } d\rho.
      \end{split}
    \end{equation}

\end{frame}
% ##################





% ##################
\begin{frame}
  \frametitle{Obecny stan pracy}


  \begin{subequations}
    \begin{equation}
      \label{eq:Efekt-Casimira-41-A}
      \widehat{g}( p ) =
      \frac{ 1 }{ \sqrt{ 2 \pi }^{ 3 } } \int\limits_{ \Rbb^{ 3 } } d^{ 3 } x \,
      e^{ -i \vecp \cdot \vecx } g( \vecx ), \quad
      M_{ p } = \absOne{ \widehat{g}( p ) }^{ 2 }.
    \end{equation}
    \begin{equation}
      \begin{split}
        f( \lambda, a )
        &= f( \lambda )
          - \frac{ 1 }{ M_{ 0 } }
          \left( \int\limits_{ 0 }^{ +\infty } dp M_{ p } \right)
          \left( \frac{ 2 \pi^{ 2 } } { a } \right)^{ 2 n } \\
        &\hphantom{=}
          \times \Imag \sum\limits_{ n = 1 }^{ \infty }
          \int\limits_{ 0 }^{ \infty }
          \frac{ 1 }{ ( -\alpha + i 2 \pi^{ 2 } \rho )^{ 2n + 1 } } e^{ i 2 na \rho } d\rho.
      \end{split}
    \end{equation}
  \end{subequations}



  Na razie brak satysfakcjonującej postaci tego wyrażenia. Wstępne
  badania z~pomocą Mathematici sugerują, że~powyższe całki stają~się
  rozbieżne, gdy $\alpha a \leq 2 \pi^{ 2 }$

\end{frame}
% ##################





% ##################
\begin{frame}
  \frametitle{Pytania, problemy, wątpliwości}


  \begin{itemize}
    \RaggedRight

  \item Jak przedstawiona tu teoria ma~się do tego co istnieje
    w~przyrodzie?

  \item Czy dla $\lambda \in ( 0, \eta )$ również istnieją stany związane?

  \item Jaka jest fizyka tych układów, gdy pojawia~się stan
    związany o~ujemnej energii?

  \item W~otrzymanych wzorach nie widać jakiegoś uniwersalnego
    zachowania. Czy to oznacza, że~podejście jest niepoprawne,
    wyprowadzenie otrzymanych wyników zawiera błąd, czy też
    takie uniwersalne zachowanie w~naturze nie
    zachodzi?

  \item Dlaczego energia Casimira tak silnie zależy od kształtu
    pseudopotencjału?

  \end{itemize}

\end{frame}
% ##################










% ######################################
\appendix
% ######################################





% ######################################
\EndingSlide{Dziękuję! Pytania?}
% ######################################










% ##################
\begin{frame}
  \frametitle{Bibliografia}


  \begin{itemize}
    \RaggedRight

  \item Andrzej Herdegen, \textit{Quantum backreaction (Casimir) effect~I.
      What are admissible idealization}, arXiv:~hep-th/0412132 (2004).

  \item Andrzej Herdegen, \textit{Quantum backreaction (Casimir) effect~II.
      Scalar and~electric fields}, arXiv:~hep-th/0507023v1 (2005).

  \item A~Herdegen, M.~Stopa, \textit{Global vs~local Casimir effect},
    arXiv:~1007.2139v1 (2010).

  \item A. Scardicchio, \textit{Casimir dynamic: Interactions~of
      surface with codi >1 due to quantum fluctuations}, Physical
    Reviev D 5004 (2005).

  \item John Taylor, \textit{Scatering theory}.

  \end{itemize}

\end{frame}
% ##################





% ############################

% Koniec dokumentu
\end{document}