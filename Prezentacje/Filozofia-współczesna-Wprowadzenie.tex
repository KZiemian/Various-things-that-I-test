% ---------------------------------------------------------------------
% Basic configuration of Beamera and Jagiellonian
% ---------------------------------------------------------------------
\RequirePackage[l2tabu, orthodox]{nag}



\ifx\PresentationStyle\notset
\def\PresentationStyle{dark}
\fi



\documentclass[10pt,t]{beamer}
\mode<presentation>
\usetheme[style=\PresentationStyle,logoLang=Latin,logoColor=monochromaticJUwhite,JUlogotitle=yes]{jagiellonian}



% ---------------------------------------
% Configuration files of Jagiellonian loceted in catalog preambule
% ---------------------------------------
\input{./preambule/LanguageSettings/JagiellonianPolishLanguageSettings.tex}
\input{./preambule/TextposConfiguration/TextposConfiguration.tex}

\input{./preambule/JagiellonianCustomizationGeneral.tex}
\input{./preambule/JagiellonianCustomizationCommands.tex}










% ---------------------------------------
% Packages, libraries and their configuration
% ---------------------------------------





% ---------------------------------------
% Configuration for this particular presentation
% ---------------------------------------










% ---------------------------------------------------------------------
\title{Filozofia współczesna. Wprowadzenie}

\author{Kamil Ziemian \\
  \texttt{kziemianfvt@gmail.com}}

% \date{}
% ---------------------------------------------------------------------










% ####################################################################
% Początek dokumentu
\begin{document}
% ####################################################################





% Wyrównanie do lewej z łamaniem wyrazów

\RaggedRight





% ######################################
\maketitle
% ######################################





% ######################################
\begin{frame}
  \frametitle{Spis treści}


  \tableofcontents % Spis treści

\end{frame}
% ######################################










% ######################################
\section{Wprowadzenie}
% ######################################



% ##################
\begin{frame}
  \frametitle{Kilka uwag i~ostrzeżeń}


  Chciałbym bardzo pobieżnie poopowiadać trochę o filozofii współczesnej,
  by zainteresowani mieli jakiś punkt startowy dla jej dalszego zgłębiania.

  Co nie znaczy, że jest to najlepszy punkt startowy.
  \begin{itemize}
    \RaggedRight

  \item W~wystąpieniu jest wiele luk, zasób mojej wiedzy nie jest
    najwyższej jakości. Jest wiele białych plan i~błędów, których wciąż
    nie jestem świadomy.

  \item Filozofia po 1990 roku, jest w ilościach śladowych, jeśli w~ogóle.

  \item Proszę również o~wybaczenie i wskazywanie literówek, niepoprawnej
    składni, błędów w~układzie materiału etc.

  \end{itemize}

  Siłą rzeczy, wszystkie uwagi są mile widziane. Warto uczyć się na własnych
  błędach. :)

\end{frame}
% ##################










% ######################################
\section{Zacznijmy prawie od początku}
% ######################################



% ##################
\begin{frame}
  \frametitle{Zacznijmy prawie od początku}


  \begin{figure}

    \centering

    \includegraphics[height=4cm]
    {./PresentationPictures/Plato\_and\_Aristotle.jpg}


    \uncover<2->{\caption{Platon (427--347 BC), Arystoteles (384--322)}}

  \end{figure}



  Złota zasada filozofii Europejskiej/Zachodniej. O czymkolwiek mówi
  się w filozofii, zawsze można zacząć od tych dwóch panów i dojść
  logicznie do wybranego tematu.

\end{frame}
% ##################





% ##################
\begin{frame}
  \frametitle{Aż tyle czasu nie mamy}


  \begin{figure}

    \centering

    \includegraphics[height=4cm]
    {./PresentationPictures/Immanuel_Kant_01.jpg}


    \caption{Immanuel Kant (1724--1804)}

  \end{figure}

  W mojej opinii stworzył on znaczną część fundamentów na których stoi
  filozofia ostatnich XIX, XX, a nawet XXI wieku.

\end{frame}
% ##################





% ##################
\begin{frame}
  \frametitle{Podstawowe informacje o~Kancie}



  \begin{itemize}
    \RaggedRight

  \item Urodził się, żył i umarł w mieście K\"{o}nisberg (Królewiec)
    w~Prusach.

  \item Profesor na tamtejszym uniwersytecie.

  \item Tym samy jeden z dwóch wybitnych filozofów którzy pomiędzy XV,
    a~XX wiekiem parali się tym zajęciem.

  \item Nigdy się nie ożenił.

  \item Legendarny pedant.

  \item Prowincjusz pierwszej kategorii.

  \end{itemize}

  Trzy główne dzieła Kanta.
  \begin{itemize}
    \RaggedRight

  \item \textit{Krytyka czystego rozumu}, 1781;

  \item \textit{Krytyka praktycznego rozumu}, 1788;

  \item \textit{Krytyka władzy sądzenia}, 1791.

  \end{itemize}
  Kant napisał wiele innych dzieł, te trzy powszechnie uchodzą za
  centralne w~jego dorobku.

\end{frame}
% ##################





% ##################
\begin{frame}
  \frametitle{Podstawowe informacje o~Kancie}


  \begin{itemize}
    \RaggedRight

  \item \textit{Krytyka czystego rozumu}, 1781;

  \item \textit{Krytyka praktycznego rozumu}, 1788;

  \item \textit{Krytyka władzy sądzenia}, 1791.

  \end{itemize}

  Można powiedzieć, że \textit{Krytyka czystego rozumu} jest wykładem
  metafizyki, \textit{Krytyka praktycznego rozumu} etyki, zaś
  \textit{Krytyka władzy sądzenia} -- estetyki. Wszystkie te nurty
  myśli Kanta są ważne, jednak tylko pierwszą będę w~stanie naprawdę
  poruszyć.

  „Są oni [Luter, Kartezjusz i~Rousseau] ojcami tego co Gabriel
  S\'{e}ailles nazwał współczesnym sumieniem. Pomijam Immanuela Kanta,
  który stoi na skrzyżowaniu prądów duchowych, płynących od tych
  trzech ludzi. Filozof z~Królewca stworzył niejako scholastyczną
  armaturę współczesnej myśli.”. Jacques Maritain \textit{Trzej
    reformatorzy} \cite{}.

\end{frame}
% ##################





% ##################
\begin{frame}
  \frametitle{Główne źródła myśli Kanta}


  \begin{itemize}
    \RaggedRight

  \item Pietystyczne wychowanie.

  \item Akademicka filozofia niemiecka.

  \item Polemika z~Humem.

  \item Potrzeba pogodzenia nauki i wiary. Chodzi tu raczej o fizyke
    niż biologię, to jednak temat na osobny wykład.

  \item Filozofia oświecenia. Warty jest podkreślenia wpływ
    Rousseau.

  \item Prawdopodobnie polemika z scholastyczną filozofią.

  \item Pietystyczne wychowanie.

  \item Akademicka filozofia niemiecka.

  \item Fizyka Newtona.

  \item Emanuel Swedenborg?

  \end{itemize}

\end{frame}
% ##################





% ##################
\begin{frame}
  \frametitle{Crashcourse filozofii Kanta}


  \begin{itemize}
    \RaggedRight

  \item Po pierwsze i~najważniejsze: Kant był idealistą! Ale co to
    znaczy? I dlaczego nim została?

  \item Kantowska klasyfikacja sądów: \\
    -- analityczne, syntetyczne; \\
    -- a priori (łac. z tego, co wcześniej), a posteriori (łac.
    z~tego, co potem).

  \item Czas i przestrzeń.

  \item Ding an sich. Wielki niszczyciel.

  \item Koncepcja moralna: uniwersalizm.

  \item Dwa problemy: słuszność fizyki Newtona i idee moralne
    Rousseau.

  \item Metoda transcendentalna.

  \end{itemize}

\end{frame}
% ##################










% ######################################
\section{Kantyzmu ciąg dalszy, czyli niemiecki idealizm}
% ######################################



% ##################
\begin{frame}
  \frametitle{Niemiecki idealizm}


  Idealizm Kanta dał, między innymi, początek ruchowi jaki nosi
  nazwę Idealizm Niemiecki. Oprócz samej myśli Kanta oddziaływały tu
  niewątpliwie jego źródła, takie jak pietyzm.

  Warto się zastanowić nad następującą sprawą. Kant był pierwszym
  wielkim filozofem piszącym po niemiecku (bardzo nieczytelnym
  niemieckim), w~czasach gdy zaczeła się rodzić współczesna
  koncepcja niemieckości.

\end{frame}
% ##################





% ##################
\begin{frame}
  \frametitle{Niemiecki idealizm}


  \begin{figure}

    \centering

    \includegraphics[height=4cm]
    {./PresentationPictures/Johann_Gottlieb_Fichte.jpg}
    \includegraphics[height=4cm]
    {./PresentationPictures/Friedrich_Schelling.jpg}


    \caption{Johann Gottlieb Fichte (1762--1814), Friedrich Schelling
    (1775--1854)}

  \end{figure}


  Można nazwać ich pierwszą falą idealizmu niemieckiego.

\end{frame}
% ##################





% ##################
\begin{frame}
  \frametitle{Kulminacja}


  \begin{figure}

    \centering

    \includegraphics[height=4cm]
    {./PresentationPictures/Georg_Wilhelm_Friedrich_Hegel.jpg}


    \caption{Georg Wilhelm Friedrich Hegel (1770--1831)}

  \end{figure}


  Jak wielu twierdzi, filozof bardzo wpływowy nawet dzisiaj.

\end{frame}
% ##################





% ##################
\begin{frame}
  \frametitle{\textit{Fenomenologia Ducha}}


  Wydane w~1807 roku książka Hegla \textit{Die Ph\"{a}nomenologie
  des Geistes} to jedno z~najważniejszych dzieł filozoficznych
  opublikowane po 1750 roku. Według Hegla tytułowa fenomenologia to
  „nauka o~doświadczaniu świadomości”.

  \textbf{Phainomenon} -- z greckiego „obserwowany”, „zjawisko”.
  Istniej również kierunek filozofii zwany fenomenologią (o~tym
  potem).

  \begin{itemize}

  \item Czy jest „Des Geistes”?

  \item Hegel próbował w książce poruszyć chyba każdym możliwy
    temat.

  \item Rozwój myślenia dialektycznego.

  \item Samoświadomość.

  \item Dialektyka „pana i~niewolnika” (bardziej kmiecia, knechta).

  \end{itemize}

\end{frame}
% ##################





% ##################
\begin{frame}
  \frametitle{Niemiecki idealizm}


  Wpływ tego nurtu ciężko oszacować i~przecenić.
  \begin{itemize}
  \item Wpływ na wiek XIX był poprostu ogromny.

  \item  Jest jednym z~najważniejszych źródeł romantyzmu
    europejskiego. Niewątpliwy związek z nim mieli Samuel Taylor
    Coleridge, Thomas De Quincey, Juliusz Słowacki, jednak na nich
    lista się na pewno nie kończy.

  \item Filozofia historii.

  \item Rekonstrukcja gnostycyzmu przez Schellinga?

  \item Relacja człowiek-świat.

  \item Koncepcja filozofii dziejów w zupełnie nowym ujęciu.

  \item Miejsce jednostki w świecie i państwie.

  \item Osiągnięcie absolutu.

  \item Neoheglizm był żywy co najmniej do końca XIX wieku.

  \end{itemize}

\end{frame}
% ##################





% ##################
\begin{frame}
  \frametitle{Można wysunąć zastrzeżenie}


  \begin{figure}

    \centering

    \includegraphics[height=4cm]
    {./PresentationPictures/Caspar_David_Friedrich_Wanderer_above.jpg}
    \includegraphics[height=4cm]
    {./PresentationPictures/Immanuel_Kant_02.jpg}

  \end{figure}



  Kant jakoś nie przypomina stereotypowego romantyka. Posłuchajmy jednak
  jego własnych słów.

  „Dwie rzeczy napełniają umysł coraz to nowym i~wzmagającym się
  podziwem i czcią, im częściej i trwalej się nad nimi
  zastanawiamy: niebo gwiaździste nade mną i prawo moralne we
  mnie.”. Kant \textit{Krytyka praktycznego rozumu}.
  % (Kto czytał \textit{Linię oporu} Dukaja?)

\end{frame}
% ##################





% ##################
\begin{frame}
  \frametitle{Zatrzymajmy się na chwilę na Emanuel Swedenborg}


  \begin{figure}

    \centering

    \includegraphics[height=4cm]
    {./PresentationPictures/Emanuel_Swedenborg.png}


    \caption{Emanuel Swedenborg (1688--1772)}

  \end{figure}


  To chyba najbardziej wpływowy mag jaki tworzył od 1700 roku do
  dziś.

\end{frame}
% ##################





% ##################
\begin{frame}
  \frametitle{Różne dzieła o~Swedenborgu}


  Immanuel Kant \textit{Träume eines Geistersehers} (pl. \textit{Sny o~duchu
    widzącym}), 1766 r.

  Ralph Waldo Emerson \textit{Swedenborg; or, the Mystic}.

  Daisetsu Teitaro Suzuki (1870--1966), \textit{Swedenborg~--
    Budda Północy}.

  Czesław Miłosz co najmilej kilka razy do niego nawiązywał. Wiersz
  \textit{Piekła Swedenborga}, esej w~\textit{Ogrodzie nauk}, wywiad we
  wstępie do książki S.~Toksvig \textit{Emanuel Swedenborg~-- uczony
    i~mistyk}. Według Miłosza „Nawet w~jego opisach
  piekła widać oko inżyniera”.

  Joseph Campbell (1904--1987) \textit{Bohater o~tysiącu twarzy}, 1949~r.
  Ale to temat na inne spotkanie.

\end{frame}
% ##################










% ######################################
\section{Reakcja przeciw Heglowi}
% ######################################



% ##################
\begin{frame}
  \frametitle{Antyheglizm}


  \begin{figure}

    \centering

    \includegraphics[height=4cm]
    {./PresentationPictures/Soren_Kierkegaard_01.jpg}
    \includegraphics[height=4cm]{./PresentationPictures/Karl_Marx_01.jpg}


    \caption{Søren Kierkegaard (1813--1855), Karl Marx (1818--1883)}

  \end{figure}

\end{frame}
% ##################





% ##################
\begin{frame}
  \frametitle{Ich bardziej znane wizerunki}


  \begin{figure}

    \centering

    \includegraphics[height=4cm]
    {./PresentationPictures/Soren_Kierkegaard_02.jpg}
    \includegraphics[height=4cm]{./PresentationPictures/Karl_Marx_02.jpg}


    \caption{Søren Kierkegaard (1813--1855), Karl Marx (1818--1883)}

  \end{figure}

\end{frame}
% ##################





% ##################
\begin{frame}
  \frametitle{Kantowski antyheglizm Schopenhauer i jego następstwa}


  \begin{figure}

    \centering

    \includegraphics[height=4cm]
    {./PresentationPictures/Arthur_Schopenhauer_01.jpg}
    \includegraphics[height=4cm]
    {./PresentationPictures/Friedrich_Nietzsche_01.jpg}


    \caption{Arthur Schopenhauer (1788--1860), Friedrich Nietzsche
      (1844--1900)}

  \end{figure}

\end{frame}
% ##################





% ##################
\begin{frame}
  \frametitle{Ich bardziej znane wizerunki}


  \begin{figure}

    \centering

    \includegraphics[height=4cm]
    {./PresentationPictures/Arthur_Schopenhauer_02.jpg}
    \includegraphics[height=4cm]
    {./PresentationPictures/Friedrich_Nietzsche_02.jpg}


    \caption{Arthur Schopenhauer (1788--1860), Friedrich Nietzsche
      (1844--1900)}

  \end{figure}

\end{frame}
% ##################





% ##################
\begin{frame}
  \frametitle{Po drugiej stronie Atlantyku}


  \begin{figure}

    \centering

    \includegraphics[height=4cm]
    {./PresentationPictures/Ralph_Waldo_Emerson.jpg}


    \caption{Ralph Waldo Emerson (1803--1882)}

  \end{figure}



  Twórca nurtu filozofii nazwanego \textit{transcendetalizmem}. To chyba
  pierwszy nurty filozofii, który stworzyli w~pełnym tego słowa znaczenia
  Amerykanie. Do dziś można znaleźć jego wpływy w~USA.

\end{frame}
% ##################





% ##################
\begin{frame}
  \frametitle{Czy Freud był kantystą?}


  \begin{figure}

    \centering

    \includegraphics[height=4cm]{./PresentationPictures/Sigmund_Freud.jpg}


    \caption{Sigmund Freud (1856--1939)}

  \end{figure}

\end{frame}
% ##################










% ######################################
\section{Filozofia analityczna i kontynentalna}
% ######################################



% ##################
\begin{frame}
  \frametitle{Sytuacja na przełomie XIX i~XX wieku}


  Kant i~Hegel zdefiniowali filozofię XIX stulecie w~stopniu
  niebywałym. Skutkiem tego na przełomie wieku XIX i~XX wieku działali
  ważni myśliciel silnie zarówno odrzucający to kantowsko-heglowską
  myśl jak i~silnie z~niej czerpiący.

  Standardowa wersja historii mówi, że~pojawiły się wtedy dwie szkoły,
  które w różnym stopniu są z nami do dziś. Będę używał terminologi,
  wedle której szkoły te noszą nazwy \textbf{analitycznej}
  i~\textbf{kontynentalnej}.

\end{frame}
% ##################





% ##################
\begin{frame}
  \frametitle{Pierwsze fala}


  \begin{figure}

    \centering

    \includegraphics[height=4cm]{./PresentationPictures/Gottlob_Frege.jpg}
    \includegraphics[height=4cm]{./PresentationPictures/Edmund_Husserl.jpg}


    \caption{Gottlob Frege (1848--1925), Edmund Husserl (1859--1938)}

  \end{figure}


\end{frame}
% ##################





% ##################
\begin{frame}
  \frametitle{Pierwsza fala}


  \begin{figure}

    \centering

    \includegraphics[height=4cm]{./PresentationPictures/G_E_Moore.jpeg}
    \includegraphics[height=4cm]
    {./PresentationPictures/Bertrand_Russell.jpeg}


    \caption{G. E. Moore (1873--1958), Bertrand Russell (1872--1970)}

  \end{figure}

\end{frame}
% ##################





% ##################
\begin{frame}
  \frametitle{Podobieństwa i~różnice}


  \begin{itemize}
    \RaggedRight

  \item Obie szkoły zaczęły się w Niemczech.

  \item U~swych początków obie były mocno związane z~logiką. Jednak
    szkoła kontynentalna w~pewnym momencie

  \item Filozofia analityczna zdominowała do lat 60-tych świat
    anglosaski, kontynentalna Francję i Niemcy Zachodnie. Jej droga do
    krajów bloku komunistycznego była zawikłana, podobnie jak na
    południe Europy. Co nie znaczy, że nie było „kontynentalnych”
    anglików, czy „analitycznych” francuzów.

  \item Plany ojców w~obu przypadkach, zostały podważone przez ich
    najwybitniejszych uczniów.

  \item Jeżeli słyszał ktoś, że~coś głupiego zrobiono
    w~średniowiecznej filozofii, oni co najmniej temu dorównali.

  \end{itemize}

\end{frame}
% ##################





% ##################
\begin{frame}
  \frametitle{Szkoła analityczna}


  \begin{itemize}

  \item Początek związany jest z~badaniami logicznymi, głównie pracami
    Fregego i Russela (Whitehead poszedł swoją drogą).

  \item Po ogromnym rozwoju logiki w XIX wieku, powstał pomysł by
    zaaplikowanie jej aparat do analizy innych problemów, w~tym
    filozoficznych.

  \item W naturalny sposób doprowadziło to do wielkich badań
    na~logiczną strukturą języka.

  \item Filozofie tej szkoły mają zwyczaj rozbijać wszystkie problemy
    na~czynniki pierwsze, stąd zasłużone miano filozofii analitycznej.

  \item „Philosophical problems should not be solved, but dissolved”,
    Ludwig Wittgenstein.

  \item Ich konkretny wpływ na~filozofię i kulturę jest bardzo trudny
    do prześledzenia. Od tytułu książki jednego z~członków tej szkoły
    Gilbert Ryle wziął swój tytuł film anime \textit{Ghost in the
      Shell}.

  \end{itemize}

\end{frame}
% ##################





% ##################
\begin{frame}
  \frametitle{Skupienie na języku}


  \begin{figure}

    \centering

    \includegraphics[height=4cm]
    {./PresentationPictures/Gottlob_Frege.jpg}


    \caption{Gottlob Frege (1848--1925)}

  \end{figure}



  Obie szkoły poważnie zajęły się problemami języków, zarówno
  naturalnych, jak różnego rodzaju języków formalnych. Według Michael
  Dummettem pierwszy tzw. \textbf{zwrotu lingwistycznego} (ang.
  \textit{linguistic turn}) dokonał Gottlob Frege w~swoim dziele
  \textit{Begriffsschrift}.

  Alisdair MacIntyre w~swoim dziel \textit{Dziedzictwo cnoty} uważa
  też, że~książka G.E. Moore poświęconą etyce, za ważny etap rozwoju,
  albo upadku, etyki europejskiej.

\end{frame}
% ##################





% ##################
\begin{frame}
  \frametitle{Szkoła kontynentalna}


  \begin{figure}

    \centering

    % \includegraphics[height=4cm]{./PresentationPictures/Gottlob_Frege.jpg}
    \includegraphics[height=4cm]{./PresentationPictures/Edmund_Husserl.jpg}


    \caption{Edmund Husserl (1859--1938)}

  \end{figure}



  W moim mniemaniu ta szkoła zaczyna się wraz z~fenomenologią Edmunda
  Husserla. Pierwszym ważnym dziełem Husserla są \textit{Logische
    Untersuchungen} (pl. \textit{Dociekania logiczne}), dwa tomy,
  1900, 1901.

  Filozofia Husserla była inspirowana Kartezjusza, mówienie o~wpływach
  Kanta nie jest bezpodstawne. Hegel? Nie wiem.

\end{frame}
% ##################





% ##################
\begin{frame}
  \frametitle{Czym jest fenomenologia?}


  „Doświadczając, wyobrażając sobie coś, myśląc czy mówiąc o~czymś
  napotykamy różnego rodzaju przedmioty. Jak to się dzieje,
  że~napotykają nas one w~takim właśnie kształcie: jako rzeczy,
  ludzie, zdarzenia, wartości itd.? Oto pytanie fenomenologii. Metoda
  fenomenologiczna jest więc analizą i~opisem samego procesu zjawiania
  się czegoś, badaniem stosunku różnych perspektyw i~sposobów
  prezentacji do~tego, co w~nich dane, nie~analizą przedmiotu, lecz
  analizą projektu wszelkiego możliwego spotkania z~przedmiotem (Jan
  Potocka). [\ldots] Inaczej mówiąc: odkrycie \textit{Badań logicznych}
  polega na ujawnieniu różnicy między tym, co się zjawia, a~samym
  zjawianiem się.” Krzysztof Michalski \textit{Heidegger i filozofia
    współczesna}.

\end{frame}
% ##################





% ##################
\begin{frame}
  \frametitle{Fenomenologia}


  \begin{itemize}

  \item Fenomenologia to zasadniczo filozofia zajmująca się
    „zjawiskami takimi jakie się jawią” poprzez przeprowadzenie
    odpowiednich redukcji oraz poszukiwanie naoczności (intuicji)
    zjawisk.

  \item Krytyka ówczesnego racjonalizmu i~„Z powrotem do rzeczy!”.

  \item Fenomenologia Husserla zrodziła współczesny egzystencjalizm
    (Heidegger).

  \item Czy program Husserla się udał? Raczej nie.

  \item Wydaje mi się, że~powrót do rzeczy i~idealizm nie pasują do
    siebie. „Fenomenologia to system filozoficzny w którym najbardziej
    liczą się nie rzeczy, lecz wrażenia rzeczy.”. Tak skrytykował
    fenomenologię amerykański dziennikarz Michael Voris.

  \end{itemize}

  Filozofia Husserla miała być krytyką i~reformą europejskiego
  racjonalizmu, lecz czy była czymś innym, niż logiczną jej
  kontynuacją?

\end{frame}
% ##################










% ######################################
\section{Dwaj niszczyciele marzeń}
% ######################################



% ##################
\begin{frame}
  \frametitle{Druga fala, dwaj bohaterowie}


  \begin{figure}

    \centering

    \includegraphics[height=4cm]
    {./PresentationPictures/Martin_Heidegger.jpg}
    \includegraphics[height=4cm]
    {./PresentationPictures/Ludwig_Wittgenstein.jpg}


    \caption{Martin Heidegger (1889--1976), Ludwig Wittgenstein
      (1889--1951)}

  \end{figure}

\end{frame}
% ##################





% ##################
\begin{frame}
  \frametitle{Martin Heidegger}

  Krótki i może niesprawiedliwy życiorys Heidggera.
  \begin{itemize}

  \item Urodził się.

  \item Był nazistą.

  \item Umarł.

  \end{itemize}

  Życiorys filozoficzny.
  \begin{itemize}

  \item Student Husserla.

  \item Ustanowił dwudziestowieczny egzystencjalizm
    i~hermeneutykę.

  \item Miał obsesje na~punkcie etymologii i~pisał w~niezwykle trudny
    sposób.

  \item Bardzo Niemiecki, przynajmniej w~pewnym okresie.

  \item Dokonał „zwrotu” w~swej filozofii. Nie do końca jasne jakiego.

  \item Główne dzieło: \textit{Seit und Zeit} (pl. \textit{Bycie i~czas}),
    1927.

  \end{itemize}

\end{frame}
% ##################

% \begin{frame}{Seit und Zeit}
%   \begin{block}{Co warto wiedzieć}
%     \begin{itemize}
%     \item Jedna z~dwóch najważniejszych książek filozoficznych po
%       roku 1925.
%     \item Światowy bestseller. M.in. sześć tłumaczeń na~język
%       japoński do 1970 roku.
%     \item Zadedykowane Husserlowi. Dedykacja usunięta w~wydaniu
%       z~1939 r.(?)
%     \item Tekst założycielski współczesnego egzystencjalizmu.
%     \item Miała być pierwszą częścią dwutomowego dzieła. Drugi tom
%       nigdy się nie ukazał.
%     \end{itemize}
%   \end{block}

%   \begin{block}{Ciekawostka}
%     Światowa twarz egzystencjalizmu Jean-Paul Sartre wydał w 1943 r.
%     książkę ,,L'étre et le néant'' (Bycie i nicość).
%   \end{block}
% \end{frame}

% \begin{frame}{Seit und Zeit}
%   \begin{block}{Uwagi techniczne}
%     \begin{itemize}
%     \item Heidegger używał silnie neologizmów (Husserl), lub
%       zwykłych słów niemieckich w zupełnie innym sensie.
%     \item Przykłady: Dasein, Lichtung (,,Bycie w prześwicie'',
%       za~Nonsensopedią). \pause
%     \item Jego styl jest trudny/nieludzko~zagmatwany/niejasny/
%       zagadkowy/bezsensowny/maskujący brak treści (niepotrzebne
%       skreślić). \newline ,,Wszechobecność wyistaczania ukazuje nam się
%       najwyraźniej wówczas, gdy zauważymy, że~również i~właśnie
%       odistaczanie określone jest poprzez owo, tajemnicze niekiedy
%       wyistaczanie.'' ,,Zeit und Seit'' (,,Czas i bycie'').
%     \end{itemize}
%   \end{block}
% \end{frame}


% \begin{frame}{Seit und Zeit}
%   \begin{block}{Pytanie o sens bycia}
%     ,,Czy w dzisiejszych czasach dysponujemy odpowiedzią na~pytanie,
%     co właściwie rozumiemy pod słowem ,,bytujący''? W~żadnym razie
%     nie. Dlatego trzeba na nowo postawić \emph{pytanie o sens
%     bycia}. Czy~jednak dzisiaj fakt, że~nie rozumiemy wyrażenia
%     ,,bycie'' jest naszym jedynym kłopotem? Bynajmniej. Dlatego też
%     przede wszystkim musimy na nowo obudzić zrozumienie dla sensu
%     tego pytania. Zamiarem niniejszej rozprawy jest konkretne
%     opracowanie pytania o~sens ,,bycia''. Prowizorycznym celem zaś
%     będzie interpretacja \emph{czasu} jako możliwego horyzontu
%     wszelkiego w ogóle rozumienia bycia.''
%   \end{block}
% \end{frame}

% \begin{frame}{Seit und Zeit}
%   \begin{block}{Treść}
%     \begin{itemize}
%     \item To nie jest pierwsze dzieło egzystencjalizmu, jednak
%       stanowi w~nim przełom.
%     \item Aby zbadać co to jest bycie należy zbadać jakiś obiekt
%       który jest. Najprościej zbadać samego siebie.
%     \item ,,Bycie w świecie'' i~bycie w czasie. \pause
%     \item Tom II nie został napisany bowiem Heidegger zmienił swoje
%       poglądy filozoficzne. Dokonał zwrotu.
%     \end{itemize}
%   \end{block}
% \end{frame}

% \begin{frame}{Ludwig Wittgenstein}
%   \begin{block}{Życiorys}
%     \begin{itemize}
%     \item Urodził się w Wiedniu, w Cesarstwie Austro-Węgierskim.
%     \item Jego rodzina ze~strony ojca miała żydowskie pochodzenie,
%       ze~strony matki czesko-słoweńskie
%     \item Był jednym z~dziewięciorga dzieci. Jego ojciec praktycznie
%       władał przemysłem stalowym w Austro\dywiz Węgrzech.
%     \item Początkowo pragnął zostać inżynierem aeronautyki.
%     \item Nauka potrzebnej mu matematyki doprowadziło go do pytań
%       o~jej podstawy i~filozofii matematyki.
%     \end{itemize}
%   \end{block}
% \end{frame}

% \begin{frame}{Wczesny Wittgenstein}
%   \begin{block}{Życiorys filozoficzny}
%     \begin{itemize}
%     \item Jako jedyny, chyba, człowiek w pierwszej połowie XX w.
%       który czytał i brał na serio Schopenhauera.
%     \item Problemy z~podstawami matematyki prowadzą go do prac
%       Fregego i Russela. Poznaje i zaprzyjaźnia się z~oboma.
%     \item Studiuje u~Russela i~Moora.
%     \item 1914 wybucha I Wojna Światowa, Wittgenstein zaciąga się do
%       wojsk cesarstwa Austro-Węgierskiego.
%     \item W przerwach działań wojennych tworzy ,,Tractatus\ldots ''
%       i~wiele notatek filozoficznych, zostaje też dwukrotnie
%       wyznaczony do medalu za odwagę.
%     \item Uważając, że rozwiązał już wszystko co \emph{dało się
%       rozwiązać} po wojnie porzuca filozofię. Tu kończy się wczesny
%       Wittgenstein.
%     \end{itemize}
%   \end{block}
% \end{frame}

% \begin{frame}{Tractatus logico-philosophicus}
%   \begin{block}{Co warto wiedzieć}
%     \begin{itemize}
%     \item Napisany jako ,,Logisch - philosophische Abhandlung''.
%       Pierwotnym językiem twórczości Wittgensteina pozostał zawsze
%       wysokiej klasy niemiecki.
%     \item Pierwszy raz wydany w 1921 r. Tytuł pod którym obecnie
%       słynie zaproponował G.E. Moore dla wydania angielskiego z~1922
%       r.
%     \item Książka bardzo małej objętości, w polskim wydaniu zajmuje
%       84 strony.
%     \item Zawiera m.in dyskusje nad problemami innych filozofów
%       np.~z~Kanta, rozważania lingwistyczne, matematyczne,
%       egzystencjalne i~teologiczne.
%     \item Podobno bardzo niewielu przeczytało go w całości.
%     \end{itemize}
%   \end{block}
% \end{frame}

% \begin{frame}{Tractatus logico-philosophicus}
%   \begin{block}{Co warto wiedzieć}
%     \begin{itemize}
%     \item Oddziaływanie traktatu było ogromnie. Stał się jedną
%       z~ulubionych lektur pozytywistów logicznych (m.in Koła
%       Wiedeńskiego).
%     \item Wciąż wielu naukowców przyjmuje ich interpretacje
%       ,,Tractatusa\ldots ''.
%     \item Odegrał również wielką rolę w filozofii języka.
%     \end{itemize}
%   \end{block}
% \end{frame}

% \begin{frame}{Tractatus logico-philosophicus}
%   \begin{block}{Początek}
%     1. Świat jest wszystkim co jest faktem\\
%     1.1 Świat jest ogółem faktów nie rzeczy.\\
%     1.11 Świat jest wyznaczony przez fakty oraz przez to, że są to \emph{wszystkie} fakty.\\
%     1.12. Ogół faktów wyznacza bowiem, co jest faktem, a także wszystko, co faktem nie jest.\\
%     1.13. Światem są fakty w przestrzeni logicznej.\\
%     1.2. Świat rozpada się na fakty.\\
%     1.21. Jedno może być być faktem lub nie być, a wszystko inne pozostanie takie samo.\\
%     2. To, co jest faktem - fakt - jest istnieniem stanów rzecz.
%   \end{block}
% \end{frame}

% \begin{frame}{Traktatus logico-philosophicus}
%   \begin{block}{Zakończenie}
%     6.54. Tezy moje wnoszą jasność przez to, że kto mnie rozumie, rozpozna je w końcu jako niedorzeczne; gdy przez nie - po nich - wyjdzie ponad nie. (Musi niejako odrzucić drabinę, uprzednio po niej się wspiąwszy.) Musi te tezy przezwyciężyć, wtedy świat przedstawi mu się właściwie.\\
%     7. O czym nie można mówić o tym trzeba milczeć.
%   \end{block}
% \end{frame}

% \begin{frame}{Dla przestrogi}
%   \begin{block}{Dwie wypowiedzi Wittgensteina}
%     \begin{itemize}
%       \pause
%     \item Z listu do wydawcy:\\
%       ,,Otóż chciałem napisać, że moja praca składa się z~dwu
%       części: z~tego co w niej napisałem, oraz tego wszystkiego,
%       czego nie napisałem. I właśnie ta druga część jest ważna.''
%       \pause
%     \item Zakończenie wstępu do Tractatusa, napisanego w 1918 roku:\\
%       ,,Natomiast \emph{prawdziwość} komunikowanych tu myśli zdaje
%       mi się niepodważalna i definitywna. Sądzę więc, że w istotnych
%       punktach problemy zostały rozwiązane ostatecznie. A jeżeli się
%       tu nie mylę, to wartością niniejszej pracy jest - po wtóre -
%       to, że~widać z~niej, jak mało się przez ich rozwiązanie
%       osiągnęło.''
%     \end{itemize}
%   \end{block}
% \end{frame}

% \begin{frame}{Dla przestrogi}
%   \begin{block}{Treść ,,Tractatusa\ldots ''}
%     Najwłaściwszy chyba opis jego treści, to zanalizowanie logicznej
%     treści świata, tego co mieści się w kompetencji nośnika prawd
%     logicznych (języka), a tego co poza niego wykracza.
%   \end{block}

%   \begin{block}{Kluczowe}
%     W tym momencie kluczowym jest punkt, że język odbija w sobie
%     strukturę świata, która jest strukturą logiczną. Jest to jednak
%     temat trudny i~nie pozbawiony kontrowersji.
%   \end{block}
% \end{frame}

% \begin{frame}{Późny Wittgenstein}
%   \begin{block}{Życiorys filozoficzny}
%     \begin{itemize}
%     \item Po dziesięciu latach zajmowania najprzeróżniejszymi
%       zajęciami, wysłuchuje w Wiedniu przez przypadek
%       wykładu\linebreak L. E. J. Brouwera na temat filozofii
%       matematyki. W wyniku tego wraca do filozofii.
%     \item 1929 publikuje swoją drugą i ostatnią ogłoszoną za życia
%       pracę.
%     \item Jeśli wierzyć Wikipedii to oprócz tego jego opublikowany
%       dorobek zawiera jeszcze: \emph{jedną} recenzję książki i
%       słownik dla dzieci.
%     \item Uzyskuje tytuł naukowy w Cambridge składają ,,Tractatusa\ldots
%       '' (wówczas już uznanego za wybitne dzieło filozoficzne) jako
%       swoją pracę dyplomową.
%     \end{itemize}
%   \end{block}
% \end{frame}

% \begin{frame}{Późny Wittgenstein}
%   \begin{block}{Życiorys filozoficzny}
%     \begin{itemize}
%     \item Od tego czasu naucza z~przerwami na angielskich
%       uniwersytetach i rozwija zupełnie nową filozofię.
%     \item Umiera w 1951 roku.
%     \item 1953 ukazują się ,,Philosophische Untersuchungen'', druga
%       najważniejsza książka filozoficzna po 1925 r.
%     \item Większość prac Wittgensteina ukazało się po jego śmierci.
%     \end{itemize}
%   \end{block}
% \end{frame}

% \begin{frame}{Późny Wittgenstein}
%   \begin{block}{Dociekania filozoficzne}
%     Dzieło to podważa koncepcje języka jako odbicia rzeczywistości,
%     zastępuję je pojęciem gry językowej. Język w~tym sensie jest
%     ludzką aktywnością z~której czerpie swój sens.
%   \end{block}

%   \begin{block}{}
%     Końcowy jest jednak taki sam, że~problemy filozoficzne (etyczne,
%     teologiczne etc.) są nierozwiązywalne. Tu jednak też są ogromne
%     niejasności i kontrowersje.
%   \end{block}

% \end{frame}

% \begin{frame}{Na zakończenie}
%   \frametitle{Krótki przegląd pozostałych filozofów XX w.}
%   \begin{block}{Fenomenologia poza szkołą Husserela}
%     \begin{figure}
%       \centering \includegraphics[height=1.6in, width=1.2in]{MS}
%       \caption{Max Scheler (1874 - 1928).}
%     \end{figure}
%   \end{block}
% \end{frame}

% \begin{frame}{Na zakończenie}
%   \frametitle{Krótki przegląd pozostałych filozofów XX w.}
%   \begin{block}{Francuska szkoła kontynentalna}
%     \begin{figure}
%       \centering \includegraphics[height=1.6in, width=1.1in]{JPS}
%       \includegraphics[height=1.6in, width=1.2in]{MP}
%       \includegraphics[height=1.6in, width=1.3in]{SB}
%       \caption{Jean-Paul Sartre (1905 - 1980), Maurice
%       Merleau-Ponty\linebreak (1908 - 1961), Simone de Beauvoir
%       (1908 - 1986).}
%     \end{figure}
%   \end{block}

% \end{frame}

% \begin{frame}{Na zakończenie}
%   \frametitle{Krótki przegląd pozostałych filozofów XX w.}
%   \begin{block}{Pozytywizm logiczny $\subset$ Koło Wiedeńskie}
%     \begin{figure}
%       \centering \includegraphics[height=1.6in, width=1.2in]{RC}
%       \includegraphics[height=1.6in, width=1.1in]{ON}
%       \caption{Rudolf Carnap (1891 - 1970), Otto Neurath (1882 -
%       1945).}
%     \end{figure}
%   \end{block}
% \end{frame}

% \begin{frame}{Na zakończenie}
%   \frametitle{Krótki przegląd pozostałych filozofów XX w.}
%   \begin{block}{Po-Wittgensteinie i filozofia \emph{nauki}}
%     \begin{figure}
%       \centering \includegraphics[height=1.6in, width=1.1in]{EA}
%       \includegraphics[height=1.6in, width=1.2in]{KP}
%       \caption{Gertrude Elizabeth Margaret Anscombe (1919 -
%       2001),\linebreak Karl Popper (1902 - 1994).}
%     \end{figure}
%   \end{block}
% \end{frame}

% \begin{frame}{Na zakończenie}
%   \frametitle{Krótki przegląd pozostałych filozofów XX w.}
%   \begin{block}{Francuski Hegel}
%     \begin{figure}
%       \centering \includegraphics[height=1.6in, width=1.5in]{MB2}
%       \caption{Henri-Louis Bergson (1861 - 1941).}
%     \end{figure}
%   \end{block}
% \end{frame}

% \begin{frame}{Na zakończenie}
%   \frametitle{Krótki przegląd pozostałych filozofów XX w.}
%   \begin{block}{Filozofowie procesu}
%     \begin{figure}
%       \centering \includegraphics[height=1.6in, width=1.2in]{HB}
%       \includegraphics[height=1.6in, width=1.2in]{AW}
%       \caption{Henri-Louis Bergson (1859 - 1941), Alfred North
%       Whitehead (1861 - 1947).}
%     \end{figure}
%   \end{block}
% \end{frame}

% \begin{frame}{Na zakończenie}
%   \frametitle{Krótki przegląd pozostałych filozofów XX w.}
%   \begin{block}{Filozofia amerykańska}
%     \begin{figure}
%       \centering \includegraphics[height=1.6in, width=1.3in]{JD}
%       \includegraphics[height=1.6in, width=1.5in]{WQ}
%       \caption{John Dewey (1859 - 1961), Willard Van Orman Quine
%       (1908\linebreak - 2000).}
%     \end{figure}
%   \end{block}
% \end{frame}

% \begin{frame}{Na zakończenie}
%   \frametitle{Krótki przegląd pozostałych filozofów XX w.}
%   \begin{block}{Filozofowie polityczni}
%     \begin{figure}
%       \centering \includegraphics[height=1.6in, width=1.3in]{AR}
%       \includegraphics[height=1.6in, width=1.5in]{JR}
%       \includegraphics[height=1.6in, width=1.1in]{RN}
%       \caption{Ayn Rand (1905 - 1982), John Rawls (1921 -
%       2002),\linebreak Robert Nozick (1938 - 2002).}
%     \end{figure}
%   \end{block}

% \end{frame}

% \begin{frame}{Na zakończenie}
%   \frametitle{Krótki przegląd pozostałych filozofów XX w.}
%   \begin{block}{Hermeneutyka, strukturalizm i neopragmatyzm}
%     \begin{figure}
%       \centering \includegraphics[height=1.6in, width=1.2in]{HG}
%       \includegraphics[height=1.6in, width=1.3in]{CS}
%       \includegraphics[height=1.6in, width=1.2in]{RR}
%       \caption{Hans-Georg Gadamer (1900 - 2002), Claude Lévi-Strauss
%       (1908 - 2009), Richard Rorty (1931 - 2007).}
%     \end{figure}
%   \end{block}
% \end{frame}

% \begin{frame}{Na zakończenie}
%   \frametitle{Krótki przegląd pozostałych filozofów XX w.}
%   \begin{block}{Postmodernizm i poststrukturalizm}
%     \begin{figure}
%       \centering \includegraphics[height=1.6in, width=1.2in]{MF}
%       \caption{Michel Foucault (1926 - 1984).}
%     \end{figure}
%   \end{block}
% \end{frame}

% \begin{frame}{Na zakończenie}
%   \frametitle{Krótki przegląd pozostałych filozofów XX w.}
%   \begin{block}{Premoderniści}
%     \begin{figure}
%       \centering \includegraphics[height=1.6in, width=1.2in]{LS}
%       \includegraphics[height=1.6in, width=1.3in]{HA}
%       \includegraphics[height=1.6in, width=1.2in]{AM}
%       \caption{Leo Strauss (1899 - 1973), Hannah Arendt (1906 -
%       1975), Alasdair MacIntyre (1929 - żyje).}
%     \end{figure}
%   \end{block}
% \end{frame}

% \begin{frame}{I na ostatek}
%   \begin{block}{Najważniejszy filozof drugiej połowy XX w.}
%     \pause
%     \begin{figure}
%       \centering \includegraphics[height=1.6in, width=1.2in]{JD1}
%       \includegraphics[height=1.6in, width=1.8in]{JD3}
%       \includegraphics[height=1.6in, width=1.2in]{JD2}
%       \caption{Jacques Derrida (1930 - 2004).}
%     \end{figure}
%   \end{block}
% \end{frame}

% \begin{frame}{Dla dociekliwych}

%   \begin{block}{Polecane pozycje}
%     Proszę nie podejrzewać, że ja to wszystko przeczytałem.
%     \begin{itemize}
%     \item[--] Frederick Copleston: \emph{Historia filozofii}
%     \item[--] Władysław Tatarkiewicz: \emph{Historia filozofii}
%     \item[--] Tadeusz Gadacz \emph{Historia filozofii XX wieku.
%       Nurty}
%     \item[--] Lawrence Cahoone: \emph{Modern Intellectual Tradition:
%       From Descartes to Derrida} (do zakupienia na stronie Great
%       Courses)
%     \item[--] Kanał na YouTubie Gregor'ego B. Sadler
%     \item[--] Anna Burzyńska, Michał Paweł Markowski: \emph{Teorie
%       literatury XX wieku}
%     \item[--] Leszek Kołakowski: \emph{Główne nurty marksizmu}
%     \item[--] Terence Hawkes: \emph{Strukturalizm i semiotyka}
%     \end{itemize}
%   \end{block}
% \end{frame}

% \begin{frame}{Dla dociekliwych}
%   \begin{block}{Polecane pozycje}
%     \begin{itemize}
%     \item[--] Krzysztof Michalski: \emph{Heidegger i filozofia
%       współczesna}
%     \item[--] Leszek Kołakowski ,,O co nas pytają wielcy
%       filozofowie''
%     \item[--] Julian Young: \emph{Heidegger, filozofia, nazizm}
%     \item[--] G. E. M. Anscombe, P. T. Geach: \emph{Trzej
%       filozofowie}
%     \item[--] Willard Van Orman Quine \emph{Logika matematyczna}
%     \item[--] Ian Shapiro: \emph{Moral Foundations of Politics}
%       (dostępne na YouTubie)
%     \item[--] S. Toksvig ,,Emanuel Swedenborg -- uczony i mistyk''
%     \end{itemize}
%   \end{block}
% \end{frame}



% ############################

% Koniec dokumentu
\end{document}
