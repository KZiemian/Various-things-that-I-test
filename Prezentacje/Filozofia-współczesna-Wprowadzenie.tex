% ---------------------------------------------------------------------
% Basic configuration of Beamera and Jagiellonian
% ---------------------------------------------------------------------
\RequirePackage[l2tabu, orthodox]{nag}



\ifx\PresentationStyle\notset
\def\PresentationStyle{dark}
\fi



\documentclass[10pt,t]{beamer}
\mode<presentation>
\usetheme[style=\PresentationStyle,logoLang=Latin,logoColor=monochromaticJUwhite,JUlogotitle=yes]{jagiellonian}



% ---------------------------------------
% Configuration files of Jagiellonian loceted in catalog preambule
% ---------------------------------------
\input{./preambule/LanguageSettings/JagiellonianPolishLanguageSettings.tex}
\input{./preambule/TextposConfiguration/TextposConfiguration.tex}

\input{./preambule/JagiellonianCustomizationGeneral.tex}
\input{./preambule/JagiellonianCustomizationCommands.tex}










% ---------------------------------------
% Packages, libraries and their configuration
% ---------------------------------------





% ---------------------------------------
% Configuration for this particular presentation
% ---------------------------------------










% ---------------------------------------------------------------------
\title{Filozofia współczesna. Wprowadzenie}

\author{Kamil Ziemian \\
  \texttt{kziemianfvt@gmail.com}}

% \date{}
% ---------------------------------------------------------------------










% ####################################################################
% Początek dokumentu
\begin{document}
% ####################################################################





% Wyrównanie do lewej z łamaniem wyrazów

\RaggedRight





% ######################################
\maketitle
% ######################################





% ######################################
\begin{frame}
  \frametitle{Spis treści}


  \tableofcontents % Spis treści

\end{frame}
% ######################################










% ######################################
\section{Wprowadzenie}
% ######################################



% ##################
\begin{frame}
  \frametitle{Kilka uwag i~ostrzeżeń}


  Chciałbym bardzo pobieżnie poopowiadać trochę o filozofii współczesnej,
  by zainteresowani mieli jakiś punkt startowy dla jej dalszego zgłębiania.

  Co nie znaczy, że jest to najlepszy punkt startowy.
  \begin{itemize}
    \RaggedRight

  \item W~wystąpieniu jest wiele luk, zasób mojej wiedzy nie jest
    najwyższej jakości. Jest wiele białych plan i~błędów, których wciąż
    nie jestem świadomy.

  \item Filozofia po 1990 roku, jest w ilościach śladowych, jeśli w~ogóle.

  \item Proszę również o~wybaczenie i wskazywanie literówek, niepoprawnej
    składni, błędów w~układzie materiału etc.

  \end{itemize}

  Siłą rzeczy, wszystkie uwagi są mile widziane. Warto uczyć się na własnych
  błędach. :)

\end{frame}
% ##################










% ######################################
\section{Zacznijmy prawie od początku}
% ######################################



% ##################
\begin{frame}
  \frametitle{Zacznijmy prawie od początku}


  \begin{figure}

    \centering

    \includegraphics[height=4cm]
    {./PresentationPictures/Plato\_and\_Aristotle.jpg}


    \uncover<2->{\caption{Platon (427--347 BC), Arystoteles (384--322)}}

  \end{figure}



  Złota zasada filozofii Europejskiej/Zachodniej. O czymkolwiek mówi
  się w filozofii, zawsze można zacząć od tych dwóch panów i dojść
  logicznie do wybranego tematu.

\end{frame}
% ##################





% ##################
\begin{frame}
  \frametitle{Aż tyle czasu nie mamy}


  \begin{figure}

    \centering

    \includegraphics[height=4cm]
    {./PresentationPictures/Immanuel_Kant_01.jpg}


    \caption{Immanuel Kant (1724--1804)}

  \end{figure}

  W mojej opinii stworzył on znaczną część fundamentów na których stoi
  filozofia ostatnich XIX, XX, a nawet XXI wieku.

\end{frame}
% ##################





% ##################
\begin{frame}
  \frametitle{Podstawowe informacje o~Kancie}



  \begin{itemize}
    \RaggedRight

  \item Urodził się, żył i umarł w mieście K\"{o}nisberg (Królewiec)
    w~Prusach.

  \item Profesor na tamtejszym uniwersytecie.

  \item Tym samy jeden z dwóch wybitnych filozofów którzy pomiędzy XV,
    a~XX wiekiem parali się tym zajęciem.

  \item Nigdy się nie ożenił.

  \item Legendarny pedant.

  \item Prowincjusz pierwszej kategorii.

  \end{itemize}

  Trzy główne dzieła Kanta.
  \begin{itemize}
    \RaggedRight

  \item \textit{Krytyka czystego rozumu}, 1781;

  \item \textit{Krytyka praktycznego rozumu}, 1788;

  \item \textit{Krytyka władzy sądzenia}, 1791.

  \end{itemize}
  Kant napisał wiele innych dzieł, te trzy powszechnie uchodzą za
  centralne w~jego dorobku.

\end{frame}
% ##################





% ##################
\begin{frame}
  \frametitle{Podstawowe informacje o~Kancie}


  \begin{itemize}
    \RaggedRight

  \item \textit{Krytyka czystego rozumu}, 1781;

  \item \textit{Krytyka praktycznego rozumu}, 1788;

  \item \textit{Krytyka władzy sądzenia}, 1791.

  \end{itemize}

  Można powiedzieć, że \textit{Krytyka czystego rozumu} jest wykładem
  metafizyki, \textit{Krytyka praktycznego rozumu} etyki, zaś
  \textit{Krytyka władzy sądzenia} -- estetyki. Wszystkie te nurty
  myśli Kanta są ważne, jednak tylko pierwszą będę w~stanie naprawdę
  poruszyć.

  „Są oni [Luter, Kartezjusz i~Rousseau] ojcami tego co Gabriel
  S\'{e}ailles nazwał współczesnym sumieniem. Pomijam Immanuela Kanta,
  który stoi na skrzyżowaniu prądów duchowych, płynących od tych
  trzech ludzi. Filozof z~Królewca stworzył niejako scholastyczną
  armaturę współczesnej myśli.”. Jacques Maritain \textit{Trzej
    reformatorzy} \cite{}.

\end{frame}
% ##################





% ##################
\begin{frame}
  \frametitle{Główne źródła myśli Kanta}


  \begin{itemize}
    \RaggedRight

  \item Pietystyczne wychowanie.

  \item Akademicka filozofia niemiecka.

  \item Polemika z~Humem.

  \item Potrzeba pogodzenia nauki i wiary. Chodzi tu raczej o fizyke
    niż biologię, to jednak temat na osobny wykład.

  \item Filozofia oświecenia. Warty jest podkreślenia wpływ
    Rousseau.

  \item Prawdopodobnie polemika z scholastyczną filozofią.

  \item Pietystyczne wychowanie.

  \item Akademicka filozofia niemiecka.

  \item Fizyka Newtona.

  \item Emanuel Swedenborg?

  \end{itemize}

\end{frame}
% ##################





% ##################
\begin{frame}
  \frametitle{Crashcourse filozofii Kanta}


  \begin{itemize}
    \RaggedRight

  \item Po pierwsze i~najważniejsze: Kant był idealistą! Ale co to
    znaczy? I dlaczego nim została?

  \item Kantowska klasyfikacja sądów: \\
    -- analityczne, syntetyczne; \\
    -- a priori (łac. z tego, co wcześniej), a posteriori (łac.
    z~tego, co potem).

  \item Czas i przestrzeń.

  \item Ding an sich. Wielki niszczyciel.

  \item Koncepcja moralna: uniwersalizm.

  \item Dwa problemy: słuszność fizyki Newtona i idee moralne
    Rousseau.

  \item Metoda transcendentalna.

  \end{itemize}

\end{frame}
% ##################










% ######################################
\section{Kantyzmu ciąg dalszy, czyli niemiecki idealizm}
% ######################################



% ##################
\begin{frame}
  \frametitle{Niemiecki idealizm}


  Idealizm Kanta dał, między innymi, początek ruchowi jaki nosi
  nazwę Idealizm Niemiecki. Oprócz samej myśli Kanta oddziaływały tu
  niewątpliwie jego źródła, takie jak pietyzm.

  Warto się zastanowić nad następującą sprawą. Kant był pierwszym
  wielkim filozofem piszącym po niemiecku (bardzo nieczytelnym
  niemieckim), w~czasach gdy zaczęła się rodzić współczesna
  koncepcja niemieckości.

\end{frame}
% ##################





% ##################
\begin{frame}
  \frametitle{Niemiecki idealizm}


  \begin{figure}

    \centering

    \includegraphics[height=4cm]
    {./PresentationPictures/Johann_Gottlieb_Fichte.jpg}
    \includegraphics[height=4cm]
    {./PresentationPictures/Friedrich_Schelling.jpg}


    \caption{Johann Gottlieb Fichte (1762--1814), Friedrich Schelling
    (1775--1854)}

  \end{figure}


  Można nazwać ich pierwszą falą idealizmu niemieckiego.

\end{frame}
% ##################





% ##################
\begin{frame}
  \frametitle{Kulminacja}


  \begin{figure}

    \centering

    \includegraphics[height=4cm]
    {./PresentationPictures/Georg_Wilhelm_Friedrich_Hegel.jpg}


    \caption{Georg Wilhelm Friedrich Hegel (1770--1831)}

  \end{figure}


  Jak wielu twierdzi, filozof bardzo wpływowy nawet dzisiaj.

\end{frame}
% ##################





% ##################
\begin{frame}
  \frametitle{\textit{Fenomenologia Ducha}}


  Wydane w~1807 roku książka Hegla \textit{Die Ph\"{a}nomenologie
  des Geistes} to jedno z~najważniejszych dzieł filozoficznych
  opublikowane po 1750 roku. Według Hegla tytułowa fenomenologia to
  „nauka o~doświadczaniu świadomości”.

  \textbf{Phainomenon} -- z greckiego „obserwowany”, „zjawisko”.
  Istniej również kierunek filozofii zwany fenomenologią (o~tym
  potem).

  \begin{itemize}

  \item Czy jest „Des Geistes”?

  \item Hegel próbował w książce poruszyć chyba każdym możliwy
    temat.

  \item Rozwój myślenia dialektycznego.

  \item Samoświadomość.

  \item Dialektyka „pana i~niewolnika” (bardziej kmiecia, knechta).

  \end{itemize}

\end{frame}
% ##################





% ##################
\begin{frame}
  \frametitle{Niemiecki idealizm}


  Wpływ tego nurtu ciężko oszacować i~przecenić.
  \begin{itemize}
  \item Wpływ na wiek XIX był poprostu ogromny.

  \item  Jest jednym z~najważniejszych źródeł romantyzmu
    europejskiego. Niewątpliwy związek z nim mieli Samuel Taylor
    Coleridge, Thomas De Quincey, Juliusz Słowacki, jednak na nich
    lista się na pewno nie kończy.

  \item Filozofia historii.

  \item Rekonstrukcja gnostycyzmu przez Schellinga?

  \item Relacja człowiek-świat.

  \item Koncepcja filozofii dziejów w zupełnie nowym ujęciu.

  \item Miejsce jednostki w świecie i państwie.

  \item Osiągnięcie absolutu.

  \item Neoheglizm był żywy co najmniej do końca XIX wieku.

  \end{itemize}

\end{frame}
% ##################





% ##################
\begin{frame}
  \frametitle{Można wysunąć zastrzeżenie}


  \begin{figure}

    \centering

    \includegraphics[height=4cm]
    {./PresentationPictures/Caspar_David_Friedrich_Wanderer_above.jpg}
    \includegraphics[height=4cm]
    {./PresentationPictures/Immanuel_Kant_02.jpg}

  \end{figure}



  Kant jakoś nie przypomina stereotypowego romantyka. Posłuchajmy jednak
  jego własnych słów.

  „Dwie rzeczy napełniają umysł coraz to nowym i~wzmagającym się
  podziwem i czcią, im częściej i trwalej się nad nimi
  zastanawiamy: niebo gwiaździste nade mną i prawo moralne we
  mnie.”. Kant \textit{Krytyka praktycznego rozumu}.
  % (Kto czytał \textit{Linię oporu} Dukaja?)

\end{frame}
% ##################





% ##################
\begin{frame}
  \frametitle{Zatrzymajmy się na chwilę na Emanuel Swedenborg}


  \begin{figure}

    \centering

    \includegraphics[height=4cm]
    {./PresentationPictures/Emanuel_Swedenborg.png}


    \caption{Emanuel Swedenborg (1688--1772)}

  \end{figure}


  To chyba najbardziej wpływowy mag jaki tworzył od 1700 roku do
  dziś.

\end{frame}
% ##################





% ##################
\begin{frame}
  \frametitle{Różne dzieła o~Swedenborgu}


  Immanuel Kant \textit{Träume eines Geistersehers} (pl. \textit{Sny o~duchu
    widzącym}), 1766 r.

  Ralph Waldo Emerson \textit{Swedenborg; or, the Mystic}.

  Daisetsu Teitaro Suzuki (1870--1966), \textit{Swedenborg~--
    Budda Północy}.

  Czesław Miłosz co najmilej kilka razy do niego nawiązywał. Wiersz
  \textit{Piekła Swedenborga}, esej w~\textit{Ogrodzie nauk}, wywiad we
  wstępie do książki S.~Toksvig \textit{Emanuel Swedenborg~-- uczony
    i~mistyk}. Według Miłosza „Nawet w~jego opisach
  piekła widać oko inżyniera”.

  Joseph Campbell (1904--1987) \textit{Bohater o~tysiącu twarzy}, 1949~r.
  Ale to temat na inne spotkanie.

\end{frame}
% ##################










% ######################################
\section{Reakcja przeciw Heglowi}
% ######################################



% ##################
\begin{frame}
  \frametitle{Antyheglizm}


  \begin{figure}

    \centering

    \includegraphics[height=4cm]
    {./PresentationPictures/Soren_Kierkegaard_01.jpg}
    \includegraphics[height=4cm]{./PresentationPictures/Karl_Marx_01.jpg}


    \caption{Søren Kierkegaard (1813--1855), Karl Marx (1818--1883)}

  \end{figure}

\end{frame}
% ##################





% ##################
\begin{frame}
  \frametitle{Ich bardziej znane wizerunki}


  \begin{figure}

    \centering

    \includegraphics[height=4cm]
    {./PresentationPictures/Soren_Kierkegaard_02.jpg}
    \includegraphics[height=4cm]{./PresentationPictures/Karl_Marx_02.jpg}


    \caption{Søren Kierkegaard (1813--1855), Karl Marx (1818--1883)}

  \end{figure}

\end{frame}
% ##################





% ##################
\begin{frame}
  \frametitle{Kantowski antyheglizm Schopenhauer i jego następstwa}


  \begin{figure}

    \centering

    \includegraphics[height=4cm]
    {./PresentationPictures/Arthur_Schopenhauer_01.jpg}
    \includegraphics[height=4cm]
    {./PresentationPictures/Friedrich_Nietzsche_01.jpg}


    \caption{Arthur Schopenhauer (1788--1860), Friedrich Nietzsche
      (1844--1900)}

  \end{figure}

\end{frame}
% ##################





% ##################
\begin{frame}
  \frametitle{Ich bardziej znane wizerunki}


  \begin{figure}

    \centering

    \includegraphics[height=4cm]
    {./PresentationPictures/Arthur_Schopenhauer_02.jpg}
    \includegraphics[height=4cm]
    {./PresentationPictures/Friedrich_Nietzsche_02.jpg}


    \caption{Arthur Schopenhauer (1788--1860), Friedrich Nietzsche
      (1844--1900)}

  \end{figure}

\end{frame}
% ##################





% ##################
\begin{frame}
  \frametitle{Po drugiej stronie Atlantyku}


  \begin{figure}

    \centering

    \includegraphics[height=4cm]
    {./PresentationPictures/Ralph_Waldo_Emerson.jpg}


    \caption{Ralph Waldo Emerson (1803--1882)}

  \end{figure}



  Twórca nurtu filozofii nazwanego \textit{transcendetalizmem}. To chyba
  pierwszy nurty filozofii, który stworzyli w~pełnym tego słowa znaczenia
  Amerykanie. Do dziś można znaleźć jego wpływy w~USA.

\end{frame}
% ##################





% ##################
\begin{frame}
  \frametitle{Czy Freud był kantystą?}


  \begin{figure}

    \centering

    \includegraphics[height=4cm]{./PresentationPictures/Sigmund_Freud.jpg}


    \caption{Sigmund Freud (1856--1939)}

  \end{figure}

\end{frame}
% ##################










% ######################################
\section{Filozofia analityczna i kontynentalna}
% ######################################



% ##################
\begin{frame}
  \frametitle{Sytuacja na przełomie XIX i~XX wieku}


  Kant i~Hegel zdefiniowali filozofię XIX stulecie w~stopniu
  niebywałym. Skutkiem tego na przełomie wieku XIX i~XX wieku działali
  ważni myśliciel silnie zarówno odrzucający to kantowsko-heglowską
  myśl jak i~silnie z~niej czerpiący.

  Standardowa wersja historii mówi, że~pojawiły się wtedy dwie szkoły,
  które w różnym stopniu są z nami do dziś. Będę używał terminologi,
  wedle której szkoły te noszą nazwy \textbf{analitycznej}
  i~\textbf{kontynentalnej}.

\end{frame}
% ##################





% ##################
\begin{frame}
  \frametitle{Pierwsze fala}


  \begin{figure}

    \centering

    \includegraphics[height=4cm]{./PresentationPictures/Gottlob_Frege.jpg}
    \includegraphics[height=4cm]{./PresentationPictures/Edmund_Husserl.jpg}


    \caption{Gottlob Frege (1848--1925), Edmund Husserl (1859--1938)}

  \end{figure}


\end{frame}
% ##################





% ##################
\begin{frame}
  \frametitle{Pierwsza fala}


  \begin{figure}

    \centering

    \includegraphics[height=4cm]{./PresentationPictures/G_E_Moore.jpeg}
    \includegraphics[height=4cm]
    {./PresentationPictures/Bertrand_Russell.jpeg}


    \caption{G. E. Moore (1873--1958), Bertrand Russell (1872--1970)}

  \end{figure}

\end{frame}
% ##################





% ##################
\begin{frame}
  \frametitle{Podobieństwa i~różnice}


  \begin{itemize}
    \RaggedRight

  \item Obie szkoły zaczęły się w Niemczech.

  \item U~swych początków obie były mocno związane z~logiką. Jednak
    szkoła kontynentalna w~pewnym momencie

  \item Filozofia analityczna zdominowała do lat 60-tych świat
    anglosaski, kontynentalna Francję i Niemcy Zachodnie. Jej droga do
    krajów bloku komunistycznego była zawikłana, podobnie jak na
    południe Europy. Co nie znaczy, że nie było „kontynentalnych”
    anglików, czy „analitycznych” francuzów.

  \item Plany ojców w~obu przypadkach, zostały podważone przez ich
    najwybitniejszych uczniów.

  \item Jeżeli słyszał ktoś, że~coś głupiego zrobiono
    w~średniowiecznej filozofii, oni co najmniej temu dorównali.

  \end{itemize}

\end{frame}
% ##################





% ##################
\begin{frame}
  \frametitle{Szkoła analityczna}


  \begin{itemize}

  \item Początek związany jest z~badaniami logicznymi, głównie pracami
    Fregego i Russela (Whitehead poszedł swoją drogą).

  \item Po ogromnym rozwoju logiki w XIX wieku, powstał pomysł by
    zaaplikowanie jej aparat do analizy innych problemów, w~tym
    filozoficznych.

  \item W naturalny sposób doprowadziło to do wielkich badań
    na~logiczną strukturą języka.

  \item Filozofie tej szkoły mają zwyczaj rozbijać wszystkie problemy
    na~czynniki pierwsze, stąd zasłużone miano filozofii analitycznej.

  \item „Philosophical problems should not be solved, but dissolved”,
    Ludwig Wittgenstein.

  \item Ich konkretny wpływ na~filozofię i kulturę jest bardzo trudny
    do prześledzenia. Od tytułu książki jednego z~członków tej szkoły
    Gilbert Ryle wziął swój tytuł film anime \textit{Ghost in the
      Shell}.

  \end{itemize}

\end{frame}
% ##################





% ##################
\begin{frame}
  \frametitle{Skupienie na języku}


  \begin{figure}

    \centering

    \includegraphics[height=4cm]
    {./PresentationPictures/Gottlob_Frege.jpg}


    \caption{Gottlob Frege (1848--1925)}

  \end{figure}



  Obie szkoły poważnie zajęły się problemami języków, zarówno
  naturalnych, jak różnego rodzaju języków formalnych. Według Michael
  Dummettem pierwszy tzw. \textbf{zwrotu lingwistycznego} (ang.
  \textit{linguistic turn}) dokonał Gottlob Frege w~swoim dziele
  \textit{Begriffsschrift}.

  Alisdair MacIntyre w~swoim dziel \textit{Dziedzictwo cnoty} uważa
  też, że~książka G.E. Moore poświęconą etyce, za ważny etap rozwoju,
  albo upadku, etyki europejskiej.

\end{frame}
% ##################





% ##################
\begin{frame}
  \frametitle{Szkoła kontynentalna}


  \begin{figure}

    \centering

    % \includegraphics[height=4cm]{./PresentationPictures/Gottlob_Frege.jpg}
    \includegraphics[height=4cm]{./PresentationPictures/Edmund_Husserl.jpg}


    \caption{Edmund Husserl (1859--1938)}

  \end{figure}



  W moim mniemaniu ta szkoła zaczyna się wraz z~fenomenologią Edmunda
  Husserla. Pierwszym ważnym dziełem Husserla są \textit{Logische
    Untersuchungen} (pl. \textit{Dociekania logiczne}), dwa tomy,
  1900, 1901.

  Filozofia Husserla była inspirowana Kartezjusza, mówienie o~wpływach
  Kanta nie jest bezpodstawne. Hegel? Nie wiem.

\end{frame}
% ##################





% ##################
\begin{frame}
  \frametitle{Czym jest fenomenologia?}


  „Doświadczając, wyobrażając sobie coś, myśląc czy mówiąc o~czymś
  napotykamy różnego rodzaju przedmioty. Jak to się dzieje,
  że~napotykają nas one w~takim właśnie kształcie: jako rzeczy,
  ludzie, zdarzenia, wartości itd.? Oto pytanie fenomenologii. Metoda
  fenomenologiczna jest więc analizą i~opisem samego procesu zjawiania
  się czegoś, badaniem stosunku różnych perspektyw i~sposobów
  prezentacji do~tego, co w~nich dane, nie~analizą przedmiotu, lecz
  analizą projektu wszelkiego możliwego spotkania z~przedmiotem (Jan
  Potocka). [\ldots] Inaczej mówiąc: odkrycie \textit{Badań logicznych}
  polega na ujawnieniu różnicy między tym, co się zjawia, a~samym
  zjawianiem się.” Krzysztof Michalski \textit{Heidegger i~filozofia
    współczesna}.

\end{frame}
% ##################





% ##################
\begin{frame}
  \frametitle{Fenomenologia}


  \begin{itemize}

  \item Fenomenologia to zasadniczo filozofia zajmująca się
    „zjawiskami takimi jakie się jawią” poprzez przeprowadzenie
    odpowiednich redukcji oraz poszukiwanie naoczności (intuicji)
    zjawisk.

  \item Krytyka ówczesnego racjonalizmu i~„Z powrotem do rzeczy!”.

  \item Fenomenologia Husserla zrodziła współczesny egzystencjalizm
    (Heidegger).

  \item Czy program Husserla się udał? Raczej nie.

  \item Wydaje mi się, że~powrót do rzeczy i~idealizm nie pasują do
    siebie. „Fenomenologia to system filozoficzny w którym najbardziej
    liczą się nie rzeczy, lecz wrażenia rzeczy.”. Tak skrytykował
    fenomenologię amerykański dziennikarz Michael Voris.

  \end{itemize}



  Filozofia Husserla miała być krytyką i~reformą europejskiego
  racjonalizmu, lecz czy była czymś innym, niż logiczną jej
  kontynuacją?

\end{frame}
% ##################










% ######################################
\section{Dwaj niszczyciele marzeń}
% ######################################



% ##################
\begin{frame}
  \frametitle{Druga fala, dwaj bohaterowie}


  \begin{figure}

    \centering

    \includegraphics[height=4cm]
    {./PresentationPictures/Martin_Heidegger.jpg}
    \includegraphics[height=4cm]
    {./PresentationPictures/Ludwig_Wittgenstein.jpg}


    \caption{Martin Heidegger (1889--1976), Ludwig Wittgenstein
      (1889--1951)}

  \end{figure}

\end{frame}
% ##################





% ##################
\begin{frame}
  \frametitle{Martin Heidegger}

  Krótki i~może niesprawiedliwy życiorys Heidggera.
  \begin{itemize}

  \item Urodził się.

  \item Był nazistą.

  \item Umarł.

  \end{itemize}

  Życiorys filozoficzny.
  \begin{itemize}

  \item Student Husserla.

  \item Ustanowił dwudziestowieczny egzystencjalizm
    i~hermeneutykę.

  \item Miał obsesje na~punkcie etymologii i~pisał w~niezwykle trudny
    sposób.

  \item Bardzo Niemiecki, przynajmniej w~pewnym okresie.

  \item Dokonał „zwrotu” w~swej filozofii. Nie do końca jasne jakiego.

  \item Główne dzieło: \textit{Seit und Zeit} (pl. \textit{Bycie i~czas}),
    1927.

  \end{itemize}

\end{frame}
% ##################





% ##################
\begin{frame}
  \frametitle{Styl Heideggera}


  Heidegger używał bardzo wielu neologizmów, zresztą tak samo jak Husserl,
  lub zwykłych słów niemieckich w~zupełnie innym sensie. Przykłady:
  „Dasein”, „Lichtung”. Według Nonsensopedii „Lichtun” to \textit{Bycie
    w~prześwicie}, co uważam za wspaniałe tłumaczenie.

  Jego styl jest trudny/nieludzko~zagmatwany/niejasny/
  zagadkowy/bezsensowny/maskujący brak treści (niepotrzebne
  skreślić). „Wszechobecność wyistaczania ukazuje nam się
  najwyraźniej wówczas, gdy zauważymy, że~również i~właśnie
  odistaczanie określone jest poprzez owo, tajemnicze niekiedy
  wyistaczanie.” Przykład ten pochodzi z~dzieła \textit{Zeit und Seit} (ale
  wymyślił).

\end{frame}
% ##################





% ##################
\begin{frame}
  \frametitle{\textit{Seit und Zeit}}


  \begin{itemize}

  \item Jedna z~dwóch najważniejszych książek filozoficznych po
    roku 1925.

  \item Światowy bestseller. M.in. sześć tłumaczeń na~język
    japoński do 1970 roku.

  \item Zadedykowane Husserlowi. Dedykacja usunięta w~wydaniu
    z~1939 r. Data raczej nie jest przypadkowa.

  \item Tekst założycielski współczesnego egzystencjalizmu.

  \item Miała być pierwszą częścią dwutomowego dzieła. Drugi tom
    nigdy się nie ukazał.

  \end{itemize}

  \textbf{Ciekawostka.} Światowa twarz egzystencjalizmu Jean-Paul Sartre
  wydał w 1943 r. książkę \textit{L'étre et le néant} (pl. \textit{Bycie
    i~nicość}). To nie jest przypadek.

\end{frame}
% ##################





% ##################
\begin{frame}
  \frametitle{\textit{Seit und Zeit}}


  Centralne dla całego dzieła wydaje się być pytanie o~sens bycia.

  „Czy w dzisiejszych czasach dysponujemy odpowiedzią na~pytanie,
  co właściwie rozumiemy pod słowem <<bytujący>>? W~żadnym razie
  nie. Dlatego trzeba na nowo postawić \textit{pytanie o~sens
    bycia}. Czy~jednak dzisiaj fakt, że~nie rozumiemy wyrażenia
  <<bycie>> jest naszym jedynym kłopotem? Bynajmniej. Dlatego też
  przede wszystkim musimy na nowo obudzić zrozumienie dla sensu
  tego pytania. Zamiarem niniejszej rozprawy jest konkretne
  opracowanie pytania o~sens <<bycia>>. Prowizorycznym celem zaś
  będzie interpretacja \textit{czasu} jako możliwego horyzontu
  wszelkiego w ogóle rozumienia bycia.”

\end{frame}
% ##################





% ##################
\begin{frame}
  \frametitle{\textit{Seit und Zeit}}


  Nie jest to pierwsze dzieło egzystencjalizmu, jednak stanowi przełom
  dla tego nurtu filozofii.

  Aby zbadać co to jest bycie należy zbadać jakiś obiekt
  który jest. Najprościej zbadać samego siebie. Oczywiście, milcząco
  zakładamy, że istniejemy ;). „Bycie w świecie” i~bycie w czasie to dwa
  typy bycia na których Heidegger skupił się w szczególności.

  Tom II nie został napisany bowiem Heidegger zmienił swoje
  poglądy filozoficzne. Dokonał zwrotu, cokolwiek ten „zwrot” znaczy.

\end{frame}
% ##################





% ##################
\begin{frame}
  \frametitle{Ludwig Wittgenstein, życiorys}


  \begin{itemize}

  \item Urodził się w Wiedniu, w Cesarstwie Austro-Węgierskim.

  \item Jego rodzina ze~strony ojca miała żydowskie pochodzenie,
    ze~strony matki czesko-słoweńskie

  \item Był jednym z~dziewięciorga dzieci. Jego ojciec praktycznie
    miał na własność przemysł stalowym w Austro-Węgrzech.

  \item Zwykle rozróżnia się wczesnego i~późnego Wittgensteina.

  \item Początkowo pragnął zostać inżynierem aeronautyki.

  \item Nauka matematyki potrzebnej w~aeronautce doprowadziło go do pytań
    o~jej podstawy i~tym samy do filozofii matematyki.

  \item Problemy z~podstawami matematyki prowadzą go do prac
    Fregego i Russela. Poznaje i zaprzyjaźnia się z~oboma.

  \item Studiuje u~Russela i~Moora.

  \end{itemize}

\end{frame}
% ##################





% ##################
\begin{frame}
  \frametitle{Wczesny Wittgenstein}


  \begin{itemize}

  \item Jako jedyny, chyba, poważny myśliciel w~pierwszej połowie XX w.
    który czytał i brał na serio Schopenhauera.

  \item 1914 wybucha I Wojna Światowa, Wittgenstein zaciąga się do
    wojsk cesarstwa Austro-Węgierskiego.

  \item W~przerwach od działań wojennych tworzy \textit{Tractatus
      logico-philosophicus} i~wiele notatek filozoficznych, zostaje
    też dwukrotnie wyznaczony do medalu za odwagę.

  \item Wraz z~publikacją \textit{Tractatus logico-philosophicus}
    dochodzi do wniosku, że rozwiązał już wszystko co \textit{dało się
      rozwiązać} i~porzuca filozofię. Tu kończy się wczesny
    Wittgenstein.

  \end{itemize}

\end{frame}
% ##################





% ##################
\begin{frame}
  \frametitle{\textit{Tractatus logico-philosophicus}}

  \begin{itemize}

  \item Napisany po niemiecku jako \textit{Logisch~-- philosophische
      Abhandlung}. Pierwotnym językiem twórczości Wittgensteina pozostał
    zawsze wysokiej klasy niemiecki. Zapewne niemiecki w~wersji
    austriackiej, różniący się sporo od tego Niemieckiego niemieckiego

  \item Pierwszy raz wydany w 1921 r. Tytuł łaciński pod którym obecnie
    jest znany zaproponował G.E. Moore dla wydania angielskiego z~1922~r.

  \item Książka bardzo małej objętości, w polskim wydaniu ma około 80 stron.

  \item Zawiera m.in dyskusje nad problemami innych filozofów
    np.~z~Kanta, rozważania lingwistyczne, matematyczne,
    egzystencjalne i~teologiczne.

  \item Podobno bardzo niewielu przeczytało go w całości.

  \end{itemize}

\end{frame}
% ##################





% ##################
\begin{frame}
  \frametitle{\textit{Tractatus logico-philosophicus}}


  \begin{itemize}

  \item Oddziaływanie dzieła było ogromne. Stał się jedną z~ulubionych
    lektur pozytywistów logicznych m.in Koła Wiedeńskiego. Wiele osób
    wciąż przyjmuje interpretację \textit{Tractatus\ldots} sformułowaną przez
    Koło Wiedeńskie.

  \item Odegrał również wielką rolę w filozofii języka.

  \end{itemize}

\end{frame}
% ##################





% ##################
\begin{frame}
  \frametitle{\textit{Tractatus logico-philosophicus}}


  Jego początek.

  1. Świat jest wszystkim co jest faktem. \\
  1.1 Świat jest ogółem faktów nie rzeczy. \\
  1.11 Świat jest wyznaczony przez fakty oraz przez to, że są to
  \textit{wszystkie} fakty. \\
  1.12. Ogół faktów wyznacza bowiem, co jest faktem, a także wszystko, co
  faktem nie jest. \\
  1.13. Światem są fakty w przestrzeni logicznej. \\
  1.2. Świat rozpada się na fakty. \\
  1.21. Jedno może być faktem lub nie być, a~wszystko inne pozostanie
  takie samo. \\
  2. To, co jest faktem -- fakt -- jest istnieniem stanów rzecz.

\end{frame}
% ##################





% ##################
\begin{frame}
  \frametitle{\textit{Traktatus logico-philosophicus}}

  Jego zakończenie.

  6.54. Tezy moje wnoszą jasność przez to, że kto mnie rozumie, rozpozna je
  w~końcu jako niedorzeczne; gdy przez nie -- po nich -- wyjdzie ponad nie.
  (Musi niejako odrzucić drabinę, uprzednio po niej~się wspiąwszy.) Musi
  te tezy przezwyciężyć, wtedy świat przedstawi mu się właściwie. \\
  7. O czym nie można mówić o tym trzeba milczeć.

\end{frame}
% ##################





% ##################
\begin{frame}
  \frametitle{Dla przestrogi}


  Dwie wypowiedzi Wittgensteina.


  „Otóż chciałem napisać, że moja praca składa się z~dwu części:
  z~tego co w~niej napisałem, oraz tego wszystkiego, czego nie napisałem.
  I~właśnie ta druga część jest ważna.” Z listu do wydawcy.

  „Natomiast \textit{prawdziwość} komunikowanych tu myśli zdaje
  mi się niepodważalna i~definitywna. Sądzę więc, że w istotnych
  punktach problemy zostały rozwiązane ostatecznie. A jeżeli się
  tu nie mylę, to wartością niniejszej pracy jest -- po wtóre --
  to, że~widać z~niej, jak mało się przez ich rozwiązanie
  osiągnęło.” Zakończenie wstępu do \textit{Tractatusa\ldots}.

\end{frame}
% ##################





% ##################
\begin{frame}
  \frametitle{\textit{Traktatus logico-philosophicus}}


  Najlepszy opis tej książki jaki mogę podać, to taki, że jest to traktat
  analizujący logiczną treści świata. Próbuje zbadać co leży w~zasięgu
  nośnika prawd logicznych, czyli języka, a tego co poza język wykracza.
  „O~czym nie można mówić, o tym trzeba milczeć.”

  Dla całego projektu \textit{Traktatus\ldots} kluczowym jest założenie, że
  język odbija w sobie strukturę świata, która jest strukturą logiczną.
  Jest to jednak temat trudny i~nie pozbawiony kontrowersji.

\end{frame}
% ##################





% ##################
\begin{frame}
  \frametitle{Późny Wittgenstein}


  \begin{itemize}

  \item Po dziesięciu latach zajmowania najprzeróżniejszymi
    zajęciami, wysłuchuje w Wiedniu wykładu L. E. J. Brouwera na temat
    filozofii matematyki. W wyniku tego wraca do filozofii.

  \item 1929 publikuje swoją drugą i~ostatnią ogłoszoną za życia
    pracę.

  \item Jeśli wierzyć Wikipedii to oprócz tego jego opublikowany
    dorobek zawiera jeszcze: \textit{jedną} recenzję książki
    i~słownik dla dzieci.

  \item Uzyskuje tytuł naukowy w Cambridge składają \textit{Tractatusa\ldots}
    (wówczas już uznanego za wybitne dzieło filozoficzne) jako
    swoją pracę dyplomową.

  \item Od tego czasu naucza z~przerwami na angielskich uniwersytetach
    i~rozwija zupełnie nową filozofię.

  \item Umiera w 1951 roku.

  \end{itemize}

\end{frame}
% ##################





% ##################
\begin{frame}
  \frametitle{Późny Wittgenstein}


  \begin{itemize}

  \item 1953 ukazują się \textit{Philosophische Untersuchungen}
    (pl.~\textit{Dociekania filozoficzne}), prawdopodobnie druga
    najważniejsza książka filozoficzna która ukazała~się po 1925 r.

  \item Większość prac Wittgensteina ukazało się po jego śmierci.

  \end{itemize}

\end{frame}
% ##################





% ##################
\begin{frame}
  \frametitle{\textit{Dociekania filozoficzne}}


  Dzieło to podważa koncepcje języka jako odbicia rzeczywistości,
  zastępuję je pojęciem \textbf{gry językowej}. Język w~tym sensie jest
  ludzką aktywnością, z~której to aktywności czerpie on swój sens.

  Końcowy efekt jest jednak taki sam: problemy
  filozoficzne/etyczne/teologiczne etc., są nierozwiązywalne. Jednak
  również tutaj są ogromne niejasności i~kontrowersje.

\end{frame}
% ##################










% ######################################
\section{Zamiast zakończenia}
% ######################################



% ##################
\begin{frame}
  \frametitle{Krótki przegląd pozostałych filozofów i~myślicieli}


  \begin{figure}

    \centering

    \includegraphics[height=4cm]{./PresentationPictures/Max_Scheler.jpg}


    \caption{Max Scheler (1874--1928)}

  \end{figure}



  Przedstawiciel fenomenologi nienależący do~ścisłej szkoły Husserela.

\end{frame}
% ##################





% ##################
\begin{frame}
  \frametitle{Krótki przegląd pozostałych filozofów i~myślicieli}


  \begin{figure}

    \centering

    \includegraphics[height=4cm]{./PresentationPictures/Jeremy_Bentham.jpg}
    \includegraphics[height=4cm]
    {./PresentationPictures/John_Stuart_Mill.jpg}


    \caption{Jeremy Bentham (1748--1832), John Stuart Mill (1806--1873)}

  \end{figure}



  Utylitaryzm.

\end{frame}
% ##################





% ##################
\begin{frame}
  \frametitle{Krótki przegląd pozostałych filozofów i~myślicieli}



  \begin{figure}

    \centering

    \includegraphics[height=4cm]{./PresentationPictures/Herbert_Spencer.jpg}


    \caption{Herbert Spencer (1820--1903)}

  \end{figure}


  Zapomniany Wiktoriański gigant.

\end{frame}
% ##################





% ##################
\begin{frame}
  \frametitle{Krótki przegląd pozostałych filozofów i~myślicieli}


  \begin{figure}

    \centering

    \includegraphics[height=4cm]{./PresentationPictures/Franz_Brentano.jpg}


    \caption{Franz Brentano (1838--1917)}

  \end{figure}



  Wielki austro-węgierski nauczyciel.

\end{frame}
% ##################





% ##################
\begin{frame}
  \frametitle{Krótki przegląd pozostałych filozofów i~myślicieli}


  \begin{figure}

    \centering

    \includegraphics[height=4cm]{./PresentationPictures/August_Comte.jpg}
    \includegraphics[height=4cm]
    {./PresentationPictures/Charles_Sanders_Peirce.jpg}
    \includegraphics[height=4cm]
    {./PresentationPictures/William_James.jpg}


    \caption{Auguste Comte (1798--1857), Charles Sanders Peirce (1839--1914),
    William James (1842 - 1910)}

  \end{figure}



  Pozytywizm i~pragmatyzm.

\end{frame}
% ##################





% ##################
\begin{frame}
  \frametitle{Krótki przegląd pozostałych filozofów i~myślicieli}


  \begin{figure}

    \centering

    \includegraphics[height=4cm]
    {./PresentationPictures/Henry_David_Thoreau.jpg}


    \caption{Henry David Thoreau (1748--1832)}

  \end{figure}



  Amerykański transcendentalizm.

\end{frame}
% ##################





% ##################
\begin{frame}
  \frametitle{Krótki przegląd pozostałych filozofów i~myślicieli}


  \begin{figure}

    \centering

    \includegraphics[height=3.7cm]
    {./PresentationPictures/Ludwig_von_Mises.jpg}
    \includegraphics[height=3.7cm]
    {./PresentationPictures/Hans_Hermann_Hoppe.jpg}
    \includegraphics[height=3.7cm]
    {./PresentationPictures/Bronislaw_Malinowski.jpg}


    \caption{Ludwig von Mises (1881--1973), Hans-Hermann Hoppe (ur. 1949),
      Bronisław Malinowski (1884--1942)}

  \end{figure}



  Ekonomia i~antropologia.

\end{frame}
% ##################





% ##################
\begin{frame}
  \frametitle{Krótki przegląd pozostałych filozofów i~myślicieli}


  \begin{figure}

    \centering

    \includegraphics[height=4cm]
    {./PresentationPictures/Jean_Paul_Sartre.jpg}
    \includegraphics[height=4cm]
    {./PresentationPictures/Maurice_Merleau_Ponty.jpg}
    \includegraphics[height=4cm]
    {./PresentationPictures/Simone_de_Beauvoir.jpg}


    \caption{Jean-Paul Sartre (1905--1980), Maurice
      Merleau-Ponty (1908--1961), Simone de Beauvoir (1908--1986)}

    \end{figure}



  Przedstawiciele francuskiej szkoła kontynentalnej.

\end{frame}
% ##################





% ##################
\begin{frame}
  \frametitle{Krótki przegląd pozostałych filozofów i~myślicieli}


  \begin{figure}

    \centering

    \includegraphics[height=4cm]{./PresentationPictures/Rudolf_Carnap.jpg}
    \includegraphics[height=4cm]{./PresentationPictures/Otto_Neurath.jpg}


    \caption{Rudolf Carnap (1891--1970), Otto Neurath (1882--1945)}

  \end{figure}



  Pozytywizm logiczny $\subset$ Koło Wiedeńskie

\end{frame}
% ##################





% ##################
\begin{frame}
  \frametitle{Krótki przegląd pozostałych filozofów XX w.}


  \begin{figure}

    \centering

    \includegraphics[height=4cm]
    {./PresentationPictures/Elizabeth_Anscombe.jpg}
    \includegraphics[height=4cm]{./PresentationPictures/Karl_Poper.jpg}


    \caption{Gertrude Elizabeth Margaret Anscombe (1919--2001),
      Karl Popper (1902--1994)}

  \end{figure}



  Filozofia analityczna po Wittgensteinie, filozofia nauki i~społeczeństwa.

\end{frame}
% ##################





% ##################
\begin{frame}
  \frametitle{Krótki przegląd pozostałych filozofów i~myślicieli}


  \begin{figure}

    \centering

    \includegraphics[height=4cm]{./PresentationPictures/Maurice_Blondel.jpg}


    \caption{Maurice Blondel (1861--1941)}

  \end{figure}



    Zwany „francuskim Heglem”.

\end{frame}
% ##################





% ##################
\begin{frame}
  \frametitle{Krótki przegląd pozostałych filozofów i~myślicieli}


  \begin{figure}

    \centering

    \includegraphics[height=4cm]
    {./PresentationPictures/Henri_Louis_Bergson.jpg}
    \includegraphics[height=4cm]
    {./PresentationPictures/Alfred_North_Whitehead.jpg}


    \caption{Henri-Louis Bergson (1859--1941), Alfred North
      Whitehead (1861--1947)}

  \end{figure}



  Filozofowie procesu.

\end{frame}
% ##################





% ##################
\begin{frame}
  \frametitle{Krótki przegląd pozostałych filozofów i~myślicieli}


  \begin{figure}
    \centering

    \includegraphics[height=4cm]{./PresentationPictures/John_Dewey.jpg}
    \includegraphics[height=4cm]
    {./PresentationPictures/Willard_Van_Orman_Quine.jpg}


    \caption{John Dewey (1859--1961), Willard Van Orman Quine (1908--2000)}

  \end{figure}



  Filozofia amerykańska.

\end{frame}
% ##################





% ##################
\begin{frame}
  \frametitle{Krótki przegląd pozostałych filozofów i~myślicieli}


  \begin{figure}

    \centering

    \includegraphics[height=4cm]{./PresentationPictures/Ayn_rand.jpg}
    \includegraphics[height=4cm]{./PresentationPictures/John_Rawls.jpg}
    \includegraphics[height=4cm]{./PresentationPictures/Robert_Nozik.jpg}


    \caption{Ayn Rand (1905--1982), John Rawls (1921--2002),
      Robert Nozick (1938--2002)}

  \end{figure}



  Filozofowie polityki.

\end{frame}
% ##################





% ##################
\begin{frame}
  \frametitle{Krótki przegląd pozostałych filozofów i~myślicieli}


  \begin{figure}

    \centering

    \includegraphics[height=4cm]
    {./PresentationPictures/Hans_Georg_Gadamer.jpg}
    \includegraphics[height=4cm]
    {./PresentationPictures/Claude_Levi_Strauss.jpg}
    \includegraphics[height=4cm]{./PresentationPictures/Richard_Rorty.jpg}


    \caption{Hans-Georg Gadamer (1900--2002), Claude Lévi-Strauss
      (1908--2009), Richard Rorty (1931--2007)}

  \end{figure}



  Hermeneutyka, strukturalizm i neopragmatyzm.

\end{frame}
% ##################





% ##################
\begin{frame}
  \frametitle{Krótki przegląd pozostałych filozofów i~myślicieli}


  \begin{figure}

    \centering

    \includegraphics[height=4cm]{./PresentationPictures/Michel_Foucault.jpg}


    \caption{Michel Foucault (1926--1984)}

  \end{figure}



  Postmodernizm i~poststrukturalizm.

\end{frame}
% ##################





% ##################
\begin{frame}
  \frametitle{Krótki przegląd pozostałych filozofów i~myślicieli}


  \begin{figure}

    \centering

    \includegraphics[height=4cm]{./PresentationPictures/Leo_Strauss.jpg}
    \includegraphics[height=4cm]
    {./PresentationPictures/Hannah_Ardent.jpeg}
    \includegraphics[height=4cm]
    {./PresentationPictures/Alisdair_MacIntyre.jpg}


    \caption{Leo Strauss (1899--1973), Hannah Arendt (1906--1975),
      Alasdair MacIntyre (ur. 1929)}

  \end{figure}



  Premoderniści.

\end{frame}
% ##################





% ##################
\begin{frame}
  \frametitle{I na ostatek}


  \begin{figure}

    \centering

    \includegraphics[height=3.7cm]
    {./PresentationPictures/Jacques_Derrida_01.jpg}
    \includegraphics[height=3.7cm]
    {./PresentationPictures/Jacques_Derrida_02.jpg}
    \includegraphics[height=3.7cm]
    {./PresentationPictures/Jacques_Derrida_03.jpg}


    \caption{Jacques Derrida (1930--2004)}

  \end{figure}



  Jeden z~głośniejszych filozofów drugiej połowy XXI wieku.

\end{frame}
% ##################










% ##################
\EndingSlide{Dziękuję! Pytania?}
% ##################










% ##################
\begin{frame}
  \frametitle{Dla dociekliwych}


  Proszę nie podejrzewać, że ja to wszystko przeczytałem.
  \begin{itemize}

  \item Frederick Copleston, \textit{Historia filozofii};

  \item Władysław Tatarkiewicz, \textit{Historia filozofii};

  \item Tadeusz Gadacz \textit{Historia filozofii XX wieku. Nurty}

  \item Lawrence Cahoone, \textit{Modern Intellectual Tradition:
      From Descartes to Derrida} (do zakupienia na stronie Great
    Courses)

  \item Kanał na YouTubie Gregor’ego B. Sadler;

  \item Anna Burzyńska, Michał Paweł Markowski, \textit{Teorie
      literatury XX wieku};

  \item Leszek Kołakowski, \textit{Główne nurty marksizmu};

  \item Terence Hawkes, \textit{Strukturalizm i semiotyka}.

  \end{itemize}

\end{frame}
% ##################





% ##################
\begin{frame}
  \frametitle{Dla dociekliwych}


  \begin{itemize}

  \item Krzysztof Michalski: \textit{Heidegger i~filozofia
      współczesna};

  \item Leszek Kołakowski, \textit{O~co nas pytają wielcy
    filozofowie};

  \item Julian Young, \textit{Heidegger, filozofia, nazizm};

  \item G. E. M. Anscombe, P. T. Geach: \textit{Trzej
      filozofowie};

  \item Willard Van Orman Quine, \textit{Logika matematyczna};

  \item Ian Shapiro, \textit{Moral Foundations of Politics}
    (dostępne na YouTubie);

  \item S. Toksvig, \textit{Emanuel Swedenborg -- uczony i~mistyk}.

  \end{itemize}

\end{frame}
% ##################










% ############################

% Koniec dokumentu
\end{document}
