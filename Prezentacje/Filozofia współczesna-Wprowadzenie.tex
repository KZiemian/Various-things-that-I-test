\begin{frame}
\frametitle{Ale, natura ludzka jest taka\ldots}
\begin{block}{Antyheglizm heglowski}
\begin{figure}
\centering
\includegraphics[height=1.6in, width=1.4in]{SK}
\includegraphics[height=1.6in, width=1.2in]{KM.jpg}
\uncover<2->{\caption{Søren Kierkegaard (1813 - 1855), Karl Marx (1818 - 1883).}}
\end{figure}
\end{block}
\end{frame}

\begin{frame}
\frametitle{Ich bardziej znane wizerunki}
\begin{block}{Antyheglizm heglowski}
\begin{figure}
\centering
\includegraphics[height=1.6in, width=1.2in]{SK1}
\includegraphics[height=1.6in, width=1.2in]{KM1}
\caption{Søren Kierkegaard (1813 - 1855), Karl Marx (1818 - 1883).}
\end{figure}
\end{block}
\end{frame}

\begin{frame}
\frametitle{Antyheglizm kantowski Schopenhauer i jego następstwa}
\begin{block}{}
\begin{figure}
\centering
\includegraphics[height=1.6in, width=1.4in]{AS}
\includegraphics[height=1.6in, width=1.2in]{FN.jpg}
\pause
\caption{Arthur Schopenhauer (1788 - 1860), Friedrich Nietzsche (1844 - 1900).}
\end{figure}
\end{block}
\end{frame}

\begin{frame}
\frametitle{I ich bardziej znane wizerunki}
\begin{block}{}
\begin{figure}
\centering
\includegraphics[height=1.6in, width=1.4in]{AS1}
\includegraphics[height=1.6in, width=1.4in]{FN1}
\caption{Arthur Schopenhauer (1788 - 1860), Friedrich Nietzsche (1844 - 1900).}
\end{figure}
\end{block}
\end{frame}

\begin{frame}
\frametitle{A że wpływowi Kanta nie ma końca}
\begin{block}{Ameryka i Emerson}
\begin{figure}
\centering
\includegraphics[height=1.6in, width=1.1in]{RE}
\caption{Ralph Waldo Emerson (1803 - 1882).}
\end{figure}
\end{block}

\begin{block}{}
Twórca pierwszej amerykańskiej filozofii zwanej \emph{transcendentalizmem}. Jej znajomość jest absolutnie niezbędna przy studiowaniu kultury Stanów Zjednoczonych.
\end{block}
\end{frame}


\begin{frame}
\frametitle{Zanim przejdziemy do zakończenia, mała odskocznia}
\begin{block}{Czy Freud był kantystą?}
\begin{figure}
\centering
\includegraphics[height=1.6in, width=1.2in]{SF}
\caption{Sigmund Freud  (1856 - 1939).}
\end{figure}
\end{block}
\end{frame}

\begin{frame}
\frametitle{Zanim przejdziemy do zakończenia, mała odskocznia}

\begin{block}{Kto jeszcze?}
\begin{figure}
\centering
\includegraphics[height=1.6in, width=1.9in]{LvM}
\includegraphics[height=1.6in, width=1.2in]{HHH}
\includegraphics[height=1.6in, width=1.2in]{BM}
\uncover<2->{\caption{Ludwig von Mises  (1881 - 1973), Hans-Hermann Hoppe\newline (1949 - ), Bronisław Malinowski (1884 - 1942).}}
\end{figure}
\end{block}
\end{frame}

\begin{frame}
\frametitle{Krótki przegląd pozostałych filozofów XIX w.}
\begin{block}{Utylitaryzm i pozytywizm}
\begin{figure}
\centering
\includegraphics[height=1.6in, width=1.2in]{JB}
\includegraphics[height=1.6in, width=1.2in]{JM}
\includegraphics[height=1.6in, width=1.2in]{AC}
\caption{Jeremy Bentham (1748 - 1832), John Stuart Mill (1806 \newline- 1873), Auguste Comte (1798 - 1857).}
\end{figure}
\end{block}
\end{frame}

\begin{frame}
\frametitle{Krótki przegląd pozostałych filozofów XIX w.}
\begin{block}{Ameryka: transcendentalizm i pragmatyzm}
\begin{figure}
\centering
\includegraphics[height=1.6in, width=1.2in]{HT}
\includegraphics[height=1.6in, width=1.2in]{CP}
\includegraphics[height=1.6in, width=1.2in]{WJ}
\caption{Henry David Thoreau (1748 - 1832), Charles Sanders Peirce (1839 - 1914), William James (1842 - 1910).}
\end{figure}
\end{block}
\end{frame}

\begin{frame}
\frametitle{Krótki przegląd pozostałych filozofów XIX w.}
\begin{block}{Wielki nauczycie}
\begin{figure}
\centering
\includegraphics[height=1.6in, width=1.2in]{FB}
\caption{Franz Brentano (1838 - 1917).}
\end{figure}
\end{block}
\end{frame}

\begin{frame}{Na zakończenie}
\frametitle{Krótki przegląd pozostałych filozofów XIX w.}
\begin{block}{Zapomniany Wiktoriański gigant}
\begin{figure}
\centering
\includegraphics[height=1.6in, width=1.1in]{HS}
\caption{Herbert Spencer (1820 - 1903).}
\end{figure}
\end{block}
\end{frame}

%\begin{frame}
%\begin{block}
%\centering
%\LARGE{Koniec części pierwszej.}
%\end{block}
%\end{frame}

\begin{frame}
\frametitle{Pole walki ustawione, możemy przejść do rzeczy}
\begin{block}{Wielki dwudziestowieczny rozłam}
\begin{itemize}
\item Kant i~Hegel zdefiniowali filozofię XIX stulecie w~stopniu niebywałym.
\item Na początku XX wieku obecni są silnie zarówno filozofowie odrzucający to dziedzictwo jak i~silnie z~niego czerpiący.
\item W powietrzu można było wyczuć oczekiwanie na przełom który obali psychologizm (należy zbadać jego związek z pozytywizmem).
\item W tej atmosferze zrodziły się wtedy dwie wielkie szkoły filozoficzne, które zdominowały wiek XX.
\item Przynajmniej w~świecie anglojęzycznym, są znane jako filozofia analityczna i kontynentalna. Ja też będę się posługiwał tymi nazwami.
\end{itemize}
\end{block}
\end{frame}

\begin{frame}
\frametitle{Szkic stron konfliktu}
\begin{block}{Podobieństwa i różnice}
\begin{itemize}
\item Obie zaczęły się w Niemczech.
\item Ich punktem wyjścia była logika. Jednak szkoła kontynentalna znalazła szybko inne tematy zainteresowań.
\item Filozofia analityczna zdominowała do lat 60-tych świat anglosaski, kontynentalna -- całą resztę. Co nie znaczy, że nie było kontynentalnych anglików, czy analitycznych francuzów.
\item Plany ojców, w~obu przypadkach, zostały zniszczone przez ich najwybitniejszych uczniów.
\item Jeżeli słyszał ktoś, że~coś głupiego zrobiono w~średniowiecznej filozofii, oni co najmniej temu dorównali.
\end{itemize}
\end{block}
\end{frame}

\begin{frame}
\frametitle{Szkic stron konfliktu}
\begin{block}{Język}
Rola języka w obu filozofiach jest niebagatelna, jednak ciężka do opisania. Pierwszym który dokonał ,,zwrotu lingwistycznego''\linebreak (ang. linguistic turn) był G. Frege w swoim pierwszym dziele ,,Begriffsschrift'' (za M. Dummettem).
\end{block}
\end{frame}

\begin{frame}
\frametitle{Osoby dramatu}
\begin{block}{Pierwsze fala}
\begin{figure}
\centering
\includegraphics[height=1.6in, width=1.4in]{GF}
\includegraphics[height=1.6in, width=1.4in]{EH}
\caption{Gottlob Frege  (1848 - 1925), Edmund Husserl (1859 - 1938).}
\end{figure}
\end{block}
\end{frame}

\begin{frame}
\frametitle{Osoby dramatu}
\begin{block}{Pierwsze fala}
\begin{figure}
\centering
\includegraphics[height=1.6in, width=1.2in]{GM}
\includegraphics[height=1.6in, width=1.4in]{BR}
\caption{G. E. Moore (1873 - 1958), Bertrand Russell (1872 - 1970).}
\end{figure}
\end{block}
\end{frame}

%\begin{frame}
%\frametitle{Szkoła analityczna}
%\begin{block}{Krótkie omówienie}
%\begin{itemize}
%\item Początek związany jest z~badaniami logicznymi, głównie Fregego i Russela (Whitehead poszedł swoją drogą).
%\item Naturalnym następnym krokiem było zaaplikowanie osiągnięć w~dziedzinie logiki do analizy innych problemów, w~tym filozoficznych.
%\item W naturalny sposób doprowadziło to do wielkich badań na~logiczną strukturą języka.
%\item Filozofie tej szkoły mają zwyczaj rozbijać wszystkie problemy na~czynniki pierwsze, stąd zasłużone miano filozofii analitycznej.
%\item ,,Philosophical problems should not be solved, but dissolved'' (Wittgenstein).
%\item Ich konkretny wpływ na~filozofię i kulturę jest bardzo trudny do prześledzenia (Gilbert Ryle i ,,Ghost in the Shell'').
%\end{itemize}
%\end{block}
%\end{frame}

\begin{frame}
\frametitle{Szkoła analityczna}
\begin{block}{Krótkie omówienie}
\begin{itemize}
\item Początek związany jest z~badaniami logicznymi, głównie Fregego i Russela (Whitehead poszedł swoją drogą).
\item Kolejnym krokiem rozwoju było zaaplikowanie osiągnięć w~dziedzinie logiki do analizy innych problemów, w~tym filozoficznych.
\item W naturalny sposób doprowadziło to do wielkich badań na~logiczną strukturą języka.
\item Filozofie tej szkoły mają zwyczaj rozbijać wszystkie problemy na~czynniki pierwsze, stąd zasłużone miano filozofii analitycznej.
\item ,,Philosophical problems should not be solved, but dissolved'' (Wittgenstein).
\item Ich konkretny wpływ na~filozofię i~kulturę jest bardzo trudny do prześledzenia (Gilbert Ryle i~,,Ghost in the Shell'').
\end{itemize}
\end{block}
\end{frame}

\begin{frame}
\frametitle{Szkoła analityczna}
\begin{block}{Krótkie omówienie}
\begin{itemize}
\item G. E. Moor napisał też niezwykle ważną książkę o etyce\linebreak (za A. MacIntyre ,,Dziedzictwo cnoty'').
\end{itemize}
\end{block}
\end{frame}

\begin{frame}
\frametitle{Szkoła kontynentalna}
\begin{block}{Krótkie omówienie}
\begin{itemize}
\item W moim mniemaniu ta szkoła zaczyna się wraz z~fenomenologią E. Husserla.
\item Tekst założycielski: ,,Logische Untersuchungen'' (dwa tomy 1900, 1901).
\item Filozofia Husserla była inspirowana Kartezjusza, mogła też zawierać wpływy Kanta. Hegel?
\end{itemize}
\end{block}

\end{frame}

\begin{frame}
\frametitle{Szkoła kontynentalna}
\begin{block}{Czym jest fenomenologia?}
Doświadczając, wyobrażając sobie coś, myśląc czy mówiąc o~czymś napotykamy różnego rodzaju przedmioty. Jak to się dzieje, że~napotykają nas one w~takim właśnie kształcie: jako rzeczy, ludzie, zdarzenia, wartości itd.? Oto pytanie fenomenologii. Metoda fenomenologiczna jest więc analizą i~opisem samego procesu zjawiania się czegoś, badaniem stosunku różnych perspektyw i~sposobów prezentacji do~tego, co w~nich dane, nie~analizą przedmiotu, lecz analizą projektu wszelkiego możliwego spotkania z~przedmiotem (Jan Potocka). (\ldots) Inaczej mówiąc: odkrycie ,,Badań logicznych'' polega na ujawnieniu różnicy między tym, co się zjawia, a~samym zjawianiem się. -- Krzysztof Michalski ,,Heidegger i filozofia współczesna''.
\end{block}
\end{frame}

\begin{frame}
\frametitle{Szkoła kontynentalna}
\begin{block}{Krótkie omówienie}
\begin{itemize}
\item Fenomenologia to zasadniczo filozofia zajmująca się ,,zjawiskami takimi jakie się jawią'' poprzez przeprowadzenie odpowiednich redukcji oraz poszukiwanie naoczności (intuicji) zjawisk.
\item Krytyka ówczesnego racjonalizmu i~,,Z powrotem do rzeczy!''.
\item Fenomenologia Husserla zrodziła współczesny egzystencjalizm (Heidegger).
\pause
\item Czy program Husserla się udał?
\pause
\item Jako całość nie jestem w stanie powiedzieć, jednak powrót do rzeczy i~idealizm nie wydają się zgrywać dobrze.
,,Fenomenologia to system filozoficzny w którym najbardziej liczą się nie rzeczy, lecz wrażenia rzeczy.'' -- M. Voris.
\end{itemize}
\end{block}

\end{frame}

\begin{frame}
\frametitle{Szkoła kontynentalna}

\begin{block}{Prywatny komentarz}
Filozofia Husserla miała być krytyką i~reformą europejskiego racjonalizmu, lecz czy nie jest tylko jego siostrą? Kartezjusz, Kant.
\end{block}
\end{frame}

\begin{frame}
\frametitle{Dwaj giganci -- niszczyciele}
\begin{block}{Czyli\ldots}
\begin{figure}
\centering
\includegraphics[height=1.6in, width=1.2in]{MH}
\includegraphics[height=1.6in, width=1.4in]{LW}
\pause
\caption{Martin Heidegger (1889 - 1976), Ludwig Wittgenstein\newline (1889 - 1951).}
\end{figure}
\end{block}
\end{frame}

\begin{frame}{Martin Heidegger}
\begin{block}{Życiorys esencjonalny}
\begin{itemize}
\pause
\item Urodził się.
\pause
\item Był nazistą.
\pause
\item Umarł.
\end{itemize}
\end{block}
\pause

\begin{block}{Życiorys filozoficzny}
\begin{itemize}
\item Student Husserla.
\item Ustanowił dwudziestowieczny egzystencjalizm i~hermeneutykę.
\item Miał obsesje na~punkcie etymologii i~pisał niezwykle ciężko.
\item Bardzo Niemiecki, przynajmniej w~pewnym okresie.
\item Dokonał ,,zwrotu'' w~swej filozofii.
\item Główne dzieło:\newline
Seit und Zeit (Bycie i czas), 1927.
\end{itemize}
\end{block}
\end{frame}

\begin{frame}{Seit und Zeit}
\begin{block}{Co warto wiedzieć}
\begin{itemize}
\item Jedna z~dwóch najważniejszych książek filozoficznych po roku 1925.
\item Światowy bestseller. M.in. sześć tłumaczeń na~język japoński do 1970 roku.
\item Zadedykowane Husserlowi. Dedykacja usunięta w~wydaniu z~1939 r.(?)
\item Tekst założycielski współczesnego egzystencjalizmu.
\item Miała być pierwszą częścią dwutomowego dzieła. Drugi tom nigdy się nie ukazał.
\end{itemize}
\end{block}

\begin{block}{Ciekawostka}
Światowa twarz egzystencjalizmu Jean-Paul Sartre wydał w 1943 r. książkę ,,L'étre et le néant'' (Bycie i nicość).
\end{block}
\end{frame}

\begin{frame}{Seit und Zeit}
\begin{block}{Uwagi techniczne}
\begin{itemize}
\item Heidegger używał silnie neologizmów (Husserl), lub zwykłych słów niemieckich w zupełnie innym sensie.
\item Przykłady: Dasein, Lichtung (,,Bycie w prześwicie'', za~Nonsensopedią).
\pause
\item Jego styl jest trudny/nieludzko~zagmatwany/niejasny/ zagadkowy/bezsensowny/maskujący brak treści (niepotrzebne skreślić).
\newline
,,Wszechobecność wyistaczania ukazuje nam się najwyraźniej wówczas, gdy  zauważymy, że~również i~właśnie odistaczanie określone jest poprzez owo, tajemnicze niekiedy wyistaczanie.'' ,,Zeit und Seit'' (,,Czas i bycie'').
\end{itemize}
\end{block}
\end{frame}


\begin{frame}{Seit und Zeit}
\begin{block}{Pytanie o sens bycia}
,,Czy w dzisiejszych czasach dysponujemy odpowiedzią na~pytanie, co właściwie rozumiemy pod słowem ,,bytujący''? W~żadnym razie nie. Dlatego trzeba na nowo postawić \emph{pytanie o sens bycia}. Czy~jednak dzisiaj fakt, że~nie rozumiemy wyrażenia ,,bycie'' jest naszym jedynym kłopotem? Bynajmniej. Dlatego też przede wszystkim musimy na nowo obudzić zrozumienie dla sensu tego pytania. Zamiarem niniejszej rozprawy jest konkretne opracowanie pytania o~sens ,,bycia''. Prowizorycznym celem zaś będzie interpretacja \emph{czasu} jako możliwego horyzontu wszelkiego w ogóle rozumienia bycia.''
\end{block}
\end{frame}

\begin{frame}{Seit und Zeit}
\begin{block}{Treść}
\begin{itemize}
\item To nie jest pierwsze dzieło egzystencjalizmu, jednak stanowi w~nim przełom.
\item Aby zbadać co to jest bycie należy zbadać jakiś obiekt który jest. Najprościej zbadać samego siebie.
\item ,,Bycie w świecie'' i~bycie w czasie.
\pause
\item Tom II nie został napisany bowiem Heidegger zmienił swoje poglądy filozoficzne. Dokonał zwrotu.
\end{itemize}
\end{block}
\end{frame}

\begin{frame}{Ludwig Wittgenstein}
\begin{block}{Życiorys}
\begin{itemize}
\item Urodził się w Wiedniu, w Cesarstwie Austro-Węgierskim.
\item Jego rodzina ze~strony ojca miała żydowskie pochodzenie, ze~strony matki czesko-słoweńskie
\item Był jednym z~dziewięciorga dzieci. Jego ojciec praktycznie władał przemysłem stalowym w Austro\dywiz Węgrzech.
\item Początkowo pragnął zostać inżynierem aeronautyki.
\item Nauka potrzebnej mu matematyki doprowadziło go do pytań o~jej podstawy i~filozofii matematyki.
\end{itemize}
\end{block}
\end{frame}

\begin{frame}{Wczesny Wittgenstein}
\begin{block}{Życiorys filozoficzny}
\begin{itemize}
\item Jako jedyny, chyba, człowiek w pierwszej połowie XX w. który czytał i brał na serio Schopenhauera.
\item Problemy z~podstawami matematyki prowadzą go do prac Fregego i Russela. Poznaje i zaprzyjaźnia się z~oboma.
\item Studiuje u~Russela i~Moora.
\item 1914 wybucha I Wojna Światowa, Wittgenstein zaciąga się do wojsk cesarstwa Austro-Węgierskiego.
\item W przerwach działań wojennych tworzy ,,Tractatus\ldots '' i~wiele notatek filozoficznych, zostaje też dwukrotnie wyznaczony do medalu za odwagę.
\item Uważając, że rozwiązał już wszystko co \emph{dało się rozwiązać} po wojnie porzuca filozofię. Tu kończy się wczesny Wittgenstein.
\end{itemize}
\end{block}
\end{frame}

\begin{frame}{Tractatus logico-philosophicus}
\begin{block}{Co warto wiedzieć}
\begin{itemize}
\item Napisany jako ,,Logisch - philosophische Abhandlung''. Pierwotnym językiem twórczości Wittgensteina pozostał zawsze wysokiej klasy niemiecki.
\item Pierwszy raz wydany w 1921 r. Tytuł pod którym obecnie słynie zaproponował G.E. Moore dla wydania angielskiego z~1922 r.
\item Książka bardzo małej objętości, w polskim wydaniu zajmuje 84 strony.
\item Zawiera m.in dyskusje nad problemami innych filozofów np.~z~Kanta, rozważania lingwistyczne, matematyczne, egzystencjalne i~teologiczne.
\item Podobno bardzo niewielu przeczytało go w całości.
\end{itemize}
\end{block}
\end{frame}

\begin{frame}{Tractatus logico-philosophicus}
\begin{block}{Co warto wiedzieć}
\begin{itemize}
\item Oddziaływanie traktatu było ogromnie. Stał się jedną z~ulubionych lektur pozytywistów logicznych (m.in Koła Wiedeńskiego).
\item Wciąż wielu naukowców przyjmuje ich interpretacje ,,Tractatusa\ldots ''.
\item Odegrał również wielką rolę w filozofii języka.
\end{itemize}
\end{block}
\end{frame}

\begin{frame}{Tractatus logico-philosophicus}
\begin{block}{Początek}
1. Świat jest wszystkim co jest faktem\\
1.1 Świat jest ogółem faktów nie rzeczy.\\
1.11 Świat jest wyznaczony przez fakty oraz przez to, że są to \emph{wszystkie} fakty.\\
1.12. Ogół faktów wyznacza bowiem, co jest faktem, a także wszystko, co faktem nie jest.\\
1.13. Światem są fakty w przestrzeni logicznej.\\
1.2. Świat rozpada się na fakty.\\
1.21. Jedno może być być faktem lub nie być, a wszystko inne pozostanie takie samo.\\
2. To, co jest faktem - fakt - jest istnieniem stanów rzecz.
\end{block}
\end{frame}

\begin{frame}{Traktatus logico-philosophicus}
\begin{block}{Zakończenie}
6.54. Tezy moje wnoszą jasność przez to, że kto mnie rozumie, rozpozna je w końcu jako niedorzeczne; gdy przez nie - po nich - wyjdzie ponad nie. (Musi niejako odrzucić drabinę, uprzednio po niej się wspiąwszy.) Musi te tezy przezwyciężyć, wtedy świat przedstawi mu się właściwie.\\
7. O czym nie można mówić o tym trzeba milczeć.
\end{block}
\end{frame}

\begin{frame}{Dla przestrogi}
\begin{block}{Dwie wypowiedzi Wittgensteina}
\begin{itemize}
\pause
\item Z listu do wydawcy:\\
,,Otóż chciałem napisać, że moja praca składa się z~dwu części: z~tego co w niej napisałem, oraz tego wszystkiego, czego nie napisałem. I właśnie ta druga część jest ważna.''
\pause
\item Zakończenie wstępu do Tractatusa, napisanego w 1918 roku:\\
,,Natomiast \emph{prawdziwość} komunikowanych tu myśli zdaje mi się niepodważalna i definitywna. Sądzę więc, że w istotnych punktach problemy zostały rozwiązane ostatecznie. A jeżeli się tu nie mylę, to wartością niniejszej pracy jest - po wtóre - to, że~widać z~niej, jak mało się przez ich rozwiązanie osiągnęło.''
\end{itemize}
\end{block}
\end{frame}

\begin{frame}{Dla przestrogi}
\begin{block}{Treść ,,Tractatusa\ldots ''}
Najwłaściwszy chyba opis jego treści, to zanalizowanie logicznej treści świata, tego co mieści się w kompetencji nośnika prawd logicznych (języka), a tego co poza niego wykracza.
\end{block}

\begin{block}{Kluczowe}
W tym momencie kluczowym jest punkt, że język odbija w sobie strukturę świata, która jest strukturą logiczną. Jest to jednak temat trudny i~nie pozbawiony kontrowersji.
\end{block}
\end{frame}

\begin{frame}{Późny Wittgenstein}
\begin{block}{Życiorys filozoficzny}
\begin{itemize}
\item Po dziesięciu latach zajmowania najprzeróżniejszymi zajęciami, wysłuchuje w Wiedniu przez przypadek wykładu\linebreak L. E. J. Brouwera na temat filozofii matematyki. W wyniku tego wraca do filozofii.
\item 1929 publikuje swoją drugą i ostatnią ogłoszoną za życia pracę.
\item Jeśli wierzyć Wikipedii to oprócz tego jego opublikowany dorobek zawiera jeszcze: \emph{jedną} recenzję książki i słownik dla dzieci.
\item Uzyskuje tytuł naukowy w Cambridge składają ,,Tractatusa\ldots '' (wówczas już uznanego za wybitne dzieło filozoficzne) jako swoją pracę dyplomową.
\end{itemize}
\end{block}
\end{frame}

\begin{frame}{Późny Wittgenstein}
\begin{block}{Życiorys filozoficzny}
\begin{itemize}
\item Od tego czasu naucza z~przerwami na angielskich uniwersytetach i rozwija zupełnie nową filozofię.
\item Umiera w 1951 roku.
\item 1953 ukazują się ,,Philosophische Untersuchungen'', druga najważniejsza książka filozoficzna po 1925 r.
\item Większość prac Wittgensteina ukazało się po jego śmierci.
\end{itemize}
\end{block}
\end{frame}

\begin{frame}{Późny Wittgenstein}
\begin{block}{Dociekania filozoficzne}
Dzieło to podważa koncepcje języka jako odbicia rzeczywistości, zastępuję je pojęciem gry językowej. Język w~tym sensie jest ludzką aktywnością z~której czerpie swój sens.
\end{block}

\begin{block}{}
Końcowy jest jednak taki sam, że~problemy filozoficzne (etyczne, teologiczne etc.) są nierozwiązywalne. Tu jednak też są ogromne niejasności i kontrowersje.
\end{block}

\end{frame}

\begin{frame}{Na zakończenie}
\frametitle{Krótki przegląd pozostałych filozofów XX w.}
\begin{block}{Fenomenologia poza szkołą Husserela}
\begin{figure}
\centering
\includegraphics[height=1.6in, width=1.2in]{MS}
\caption{Max Scheler (1874 - 1928).}
\end{figure}
\end{block}
\end{frame}

\begin{frame}{Na zakończenie}
\frametitle{Krótki przegląd pozostałych filozofów XX w.}
\begin{block}{Francuska szkoła kontynentalna}
\begin{figure}
\centering
\includegraphics[height=1.6in, width=1.1in]{JPS}
\includegraphics[height=1.6in, width=1.2in]{MP}
\includegraphics[height=1.6in, width=1.3in]{SB}
\caption{Jean-Paul Sartre (1905 - 1980), Maurice Merleau-Ponty\linebreak (1908 - 1961), Simone de Beauvoir (1908 - 1986).}
\end{figure}
\end{block}

\end{frame}

\begin{frame}{Na zakończenie}
\frametitle{Krótki przegląd pozostałych filozofów XX w.}
\begin{block}{Pozytywizm logiczny $\subset$ Koło Wiedeńskie}
\begin{figure}
\centering
\includegraphics[height=1.6in, width=1.2in]{RC}
\includegraphics[height=1.6in, width=1.1in]{ON}
\caption{Rudolf Carnap (1891 - 1970), Otto Neurath (1882 - 1945).}
\end{figure}
\end{block}
\end{frame}

\begin{frame}{Na zakończenie}
\frametitle{Krótki przegląd pozostałych filozofów XX w.}
\begin{block}{Po-Wittgensteinie i filozofia \emph{nauki}}
\begin{figure}
\centering
\includegraphics[height=1.6in, width=1.1in]{EA}
\includegraphics[height=1.6in, width=1.2in]{KP}
\caption{Gertrude Elizabeth Margaret Anscombe (1919 - 2001),\linebreak Karl Popper (1902 - 1994).}
\end{figure}
\end{block}
\end{frame}

\begin{frame}{Na zakończenie}
\frametitle{Krótki przegląd pozostałych filozofów XX w.}
\begin{block}{Francuski Hegel}
\begin{figure}
\centering
\includegraphics[height=1.6in, width=1.5in]{MB2}
\caption{Henri-Louis Bergson (1861 - 1941).}
\end{figure}
\end{block}
\end{frame}

\begin{frame}{Na zakończenie}
\frametitle{Krótki przegląd pozostałych filozofów XX w.}
\begin{block}{Filozofowie procesu}
\begin{figure}
\centering
\includegraphics[height=1.6in, width=1.2in]{HB}
\includegraphics[height=1.6in, width=1.2in]{AW}
\caption{Henri-Louis Bergson (1859 - 1941), Alfred North Whitehead (1861 - 1947).}
\end{figure}
\end{block}
\end{frame}

\begin{frame}{Na zakończenie}
\frametitle{Krótki przegląd pozostałych filozofów XX w.}
\begin{block}{Filozofia amerykańska}
\begin{figure}
\centering
\includegraphics[height=1.6in, width=1.3in]{JD}
\includegraphics[height=1.6in, width=1.5in]{WQ}
\caption{John Dewey (1859 - 1961), Willard Van Orman Quine (1908\linebreak - 2000).}
\end{figure}
\end{block}
\end{frame}

\begin{frame}{Na zakończenie}
\frametitle{Krótki przegląd pozostałych filozofów XX w.}
\begin{block}{Filozofowie polityczni}
\begin{figure}
\centering
\includegraphics[height=1.6in, width=1.3in]{AR}
\includegraphics[height=1.6in, width=1.5in]{JR}
\includegraphics[height=1.6in, width=1.1in]{RN}
\caption{Ayn Rand  (1905 - 1982), John Rawls (1921 - 2002),\linebreak Robert Nozick (1938 - 2002).}
\end{figure}
\end{block}

\end{frame}

\begin{frame}{Na zakończenie}
\frametitle{Krótki przegląd pozostałych filozofów XX w.}
\begin{block}{Hermeneutyka, strukturalizm i neopragmatyzm}
\begin{figure}
\centering
\includegraphics[height=1.6in, width=1.2in]{HG}
\includegraphics[height=1.6in, width=1.3in]{CS}
\includegraphics[height=1.6in, width=1.2in]{RR}
\caption{Hans-Georg Gadamer (1900 - 2002), Claude Lévi-Strauss (1908 - 2009), Richard Rorty (1931 - 2007).}
\end{figure}
\end{block}
\end{frame}

\begin{frame}{Na zakończenie}
\frametitle{Krótki przegląd pozostałych filozofów XX w.}
\begin{block}{Postmodernizm i poststrukturalizm}
\begin{figure}
\centering
\includegraphics[height=1.6in, width=1.2in]{MF}
\caption{Michel Foucault (1926 - 1984).}
\end{figure}
\end{block}
\end{frame}

\begin{frame}{Na zakończenie}
\frametitle{Krótki przegląd pozostałych filozofów XX w.}
\begin{block}{Premoderniści}
\begin{figure}
\centering
\includegraphics[height=1.6in, width=1.2in]{LS}
\includegraphics[height=1.6in, width=1.3in]{HA}
\includegraphics[height=1.6in, width=1.2in]{AM}
\caption{Leo Strauss (1899 - 1973), Hannah Arendt (1906 - 1975), Alasdair MacIntyre (1929 - żyje).}
\end{figure}
\end{block}
\end{frame}

\begin{frame}{I na ostatek}
\begin{block}{Najważniejszy filozof drugiej połowy XX w.}
\pause
\begin{figure}
\centering
\includegraphics[height=1.6in, width=1.2in]{JD1}
\includegraphics[height=1.6in, width=1.8in]{JD3}
\includegraphics[height=1.6in, width=1.2in]{JD2}
\caption{Jacques Derrida (1930 - 2004).}
\end{figure}
\end{block}
\end{frame}

\begin{frame}{Dla dociekliwych}

\begin{block}{Polecane pozycje}
Proszę nie podejrzewać, że ja to wszystko przeczytałem.
\begin{itemize}
\item[--] Frederick Copleston: \emph{Historia filozofii}
\item[--] Władysław Tatarkiewicz: \emph{Historia filozofii}
\item[--] Tadeusz Gadacz \emph{Historia filozofii XX wieku. Nurty}
\item[--] Lawrence Cahoone: \emph{Modern Intellectual Tradition: From Descartes to Derrida} (do zakupienia na stronie Great Courses)
\item[--] Kanał na YouTubie Gregor'ego B. Sadler
\item[--] Anna Burzyńska, Michał Paweł Markowski: \emph{Teorie literatury XX wieku}
\item[--] Leszek Kołakowski: \emph{Główne nurty marksizmu}
\item[--] Terence Hawkes: \emph{Strukturalizm i semiotyka}
\end{itemize}
\end{block}
\end{frame}

\begin{frame}{Dla dociekliwych}
\begin{block}{Polecane pozycje}
\begin{itemize}
\item[--] Krzysztof Michalski: \emph{Heidegger i filozofia współczesna}
\item[--] Leszek Kołakowski ,,O co nas pytają wielcy filozofowie''
\item[--] Julian Young: \emph{Heidegger, filozofia, nazizm}
\item[--] G. E. M. Anscombe, P. T. Geach: \emph{Trzej filozofowie}
\item[--] Willard Van Orman Quine \emph{Logika matematyczna}
\item[--] Ian Shapiro: \emph{Moral Foundations of Politics} (dostępne na YouTubie)
\item[--] S. Toksvig ,,Emanuel Swedenborg -- uczony i mistyk''
\end{itemize}
\end{block}
\end{frame}



\end{document}