% ---------------------------------------------------------------------
% Basic configuration of Beamera and Jagiellonian
% ---------------------------------------------------------------------
\RequirePackage[l2tabu, orthodox]{nag}



\ifx\PresentationStyle\notset
\def\PresentationStyle{dark}
\fi



\documentclass[10pt,t]{beamer}
\mode<presentation>
\usetheme[style=\PresentationStyle,logoLang=Latin,logoColor=monochromaticJUwhite,JUlogotitle=yes]{jagiellonian}



% ---------------------------------------
% Configuration files of Jagiellonian loceted in catalog preambule
% ---------------------------------------
\input{./preambule/LanguageSettings/JagiellonianPolishLanguageSettings.tex}
\input{./preambule/TextposConfiguration/TextposConfiguration.tex}

\input{./preambule/JagiellonianCustomizationGeneral.tex}
\input{./preambule/JagiellonianCustomizationCommands.tex}










% ---------------------------------------
% Packages, libraries and their configuration
% ---------------------------------------
\usepackage{mathcommands}





% ---------------------------------------
% Configuration for this particular presentation
% ---------------------------------------










% ---------------------------------------------------------------------
\title{Grafen, czyli dynamika relatywistyczna przy trochę mniejszych prędkościach}

\author{Kamil Ziemian \\
  \texttt{kziemianfvt@gmail.com}}


\institute{II rok, fizyka teoretyczna, studia magisterskie}

\date[16 V 2013]{16 maja 2013 r.}
% ---------------------------------------------------------------------










% ####################################################################
% Początek dokumentu
\begin{document}
% ####################################################################





% Wyrównanie do lewej z łamaniem wyrazów

\RaggedRight





% ######################################
\maketitle
% ######################################





% ##################
\begin{frame}
  \frametitle{Wiadomości wstępne}


  Grafen to pojedyncza warstwa atomów węgla ułożonych w kształt plastra
  miodu.

  Kilka podstawowych informacja o~jego strukturze fizycznej.
  \begin{itemize}
    \RaggedRight

  \item Sieć Bravais'a grafenu jest siecią trójkątną zawierającą w~każdej
    komórce dwa atomy węgla.

  \item Pierwsza strefa Brillouin'a, otrzymana metodą Wigner-Seitz'a jest
    heksagonalna. Nie ma to jednak nic wspólnego z rozkładem atomów
    w~przestrzeni.

  \item Spośród jej sześciu wierzchołków jedynie dwa dają fizycznie różne
    wyniki. Noszą one nazwę K-punktów Diraca.

  \end{itemize}

\end{frame}
% ##################





% ##################
\begin{frame}
  \frametitle{Model ciasnego wiązania}


  Podstawowe założenie modelu ciasnego wiązania. Ruch elektronu w grafenie
  jest opisywany przez nierelatywistyczne równanie Schr\"{o}dingera
  z~hamiltonianem:
  \begin{equation}
    \label{eq:Grafen-czyli-01}
    \widehat{ H } =
    -\frac{ \hbar^{ 2 } }{ 2 m } \triangle + V_{ \textrm{eff} }( \vecxbold ),
  \end{equation}
  gdzie $V_{ \textrm{eff} }( \vecxbold )$ jest potencjałem efektywnym
  pochodzącym od jonów sieci i~innych elektronów \cite{EWQTN}.

  Wyprowadzenie modelu. Ruch elektronu polega na przeskokach między stanami
  związanymi $2 p_{ z }$ poszczególnych atomów węgla. Przyjmujemy następujące
  oznaczenie:
  \begin{equation}
    \label{eq:Grafen-czyli-02}
    \psi_{ i, \alpha }( \vecxbold ) = | i, \alpha \rangle,
  \end{equation}
  na funkcję falową elektronu związanego na $i$-węźle, atomu $\alpha$.

\end{frame}
% ##################





% ##################
\begin{frame}
  \frametitle{Model ciasnego wiązania}


  Ze względu na naturę funkcji $2 p_{ z }$ dochodzi do ich przekrywania,
  niemniej przyjmiemy iż:
  \begin{equation}
    \label{eq:Grafen-czyli-03}
    \langle i, \alpha \, | \, j, \beta \rangle = \delta_{ i j } \, \delta_{ \alpha \beta }.
  \end{equation}

  Ze względu na standardowy rozkład hamiltonianu:
  \begin{equation}
    \label{eq:Grafen-czyli-04}
    \widehat{ H } =
    \sum_{ i, \alpha, \, j, \beta } \langle i, \alpha \, | \widehat{ H } | \, j, \beta \rangle \,
    | \, i, \alpha \rangle \langle \, j,\beta |,
  \end{equation}
  do pełnego zdefiniowania problemu potrzebujemy jedynie określić elementy
  macierzowe.

\end{frame}
% ##################





% ##################
\begin{frame}
  \frametitle{Model ciasnego wiązania}


  Elementy macierzowe
  \begin{equation}
    \label{eq:Grafen-czyli-05}
    \langle i, \alpha \, | \widehat{ H } | \, j, \beta \rangle =
    \begin{cases}
      V_{ 0 } + M, & i = j, \quad \alpha = \beta = 1, \\
      V_{ 0 } - M, & i = j, \quad \alpha = \beta = 2, \\
      -t,  & \textrm{dla najbliższych sąsiadów,} \\
      -t', & \textrm{dla prawie najbliższych sąsiadów,}
    \end{cases}
  \end{equation}
  gdzie $t = 2.7 \textrm{ eV}$, $t' = 0.1 \, t$.

  Ansatz. Stany własne są postaci:
  \begin{subequations}
    \begin{align}
      \label{eq:Grafen-czyli-06-A}
      &\varphi( \veckbold, \vecxbold ) =
        \varphi_{ 1 }( \veckbold ) \, \Phi_{ 1 }( \veckbold, \vecxbold )
        + \varphi_{ 2 }( \veckbold ) \, \Phi_{ 2 }( \veckbold, \vecxbold ), \\
      \label{eq:Grafen-czyli-06-B}
      &\Phi_{ \alpha }( \veckbold , \vecxbold ) =
        \frac{ 1 }{ \sqrt{ N } }
        \sum_{ i } e^{ i \veckbold ( \vecRbold_{ i } + \vecdbold_{ \alpha } ) } \,
        \psi_{ i, \alpha }( \vecxbold ).
    \end{align}
  \end{subequations}

\end{frame}
% ##################





% ##################
\begin{frame}
  \frametitle{Rozwiązanie analityczne}


  Pasma energetyczne
  \begin{equation}
    \begin{split}
      E_{ \pm }( \veckbold )
      &=
        V_{ 0 } - t' \left( 2 \cos( k_{ x } a )
        + 4 \, \cos\left( \frac{ k_{ x } a }{ 2 } \right) \,
        \cos\left( \frac{ \sqrt{ 3 } k_{ y } a }{ 2 } \right)
        \right) \pm \\[0.3em]
      &\pm \, \sqrt{ M^{ 2 } + t^{ 2 } \left( 3 + 2 \cos( k_{ x } a )
        + 4 \cos\left( \frac{ k_{ x } a }{ 2 } \right)
        \cos\left( \frac{ \sqrt{ 3 } k_{ y } a }{ 2 } \right) \right) }.
    \end{split}
  \end{equation}

  Rozwinięcie wokół K-punktów Diraca
  \begin{equation}
    E_{ \pm }( \vecKbold + \vecqbold ) \approx
    \begin{cases}
      \hbar v_{ \textrm{F} } | \vecqbold |, & M = 0, \\[0.3em]
      \pm\sqrt{ \left( \frac{ M }{ v_{ \textrm{F} }^{ 2 } } \right)^{ 2 }
      v_{ \textrm{F} }^{ 4 }
      + v_{ \textrm{F} }^{ 2 } \, ( \, \hbar | \vecqbold | )^{ 2 } }, & M \neq 0,
    \end{cases}
  \end{equation}
  gdzie $v_{ \textrm{F} } = ( \sqrt{ 2 } t a ) / ( 2 \; \hbar ) \approx 10^{ 6 } \,
  \frac{ \textrm{m} }{ \textrm{s} }$.

\end{frame}
% ##################










% ######################################
\appendix
% ######################################





% ######################################
\EndingSlide{Dziękuję! Pytania?}
% ######################################










% ##################
\begin{frame}
  \frametitle{Bibliografia}


  \bibliographystyle{plalpha}

  \bibliography{VariousFieldsBooks}{}

\end{frame}
% ##################





% ############################

% Koniec dokumentu
\end{document}
