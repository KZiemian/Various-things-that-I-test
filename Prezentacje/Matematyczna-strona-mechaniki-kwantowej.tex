% ---------------------------------------------------------------------
% Basic configuration of Beamera and Jagiellonian
% ---------------------------------------------------------------------
\RequirePackage[l2tabu, orthodox]{nag}



\ifx\PresentationStyle\notset
\def\PresentationStyle{dark}
\fi



\documentclass[10pt,t]{beamer}
\mode<presentation>
\usetheme[style=\PresentationStyle,logoLang=Latin,logoColor=monochromaticJUwhite,JUlogotitle=yes]{jagiellonian}



% ---------------------------------------
% Configuration files of Jagiellonian loceted in catalog preambule
% ---------------------------------------
\input{./preambule/LanguageSettings/JagiellonianPolishLanguageSettings.tex}
\input{./preambule/TextposConfiguration/TextposConfiguration.tex}

\input{./preambule/JagiellonianCustomizationGeneral.tex}
\input{./preambule/JagiellonianCustomizationCommands.tex}










% ---------------------------------------
% Packages, libraries and their configuration
% ---------------------------------------
\usepackage{mathcommands}





% ---------------------------------------
% Configuration for this particular presentation
% ---------------------------------------










% ---------------------------------------------------------------------
\title{Alistera McGratha krytyka memetyki}

\author{Kamil Ziemian \\
  \texttt{kziemianfvt@gmail.com}}

% \institute{Uniwersytet Jagielloński w~Krakowie}

\date[19 XII 2016]{Seminarium Międzywydziałowego Koła Naukoznawstwa $\Omega$, \\
  19 grudnia 2016}
% ---------------------------------------------------------------------
% \documentclass{beamer}
% \mode<presentation>
% %\usepackage{beamerthemesplit}
% \usepackage{epsfig}
% \usetheme{Warsaw}
% %\usecolortheme{albatross}
% %\usecolortheme{beetle}
% %\usecolortheme{crane}
% %\usecolortheme{dolphin}
% %\usecolortheme{dove}
% %\usecolortheme{fly}
% %\usecolortheme{seagull}
% %\usecolortheme{lily}
% \usecolortheme{orchid}
% %\usecolortheme{whale}
% %\usecolortheme{seahorse}
% \usepackage[]{graphicx}
% \usepackage{tensor}
% \usepackage{subfigure}%Pozwala dzielić figury na podfigury.
% \usepackage[polish]{babel}%Tłumaczy na polski teksty automatyczne LaTeXa i pomaga z typografią.
% \usepackage{amsfonts}%Czcionki matematyczne od American Mathematic Society.
% \usepackage{amsmath}%Dalsze wsparcie od AMS. Więc tego, co najlepsze w LaTeX, czyli trybu matematycznego.
% \usepackage[plmath,OT4,MeX]{polski}%Polska notacja we wzorach matematycznych. Ładne polskie czcionki i więcej cudzysłowów. Pełna polonizacja.
% \usepackage[utf8]{inputenc}%Pozwala pisać polskie znaki bezpośrednio.
% \usepackage{latexsym}%
% \usepackage{indentfirst}%Sprawia że jest wcięcie w pierwszym akapicie.
% \usepackage{textcomp}%Pakiet z dziwnymi symbolami.
% \usepackage{xy}%Pozwala rysować grafy.
% \frenchspacing%Wyłącza duże odstępy na końcu zdania. Podobno pakiet polski robi to samo, ale zostawić nie zaszkodzi.






% ---------------------------------------------------------------------
\title{Matematyczna strona mechaniki kwantowej}
\subtitle{Czyli fizyka matematyczna spotyka związki komutacji}

\author{Kamil Ziemian,
  \texttt{kziemianfvt@gmail.com}}


\institute{II rok, fizyka teoretyczna, studia magisterskie}

\date[15.11.2012]{15 listopada 2012 r.}
% ---------------------------------------------------------------------










% ####################################################################
% Początek dokumentu
\begin{document}
% ####################################################################





% Wyrównanie do lewej z łamaniem wyrazów

\RaggedRight





% ######################################
\maketitle % Tytuł całego tekstu
% ######################################





% ######################################
\begin{frame}
  \frametitle{Plan prezentacji}


  \tableofcontents % Spis treści

\end{frame}
% ######################################










% ######################################
\section[Podstawowe pojęcia]{Podstawowe pojęcia matematyczne}
% ######################################



% ##################
\begin{frame}
  \frametitle{Podstawowe pojęcia. Przestrzenie metryczne}


  Jest to zbiór $X$, na którym zdefiniowana jest odległość $\rho( x, y )$ $\Rightarrow$
  określona jest zbieżność ciągów i funkcji.

  Słownik pojęć
  \begin{itemize}
    \RaggedRight

  \item Kula otwarta, $K( x, r ) = \{ y \in X \; | \; \rho( x, y ) < r \}$,
    $r \in ( 0, +\infty )$.

  \item Zbiór otwarty, zbiór domknięty.

  \item Zbiór zwarty - analog zbioru domkniętego i~ograniczonego
    w~$\Rbb^{ n }$. Klasyczny przykład: odcinek domknięty $[ a, b ]$.

  \item Funkcja ciągła.

  \item Ciąg zbieżny.

\end{itemize}

\end{frame}
% ##################





% ##################
\begin{frame}
  \frametitle{Podstawowe pojęcia, przestrzenie metryczne}


  Słownik pojęć, ciąg dalszy.
  \begin{itemize}

  \item Ciąg Cauchy’ego $x_{ n }$: $\forall \; \varepsilon > 0, \exists \; N: \forall \; n, m > N$, $
    \rho ( x_{ n }, x_{ m } ) < \varepsilon $.

  \item Przestrzeń zupełna: każdy ciąg Cauchy’ego jest zbieżny.

  \item Podzbiór $D$ przestrzeni metrycznej $\Tcal$ nazywamy gęstym, jeżeli
    każdy jej element można dowolnie dokładnie przybliżyć elementem tego
    zbioru.

  \end{itemize}

  \textbf{Uwaga.}
  Od tej pory $X$ jest zawsze przestrzenią wektorową nad $\Cbb$ lub $\Rbb$.


\end{frame}
% ##################





% ##################
\begin{frame}
  \frametitle{Podstawowe pojęcia, przestrzenie liniowo-topologiczne}


  \textbf{Promieniem} w~zespolonej przestrzeni wektorowej nazywam zbiór
  $\Rcal ( \psi ) = \{ \phi : \; \phi = e^{ i \alpha } \psi ,\; ( \psi, \psi ) = 1 \}$.

  \textbf{Przestrzeń liniowo-topologiczne} to taka przestrzeń liniowa,
  w~której dodawanie wektorów i mnożenie przez liczbę są operacjami ciągłymi.
  Trzy najważniejsze typy tych przestrzeni to: przestrzenie unormowane,
  przestrzenie Banacha, przestrzenie Hilberta.

  Zachodzi: przestrzeń Hilberta $\Rightarrow$ przestrzeń Banacha $\Rightarrow$ przestrzeń
  unormowana.

\end{frame}
% ##################





% ##################
\begin{frame}
  \frametitle{Przestrzenie liniowo-topologiczne, przyłady}


  \begin{itemize}
    \RaggedRight

  \item Zbiór wielomianów określonych na odcinku $[ 0, 1 ]$ z~normą \\
    $\Vert \, f \Vert_{ 0 } = \sup_{ x \in [ 0, 1 ] } | \, f( x ) |$.

  \item Zbiór funkcji ciągłych określonych na odcinku $[ 0, 1 ]$ z~normą
    $\Vert \, f \Vert_{ 1 } = \sup_{ x \in [ 0, 1 ] } | \, f( x ) |$.
    Przestrzeń tą oznaczamy $\Ccal[ 0, 1 ]$.

  \item Przestrzeń $L^{ 2 }( \Rbb^{ n }, dx) $. Jest to najczęściej używana
    przestrzeń w~nierelatywistycznej mechanice kwantowej.

  \end{itemize}

\end{frame}
% ##################










% ######################################
\section{Postulaty mechaniki kwantowej}
% ######################################



% ##################
\begin{frame}
  \frametitle{Postulaty mechaniki kwantowej}


  Postulaty te pochodzą z książki Stevena Weinberga \textit{Teoria pól
    kwantowych. Tom I: Podstawy},
  \cite{WeinbergTeoriaPolKwantowychPodstawy2012}.


  \textbf{Postulat~I.}
  Stany fizyczne są reprezentowane przez promienie w~przestrzeni Hilberta.

  \textbf{Postulat II.}
  Obserwable są reprezentowane przez operatory samosprzężone. Stan
  reprezentowany przez promień $\Rcal( \psi )$ ma określoną wartość $\alpha \in \Rbb$
  obserwabli reprezentowanej przez operator $A$, jeżeli
  \begin{equation}
    \label{eq:Matematyczna-strona-01}
    A \psi = \alpha \psi.
  \end{equation}

\end{frame}
% ##################





% ##################
\begin{frame}
  \frametitle{Postulaty mechaniki kwantowej}


  \textbf{Postulat III.}
  Jeśli układ jest w stanie reprezentowanym przez promień $\Rcal$
  i~przeprowadzane jest doświadczenie sprawdzające, czy jest on w~jednym ze
  stanów reprezentowanych przez wzajemnie ortogonalne promienie
  $\Rcal_{ 1 }, \Rcal_{ 2 }, \ldots$, to prawdopodobieństwo znalezienia go w~stanie
  reprezentowanym przez $\Rcal_{ n }$ wynosi:
  \begin{equation}
    \label{eq:Matematyczna-strona-02}
    \Pcal( \Rcal \rightarrow \Rcal_{ n }) = | ( \psi, \psi_{ n } ) |^{ 2 }.
  \end{equation}

\end{frame}
% ##################










% ######################################
\section{Fizyka matematyczna a~mechanika kwantowa}
% ######################################



% ##################
\begin{frame}
  \frametitle{Fizyka matematyczna a~mechanika kwantowa}


  Pytanie: co można zakwestionować w tych postulatach?


  Z punktu widzenia matematyki\ldots
  \begin{itemize}

  \item czy przestrzeń Hilberta ma jakąś dodatkową strukturę?

  \item jakimi dokładnie operatorami są obserwable?

  \end{itemize}

\end{frame}
% ##################





% ##################
\begin{frame}
  \frametitle{Operatory samosprzężone}


  W~przestrzeni Hilberta istnieją trzy podstawowe klasy operatorów, mających
  diametralnie różne własności. Musimy więc ustalić, do której z~tych klas
  należą operatory reprezentujące obserwable.

  Klasy operatorów
  \begin{itemize}

  \item Operatory zwarte.

  \item Operatory ograniczone.

  \item Operatory nieograniczone.

  \end{itemize}

\end{frame}
% ##################





% ##################
\begin{frame}
  \frametitle{Operatory samosprzężone}


  Uwagi
  \begin{itemize}

  \item Dla operatora ograniczonego $A$ można zdefiniować normę operatora
    $\Vert A \Vert < \infty$ jako najmniejszą liczbę spełniającą:
    \begin{equation}
      \label{eq:Matematyczna-strona-03}
      \Vert A x \Vert \leq \Vert A \Vert \, || x ||, \quad
      \forall x \in \Hcal.
    \end{equation}

  \item Operatory zwarte są ograniczone.

  \item Matematycznie trzeba inaczej traktować wartości własne i~widmo
    ciągłe.

  \end{itemize}

\end{frame}
% ##################





% ##################
\begin{frame}
  \frametitle{Operatory zwarte}


  \textbf{Twierdzenie Riesza o rozkładzie spektralnym}

  Widmo samosprzężonego operatora zwartego składa się z samych wartości
  własnych, zawartych w~odcinku $[ -\Vert A \Vert, \Vert A \Vert ]$. Zbiór wartości
  własnych jest albo skończony albo można go ustawić w~ciąg
  $\lambda_{ 1 } \geq \lambda_{ 2 } \geq \lambda_{ 3 } \geq \ldots \to 0$. Ponadto wszystkie podprzestrzenie
  do wartości własnej $\lambda_{ n } \neq 0$ mają wymiar skończony.

\end{frame}
% ##################





% ##################
\begin{frame}
  \frametitle{Operatory samosprzężone, wnioski}


  \begin{itemize}

  \item Operatory zwarte samosprzężone są najbliższymi krewnymi macierzy
    hermitowskich w~przypadku, gdy wymiar $\Hcal$ jest nieskończony.

  \item Działanie operatora zwartego $\Lcal$ można w poprawnie matematyczny
    sposób zapisać jako:
    \begin{equation}
      \label{eq:Matematyczna-strona-03}
      \Lcal | V \rangle = \sum_{ \lambda } \lambda \, a_{ \lambda } \, | \lambda \rangle
    \end{equation}

  \item Absolutnie nie nadają się jako kandydaci na obserwable takie jak:
    $\widehat{ P }$, $\widehat{ X }$, $a$, $a^{ \dagger }$, etc.

  \end{itemize}

  \textbf{Dlaczego?} Ograniczoność widma operatorów zwartych nie zgadza~się
  z~dobrze znanymi własnościami tych obserwabli.

\end{frame}
% ##################





% ##################
\begin{frame}
  \frametitle{Czy w takim razie operatory zwarte nie mają tu zastosowań?}


  \textbf{Kontrprzykład.}

  Załóżmy, że nie wiemy w jakim konkretnie stanie znaduje się nasz układ,
  wiemy jednak, że z prawdopodobieństwem $p_{ i }$ znajduje się w stanie
  $\psi_{ i }$. Ponieważ $p_{ i }$ są nieujemne i ich suma jest równa jeden,
  operator dany wzorem
  \begin{equation}
    \label{eq:Matematyczna-strona-04}
    \rho = \sum_{ i = 0 }^{ \infty } p_ { i } \, ( \psi, \cdot \, ) \, \psi.
  \end{equation}
  jest poprawnie określony i~jest operatorem zwartym.

  Z~tego względu teoria operatorów okazała się wartościowym podejściem
  do~problemów kwantowej fizyki statystycznej.

\end{frame}
% ##################





% ##################
\begin{frame}
  \frametitle{Operatory nieograniczone}


  Ponieważ dla ograniczonych operatorów samosprzężonych istnieje analogiczne
  twierdzenie o~lokalizacji widma na odcinku $[ -\Vert A \Vert, \Vert A \Vert ]$,
  pozostaje nam zająć się operatorami nieograniczonymi.

  \textbf{Rozwiązanie.}
  Rzeczywiście okazuje się, że większość operatorów znanych z mechaniki
  kwantowej, takich jak $\widehat{ P }$, $\widehat{ X }$, $a$, $a^{ \dagger }$,
  etc., okazuje~się być samosprzężonymi operatorami nieograniczonymi.

  Z~punktu widzenia fizyki matematycznej to krótkie stwierdzenie zawiera
  w~sobie ogromną ilość skomplikowanej matematyki
  i~nietrywialnych/ciekawych/irytujących problemów do rozwiązania.

\end{frame}
% ##################

% \begin{frame}{Źródło problemów}
% \pause


% \begin{block}{Twierdzenie Hellingera - Toeplitza}
% Operator samosprzężony określony na całej przestrzeni Hilberta jest ograniczony.
% \end{block}
% \pause

% \begin{block}{Konsekwencje}
% Operator reprezentujący najbardziej podstawowe obserwable możemy określić jedynie na podprzestrzeniach wektorowych, które są gęste w $\mathcal{ H }$. Podprzestrzenie takie będziemy oznaczać jako $\mathcal{ D }$, zaś dziedzinę obserwabli $A$ przez $\mathcal{ D }( A )$.
% \end{block}

% \begin{block}{Własności gęstych podprzestrzeni}
% W przestrzeni skończenie wymiarowej takie podprzestrzenie nie istnieją. Gdy zaś istnieją, potrafią mieć bardzo nieoczekiwane i wyjątkowe własności. W szczególności nie można z nich wnioskować o własnościach całej przestrzeni.
% \end{block}

% \end{frame}


% \begin{frame}{Podprzestrzenie gęste}
% \pause


% \begin{block}{Własności gęstych podprzestrzeni}
% W przestrzeni skończenie wymiarowej takie podprzestrzenie nie istnieją. Gdy zaś istnieją, potrafią mieć bardzo nieoczekiwane i wyjątkowe własności.
% \end{block}
% \pause

% \begin{block}{Przykład z przestrzeni Banacha}
% W przestrzeni Banacha $\mathcal{ C } [0, 1]$ z normą $||f|| = \sup_{ x \in [0, 1] } |f(x)| $, wielomiany tworzą podprzestrzeń gęstą na mocy klasycznego twierdzenia Weierstrassa o aproksymacji funkcji ciągłych.
% \end{block}
% \pause

% \begin{block}{}
% Z drugiej strony Banach udowodnił, że większość funkcji ciągłych nie posiada pochodnych w żadnym punkcie swojej dziedziny. Widzimy więc, że z faktu, iż możemy jakąś funkcję z dowolną dokładnością przybliżyć funkcją analityczną (wielomianem!), nie wynika nawet jej różniczkowalność.
% \end{block}

% \end{frame}


% \begin{frame}{Problemy w mechanice kwantowej}
% \pause


% \begin{block}{Podstawowe problemy}
% \begin{itemize}
% \item[--]Korzystając z reguł Borna, nic nie wiem o dziedzinach danych obserwabli.
% \pause
% \item[--]Może się zdarzyć, że $D( A ) \cap D( B ) = {0}$, więc suma dwóch obserwabli $A$ i $B$, może mieć sens tylko w działaniu na wektor zerowy.
% \pause
% \item[--]W szczególności nic nie wiem o tym, na jakim zbiorze mają być spełnione relacje komutacji $[\hat{ X }, \hat{ P }] = i \hbar$. Co więcej, można pokazać, że na odpowiednio sensownej dziedzinie nie istnieją operatory ograniczone realizujące te związki.
% \pause
% \item[--]Widmo takiego operatora zależy od wyboru dziedziny.
% \item[--]Tak samo jak samosprzężoność.
% \end{itemize}
% \end{block}

% \end{frame}


% \begin{frame}


% \begin{block}{Ciekawsze problemy}
% \begin{itemize}
% \item[--]Załóżmy, że mamy standardowy problem, w którym hamiltonian ma postać $H = H_{ 0 } + V$, gdzie $V$ jest odpowiednią poprawką i znamy $\mathcal{ D }(H_{ 0 })$. Dodanie takiej poprawki może sprawić, że $\mathcal{ D }(H) \neq \mathcal{ D }(H_{ 0 })$.
% \pause
% \item[--]Jeżeli hamiltonian (jak to zwykle bywa) jest nieograniczony, to równanie Schr\"{o}dingera
% $$i \hbar \partial_{ t } | \psi \rangle = H | \psi \rangle$$
% jest dla wielu wektorów po prostu pozbawione sensu.
% \end{itemize}
% \end{block}

% \begin{block}{Osiągnięcia}
% W ramach operatorowego podejścia do mechaniki kwantowej wiele z tych problemów udało się w dużym stopniu rozwiązać.
% \end{block}

% \end{frame}



% \section[]{Podejście algebraiczne}



% \begin{frame}{Podejście algebraiczne}


% \begin{block}{Podsumowanie teorii operatorów}
% Zapoczątkowana przez John von Neumanna teoria operatorów nieograniczonych stanowi do dziś najpotężniejsze podejście do uzyskiwania matematycznych rozwiązań problemów mechaniki kwantowej. Z innych jej osiągnięć warto wspomnieć rozwiniętą przez T. Kato teorię zaburzeń.
% \end{block}

% \begin{block}{Alternatywne podejście}
% Formalizm ten jest jednak jak widać dość delikatny. Pytanie, czy można znaleźć formalizm, który jest bardziej stabilny.
% Poprawną odpowiedź na to pytanie zapoczątkował pomysł wielkiego XX w. matematyka H. Weyla.
% \end{block}

% \end{frame}

% \begin{frame}{Podejście algebraiczne}


% \begin{block}{Związki Weyla}
% Weyl, wspierając się na poprzednich pracach von Neumanna, zauważył, że korzystając rachunku operatorowego i funkcji ograniczonej możemy zbudować operatory ograniczone z operatorów $ \hat{ X } $ i $ \hat{ P } $. Korzystając z funkcji ograniczonych $\exp(i p x)$, związki komutacji przechodzą w następujące \emph{związki Weyla} dla operatorów ograniczonych:
% \pause
% $$e^{ i \alpha \hat{ X } } e^{ i \beta \hat{ P } } e^{ -i \alpha \hat{ X } } = e^{ i \beta (\hat{ P } - \alpha) } \textrm{.}$$

% \end{block}
% \pause

% \begin{block}{}
% Można pokazać, że tak zdefiniowane operatory zwierają tyle samo informacji co pierwotne $\hat{ X }$ i $\hat{ P }$, które można otrzymać z form eksponencjalnych poprzez różniczkowanie.
% \end{block}

% \end{frame}


% \begin{frame}{Podejście algebraiczne}


% \begin{block}{}
% Sytuacja ta wraz z równoważnością różnych realizacji mechaniki kwantowej, takich jak reprezentacja położeniowa, pędowa, czy obsadzeń w przypadku oscylatora, zasugerowało następujące podejście.
% \end{block}
% \pause

% \begin{block}{Algebraiczna mechanika kwantowa}
% Podstawowym obiektem jest odpowiednia algebra $\mathcal{ A }$ (*-algebra\linebreak lub $C^{*}$), stany zaś odpowiadają funkcjonałom liniowym na $\mathcal{ A }$. Przy wybraniu pewnej reprezentacji elementy algebry przechodzą w operatory na przestrzeni Hilbert, a stany w wektory tej przestrzeni. Reprezentacje równoważne odpowiadają tej samej sytuacji fizycznej.
% \end{block}

% \end{frame}


% \begin{frame}{AQFT}


% \begin{block}{AQFT}
% To podejście zostało zastosowane do QFT m.in. prze Haaga, Kastlera, Wightmana i Gardinga, w rezultacie czego otrzymano teorie zwaną algebraiczną kwantową teorią pola.
% \end{block}
% \pause

% \begin{block}{}
% Przedstawione tu sformułowanie mechaniki kwantowej jest chyba najbardziej udaną teorią we współczesnej fizyce matematycznej. Jednak w przypadku QFT sytuacja nie jest już tak dobra.
% \end{block}
% \pause

% \end{frame}


% \begin{frame}{Sukcesy i porażki}


% \begin{block}{Sukcesy i porażki}
% \begin{itemize}
% \pause
% \item[--]Nie istnieje pełne algebraiczne sformułowanie jakiejkolwiek rzeczywistej teorii pola.
% \pause
% \item[--]Istnieją modele konkretnych realnych efektów.
% \pause
% \item[--]Bliski związek z teorią grup.
% \pause
% \item[--]Kwantyzacja pola Diraca.
% \pause
% \item[--]Reguły nadwyboru.
% \pause
% \item[--]Brak nieskończoności.
% \pause
% \item[--]Brak nieskończoności.
% \end{itemize}
% \end{block}

% \end{frame}



\end{document}