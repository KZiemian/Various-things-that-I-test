% ---------------------------------------------------------------------
% Basic configuration of Beamera and Jagiellonian
% ---------------------------------------------------------------------
\RequirePackage[l2tabu, orthodox]{nag}



\ifx\PresentationStyle\notset
\def\PresentationStyle{dark}
\fi



\documentclass[10pt,t]{beamer}
\mode<presentation>
\usetheme[style=\PresentationStyle,logoLang=Latin,logoColor=monochromaticJUwhite,JUlogotitle=yes]{jagiellonian}



% ---------------------------------------
% Configuration files of Jagiellonian loceted in catalog preambule
% ---------------------------------------
\input{./preambule/LanguageSettings/JagiellonianPolishLanguageSettings.tex}
\input{./preambule/TextposConfiguration/TextposConfiguration.tex}

\input{./preambule/JagiellonianCustomizationGeneral.tex}
\input{./preambule/JagiellonianCustomizationCommands.tex}










% ---------------------------------------
% Packages, libraries and their configuration
% ---------------------------------------
\usepackage{mathcommands}





% ---------------------------------------
% Configuration for this particular presentation
% ---------------------------------------





% \documentclass[handout]{beamer} \mode<presentation>
% % \usepackage{beamerthemesplit}
% \usepackage{epsfig}
% \usetheme{Warsaw}
% \usecolortheme{orchid}
% \usepackage[utf8]{inputenc}
% \usepackage[polish]{babel}
% \usepackage[MeX]{polski}

% \usepackage{verbatim}
% \usepackage{subfigure}%Pozwala dzielić figury na podfigury.
% \usepackage{amsfonts, amsmath}


% \usepackage{amscd}% Jeszcze wsparcie od AMS.
% \usepackage{latexsym}% Więcej symboli.
% \usepackage{textcomp}% Pakiet z dziwnymi symbolami.
% \usepackage{xy}% Pozwala rysować grafy.
% \usepackage{tensor}% Pozwala prosto używać notacji tensorowej. Albo nawet pięknej notacji %tensorowej:).
% \usepackage{graphicx}% Pozwala wstawiać grafikę.
% \usepackage{xcolor}
% \usepackage{hyperref}
% \usepackage{verse}
% \usepackage{siunitx}
% \newcommand{\de}[1]{\mathrm{d} #1 \;}
% \newcommand{\lra}{\longrightarrow}
% \newcommand{\tbs}{\textbackslash}
% \newcommand{\Google}[3]{\begin{frame}
% \frametitle{#1}
% \begin{block}{Wszystko co warto wiedzieć o~\LaTeX u}
% {Da się {\color{#2} wygooglować}.}
% \end{block}

% #3

% \end{frame}}
% \newcommand{\pd}[3]{\frac{ \partial^{ #1 } #2 }{ \partial {#3}^{ #1 }}}

% \newcommand{\tb}{\textbf}

% \newcommand{\attribA}[1]{%
%   \nopagebreak{\vspace{2mm}\raggedleft\footnotesize #1\par\vspace{2em}}}
% \newcommand{\attribB}[1]{%
%   \nopagebreak{\raggedleft\footnotesize #1\par \vspace{2em}}}

\LetLtxMacro{\oldsqrt}{\sqrt}
\def\sqrt{\mathpalette\DHLhksqrt}
\def\DHLhksqrt#1#2{%
\setbox0=\hbox{$#1\oldsqrt{#2\,}$}\dimen0=\ht0
\advance\dimen0-0.2\ht0
\setbox2=\hbox{\vrule height\ht0 depth -\dimen0}%
{\box0\lower0.4pt\box2}}










% ---------------------------------------------------------------------
\title{O~różnych spojrzeniach na~QFT i~efekt Casimira}

\author{Kamil Ziemian \\
  \texttt{kziemianfvt@gmail.com} }


\date[21 I 2016]{21 stycznia 2016 r.}
% ---------------------------------------------------------------------










% ####################################################################
% Początek dokumentu
\begin{document}
% ####################################################################





% Wyrównanie do lewej z łamaniem wyrazów

\RaggedRight





% ######################################
\maketitle
% ######################################





% ######################################
\section{Osobiste poglądy}
% ######################################



% ##################
\begin{frame}
  \frametitle{Ostrzeżenie i~osobiste przemyślenia}


  To wystąpienie to bardziej zbiór pytań niż odpowiedzi. Może ktoś z~was
  będzie umiał powiedzieć jak je~rozwiązać, ja~nie jestem w~stanie.

  Celem nauk przyrodniczych nie jest
  \begin{itemize}

  \item zajęcie bardziej prestiżowego miejsca w~społeczeństwie;

  \item zdawanie egzaminów;

  \item zrobienie kariery naukowej;

  \item zdobycie ogromnych grantów;

  \item stanie~się sławnym;

  \item podrywanie kobiet;

  \item picie alkoholu w~dobrym/niedobrym towarzystwie;

  \item oglądanie seriali, granie w~gry video (mea culpa), etc.;

  \item zajmowanie~się czymś, bo na Zachodzie się tym zajmują
    („Co~ Francuz wymyśli to Polak polubi”, \textit{Pan Tadeusz}, Adam~M.);

  \item nie zajmowanie~się czymś, bo lepsi od nas tego nie
    robili.

  \end{itemize}

\end{frame}
% ##################





% ##################
\begin{frame}
  \frametitle{Ostrzeżenie i~osobiste przemyślenia}


  Celem nauk przyrodniczych nie jest
  \begin{itemize}

  \item nabicie wysokiego indeksu Hirscha;

  \item ani nawet, szukanie zastosowań technologicznych (to nie
    {\color{red} \textit{cel}}, acz też jest ważne).

  \end{itemize}

\end{frame}
% ##################





% ##################
\begin{frame}
  \frametitle{Ostrzeżenie i~osobiste przemyślenia}


  Cel pierwszy i~najważniejszy: lepsze zrozumienie otaczającego nas świata.

  Cel drugorzędny: zrozumienie wszystkich konsekwencji, jakie z~tej wiedzy
  naprawdę wynikają. To łączy~się z~pytaniami o~zastosowania.

  Po cóż jest więc kwantowa teorii pola{\LARGE \color{red} ???}

  Uwaga o~kryteriach filozoficznych i~estetycznych.
  Absurdalność, matematyczna nonsensowność i~największa nawet
  brzydota, nie~są dobrymi powodami by~odrzucić teorię która
  w~poprawny sposób opisuje świat wokół nas. Tym bardziej, jeśli żadna
  inna nie~była w~stanie powtórzyć jej sukcesów.

\end{frame}
% ##################










% ######################################
\section{QFT}
% ######################################



% ##################
\begin{frame}
  \frametitle{Potrzeba istnienia jakiejś teorii}


  Fakty doświadczalne
  \begin{itemize}

  \item Widma atomowe.

  \item Istnieją elektrony.

  \item Istnieje pole elektromagnetyczne, bardzo dobrze
    opisywane teorią Clerka-Maxwella.

  \item Cząstki mogą być kreowane lub anihilowanie (dziwne).

  \item Cząstki mogą kreować cząstki innych typów (bardzo
    dziwne).

  \end{itemize}

  Wnioski
  \begin{itemize}

  \item Widma atomowe i~elektrony $\Rightarrow$ mechanika kwantowa.

  \item Pole elektromagnetyczne $\Rightarrow$ szczególna teoria względności.

  \end{itemize}



  Problemy
  \begin{itemize}

  \item Co~się dzieje, gdy elektron porusza~się naprawdę szybko?

  \item Skoro cząstki~są kwantowe, to czy pole elektromagnetyczne też?

  \end{itemize}

\end{frame}
% ##################





% ##################
\begin{frame}
  \frametitle{Potrzeba istnienia jakiejś teorii}


  Problem profesora Sadzikowskiego.
  Jeśli by ktoś podszedł i~walnął bardzo szybko patelnią w~ścianę
  wydziału FAIS, to~nie~oczekuję, że~zostanie po tym laptop, lody
  cytrynowe, dwie zapałki i~długopis, lecące we wszystkie strony z~dużą
  prędkością.

\end{frame}
% ##################





% ##################
\begin{frame}
  \frametitle{Standardowe podejście do QFT}


  Lagranżjan elektrodynamiki kwantowej (QED)
  \begin{equation}
    \label{eq:O-roznych-01}
    \Lcal =
    -\frac{ 1 }{ 4 } F_{ \mu \nu }( x ) F^{ \mu \nu }( x )
    + i \overline{\psi( x )} ( \gamma^{ \mu } \partial_{ \mu } - m ) \psi( x )
    - e \overline{\psi( x )} \gamma^{ \mu } A_{ \mu }( x ) \psi( x ).
  \end{equation}

  Uwagi
  \begin{itemize}

  \item \textbf{SWOBODNE} i~\textbf{PRAWIE SWOBODNE} pole
    elektromagnetyczne $F_{ \mu \nu }( x )$ można rozłożyć na fotony.

  \item Jeśli chodzi o~naprawdę oddziałujące, to nie~wiem czy ktoś potrafi
    rozstrzygnąć czy również jest to wykonalne (może jestem ignorantem).

  \item \textbf{SWOBODNE} i~\textbf{PRAWIE SWOBODNE} pole Diraca $\psi( x )$
    można rozłożyć na elektrony i~pozytony.

  \item Reszta jak powyżej.

  \end{itemize}

\end{frame}
% ##################





% ##################
\begin{frame}
  \frametitle{Standardowe podejście do QFT}


  Uwagi cd.
  \begin{itemize}

  \item Jeżeli ma sens powiedzenie, że~foton jest kwantem pola
    elektromagnetycznego, to elektron i~pozyton~są kwantami pola
    Diraca. Jest to dobra wiadomość, bowiem wyjaśnia dlaczego
    elektrony w~przyrodzie są tak podobne do siebie.

  \item Każde pole kwantowe kreuje i~anihiluje cząstki.

  \item Teoria ta działa rewelacyjnie, choć trzeba renormalizować.

  \item Dirac ją odrzucił twierdząc, że~jest brzydka.

  \end{itemize}



  \textbf{Odpowiedź fizyka.} Obiektem który reprezentuje cząstki.

  Więcej~się nie dowiedziałem, ale może za~mało pytałem?

\end{frame}
% ##################





% ##################
\begin{frame}
  \frametitle{Moja ulubiona, nieudana, odpowiedź}


  Aksjomaty Wightmana.
  $M$ „$=$” $\Rbb^{ 4 }$ przestrzeń Minkowskiego, $\Hcal$ przestrzeń
  Hilberta, $S( M )$ przestrzeń „dobrych” funkcji zespolonych, najlepiej
  klasy Schwartza (jeden z~powodów dlaczego prof. Herdegen je tak lubi).
  Pole kwantowe $F_{ \mu \nu }( x )$ to dystrybucja która funkcji
  $f( x ) \in S( M )$ przyporządkowuje operator $F_{ \mu \nu }( f( x ) )$ na
  $\Hcal$.

  Aksjomatów tych jest więcej, ale~nam nie będą one potrzebne.

  { \color{red} Porażka. }
  Do dziś nikt nie był w~stanie podać w~ramach podejścia Wightmana
  teorii w~czterech wymiarach, która nie jest \textbf{swobodna}!!!

\end{frame}
% ##################





% ##################
\begin{frame}
  \frametitle{Moja ulubiona, nieudana, odpowiedź}


  Dobre strony podejścia Wightmana i~pokrewnych
  \begin{itemize}

  \item Mówi czym pole kwantowe jest :).

  \item W~mechanice kwantowej fundamentem jest przestrzeń
    Hilberta $\Hcal$ i~czas $t$, przestrzeń pojawia~się jako
    sposób diagonalizacji operatorów~$\widehat{x}_{ i }$. Podejście
    Wightmana w~przejrzysty sposób pokazuje, że~w~QFT musimy
    dołączyć nowy byt: czasoprzestrzeń Minkowskiego.

  \item \textbf{Udowodniono} twierdzenia CPT i~o związku spinu ze~statystyką.

  \item Udało~się sformułować najlepsze, w~mojej opinii, wersje
    QFT na zadanych zakrzywionych czasoprzestrzeniach.

  \end{itemize}

\end{frame}
% ##################





% ##################
\begin{frame}
  \frametitle{Zaraz, zaraz, ty coś ściemniasz}


  Przecież napisałeś
  \begin{equation}
    \label{eq:O-roznych-02}
    \Lcal =
    -\frac{ 1 }{ 4 } F_{ \mu \nu }( x ) F^{ \mu \nu }( x )
    + i \overline{\psi( x )} ( \gamma^{ \mu } \partial_{ \mu } - m ) \psi( x )
    - e \overline{\psi( x )} \gamma^{ \mu } A_{ \mu }( x ) \psi( x ),
  \end{equation}
  a~przecież ktoś taki jak ty, powinien wiedzieć, że~dystrybucji~się
  nie~mnoży.

  Zgadza~się.
  Co najmniej od lat 50 XX wieku, wielu autorów, że~właśnie mnożenie
  dystrybucji jest przyczyną istnienia rozbieżności w~ultrafiolecie,
  podczas gdy~rozbieżności podczerwone niosą treść fizyczną.

  Osiągnięcia.
  Epstein i~Glaser w~latach 70 XX wieku pokazali, że~na poziomie
  grafów Feynmana używając poprawnej teorii mnożenia i~rozszerzania
  dystrybucji można wyeliminować wszystkie rozbieżności
  w~ultrafiolecie.

\end{frame}
% ##################










% ######################################
\section{Efekt Casimira -- standardowe podejście}
% ######################################



% ##################
\begin{frame}
  \frametitle{Efekt Casimira -- standardowe podejście}


  Masywne pole skalarne.
  Lagrażjan:
  \begin{equation}
    \label{eq:O-roznych-03}
    \Lcal =
    \partial_{ \mu } \varphi( t, x ) \partial^{ \mu } \varphi( t, x )
    + \frac{ m^{ 2 } c^{ 2 } }{ \hbar^{ 2 } } \varphi^{ 2 }( t, x ).
  \end{equation}

  Równanie Kleina-Gordona:
  \begin{equation}
    \label{eq:O-roznych-04}
    \frac{ 1 }{ c^{ 2 } }
    \frac{ \partial^{ 2 } \varphi( t, x ) }{ \partial t^{ 2 } }
    - \frac{ \partial^{ 2 } \varphi( t, x ) }{ \partial x^{ 2 } }
    + \frac{ m^{ 2 } c^{ 2 } }{ \hbar^{ 2 } } \varphi( t, x ) = 0.
  \end{equation}

  Hamiltonian bezpośredni skwantowanej teorii w~przestrzeni pędów
  \begin{subequations}
    \begin{equation}
      \label{eq:O-roznych-04-A}
      \widehat{H} =
      \frac{ 1 }{ 2 } \int dk \, dk' \, \hbar
      \sqrt{ \omega_{ k } \omega_{ k' } } ( a_{ k' }^{ \dagger } a_{ k } + a_{ k' } a_{ k }^{ \dagger } ),
    \end{equation}
    \begin{equation}
      \label{eq:O-roznych-04-B}
      \omega_{ k } = \sqrt{ c^{ 2 } k^{ 2 } + m^{ 2 } c^{ 4 } }.
    \end{equation}
  \end{subequations}

\end{frame}
% ##################





% ##################
\begin{frame}
  \frametitle{Efekt Casimira -- standardowe podejście}


  Z~tamtym hamiltonianem ciężko cokolwiek zrobić.
  Standardowa rozwiązanie (ja nie umiem inaczej):
  \begin{equation}
    \label{eq:O-roznych-05}
    \widehat{H} =
    \int dk \; \hbar \, \omega_{ k } a_{ k }^{ \dagger } a_{ k }
    + \frac{ \hbar }{ 2 } \int dk \, \omega_{ k }.
  \end{equation}
  Co zrobić z~nieskończoną energią stanu podstawowego (zwaną zwykle
  energią próżni)?

  Recepta Diraca.
  „Ponieważ w~fizyce mają znaczenie tylko różnice energii,
  więc~energię stanu podstawowego należy przyjąć za punkt
  odniesienia i~położyć 0.”

  Riposta Pauli’ego. „Tylko dlatego, że~coś jest nieskończone, to jeszcze
  nie~znaczy, że~możesz je przyjąć równe zeru.”n

\end{frame}
% ##################





% ##################
\begin{frame}
  \frametitle{Efekt Casimira -- standardowe podejście}


  W~praktyce recepta Diraca prawie zawsze działa. Efekt Casimira to jeden
  z~przypadków, gdy nie działa.

  Jeżeli ustawimy w~punktach $x = 0$ i~$x = a$, dwie płyty metalowe,
  to pole $\varphi$ musi na nich znikać (bo ma~się zachowywać jak
  pole elektryczne na granicy przewodnika):
  \begin{equation}
    \label{eq:O-roznych-06}
    \varphi( t, 0 ) = \varphi( t, a ) = 0.
  \end{equation}

  Takie pole ma postać:
  \begin{subequations}
    \begin{equation}
      \label{eq:O-roznych-07-A}
      \varphi_{ n }^{ ( \pm ) }( t, x ) =
      \left( \frac{ c }{ a \omega_{ n } } \right)^{ 1 / 2 }
      e^{ \pm i \omega_{ k } t } \sin( k_{ n } x );
    \end{equation}
    \begin{equation}
      \label{eq:O-roznych-07-B}
      \omega_{ n } =
      \left( \frac{ m^{ 2 } c^{ 4 } }{ \hbar^{ 2 } } \right)^{ 1 / 2 }, \quad
      k_{ n } = \frac{ n \pi }{ a }, \quad n = 1, 2, 3, \ldots
    \end{equation}
  \end{subequations}

\end{frame}
% ##################





% ##################
\begin{frame}
  \frametitle{Efekt Casimira -- standardowe podejście}


  Nieskończona energia próżni na odcinku $( 0, a )$
  \begin{equation}
    \label{eq:O-roznych-08}
    E_{ 0 }( a ) = \frac{ \hbar }{ 2 } \sum_{ n = 1 }^{ \infty } \omega_{ n }.
  \end{equation}

  Hendrik Casimir stwierdził, że~nieskończoność nieskończoności nie~równa,
  więc jeśli \textit{nieskończone} energia pustej przestrzeni przypadająca
  na przedział $( 0, a )$, jest różna od \textit{nieskończonej} energii
  między płytami, powinna pojawić~się siła dążąca do korzystniejszej
  konfiguracji energetycznej.

\end{frame}
% ##################





% ##################
\begin{frame}
  \frametitle{Efekt Casimira -- standardowe podejście}


  Aby odjąć dwie nieskończoności, trzeba najpierw uczynić je skończonymi
  \begin{equation}
    \label{eq:O-roznych-09}
    E( \textrm{prostej} ) =
    L \frac{ \hbar }{ 2 \pi } \int\limits_{ 0 }^{ +\infty } dk \, \omega_{ k },
  \end{equation}
  gdzie $L$, to długość prostej. Na rozpatrywanym odcinku
  zgromadzona jest energia:
  \begin{equation}
    \label{eq:O-roznych-10}
    \frac{ a }{ L } E( \textrm{prostej} ) =
    \frac{ \hbar a }{ 2 \pi } \int\limits_{ 0 }^{ +\infty } dk \, \omega_{ k }.
  \end{equation}

  No, już trochę lepiej, ale wciąż jest za duża. Przyjmijmy dla
  prostoty $m = 0$
  \begin{equation}
    \label{eq:O-roznych-11}
    \frac{ \hbar a }{ 2 \pi } \int\limits_{ 0 }^{ +\infty } dk \, \omega_{ k } =
    \lim_{ \delta \rightarrow 0 } \frac{ \hbar c a }{ 2 \pi } \int\limits_{ 0 }^{ +\infty } dk \,
    k e^{ -\delta c k }
    =
    \lim_{ \delta \rightarrow 0 } \frac{ \hbar a }{ 2 \pi c \delta^{ 2 } }.
  \end{equation}

\end{frame}
% ##################





% ##################
\begin{frame}
  \frametitle{Efekt Casimira -- standardowe podejście}


  Potrzebujemy zregularyzować jeszcze jedną nieskończoność
  \begin{equation}
    \label{eq:O-roznych-12}
    E( \textrm{odcinak} ) =
    \frac{ \hbar }{ 2 } \sum_{ n = 1 }^{ \infty } \frac{ c \pi n }{ a } \lim_{ \delta \rightarrow 0 }
    \exp\left( -\frac{ \delta c \pi n }{ a } \right)
  \end{equation}
  Znów $m = 0$.

  Wychodzi
  \begin{equation}
    \label{eq:O-roznych-13}
    E( a ) =
    \lim_{ \delta \rightarrow 0 } [ E( \textrm{odcinek}, \delta)
    - E( \textrm{prosta}, \delta ) ]
    = -\frac{ \pi \hbar c }{ 24 a }.
  \end{equation}

  Co daje siłę:
  \begin{equation}
    \label{eq:O-roznych-14}
    F( a ) =
    -\frac{ \partial^{ 2 } E( a ) }{ \partial a^{ 2 } } =
    -\frac{ \pi \hbar c }{ 24 a^{ 2 } }.
  \end{equation}

\end{frame}
% ##################





% ##################
\begin{frame}
  \frametitle{Efekt Casimira -- standardowe podejście}


  Problemy i~uwagi
  \begin{itemize}

  \item Bardzo dobrze zgadza~się z~rzeczywistością.

  \item Ponieważ liczymy tylko energię zawartą w~naszym odcinku,
    to nie~jest zaskakujące, że~opłaca mu~się kurczyć. To co mogłoby
    być dziwne, to~fakt, iż~ta energia ta dąży do $-\infty$.

  \item Pole swobodne ma~zawsze „więcej” częstości niż~zamknięte
    w~pudle, „dlatego” ma~wyższą energię.

  \end{itemize}

\end{frame}
% ##################

% % \section{Podejście prof. Herdegena}


% % \begin{frame}
% %   \frametitle{Algebraiczna kwantowa teoria pola}

% %   \begin{block}{Matematyczne zawiłości na bok}
% %     Rozpatrywany system jest opisywany przez
% %   \end{block}

% % \end{frame}


% % \usepackage{vmargin}
% % % ---------------------------------------------------
% % MARGINS
% % % ---------------------------------------------------
% % \setmarginsrb { 0.7in} % left margin
% % { 0.6in} % top margin
% % { 0.7in} % right margin
% % { 0.8in} % bottom margin
% % { 20pt} % head height
% % {0.25in} % head sep
% % { 9pt} % foot height
% % { 0.3in} % foot sep
% % \end{verbatim}
% % \end{block}

% % \end{frame}


% % \begin{frame}[fragile]
% %   \frametitle{Ciekawostka: verse}

% %   \begin{block}{Wpisujemy}
% %     {\tiny
% % \begin{verbatim}
% % \settowidth{\versewidth}{Haniebnym marnotrawstwem sił ducha i ciała}
% % \begin{verse}[\versewidth]
% %   \poemlines{20}

% %   Haniebnym marnotrawstwem sił ducha i ciała \\
% %   Jest żądza, gdy osiąga cel; przedtem zaś skłania \\
% %   Do kłamstwa, morderstw, podłości -- zaborcza, zuchwała, \\
% %   Dzika, sroga, niegodna krztyny zaufania; \\
% %   Ledwie rozkosz przyniesie, już czyni ją wstrętem; \\
% %   Ledwie w obłędnych łowach dopadnie zwierzyny -- \\
% %   Z obłędną nienawiścią widzi w niej przynętę, \\
% %   Co szaleństwem skaziła jej plany i czyny: \\
% %   Burzy się, gdy chce posiąść i gdy już posiada; \\
% %   Sama szalona, chętkom folguje szalonym; \\
% %   Jęk szczęścia, ledwie zabrzmiał, przemienia w jęk ,,Biada!'', \\
% %   Marzenie w jednej chwili czyni snem prześnionym. \\
% %   \vin Wszystko to wiemy; lecz choć wiemy, wciąż nas wściekła \\
% %   \vin Żądza, raj obiecując, wiedzie w otchłań piekła.
% % \end{verse}
% % \attribB{W. Shakespeare \emph{Sonet 129}. \\
% %   Tłumaczył S. Barańczak.}
% % \end{verbatim}
% %   }

% %   \end{block}

% % \end{frame}

% % \begin{frame}[fragile]
% %   \frametitle{Ciekawostka: verse}

% %   {\tiny \settowidth{\versewidth}{Haniebnym marnotrawstwem sił ducha
% %   i ciała}
% %   \begin{verse}[\versewidth]
% %     \poemlines{20}

% %     Haniebnym marnotrawstwem sił ducha i ciała \\
% %     Jest żądza, gdy osiąga cel; przedtem zaś skłania \\
% %     Do kłamstwa, morderstw, podłości -- zaborcza, zuchwała, \\
% %     Dzika, sroga, niegodna krztyny zaufania; \\
% %     Ledwie rozkosz przyniesie, już czyni ją wstrętem; \\
% %     Ledwie w obłędnych łowach dopadnie zwierzyny -- \\
% %     Z obłędną nienawiścią widzi w niej przynętę, \\
% %     Co szaleństwem skaziła jej plany i czyny: \\
% %     Burzy się, gdy chce posiąść i gdy już posiada; \\
% %     Sama szalona, chętkom folguje szalonym; \\
% %     Jęk szczęścia, ledwie zabrzmiał, przemienia w jęk ,,Biada!'', \\
% %     Marzenie w jednej chwili czyni snem prześnionym. \\
% %     \vin Wszystko to wiemy; lecz choć wiemy, wciąż nas wściekła \\
% %     \vin Żądza, raj obiecując, wiedzie w otchłań piekła.
% %   \end{verse}
% % }
% %   \attribB{\tiny W. Shakespeare \emph{Sonet 129}. \\
% %   Tłumaczył S. Barańczak.}

% % \end{frame}


% % \section{Praca z~,,kodem''}

% % \begin{frame}
% %   \frametitle{Przykładowy slajd}

% %   \begin{block}{Uwagi o Python}
% %     \begin{itemize}
% %     \item[--] Jest dwustandardowy: 2.7x, 3.x (3.3.x). Wersje te nie
% %       są wstecznie kompatybilne.
% %     \item[--] ,,Rzeczy niebezpieczne mają być utrudnione, lecz nie
% %       zabronione.''
% %     \item[--] Doświadczenie z nauki można łatwo przenieść na wiele
% %       innych języków.
% %     \end{itemize}
% %   \end{block}

% %   \begin{block}{Uwagi o Terrarium}
% %     \begin{itemize}
% %     \item[--] Będzie trochę matematyki.
% %     \item[--] Prowadzący spotkania nie jest najlepszy. Za to ma
% %       oddane wsparcie.
% %     \end{itemize}
% %   \end{block}
% % \end{frame}



% % \begin{frame}[fragile]
% %   \frametitle{Dobrych wcięć nigdy nie~za~wiele}

% %   \begin{block}{,,Kod'' poprzedniego slajdu może wyglądać tak}
% %     {\tiny
% % \begin{verbatim}
% % \begin{frame}
% %   \frametitle{Przykładowy slajd}
% %   \begin{block}{Uwagi o Python}
% %     \begin{itemize}
% %     \item[--] Jest dwustandardowy: 2.7x, 3.x (3.3.x). Wersje te nie
% %       są wstecznie kompatybilne.
% %     \item[--] ,,Rzeczy niebezpieczne mają być utrudnione, lecz nie
% %       zabronione.''
% %     \item[--] Doświadczenie z nauki można łatwo przenieść na wiele
% %       innych języków.
% %     \end{itemize}
% %   \end{block}
% %   \begin{block}{Uwagi o Terrarium}
% %     \begin{itemize}
% %     \item[--] Będzie trochę matematyki.
% %     \item[--] Prowadzący spotkania nie jest najlepszy. Za to ma
% %       oddane wsparcie.
% %     \end{itemize}
% %   \end{block}
% % \end{verbatim}
% % }
% % \end{block}

% % \end{frame}

% % \begin{frame}[fragile]
% %   \frametitle{Dobrych wcięć nigdy nie~za~wiele}

% %   \begin{block}{Albo tak}
% %     {\tiny
% % \begin{verbatim}
% % \begin{frame}
% %   \frametitle{Przykładowy slajd}

% %   \begin{block}{Uwagi o Python}
% %     \begin{itemize}
% %     \item[--] Jest dwustandardowy: 2.7x, 3.x (3.3.x). Wersje te nie
% %       są wstecznie kompatybilne.
% %     \item[--] ,,Rzeczy niebezpieczne mają być utrudnione, lecz nie
% %       zabronione.''
% %     \item[--] Doświadczenie z nauki można łatwo przenieść na wiele
% %       innych języków.
% %     \end{itemize}
% %   \end{block}

% %   \begin{block}{Uwagi o Terrarium}
% %     \begin{itemize}
% %     \item[--] Będzie trochę matematyki.
% %     \item[--] Prowadzący spotkania nie jest najlepszy. Za to ma
% %       oddane wsparcie.
% %     \end{itemize}
% %   \end{block}
% % \end{verbatim}
% % }
% % \end{block}

% % \begin{block}{Edytor może pomóc}
% %   -- Wcięcie ,,kodu'' (dla \TeX Makera): Ctrl-$>$; \\
% %   -- cofnięcie ,,kodu'': Ctrl-$<$.
% % \end{block}

% % \end{frame}

% % \Google{Podstawowa technika usuwania błędów kompilacji}{teal}{

% % \begin{block}{Czytelny ,,kod''}
% %   Pozwala zorientować~się co gdzie jest. Ponieważ często nie umiemy
% %   od razu zlokalizować zbyt źródła najlepiej jest wykomentować
% %   podejrzane partie ,,kodu'', aż~kompilator ruszy. Gdy już wiemy
% %   gdzie jest błąd, jego znalezienie jest prostsze.
% % \end{block}

% % \begin{block}{Edytor może pomóc}
% %   W~\TeX Makerze: \\
% %   -- komentarz: Ctrl-T; \\
% %   -- odkomentowanie: Ctrl-U.
% % \end{block}

% % }

% %   \begin{frame}[fragile]
% %     \frametitle{\LaTeX{} i~\TeX{} to również języki programowania}

% %     \begin{block}{Wszystko co warto wiedzieć o~\LaTeX u} {Da się
% %       {\color{olive} wygooglować}.}
% %     \end{block}

% %     \begin{block}{I~czasem trzeba użyć fragmentu kodu}
% %       \begin{itemize}
% %       \item[--] Pozbyć~się z~tej prezentacji niepotrzebnych dodatków
% %         beamera, które zajmują potrzebne miejsce.
% %         \begin{itemize}
% %         \item[1)] Niepotrzebny spis treści na górze:
% %           \verb+\setbeamertemplate{navigation symbols}{}+

% %         \item[2)] Przez nikogo nie używane ikonki na dole:
% %           \verb+\setbeamertemplate{navigation symbols}{}+.
% %         \end{itemize}
% %         Można było użyć stylu ,,Cracków'', ale~to rozwiązanie było
% %         prostsze.
% %       \item[--] Zwiększyć odstęp między wierszami w~tabeli:
% %         \verb+\renewcommand{\arraystretch}{1.2}+. Wartość~1.0
% %         to~standardowy ostęp.
% %       \end{itemize}



% ##################
\begin{frame}[standout]


  { \color{jFrametitleFGColor} Warto zajmować się fizyką. Jest w~niej
    wiele ciekawych problemów do rozwiązania :). }

  \vspace{5em}



  { \Large \color{jFrametitleFGColor} Dziękuję! Pytania? }

\end{frame}
% ##################










% ##################
\begin{frame}
  \frametitle{Bibliografia}


  \begin{itemize}

  \item R. F. Streater, A. S. Wightman, \textit{PCT, Spin and
    Statistics and All That} -- aksjomatyka Wightmana (zaskoczeni);

  \item R. F. Streater,
    \href{http://www.scholarpedia.org/article/Wightman_quantum_field_theory}
    {\textit{Wightman quantum field theory}};

  \item S. Hollands, R. M. Wald,
    \href{http://arxiv.org/abs/0803.2003}{\textit{Axiomatic quantum field
        theory in~curved spacetime}},
    \href{http://arxiv.org/abs/1401.2026}{\textit{Quantum fields in~curved
        spacetime}} -- QFT na zakrzywionych czasoprzestrzeniach;

  \item K. J. Keller,
    \href{http://arxiv.org/abs/1006.2148}{\textit{Dimensional
        Regularization in~Position Space and~a~Forest Formula for
        Regularized Epstein-Glaser Renormalization}};

  \item M. Duetsch, K. Fredenhagen, K. J. Keller, K. Rejzner,
    \href{http://arxiv.org/abs/1311.5424}{\textit{Dimensional
        Regularization in~Position Space, and a~Forest Formula for
        Epstein-Glaser Renormalization}};

  \item M. Bordag, U. Mohideen, V.M. Mostepanenko,
    \href{http://arxiv.org/abs/quant-ph/0106045}{\textit{New Developments
      in the Casimir Effect}}, standardowe podejście do efektu
    Casimira.

  \end{itemize}

\end{frame}
% ##################





% ##################
\begin{frame}
  \frametitle{Bibliografia}


  \begin{itemize}

  \item R. Haag,
    \href{http://link.springer.com/book/10.1007/978-3-642-61458-3}
    {\textit{Local Quantum Physics. Fields, Particles, Algebras}} --
    algebraiczne sformułowanie teorii pola.
    http://link.springer.com/book/10.1007/978-3-642-61458-3. Można
    pobrać z~komputera UJ, lub logując~się przez Extranet.

  \end{itemize}

\end{frame}
% ##################










% ############################

% Koniec dokumentu
\end{document}