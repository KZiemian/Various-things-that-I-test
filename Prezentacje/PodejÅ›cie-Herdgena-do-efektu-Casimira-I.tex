% ---------------------------------------------------------------------
% Basic configuration of Beamera and Jagiellonian
% ---------------------------------------------------------------------
\RequirePackage[l2tabu, orthodox]{nag}



\ifx\PresentationStyle\notset
\def\PresentationStyle{dark}
\fi



\documentclass[10pt,t]{beamer}
\mode<presentation>
\usetheme[style=\PresentationStyle,logoLang=Latin,logoColor=monochromaticJUwhite,JUlogotitle=yes]{jagiellonian}



% ---------------------------------------
% Configuration files of Jagiellonian loceted in catalog preambule
% ---------------------------------------
\input{./preambule/LanguageSettings/JagiellonianPolishLanguageSettings.tex}
\input{./preambule/TextposConfiguration/TextposConfiguration.tex}

\input{./preambule/JagiellonianCustomizationGeneral.tex}
\input{./preambule/JagiellonianCustomizationCommands.tex}










% ---------------------------------------
% Packages, libraries and their configuration
% ---------------------------------------
\usepackage{mathcommands}





% ---------------------------------------
% Configuration for this particular presentation
% ---------------------------------------










% ---------------------------------------------------------------------
\title{Podejście Herdegena do efektu Casimira}

\author{Kamil Ziemian}
\institute{II rok, fizyka teoretyczna, studia magisterskie.}
\date[6 VI 2013]{6 czerwca 2013 r.}
% ---------------------------------------------------------------------










% ####################################################################
% Początek dokumentu
\begin{document}
% ####################################################################





% Wyrównanie do lewej z łamaniem wyrazów

\RaggedRight





% ######################################
\maketitle % Tytuł całego tekstu
% ######################################





% ##################
\begin{frame}
  \frametitle{Podstawy matematyczne}


  Algebry $C^{ * }$.
  Niech $\Acal$ będzie przestrzenią Banacha, z normą $\Vert \cdot \Vert$. Potrzebujemy
  określić na niej dwa nowe działania. Zob.
  \cite{BratteliRobinsonOperatorAlgebrasVolI2002}.

  Pierwszym z tych działań jest mnożenie elementów algebry:
  \begin{equation}
    \label{eq:1}
    ( A, B ) \rightarrow A B \in \Acal.
  \end{equation}
  \begin{itemize}

  \item Działanie to jest łączne i dwuliniowe.

  \item Spełnia warunek $\Vert A B \Vert \leq \Vert A \Vert \, \Vert B \Vert$.

  \end{itemize}

\end{frame}
% ##################





% ##################
\begin{frame}{Podstawy matematyczne}


  Drugim potrzebnym nam działaniem jest inwolucja (sprzężenie) elementu
  algebry:
  \begin{equation}
    \label{eq:2}
    * : A \mapsto A^{ * } \in \Acal.
  \end{equation}

  \begin{itemize}

  \item $A^{ ** } = A$;

  \item $( A B )^{ * } = B^{ * } A^{ * }$;

  \item $( \alpha A + \beta B )^{ * } = \bar{ \alpha } A^{ * } + \bar{ \beta } B^{ * }$;

  \item $\Vert A^{ * } A \Vert = \Vert A \Vert^{ 2 }$.

  \end{itemize}

\end{frame}
% ##################





% ##################
\begin{frame}{Podstawy matematyczne}


  Reprezentacja algebry $C^{ * }$. Niech $\Hcal$ będzie przestrzenią
  Hilberta, zaś $\Lcal( \Hcal )$ będzie zbiorem operatorów określonym na tej przestrzeni. Nie będziemy się zagłębiać jaka dokładnie jest natura tych operatorów, czy są ograniczone, czy nieograniczone, etc.

  Reprezentacją $C^{ * }$-algebry nazywamy odwzorowanie:
  \begin{equation}
    \label{eq:3}
    \pi : \Acal \ni A \rightarrow \pi( A ) \in \Lcal( \Hcal ),
  \end{equation}
  które spełnia następujące warunki.
  \begin{itemize}
  \item $\pi( \alpha A + \beta B ) = \alpha \pi( A ) + \beta \pi( B )$;

  \item $\pi( A B ) = \pi( A ) \pi( B )$;

  \item $\pi( A^{ * } ) = \pi( A )^{ * }$.

  \end{itemize}
  Sprzężenie operatora również oznaczamy symbolem $*$.

\end{frame}
% ##################





% ##################
\begin{frame}
  \frametitle{Algebraiczne sformułowanie teorii kwantowej}


  Każdy układ kwantowy będzie opisywany przez
  \cite{HaagLocalQuantumPhysics1996}:
  \begin{itemize}

  \item algebrę $C^{ * }$, opisującej wielkości fizyczne;

  \item jej reprezentacje $( \Hcal, \pi )$;

  \item prawo ewolucji czasowej obowiązujące na $\Acal$.

  \end{itemize}


  \textbf{Filozoficzne motto tego podejścia} \\
  Każdy problem fizyczny właściwie ujęty w~ramy matematyki, powinien dawać
  sensowny, skończony i jednoznaczny wynik. Jeśli wynik, np. masa elektronu,
  nie jest skończony to problem jest źle postawiony.

\end{frame}
% ##################





% ##################
\begin{frame}
  \frametitle{Efekt Casimira}


  Standardowo efekt ten opisuje w następując sposób.
  Siłę Casimira otrzymujemy odejmując dwie nieskończone wielkości,
  wynikające z potrzeby uporządkowania normalnego dwu różnych
  hamiltonianów. Kanoniczny przykład: pole swobodne i pole z warunkami
  Dirichleta.

  W~języku algebraicznej kwantowej teorii użytym przez Herdegena formułujemy
  problem w następujący sposób. Dla problemu pola swobodnego i pola wraz
  z~obecnym ciałem makroskopowym budujemy ich $C^{ * }$ algebry wraz z ich
  reprezentacjami w oparciu o operatory: $h^{ 2 } = - \Delta$ i $h^{ 2 }_{ a }$.
  Możliwe sposoby konstrukcji $h^{ 2 }_{ a }$ podane są w pracach
  \cite{HerdegenQuantumBackreationPartI2005}
  i~\cite{HerdegenStopaGlobalVsLocal2010}.

  Pojawia się problem. W~celu porównania wartości fizycznych zarówno
  algebry, jak ich reprezentacje muszą być „identyczne”.

\end{frame}
% ##################





% ##################
\begin{frame}
  \frametitle{Warunki spójności}


  W~celu zapewnienia równoważności tych dwu sytuacji muszą być spełnione
  poniższe równość.
  \begin{equation}
    \label{eq:3}
    N_{ a } =
    \Tr[ h^{ -1/2 } ( h_{ a } - h ) h_{ a }^ { -1 } ( h_{ a } - h ) h^{ -1/2 } ]
    < +\infty .
  \end{equation}

  Postulujemy jeszcze????
  \begin{equation}
    \label{eq:34}
    E_{ a } =
    \frac{ 1 }{ 4 } \Tr[ ( h_{ a } - h ) h_{ a }^ { -1 } ( h_{ a } - h ) ]
    < +\infty ,
  \end{equation}


  Można teraz wprowadzając odpowiednie skalowanie można odtworzyć w~granicy
  warunki Dirichleta i zbadać czy otrzymany tym sposobem wynik daje
  odpowiedź pokrywającą się z wynikiem z perturbacyjnej QFT.

\end{frame}
% ##################










% ######################################
\EndingSlide{Dziękuję! Pytania?}
% ######################################










% ##################
\begin{frame}
  \frametitle{Bibliografia}


  \bibliographystyle{plalpha}

  \bibliography{MathComScienceBooks,PhilNaturArticles,PhilNaturBooks}{}

\end{frame}
% ##################










% ############################

% Koniec dokumentu
\end{document}