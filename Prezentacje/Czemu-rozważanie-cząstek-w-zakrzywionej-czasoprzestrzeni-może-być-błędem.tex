% ---------------------------------------------------------------------
% Basic configuration of Beamera and Jagiellonian
% ---------------------------------------------------------------------
\RequirePackage[l2tabu, orthodox]{nag}



\ifx\PresentationStyle\notset
\def\PresentationStyle{dark}
\fi



\documentclass[10pt,t]{beamer}
\mode<presentation>
\usetheme[style=\PresentationStyle,logoLang=Latin,logoColor=monochromaticJUwhite,JUlogotitle=yes]{jagiellonian}



% ---------------------------------------
% Configuration files of Jagiellonian loceted in catalog preambule
% ---------------------------------------
\input{./preambule/LanguageSettings/JagiellonianPolishLanguageSettings.tex}
\input{./preambule/TextposConfiguration/TextposConfiguration.tex}

\input{./preambule/JagiellonianCustomizationGeneral.tex}
\input{./preambule/JagiellonianCustomizationCommands.tex}










% ---------------------------------------
% Packages, libraries and their configuration
% ---------------------------------------
\usepackage{mathcommands}





% ---------------------------------------
% Configuration for this particular presentation
% ---------------------------------------










% ---------------------------------------------------------------------
\title{Czemu rozważanie cząstek w~zakrzywionej czasoprzestrzeni może być
  błędem?}

\author{Kamil Ziemian \\
  \texttt{kziemianfvt@gmail.com} }

% ---------------------------------------------------------------------










% ####################################################################
% Początek dokumentu
\begin{document}
% ####################################################################





% Wyrównanie do lewej z łamaniem wyrazów

\RaggedRight





% ######################################
\maketitle
% ######################################





% % ######################################
% \begin{frame}
%   \frametitle{Spis treści}


%   \tableofcontents % Spis treści

% \end{frame}
% % ######################################





% ##################
\begin{frame}
  \frametitle{Co to jest cząstka?}


  To proste: nieskończenie mała kulka poruszając~się w~przestrzeni.
  Cóż może być bardziej naturalnego?

  A~bardziej na poważnie. Koncepcja cząstki punktowej jest niezwykle
  użyteczna w~wielu rozważaniach fizycznych, ale nie powinno nam to
  przysłaniać faktu, że potrafi też prowadzić do poważnych problemów.

\end{frame}
% ##################





% ##################
\begin{frame}
  \frametitle{Klasyczne cząstki w~OTW}


  Jeśli przyjmiemy, że~klasyczne cząstki punktowe nie dają wkładu do
  tensora energii-pędu, to nie ma żadnego problemu, by~rozważać
  klasyczny foton czy elektron na zakrzywionej czasoprzestrzeni.

  Przez klasyczny foton rozumiem cząstkę punktową o~zerowej masie,
  której czteroprędkość jest zawsze wektorem świetlnym. Analogicznie
  elektron.

  Jeśli jednak będziemy chcieli uwzględnić wkład cząstki punktowej do
  pola grawitacyjnego, to spodziewamy~się, że~cząstka punktowa będzie
  czarną dziurą albo nagą osobliwością. Poza tym o~problemie niewiele
  da się powiedzieć.

  Jeśli jednak weźmiemy cząstkę kwantową, to czy nawet takie niedające
  wkładu do tensora energii-pędu mają sens?

\end{frame}
% ##################





% ##################
\begin{frame}
  \frametitle{Uwaga techniczna}


  Istnieje tylko \alert{jedna} przestrzeń Hilberta $\Hcal$ dla
  nierelatywistycznej mechaniki kwantowej. Dla kwantowej teorii pola
  jest ich nieprzeliczalnie wiele, albo więcej.

  Trochę bardziej ściśle. Twierdzenie Stone'a-von Neumanna głosi, że
  mechanika kwantowa żyje w~ośrodkowej przestrzeni Hilberta, a~ta jest
  tylko jedna (z~dokładnością do transformacji unitarnych).

  Kwantowej teorii pola nie można sformułować w~ośrodkowej przestrzeni
  Hilberta, a~przestrzeni nieośrodkowych znamy już nieprzeliczalnie
  wiele.

\end{frame}
% ##################





% ##################
\begin{frame}
  \frametitle{Zostańmy przy fotonach}


  Elektrodynamika Clerka Maxwella bez ładunków i~prądów jest teorią
  czysto geometryczną, więc jej przeniesienie do OTW nie stwarza
  problemu.

  Pole zapisujemy za pomocą tensora Clerka Maxwell.
  \begin{equation}
    \label{eq:Czemu-rozwazanie-01}
    F_{ \mu \nu }( x )
    =
    \begin{pmatrix}
      \hphantom{-} 0
      & \frac{ 1 }{ c } E_{ x }( x ) & \frac{ 1 }{ c } E_{ y }( x )
      & \frac{ 1 }{ c } E_{ z }( x ) \\
      -\frac{ 1 }{ c } E_{ x }( x )
      & \hphantom{-} 0 & -B_{ z }( x ) & \hphantom{-} B_{ y }( x ) \\
      -\frac{ 1 }{ c } E_{ y }( x )
      & \hphantom{-} B_{ z }( x ) & \hphantom{-} 0 & -B_{ z }( x ) \\
      -\frac{ 1 }{ c } E_{ z }( x )
      & -B_{ y }( x ) & \hphantom{-} B_{ x }( x ) & \hphantom{-} 0
    \end{pmatrix}
  \end{equation}
  Jeśli wprowadzimy czteroformę pola
  $F( x ) = F_{ \mu \nu }( x ) \, dx^{ \mu } \wedge dx^{ \nu }$
  i~czteroformę potencjału $A( x ) = A_{ \mu }( x ) \, dx^{ \mu }$ to
  równania Clerka Maxwella przyjmują formę (trochę upraszczam)
  \begin{align*}
    F( x ) = dA( x ), \quad dF( x ) = 0.
  \end{align*}
  Pochodna zewnętrzna $d$ istnieje na dowolnej gładkiej rozmaitości,
  nawet takiej bez tensora metrycznego.

\end{frame}
% ##################





% ##################
\begin{frame}
  \frametitle{Zostańmy przy fotonach}


  \begin{subequations}
    \begin{align}
      \label{eq:Czemu-rozwazanie-02-A}
      &F( x ) = dA( x ), \quad dF( x ) = 0, \quad
        F( x ) = F_{ \mu \nu }( x ) \, dx^{ \mu } \wedge dx^{ \nu }, \\
      \label{eq:Czemu-rozwazanie-02-B}
      &F_{ \mu \nu }( x )
        =
        \begin{pmatrix}
          \hphantom{-} 0
          & \frac{ 1 }{ c } E_{ x }( x ) & \frac{ 1 }{ c } E_{ y }( x )
          & \frac{ 1 }{ c } E_{ z }( x ) \\
          -\frac{ 1 }{ c } E_{ x }( x )
          & \hphantom{-} 0 & -B_{ z }( x ) & \hphantom{-} B_{ y }( x ) \\
          -\frac{ 1 }{ c } E_{ y }( x )
          & \hphantom{-} B_{ z }( x ) & \hphantom{-} 0 & -B_{ z }( x ) \\
          -\frac{ 1 }{ c } E_{ z }( x )
          & -B_{ y }( x ) & \hphantom{-} B_{ x }( x ) & \hphantom{-} 0
        \end{pmatrix}
    \end{align}
  \end{subequations}

  Te równania definiują nam elektrodynamikę na dowolnej gładkiej
  rozmaitości. Teraz potrzeba je skwantować.

  I tu zaczynają się problemy. Będę argumentował, że standardowa
  kwantyzacja pola pole elektromagnetycznego silnie korzysta
  z~własności czasoprzestrzeni Minkowskiego.

\end{frame}
% ##################





% ##################
\begin{frame}
  \frametitle{Uproszczona kwantyzacja w~czasoprzestrzeni Minkowskiego}


  1. Wybierz inercjalny układ współrzędnych.

  2. Wypisz w~nim równania Clerka Maxwella.
  \begin{equation}
    \label{eq:Czemu-rozwazanie-03}
    \square \, A_{ \mu }( x ) = 0, \quad \partial^{ \nu } A_{ \nu } = 0
  \end{equation}

  \vspace{-1.5em}



  3. Zapisz rozwiązanie równań Clerka Maxwella jako
  \begin{equation}
    \label{eq:Czemu-rozwazanie-04}
    \begin{split}
      A( x )
      &=
        \int d^{ 3 }k \sum_{ \mu = \pm 1 } \frac{ 1 }{ \omega( \vec{ k } ) }
        \left( \vec{ e }^{ ( \mu ) }( \vec{ k } ) a_{ \vec{ k } }^{ ( \mu ) }( t )
        e^{ -i \omega( \vec{ k } ) t + i \vec{ k } \cdot \vec{ x } } \right. \\
      &\hspace{1.5em}
        \left. + \, \overline{ \vece^{ ( \mu ) } }\!\left( \vec{ k } \right)
        \overline{ a }_{ \vec{ k } }^{ ( \mu ) }( t )
        e^{ i \omega( \vec{ k } ) t - i \vec{ k } \cdot \vec{ x } } \right).
    \end{split}
  \end{equation}

  \vspace{-1em}



  4. Podnieś $A( x )$, $a_{ \vec{ k } }^{ ( \mu ) }$ do rangi
  operatorów.

\end{frame}
% ##################





% ##################
\begin{frame}
  \frametitle{Uproszczona kwantyzacja w~czasoprzestrzeni Minkowskiego}


  \begin{equation}
    \label{eq:Czemu-rozwazanie-05}
    \begin{split}
      A( x )
      &=
        \int d^{ 3 }k \sum_{ \mu = \pm 1 } \frac{ 1 }{ \omega( \vec{ k } ) }
        \left( \vec{ e }^{ ( \mu ) }( \vec{ k } ) a_{ \vec{ k } }^{ ( \mu ) }( t )
        e^{ -i \omega( \vec{ k } ) t + i \vec{ k } \cdot \vec{ x } } \right. \\
      &\hspace{1.5em}
        \left. + \, \overline{ \vece^{ ( \mu ) } }\!\left( \vec{ k } \right)
        \overline{ a }_{ \vec{ k } }^{ ( \mu ) }( t )
        e^{ i \omega( \vec{ k } ) t - i \vec{ k } \cdot \vec{ x } } \right).
    \end{split}
  \end{equation}

  \vspace{-1.5em}



  5. Zinterpretuj operator przy
  $\exp( -i \omega( \vec{ k } ) t + i \vec{ k } \cdot \vec{ x } )$
  jako operator anihilacji. Znajdź więc stan, taki że
  \begin{equation}
    \label{eq:Czemu-rozwazanie-06}
    a_{ \vec{ k } }^{ ( \mu ) }( t ) | 0 \rangle = 0, \quad
    \forall \, \vec{ k }, \mu.
  \end{equation}

  \vspace{-1em}



  6. Zbuduj przestrzeń Hilberta ze stanów zawierających 0, 1, 2, 3, \ldots,
  cząstek.
  \begin{equation}
    \label{eq:Czemu-rozwazanie-07}
    | 0 \rangle, a^{ \dagger \, ( \mu ) }_{ \vec{ k } }( t ) | 0 \rangle,
    a^{ \dagger \, ( \mu ) }_{ \vec{ k } }( t ) a^{ \dagger \, ( \nu ) }_{ \vec{ k }' }( t )
    | 0 \rangle,
    a^{ \dagger \, ( \mu ) }_{ \vec{ k } }( t ) a^{ \dagger \, ( \nu ) }_{ \vec{ k }' }( t )
    a^{ \dagger \, ( \rho ) }_{ \vec{ k }'' }( t ) | 0 \rangle, \ldots
  \end{equation}

\end{frame}
% ##################





% ##################
\begin{frame}
  \frametitle{Co może pójść nie tak?}


  \begin{equation}
    \label{eq:Czemu-rozwazanie-08}
    a_{ \vec{ k } }^{ ( \mu ) }( t ) | 0 \rangle = 0, \quad
    \forall \, \vec{ k }, \mu
  \end{equation}
  Jest zbyt wiele stanów spełniających tą relację. Wybór każdego z
  nich daje inną przestrzeń Hilberta budowanej wedle przepisu
  \begin{equation}
    \label{eq:Czemu-rozwazanie-09}
    | 0 \rangle,
    a^{ \dagger \, ( \mu ) }_{ \vec{ k } }( t ) | 0 \rangle,
    a^{ \dagger \, ( \mu ) }_{ \vec{ k } }( t ) a^{ \dagger \, ( \nu ) }_{ \vec{ k }' }( t )
    | 0 \rangle,
    a^{ \dagger \, ( \mu ) }_{ \vec{ k } }( t ) a^{ \dagger \, ( \nu ) }_{ \vec{ k }' }( t )
    a^{ \dagger \, ( \rho ) }_{ \vec{ k }'' }( t ) | 0 \rangle, \ldots
  \end{equation}
  Każda taka przestrzeń to inna fizyka.

  Ale istnieje tylko \alert{jeden} stan $| 0 \rangle$ który jest taki sam
  w~każdym inercjalnym układzie współrzędnych. Inaczej mówiąc, który
  spełnia relację:
  \begin{equation}
    \label{eq:Czemu-rozwazanie-10}
    \widehat{U}( L, a ) | 0 \rangle = | 0 \rangle,
  \end{equation}
  gdzie $L$ to transformacja Lorentza, a~$a$ to czterowektor
  translacji.

\end{frame}
% ##################





% ##################
\begin{frame}
  \frametitle{Trochę szczegółów}


  Pole kwantowe $\widehat{A}( x )$ wygląda tak samo w~każdym
  inercjalnym układzie współrzędnych i~istnieje tylko jeden stan
  $| 0 \rangle$, który wygląda tak samo w~każdym z~tych układów.

  Mówiąc prościej, może nawet za prosto. Jest tylko jeden „stan
  próżni”, który jest pusty w~\alert{każdym} inercjalnym układzie
  odniesienia. Pozostałe są puste w~jednych, a~zawierają cząstki
  w~innych. „Dziwne stany próżni”.

  Dobrze nam znane pojęcie cząstki jest uratowane!

\end{frame}
% ##################





% ##################
\begin{frame}
  \frametitle{Teraz widać co się psuje}


  Na zakrzywionej czasoprzestrzeni nie można liczyć na istnienie grupy
  symetria tak dużej, jak grupa Poincar\'{e}’ego. Tym samym nie jest
  możliwe znalezienie stanu próżni który jest wszędzie pusty. Bardzo
  dziwne.

  Dwa poważne zastrzeżenia.
  \begin{enumerate}
    \RaggedRight

  \item Przecież cząstki, które znamy się taki nigdy nie zachowują,
    więc to musi być błąd w~teorii.

  \item To jest niuans matematyczny, bez znaczenia fizycznego.

  \end{enumerate}

\end{frame}
% ##################





% ##################
\begin{frame}
  \frametitle{Komentarz do 1}


  Ze względu na naturę grawitacji większość obserwowanej
  czasoprzestrzeni jest praktycznie płaska. Brak obserwacji tego typu
  zachowań nie powinien więc nas dziwić bardziej, niż brak
  obserwowalnego wpływu dylatacji czasu na zegarek który noszę na
  ręce.

  Powinno nam wystarczyć więc, że teoria zagwarantuje nam odtwarzanie
  fizyki cząstek w~granicy płaskiej czasoprzestrzeni.

\end{frame}
% ##################





% ##################
\begin{frame}
  \frametitle{Komentarz do 2}


  Brak istnienia stanu uniwersalnej próżni stoi za słynnym zjawiskiem
  promieniowania czarnych dziur (słynny, bo miało dobry marketing).

  Obliczenia ilościowe są skomplikowane, ale można ten efekt można
  podsumować w~następujący sposób. Jeśli wybierzemy jeden „stan
  próżni” to jeśli będzie on pusty w~układzie odniesienia
  nieskończonej przeszłości czarnej dziury, to nie będzie pusty
  w~układzie odniesienia nieskończonej przyszłości.

  A~może po prostu znaleźć taki stan, który jest pusty w~każdym
  układzie odniesienia?

\end{frame}
% ##################





% ##################
\begin{frame}
  \frametitle{Komentarz do 2}


  W~czasoprzestrzenni Minkowskiego zbiór operatorów anihilacji
  $a^{ ( \mu ) }_{ \vec{ k } }( t )$ wszystkich układów inercjalnych
  jest stosunkowo mały. Ogranicza go warunek równoważny
  niezmienniczości próżni:
  \begin{equation}
    \label{eq:Czemu-rozwazanie-10}
    U^{ \dagger }( L, a ) a^{ ( \mu ) }_{ \vec{ k } }( t ) U( L, a )
    =
    a^{ ( \mu ) }_{ \vec{ k }' }( t' ),
  \end{equation}
  przy czym $( L, a )$ przekształca $t$, $\vec{ k }$ na $t'$,
  $\vec{ k }'$.

  W~zakrzywionej czasoprzestrzenie zwykle nie będzie żadnego takiego
  ograniczenia, więc klasa operatorów anihilacji będzie zbyt duża, by
  znaleźć dla nich wspólną próżnię.

\end{frame}
% ##################





% ##################
\begin{frame}
  \frametitle{Trochę radykalnych wniosków}


  Pojęcie cząstki kwantowej ma sens w układach inercjalnych w~płaskiej
  czasoprzestrzeni.

  W~czasoprzestrzeni zakrzywionej mamy stary problem z~filmu
  „Poszukiwany-poszukiwana”: ile cząstki w~cząstce kwantowej? Można
  postawić pytanie: czy nazywanie takiego obiektu cząstką to nie jest
  tylko rezultat bezwładności języka?

  Skoro stan próżni $| 0 \rangle$ określa nam przestrzeń Hilberta, to bez
  dobrego stanu próżni niego nie mamy dobrej przestrzeni. Czyli
  z~wektorów stanu i~operatorów też.

  Możemy jednak sięgnąć do nienowego podejścia do fizyki kwantowej,
  która pozwala przynajmniej obejść te problemy.

\end{frame}
% ##################





% ##################
\begin{frame}
  \frametitle{Szkic podejścia}


  Fundamentalnymi obiektami są ustalona czasoprzestrzeń, pomiary
  w~skończonych czasowo oraz przestrzenie obszarach i~relacje
  algebraiczne między mierzonymi obiektami. Plus odpowiednie warunki
  spójności.

  Przez relacje algebraiczne rozumiemy relacje odpowiednie komutacji:
  między pędem i~położeniem, $a$ i~$a^{ \dagger }$, etc. Przy tym
  traktujemy je po prostu jako relacje w~sensie teorii grup,
  pierścieni i~ciał, bez odniesienia do przestrzeni Hilberta.

  Postulujemy, że możemy zmierzyć daną wielkość w skończonym obszarze
  przestrzeni. Na poziomie matematyki oznacza to podanie odpowiedniego
  elementu przestrzeni dualnej do przestrzeni naszych relacji
  komutacji.

\end{frame}
% ##################





% ##################
\begin{frame}
  \frametitle{Szkic podejścia}


  Konstrukcje takie jak GNS (Gelfand-Naimark-Segal), gwarantują nam,
  że~dla takich lokalnych pomiarów istnieje przestrzeń Hilberta,
  wektor stanu i~operatory, które odtwarzają cały znany nam formalizm.
  To jednak nie pozwala wyciągać wniosków globalnych, chyba że dla
  płaskiej czasoprzestrzeni.

\end{frame}
% ##################





% ##################
\begin{frame}
  \frametitle{Pola i cząstki}


  Powszechnie wiadomo, że pola kwantowe można zbudować z~cząstek,
  a~cząstki z~pól kwantowych. Przynajmniej w czasoprzestrzeni
  Minkowskiego. Ze względu na wcześniej wymienione problemy możemy
  porzucić tą starożytną mądrość i~rozumować następująco.

  Fundamentalnym obiektem fizyczny są wielkości mierzalne pól
  kwantowych, które zapisujemy jako odpowiednią lokalną algebrę
  relacji komutacji wedle opisu przedstawionego poprzednio. Przez
  lokalną algebrę rozumiem to, że relacje komutacji są określone dla
  pomiarów w danym skończonym obszarze. Porównywanie pomiarów między
  „dalekimi” zbiorami to temat na inny dzień.

  Przestrzeń Hilberta i jej formalizm staje się pochodnym aparatem
  teorii, który ma lokalny sens.

\end{frame}
% ##################





% ##################
\begin{frame}
  \frametitle{Pola i~cząstki}


  Bardziej szczegółowe omówienie tego podejścia przekracza obecnie
  moje możliwości.

  Czy to dobra droga? W~tej chwili nie umiem odpowiedzieć.

\end{frame}
% ##################










% ##################
\begin{frame}
  \frametitle{Bibliografia}


  K. Fredenhagen, \textit{The impact of~the~algebraic approach on
    perturbative quantum field theory}, [Fre09]. Wystąpienie na
  \textit{Algebraic Quantum Field Theory – the first 50 Years},
  G\"{o}ttingen 2009.

  R. M. Wald, \textit{Axiomatic Quantum Field Theory in~Curved
    Spacetime}, [Wal09]. Wystąpienie na \textit{Algebraic Quantum
    Field Theory – the first 50 Years}, G\"{o}ttingen 2009.

  R. Haag, \textit{Local Algebras: A~look back at the~early years
    and~at~some successes and~missed opportunitie} [Haa09].
  Wystąpienie na \textit{Algebraic Quantum Field Theory – the first 50
    Years}, G\"{o}ttingen 2009.

  K. Fredenhagen, K. Rejzner, \textit{Perturbative algebraic quantum
    field theory}, arXiv: 1208.1428, [FR12].

\end{frame}
% ##################





% ##################
\begin{frame}
  \frametitle{Bibliografia}


  R. Brunetti, K. Fredenhagen, \textit{Microlocal Analysis
    and~Interacting Quantum Field Theories: Renormalization
    on~Physical Backgrounds}, Commun.Math.Phys, \textbf{208} (2000)
  623-661, arXiv: 9903.028, [BF00].

  R. Brunetti, K. Fredenhagen, \textit{Quantum Field Theory on~Curved
    Backgrounds}, Proceedings of the Kompaktkurs „Quantenfeldtheorie
  auf gekruemmten Raumzeiten” held at~Universitaet Potsdam, Germany,
  in 8.-12.10.2007, arXiv: 0901.2063, [BF09].

  M. D\"{u}etsch, K. Fredenhagen, K. J. Keller, K.~Rejzner,
  \textit{Dimensional Regularization in Position Space, and~a Forest
    Formula for Epstein-Glaser Renormalization}, arXiv: 1311.5424,
  [DFKR13].

\end{frame}
% ##################










% ######################################
\appendix
% ######################################





% ######################################
\EndingSlide{Dziękuję! Pytania?}
% ######################################










% ####################################################################
% ####################################################################
% Bibliografia
% \bibliographystyle{plalpha}

% \bibliography{}{}





% ############################

% Koniec dokumentu
\end{document}
