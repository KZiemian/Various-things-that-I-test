% ####################################################################
% Author: Kamil Ziemian
% ####################################################################

% ---------------------------------------------------------------------
% Basic configuraton of LaTeX document and Polish language
% ---------------------------------------------------------------------
\RequirePackage[l2tabu, orthodox]{nag}  % Find out deprecated part of LaTeX.
% More information in l2tabu English version.


\documentclass[a4paper,11pt]{article}
% {rozmiar papieru, rozmiar fontu}[klasa dokumentu]










% ---------------------------------------
% MARGINS
% ---------------------------------------
\usepackage{vmargin}  % Pozwala na prostą kontrolę rozmiaru marginesów,
% za pomocą komend poniżej. Rozmiar odstępów jest mierzony w calach.
\setmarginsrb
{ 0.7in}  % left margin
{ 0.6in}  % top margin
{ 0.7in}  % right margin
{ 0.8in}  % bottom margin
{  20pt}  % head height
{0.25in}  % head sep
{   9pt}  % foot height
{ 0.3in}  % foot sep







% ---------------------------------------
% Configuration for this particular file
% ---------------------------------------
\usepackage{xcolor}  % Packages enabling use of many color models and many
% additional colors.










% ---------------------------------------
% Packages "hyperref"
% They say that you should put it at the end of the preamble
% ---------------------------------------
\usepackage{hyperref}  % Pozwala tworzyć hiperlinki i zamienia odwołania
% do bibliografii na hiperlinki.










% ---------------------------------------------------------------------
% Tytuł, autor, data
\title{\href{https://repo.skni.umcs.pl/ctan/macros/latex/contrib/xcolor/xcolor.pdf}{\texttt{xcolor}} testy \\
  s. 01, 04}

\author{}
% \date{}
% ---------------------------------------------------------------------










% ####################################################################
\begin{document}
% ####################################################################





% ######################################
\maketitle % Tytuł całego tekstu
% ######################################





\extractcolorspec{red}{\varOne}
\noindent
\varOne

\extractcolorspec{yellow}{\varTwo}
\noindent
\varTwo

\extractcolorspec{magenta}{\varThree}
\noindent
\varThree

\extractcolorspec{violet}{\varFour}
\noindent
\varFour

\extractcolorspecs{red}{\varModelOne}{\varColorOne}
\noindent
\varModelOne: \varColorOne

\extractcolorspecs{yellow}{\varModelTwo}{\varColorTwo}
\noindent
\varModelTwo: \varColorTwo

\extractcolorspecs{magenta}{\varModelThree}{\varColorThree}
\noindent
\varModelThree: \varColorThree

\extractcolorspecs{violet}{\varModelFour}{\varColorFour}
\noindent
\varModelFour: \varColorFour


\convertcolorspec{cmyk}{0.81,1,0,0.07}{HTML}{\tmp}
\noindent
\textbackslash tmp: \tmp










% ####################################################################
% ####################################################################

% Koniec dokumentu
\end{document}